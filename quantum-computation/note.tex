\documentclass[UTF8]{ctexart}

\usepackage{braket}
\usepackage{amsmath}

\begin{document}

\title{量子计算简介}
\author{%
Le Yang\\
yangle0125@qq.com
}
\date{}
\maketitle

%\tableofcontents


\section{量子比特}

经典比特有一个状态,或者是0,或者是1。

量子比特也有一个状态,其可能的两种取值为$\ket{0}$,$\ket{1}$。

量子比特与经典比特的区别在于,量子比特可以是上述两个状态的线性组合,称为叠加态(superposition)。例如

\begin{equation}
\ket{\psi} = \alpha\ket{0} + \beta\ket{1}
\end{equation}

其中$\alpha$和$\beta$是复数。在测量量子比特时,我们得到0的概率是$|\alpha|^2$;得到1的概率是$|\beta|^2$。

因此我们有$|\alpha|^2 + |\beta|^2 = 1$。

一个今后常用的状态称之为$\ket{+}$,其定义为

\begin{equation}
\ket{+} = \frac{1}{\sqrt{2}}\ket{0} + \frac{1}{\sqrt{2}}\ket{1}
\end{equation}

因为$\alpha$和$\beta$是模长和为1的复数,假定其幅角之差为$\phi$,我们就可以将两者写作$\alpha = \cos\frac{\theta}{2}e^{i\delta}$, $\beta = \sin\frac{\theta}{2}e^{i(\delta + \phi)}$。故而

\begin{equation}
\begin{aligned}
\ket{\psi} &= \cos\frac{\theta}{2}e^{i\delta}\ket{0} + \sin\frac{\theta}{2}e^{i(\delta + \phi)}\ket{1} \\
&= e^{i\delta}(\cos\frac{\theta}{2}\ket{0} + \sin\frac{\theta}{2}e^{i\phi}\ket{1})
\end{aligned}
\end{equation}

\begin{figure}
  \centering
  \input{BlochSphere.pdf_tex}
  \caption{Bloch Sphere}
\end{figure}

其中$e^{i\delta}$,称作共同相位(global phase)。因为对$\ket{0}$、对$\ket{1}$都一样影响,不具有观测效应,故可以将之舍弃不看。

至于相对相位(relative phase)$e^{i\phi}$,就不同了,它的影响可以在球面上表现出来。故得

\begin{equation}
\ket{\psi} = \cos\frac{\theta}{2}\ket{0} + \sin\frac{\theta}{2}e^{i\phi}\ket{1}
\end{equation}

\end{document}

