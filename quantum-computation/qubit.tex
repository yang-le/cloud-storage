\section{量子比特}

\subsection{单量子比特}

经典比特有一个状态,或者是0,或者是1。

量子比特也有一个状态,其可能的两种取值为$\ket{0}$,$\ket{1}$。

量子比特与经典比特的区别在于,量子比特可以是上述两个状态的线性组合,称为叠加态(superposition)。例如

\begin{equation}
\ket{\psi} = \alpha\ket{0} + \beta\ket{1}
\end{equation}

其中$\alpha$和$\beta$是复数。在测量量子比特时,我们得到0的概率是$|\alpha|^2$;得到1的概率是$|\beta|^2$。

因此我们有$|\alpha|^2 + |\beta|^2 = 1$。

一个今后常用的状态称之为$\ket{+}$,其定义为

\begin{equation}
\ket{+} = \frac{1}{\sqrt{2}}\ket{0} + \frac{1}{\sqrt{2}}\ket{1}
\end{equation}

因为$\alpha$和$\beta$是模长和为1的复数,假定其幅角之差为$\phi$,我们就可以将两者写作$\alpha = \cos\frac{\theta}{2}e^{i\delta}$, $\beta = \sin\frac{\theta}{2}e^{i(\delta + \phi)}$。故而

\begin{equation}
\begin{aligned}
\ket{\psi} &= \cos\frac{\theta}{2}e^{i\delta}\ket{0} + \sin\frac{\theta}{2}e^{i(\delta + \phi)}\ket{1} \\
&= e^{i\delta}(\cos\frac{\theta}{2}\ket{0} + \sin\frac{\theta}{2}e^{i\phi}\ket{1})
\end{aligned}
\end{equation}

\begin{figure}
  \centering
  \input{BlochSphere.pdf_tex}
  \caption{Bloch Sphere}
\end{figure}

其中$e^{i\delta}$,称作共同相位(global phase)。因为对$\ket{0}$、对$\ket{1}$都一样影响,不具有观测效应,故可以将之舍弃不看。

至于相对相位(relative phase)$e^{i\phi}$,就不同了,它的影响可以在球面上表现出来。故得

\begin{equation}
\ket{\psi} = \cos\frac{\theta}{2}\ket{0} + \sin\frac{\theta}{2}e^{i\phi}\ket{1}
\end{equation}

\subsection{多量子比特}

\begin{quote}
Hilbert空间很大。

--- Calton Caves
\end{quote}

一个双量子比特有四种基态,分别记为$\ket{00}$,$\ket{01}$,$\ket{10}$,$\ket{11}$。
因此,描述双量子比特的状态向量为

\begin{equation}
\ket{\psi} = a_{00}\ket{00} + a_{01}\ket{01} + a_{10}\ket{10} + a_{11}\ket{11}
\end{equation}

类似单比特的情况,测量得到$\ket{x}$的概率为$|a_x|^2$,故有归一化条件

\begin{equation}
\sum_{x\in\{0,1\}^2}|a_x|^2 = 1
\end{equation}

不过,对于双量子比特,我们还有可能只测量其中一个比特,例如第一个。如你所想,测量第一个比特得到0的概率为$|a_{00}|^2 + |a_{01}|^2$。
而测量后的状态为

\begin{equation}
\ket{\psi'} = \frac{a_{00}\ket{00} + a_{01}\ket{01}}{\sqrt{|a_{00}|^2 + |a_{01}|^2}}
\end{equation}

注意这里的概率幅仍满足归一化条件。

双量子比特中有一个重要的状态,称之为Bell态或EPR对,其定义为

\begin{equation}
\frac{\ket{00} + \ket{11}}{\sqrt{2}}
\end{equation}

Bell态的性质是当测量第一量子比特时,以1/2概率得到0,进入测后状态$\ket{\psi'} = \ket{00}$;
同样也会以1/2概率得到1,进入测后状态$\ket{\psi'} = \ket{11}$。
结果是,测量第二量子比特总会得到与第一量子比特一样的结果。也就是说,测量结果是相关的。

更一般地,可以考虑n量子比特系统。其基态形如$\ket{x_1x_2\cdots x_n}$。
其状态由$2^n$个概率幅所确定。n = 500时,这个数字就已经超过了整个宇宙中原子的估算总数。
所以Hilbert空间确实是个巨大的空间。然而即使是包含几百个原子的系统,大自然也要处理如此巨量的数据。
看起来为了进一步的计算系统的演化,大自然需要在手边放上$2^{500}$张看不见的草稿纸。

一个大胆的设想是,我们是否能够---以及如何才能---利用如此强大的计算能力呢?

