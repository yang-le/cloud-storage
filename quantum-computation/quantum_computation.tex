\section{量子计算}

\subsection{单量子比特门}

量子状态$\alpha\ket{0} + \beta\ket{1}$写成向量形式

\begin{equation}
\begin{bmatrix}
\alpha \\
\beta
\end{bmatrix}
\end{equation}

其中上面一项代表0的幅度,而下面一项代表1的幅度。容易想到矩阵

\begin{equation}
X \equiv
\begin{bmatrix}
0 & 1 \\
1 & 0
\end{bmatrix}
\end{equation}

可使两者互换,即

\begin{equation}
X
\begin{bmatrix}
\alpha \\
\beta
\end{bmatrix}
=
\begin{bmatrix}
\beta \\
\alpha
\end{bmatrix}
\end{equation}

容易验证,$X\ket{0} = \ket{1}$,以及$X\ket{1} = \ket{0}$。因此$X$就是量子非门。

如果一个矩阵可以表示一个量子门,那么对这个矩阵有什么要求呢?回忆归一化条件,归一化条件要求向量
$\begin{bmatrix}\alpha \\ \beta\end{bmatrix}$模长为1。即$\begin{bmatrix}\alpha \\ \beta\end{bmatrix}^T\begin{bmatrix}\alpha \\ \beta\end{bmatrix} = 1$。显然,经过一个合法的量子门转换后,其结果仍应满足归一化条件,我们就有$(X\begin{bmatrix}\alpha \\ \beta\end{bmatrix})^T(X\begin{bmatrix}\alpha \\ \beta\end{bmatrix}) = \begin{bmatrix}\alpha \\ \beta\end{bmatrix}^TX^TX\begin{bmatrix}\alpha \\ \beta\end{bmatrix} = 1$。因此,$X^TX = I$。其中$I$是单位矩阵。实际上,由于$X$是复矩阵,前述的转置准确来说应该是共轭转置,即表示量子门的复矩阵应满足条件

\begin{equation}
X^{\dag}X = I
\end{equation}

其中$X^{\dag}$是$X$的共轭转置。满足上述条件的矩阵数学上称为酉矩阵。
令人惊奇的是,酉性是对量子门的唯一限制;每一个酉矩阵都定义了一个有效的量子门。

让我们继续,后面我们常会用到另外两个量子门,它们分别是$Z$门

\begin{equation}
Z \equiv
\begin{bmatrix}
1 & 0 \\
0 & -1
\end{bmatrix}
\end{equation}

它保持$\ket{0}$不变而反转$\ket{1}$为$-\ket{1}$,以及Hadamard门

\begin{equation}
H \equiv \frac{1}{\sqrt{2}}
\begin{bmatrix}
1 & 1 \\
1 & -1
\end{bmatrix}
\end{equation}

它将$\ket{0}$变为$\frac{\ket{0} + \ket{1}}{\sqrt{2}}$,且将$\ket{1}$变为$\frac{\ket{0} - \ket{1}}{\sqrt{2}}$。Hadamard门是最有用的量子门之一。

