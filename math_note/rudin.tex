%\documentclass[hyperref, UTF8]{ctexart}
\documentclass[UTF8]{ctexbook}

\usepackage{amsmath}

\newtheorem{example}{例}[chapter]
\newtheorem{definition}{定义}

\begin{document}

\title{数学分析原理笔记}
\author{
Le Yang\\
yangle0125@qq.com
}
\date{}
\maketitle

\tableofcontents

\chapter{实数系和复数系}

\begin{example}
我们现在证明方程
\begin{equation}
p^2 = 2	\label{eq1}
\end{equation}
不能被任何有理数$p$满足。倘若存在那样一个$p$,我们可以把它写成$p = m / n$,其中$m$和$n$都是整数\footnote{根据有理数的定义。},而且可以选得不都是偶数\footnote{否则我们可以一直约分,即分子分母都除以2,如此这样一直进行下去,直到其中一个不是偶数。}。于是由\eqref{eq1}式得出
\begin{equation}
m^2 = 2n^2 \label{eq2}
\end{equation}
这表明$m^2$是偶数,因此$m$是偶数(如果$m$是奇数,那么$m^2$将是奇数\footnote{反证法中的反证法。}),因而$m^2$能被$4$整除。于是\eqref{eq2}式右边能被$4$整除,因而$n^2$是偶数,这又说明$n$是偶数。

假定\eqref{eq1}式成立,就导致$m$和$n$都是偶数的结论,这与$m$及$n$的选择相矛盾。因此,对于有理数$p$,\eqref{eq1}式不能成立。

现在我们把这种情况考察的更严密一些。令$A$是使$p^2 < 2$的一切正有理数$p$的集,$B$是使$p^2 > 2$的一切正有理数$p$的集。我们来证明$A$里没有最大数,$B$里没有最小数\footnote{rudin先生显然觉得前面的证明太简单了,不够过瘾。}。

更明确地说,对于$A$中的每一个$p$,能在$A$中找到一个有理数$q$,而$p < q$,并且对于$B$中的每一个$p$,能在$B$中找到一个有理数$q$,而$q < p$。

为了做这件事,给每一个有理数$p > 0$,配置一个数\footnote{对于$A$中的情况,就是我们要找这样一个$q$,使得$q = p + x$,$x > 0$, $q < \sqrt{2}$。因此$x < \sqrt{2} - p = (2 - p^2) / (\sqrt{2} + p)$。可令$x = (2 - p^2) / (2 + p)$,就得到下面的式子。对于$B$中的情况也会得到同样的式子。}
\begin{equation}
q = p - \frac{p^2 - 2}{p + 2} = \frac{2p + 2}{p + 2}\label{eq3}
\end{equation}

\end{example}

\end{document}

