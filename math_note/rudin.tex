\documentclass[a4paper,UTF8]{ctexbook}

\usepackage{amsmath,amssymb,amsthm}
\usepackage{enumerate}
\usepackage{geometry}
\geometry{a4paper}

\newtheorem{example}{例}[chapter]
\newtheorem{definition}{定义}[chapter]
\newtheorem{theorem}{定理}[chapter]

\begin{document}

\title{数学分析原理笔记}
\author{
Le Yang\\
yangle0125@qq.com
}
\date{}
\maketitle

\tableofcontents

\chapter{实数系和复数系}

\section{导引}

\begin{example}
\label{example1}
我们现在证明方程
\begin{equation}
p^2 = 2	\label{eq1}
\end{equation}
不能被任何有理数$p$满足。倘若存在那样一个$p$,我们可以把它写成$p = m / n$,其中$m$和$n$都是整数\footnote{根据有理数的定义。},而且可以选得不都是偶数\footnote{否则我们可以一直约分,即分子分母都除以$2$,如此这样一直进行下去,直到其中一个不是偶数。}。于是由\eqref{eq1}式得出
\begin{equation}
m^2 = 2n^2 \label{eq2}
\end{equation}
这表明$m^2$是偶数,因此$m$是偶数(如果$m$是奇数,那么$m^2$将是奇数\footnote{反证法中的反证法。}),因而$m^2$能被$4$整除。于是\eqref{eq2}式右边能被$4$整除,因而$n^2$是偶数,这又说明$n$是偶数。

假定\eqref{eq1}式成立,就导致$m$和$n$都是偶数的结论,这与$m$及$n$的选择相矛盾。因此,对于有理数$p$,\eqref{eq1}式不能成立。

现在我们把这种情况考察的更严密一些。令$A$是使$p^2 < 2$的一切正有理数$p$的集,$B$是使$p^2 > 2$的一切正有理数$p$的集。我们来证明$A$里没有最大数,$B$里没有最小数\footnote{rudin先生显然觉得前面的证明太简单了,不够过瘾。}。

更明确地说,对于$A$中的每一个$p$,能在$A$中找到一个有理数$q$,而$p < q$,并且对于$B$中的每一个$p$,能在$B$中找到一个有理数$q$,而$q < p$。

为了做这件事,给每一个有理数$p > 0$,配置一个数\footnote{对于$A$中的情况,就是我们要找这样一个$q$,使得$q = p + x$,$x > 0$, $q < \sqrt{2}$。因此$x < \sqrt{2} - p = (2 - p^2) / (\sqrt{2} + p)$。可令$x = (2 - p^2) / (2 + p)$,就得到下面的式子。对于$B$中的情况也会得到同样的式子。}
\begin{equation}
q = p - \frac{p^2 - 2}{p + 2} = \frac{2p + 2}{p + 2} \label{eq3}
\end{equation}
于是
\begin{equation}
q^2 - 2 = \frac{2(p^2 - 2)}{(p + 2)^2} \label{eq4}
\end{equation}
如果$p$在$A$中,那么$p^2 - 2 < 0$,\eqref{eq3}式说明$q > p$,而\eqref{eq4}式说明$q^2 < 2$,因而$q$在$A$中。

如果$p$在$B$中,那么$p^2 - 2 > 0$,\eqref{eq3}式说明$0 < q < p$,而\eqref{eq4}式说明$q^2 > 2$,因而$q$在$B$中。
\end{example}

\section{有序集}

\begin{definition}[序关系]
设$S$是一个集。$S$上的序是一种关系,记作$<$,它有下面的两个性质:
\begin{enumerate}
\item 如果$x \in S$,并且$y \in S$,那么在$$x < y, x = y, y < x$$三种述语之中,有且只有一种成立。
\item 如果$x, y, z \in S$,又如果$x < y$且$y < z$,那么$x < z$。
\end{enumerate}
\end{definition}

\begin{definition}[有序集]
在集$S$里定义了一种序,便是一个有序集。
\end{definition}

\begin{definition}[上有界]
设$S$是有序集,而$E \subset S$。如果存在$\beta \in S$,而每个$x \in E$,满足$x \leqslant \beta$,我们就说$E$上有界,并称$\beta$为$E$的一个上界。

用类似的方法定义下界(把$\leqslant$换成$\geqslant$就行了)。
\end{definition}

\begin{definition}[最小上界]
设$S$是有序集,$E \subset S$,且$E$上有界。设存在一个$\alpha \in S$,它具有以下性质:
\begin{enumerate}
\item $\alpha$是$E$的上界。
\item 如果$\gamma < \alpha$,$\gamma$就不是$E$的上界。\label{item2}
\end{enumerate}
便把$\alpha$叫做$E$的最小上界(由\ref{item2}来看,显然\footnote{假设有$\alpha_1,\alpha_2$($\alpha_1 \neq \alpha_2$)都是$E$的最小上界,根据序的性质,要么$\alpha_1 < \alpha_2$,要么$\alpha_2 < \alpha_1$,根据\ref{item2},这就表明或者$\alpha_1$不是$E$的上界,或者$\alpha_2$不是$E$的上界,这与我们的假设矛盾。}最多有一个这样的$\alpha$)或$E$的上确界,而记作$$\alpha = \sup{E}$$
类似地可以定义下有界集$E$的最大下界或下确界。述语$$\alpha = \inf{E}$$表示$\alpha$是$E$的一个下界,而任何合于$\beta > \alpha$的$\beta$,不能是$E$的下界。
\end{definition}

\begin{example}
\label{example2}\hfill
\begin{enumerate}[a]
\item 把例\ref{example1}中的集$A$与集$B$看作有序集$Q$的子集。集$A$上有界。实际上,$A$的那些上界,刚好就是$B$的那些元。因为$B$没有最小的元,所以$A$在$Q$中没有最小上界。\label{item1}
\item 如果$\alpha = \sup{E}$存在。这$\alpha$可以是$E$的元,也可以不是$E$的元。例如,假设$E_1$是所有合于$r \in Q$及$r < 0$的集。假设$E_2$是所有合于$r \in Q$及$r \leqslant 0$的集。于是$$\sup{E_1} = \sup{E_2} = 0,$$而$0 \notin E_1$,$0 \in E_2$。
\end{enumerate}
\end{example}

\begin{definition}[最小上界性]
有序集$S$,如果具有性质:
若$E \subset S$,$E$不空,且$E$上有界时,$\sup{E}$便在$S$里。就说$S$有最小上界性。
\end{definition}
例\ref{example2} \ref{item1}说明$Q$没有最小上界性。

\begin{theorem}[有最小上界性的有序集也有最大下界性]
设$S$是具有最小上界性的有序集,$B \subset S$,$B$不空且$B$下有界。令$L$是$B$的所有下界的集。那么$$\alpha = sup{L}$$在$S$存在,并且$\alpha = \inf{B}$。

特别地说就是$\inf{B}$在$S$存在。
\end{theorem}

\begin{proof}[证]
(证明$\alpha$存在)因为$B$下有界,$L$不空。$L$刚好由这样一些$y \in S$组成,他们对于每个$x \in B$,满足不等式$y \leqslant x$。
可见每个$x \in B$是$L$的上界。于是$L$上有界,因而我们对$S$的假定意味着$S$里有$L$的上确界\footnote{这里还需要保证$L \subset S$。根据$L$的定义,若$x \in L$,那么$x$就是$B$的下界。而根据下界的定义,就有$x \in S$。所以$L \subset S$是成立的。rudin先生是觉得这些太显然了就没有把它们写出来吗?};把它叫做$\alpha$。

(证明$\alpha$是$B$的下界)如果$\gamma < \alpha$,那么$\gamma$不是$L$的上界,因此$\gamma \notin B$。由此对于每个$x \in B$,$\alpha \leqslant x$。所以$\alpha \in L$。

(证明$\alpha$是$B$的最大下界)如果$\alpha < \beta$,由于$\alpha$是$L$的上界,必然$\beta \notin L$。我们已证明了:$\alpha \in L$。而当$\beta > \alpha$时,就有$\beta \notin L$。换句话说,$\alpha$是$B$的下界,但若$\beta > \alpha$,$\beta$就不是$B$的下界。这就是说$\alpha = \inf{B}$。
\end{proof}

\section{域}

\end{document}

