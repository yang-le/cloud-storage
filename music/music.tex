\documentclass[hyperref, UTF8]{ctexart}

\begin{document}

\title{音乐和数学}
\author{
Le Yang\\
yangle0125@qq.com
}
\date{}
\maketitle

%\tableofcontents

\begin{quote}
\begin{flushright}
Doe, a deer, a female deer.\\
Ray, a drop of golden sun.\\
Me, a name I call myself.\\
Far, a long long way to run.\\
Sew, a needle pulling thread.\\
La, a note to follow Sew.\\
Tea, a drink with jam and bread.\\
\dots

--- Do Re Mi
\end{flushright}
\end{quote}

\section{从标准音高说起}
根据中学物理知识我们知道,声音的音调是由发声体震动的频率决定的。
听起来很简单,但你能告诉我---以众所周之的“Do Re Mi”为例---Do应该唱多高吗?
换句话说,“Do”这个音的频率,应该是多少呢?
你可能会说,应该跟钢琴弹出来的标准音一样高。
那么,就以位于钢琴键盘正中央的C键为例,其音高又应该是多少呢?

你可能觉得我在胡搅蛮缠,但这的确是个问题。
举例来说,1720年英国音笛演奏中央A的频率为380Hz,
而巴赫在汉堡市、莱比锡及魏玛等地使用的管风琴则以480Hz表示同一个音符,
这两者约差四个半音。换句话说,
1720年的英国音笛演奏的A音在巴赫的时代,会被认为是F音。
韩德尔在1740年使用对应为A音的音叉,其频率为422.5Hz,
但在1780年时他使用同样对应A的音叉则有不同的频率:409Hz,后者低了将近一个半音。
1815年,德累斯顿歌剧院中使用的A频率为423.2Hz,七年之后,同一个地方使用的A音却升高为435Hz。
在米兰的斯卡拉大剧院中,这个中央C上的A甚至高到451Hz。
举了这么多例子,我的意思是说,在不同的年代,甚至是同一年代的不同地区,对某个特定音符的音高,是没有一个固定标准的。

不过,混乱的局面肯定不能就这样一直持续下去。
时间来到1859年,这年的2月16日,法国政府通过了一项法案定义中央C上方的A为435Hz。这是把音高尺度标准化的第一个尝试。
80年后的1939年,一个国际会议提出,把中央C上方的A定为440Hz。到了1955年,国际标准化组织采用了这个标准,将其订为ISO 16。
从此,A440就成为钢琴、小提琴以及其他乐器的频率校准标准并一直沿用到现在。

好了,现在我们知道了,按照现行的国际标准,键盘中央C上方的A音符的音高为440Hz。但是,这仍然没有回答我一开始提出的问题。
我的问题是,中央C的音高应该是多少呢?这并不是一个简单的问题。
如果说440Hz这个标准称为“绝对音高”,那么根据这个“绝对音高”去定出其他音符的频率的问题,
就需要我们知道相邻音符之间的“相对音高”。

\end{document}

