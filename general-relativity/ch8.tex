\chapter[纤维丛及其在规范场论的应用]{\\纤维丛及其在规范场论的应用}

本附录大量用到李群和李代数的知识,阅读前最好先读前一章。
由于种种原因,本章(及前一章)的某些符号与本书他处的习惯不尽相同,例如,
\textcircled{1} 除少数地方外不用抽象指标,以减轻表达式的外观复杂度。
\textcircled{2} 曲线$\eta(t)$的切矢用$\displaystyle\frac{\mathrm{d}\eta(t)}{\mathrm{d}t}$(或$\mathrm{d}\eta(t) / \mathrm{d}t$,或$(\mathrm{d} / \mathrm{d}t)\eta(t)$,只在少数情况用$\partial / \partial t$)代表。
此外,本章经常涉及丛流形$P$和底流形$M$,它们的点分别用$p \in P$和$x \in M$代表;
$p \in P$的切空间在本书他处都记作$V_p$,但在本章中$V_p$另有含义,代表$p$点切空间的竖直子空间。
因此本章把$p \in P$的切空间记作$\mathrm{T}_pP$。
类似地,$x \in M$的切空间记作$\mathrm{T}_xM$。
这不但符合大多数文献的习惯,而且也合理,因为$M$的切丛在文献中统一记作$\mathrm{T}M$,点$x \in M$的切空间既然对应于$x$上方的纤维,就对应于$\mathrm{T}M$的一个子集,自然应记作$\mathrm{T}_xM$。

\section{主纤维丛}

\subsection{主丛的定义和例子}

\begin{definition}
    李群$G$在流形$K$上的一个左作用是一个$C^\infty$映射$L \colon G \times K \to K$满足:
    \begin{enumerate}[(a)]
        \item $L_g \colon K \to K$是微分同胚 $\forall g \in G$;
        \item $L_{gh} = L_g \comp L_h, ~ \forall g, h \in G$
    \end{enumerate}
\end{definition}

\begin{note}
    与李变换群的定义比较可知李变换群$G \times M \to M$就是$G$在$M$上的左作用。
\end{note}

\begin{definition}
    李群$G$在流形$K$上的一个右作用是一个$C^\infty$映射$R \colon K \times G \to K$满足:
    \begin{enumerate}[(a)]
        \item $R_g \colon K \to K$是微分同胚 $\forall g \in G$;
        \item $R_{gh} = R_h \comp R_g, ~ \forall g, h \in G$
    \end{enumerate}  
\end{definition}

\begin{note}
    仿照李变换群,也可以证明$L_e$及$R_e$(其中$e$是$G$的恒等元)是恒等映射。
\end{note}

下面大量用到李群$G$在某流形(主丛流形)$P$上的右作用,即$R \colon P \times G \to P$。
我们将把$R_g(p)$(其中$g \in G, p \in P$)简记作$pg$。

\begin{definition}
    子集$\{pg \mid g \in G\} \subset P$称为右作用$R \colon P \times G \to P$过点$p \in P$的轨道。
    右作用$R \colon P \times G \to P$称为自由的,若$g \neq e \Rightarrow pg \neq p ~ \forall p \in P$。
    左作用的自由性和轨道仿此定义。
\end{definition}

现在介绍主纤维丛的定义。
这一定义对初学者有一定难度,我们先给出定义,再以加注的方式详加解释。

\begin{definition}
    主纤维丛由一个叫做丛流形的流形$P$、一个叫做底流形的流形$M$和一个叫做结构群的李群$G$组成,满足以下要求:
    \begin{enumerate}[(a)]
        \item $G$在$P$上有自由右作用$R \colon P \times G \to P$;
        \item 存在$C^\infty$的、到上的投影映射$\pi \colon P \to M$,满足
        $$\pi^{-1}[\pi(p)] = \{pg \mid g \in G\}, ~ \forall p \in P$$
        \item 每一$x \in M$有开邻域$U \subset M$及微分同胚$T_U \colon \pi^{-1}[U] \to U \times G$,且$T_U$取如下形式:
        $$T_U(p) = (\pi(p), S_U(p)), ~ \forall p \in \pi^{-1}[U]$$
        其中映射$S_U \colon \pi^{-1}[U] \to G$满足
        $$S_U(pg) = S_U(p)g, ~ \forall g \in G$$
    \end{enumerate}
\end{definition}

\begin{note}
    今后把主纤维丛简称主丛,并简记为$P(M, G)$,甚至简记为$P$。
\end{note}

\section{主丛上的联络}

\section{与主丛相伴的纤维丛(伴丛)}

\section{物理场的整体规范不变性}

\section{物理场的局域规范不变性}

\section{截面的物理意义}

\section{规范势与联络}

\section{规范场强与曲率}

\section{矢丛上的联络和协变导数}
