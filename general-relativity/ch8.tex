\chapter[纤维丛及其在规范场论的应用]{\\纤维丛及其在规范场论的应用}

本附录大量用到李群和李代数的知识,阅读前最好先读前一章。
由于种种原因,本章(及前一章)的某些符号与本书他处的习惯不尽相同,例如,
\textcircled{1} 除少数地方外不用抽象指标,以减轻表达式的外观复杂度。
\textcircled{2} 曲线$\eta(t)$的切矢用$\displaystyle\frac{\mathrm{d}\eta(t)}{\mathrm{d}t}$(或$\mathrm{d}\eta(t) / \mathrm{d}t$,或$(\mathrm{d} / \mathrm{d}t)\eta(t)$,只在少数情况用$\partial / \partial t$)代表。
此外,本章经常涉及丛流形$P$和底流形$M$,它们的点分别用$p \in P$和$x \in M$代表;
$p \in P$的切空间在本书他处都记作$V_p$,但在本章中$V_p$另有含义,代表$p$点切空间的竖直子空间。
因此本章把$p \in P$的切空间记作$\mathrm{T}_pP$。
类似地,$x \in M$的切空间记作$\mathrm{T}_xM$。
这不但符合大多数文献的习惯,而且也合理,因为$M$的切丛在文献中统一记作$\mathrm{T}M$,点$x \in M$的切空间既然对应于$x$上方的纤维,就对应于$\mathrm{T}M$的一个子集,自然应记作$\mathrm{T}_xM$。

\section{主纤维丛}

\subsection{主丛的定义和例子}

\begin{definition}
    李群$G$在流形$K$上的一个左作用是一个$C^\infty$映射$L \colon G \times K \to K$满足:
    \begin{enumerate}[(a)]
        \item $L_g \colon K \to K$是微分同胚 $\forall g \in G$;
        \item $L_{gh} = L_g \comp L_h, ~ \forall g, h \in G$
    \end{enumerate}
\end{definition}

\begin{note}
    与李变换群的定义比较可知李变换群$G \times M \to M$就是$G$在$M$上的左作用。
\end{note}

\begin{definition}
    李群$G$在流形$K$上的一个右作用是一个$C^\infty$映射$R \colon K \times G \to K$满足:
    \begin{enumerate}[(a)]
        \item $R_g \colon K \to K$是微分同胚 $\forall g \in G$;
        \item $R_{gh} = R_h \comp R_g, ~ \forall g, h \in G$
    \end{enumerate}  
\end{definition}

\begin{note}
    仿照李变换群,也可以证明$L_e$及$R_e$(其中$e$是$G$的恒等元)是恒等映射。
\end{note}

下面大量用到李群$G$在某流形(主丛流形)$P$上的右作用,即$R \colon P \times G \to P$。
我们将把$R_g(p)$(其中$g \in G, p \in P$)简记作$pg$。

\begin{definition}
    子集$\{pg \mid g \in G\} \subset P$称为右作用$R \colon P \times G \to P$过点$p \in P$的轨道。
    右作用$R \colon P \times G \to P$称为自由的,若$g \neq e \Rightarrow pg \neq p ~ \forall p \in P$。
    左作用的自由性和轨道仿此定义。
\end{definition}

现在介绍主纤维丛的定义。
这一定义对初学者有一定难度,我们先给出定义,再以加注的方式详加解释。

\begin{definition}
    主纤维丛由一个叫做丛流形的流形$P$、一个叫做底流形的流形$M$和一个叫做结构群的李群$G$组成,满足以下要求:
    \begin{enumerate}[(a)]
        \item $G$在$P$上有自由右作用$R \colon P \times G \to P$;
        \item 存在$C^\infty$的、到上的投影映射$\pi \colon P \to M$,满足
        $$\pi^{-1}[\pi(p)] = \{pg \mid g \in G\}, ~ \forall p \in P$$
        \item 每一$x \in M$有开邻域$U \subset M$及微分同胚$T_U \colon \pi^{-1}[U] \to U \times G$,且$T_U$取如下形式:
        $$T_U(p) = (\pi(p), S_U(p)), ~ \forall p \in \pi^{-1}[U]$$
        其中映射$S_U \colon \pi^{-1}[U] \to G$满足
        $$S_U(pg) = S_U(p)g, ~ \forall g \in G$$
    \end{enumerate}
\end{definition}

\begin{note}
    今后把主纤维丛简称主丛,并简记为$P(M, G)$,甚至简记为$P$。
\end{note}

\begin{note}
    投影映射$\pi \colon P \to M$一般不是一一映射,故逆映射$\pi^{-1} \colon M \to P$一般不存在。
    但对子集$U \subset M$而言,$\pi^{-1}[U] \equiv \{p \in P \mid \pi(p) \in U\}$有明确含义。
    同理,把$x \in M$看作$M$的独点子集$\{x\} \in M$,则$\pi^{-1}[\{x\}]$有意义(简记作$\pi^{-1}[x]$),称为点$x \in M$上方的纤维。
    注意到前述定义,条件(b)实际上就是要求任一点$p \in P$的投影$\pi(p) \in M$上方的纤维$\pi^{-1}[\pi(p)]$等于右作用$R$过点$p$的轨道。
\end{note}

\begin{note}
    映射$R \colon P \times G \to P$在给定$p \in P$后诱导出映射$R_p \colon G \to P$。
    既然$\pi^{-1}[\pi(p)]$是$R$过$p$的轨道,映射$R_p$的值域就只能是$\pi^{-1}[\pi(p)] \subset P$,故也可把$R_p$更明确地写成$R_p \colon G \to \pi^{-1}[\pi(p)]$。
    令$x \equiv \pi(p)$以便把$R_p \colon G \to \pi^{-1}[\pi(p)]$简记为$R_p \colon G \to \pi^{-1}[x]$。
    可以证明$R_p \colon G \to P$是个嵌入映射,所以$G$的流形结构可被带到$\pi^{-1}[x]$上,使$\pi^{-1}[x]$成为$P$中的嵌入子流形,\footnote{
        不难证明$\pi^{-1}[x]$由此得到的微分(流形)结构与点$p \in \pi^{-1}[x]$的选择无关。
        还可证明$\pi^{-1}[x]$是正则嵌入子流形。
    }而且$R_p \colon G \to \pi^{-1}[x]$成为微分同胚。
    进一步自然要问:$G$的群结构是否也可被带到$\pi^{-1}[x]$上?
    答案是肯定的。
    首先,$R_p(e) = p$使我们想到可选$p$作为待定义李群$\pi^{-1}[x]$的恒等元。
    其次,每一$p' \in \pi^{-1}[x]$对应于$G$的一个元素$g \equiv R_p^{-1}(p')$,故
    $$p' = R_p(g) = R_g(p) = pg$$
    因而可借用$G$的群乘法给$\pi^{-1}[x]$定义群乘法:
    $$(pg)\cdot(ph) \coloneq p(gh), ~ \forall pg, ph \in \pi^{-1}[x]$$
    于是每一纤维都可看作$G$的一个李群同构版本。
    但必须注意:由于$p$点在$\pi^{-1}[x]$上可以任取($\pi^{-1}[x]$中没有一点是天生与众不同的),在把$\pi^{-1}[x]$定义为李群时不存在一种自然的定义方式(取任意$p \in \pi^{-1}[x]$作为恒等元均可)。
    所以,与$G$不同,$\pi^{-1}[x]$不是一个自然的李群,或说它不存在天生的群结构。
    它与$G$之间的李群同构映射$R_p$是$p$点依赖的。
\end{note}

\begin{note}
    条件(c)保证每一$x \in M$都有开邻域$U$,其逆像$\pi^{-1}[U]$与乘积流形$U \times G$微分同胚。
    在特殊情况下,这个$U$可能``大''到等于$M$,这时$\pi^{-1}[U] = P$,于是$P$与乘积流形$M \times G$微分同胚,不妨写$P = M \times G$。
    这种可表为乘积流形的主丛称为平凡主丛。
    一般主丛并不平凡,但条件(c)保证$\pi^{-1}[U]$总与$U \times G$微分同胚,所以说任何主丛都是局域平凡的。
    由于微分同胚映射$T_U$在此起关键作用,所以把$T_U$称为一个局域平凡化,简称局域平凡。
\end{note}

\begin{note}
    条件(c)是对映射$T_U$的要求。
    $\forall p \in \pi^{-1}[U]$,映射的像$T_U(p)$是$U \times G$的一点,即$U$的一点与$G$的一点构成的有序对$(\bullet, \bullet)$,由此不难理解前式右边写成$(\pi(p), S_U(p))$的原因。
    括号中第一槽$\pi(p)$表明$T_U(p)$的第一要素等于$p$的投影$\pi(p)$($p \in \pi^{-1}[U]$保证$\pi(p)$的确是$U$的元素),第二槽则要灵活得多,它只规定第二要素是$p$在映射$S_U$下的像,而对映射$S_U$的唯一要求是满足前述等式。
\end{note}

\begin{note}
    $\forall x \in U$,把$S_U$的定义域限制为$\pi^{-1}[U]$的子集$\pi^{-1}[x]$,便有微分同胚$S_U \colon \pi^{-1}[x] \to G$(以保证$T_U$是微分同胚)。
    可见,选定一个局域平凡$T_U$就选定了$\pi^{-1}[x]$与$G$之间的一个微分同胚映射,因而在$\pi^{-1}[x]$上确定了一个``特殊点''$\breve{p}_U$,满足$S_U(\breve{p}_U) = e \in G$。
    相应于这个$\breve{p}_U$又有微分同胚$R_{\breve{p}_U} \colon G \to \pi^{-1}[x]$,不难证明映射$R_{\breve{p}_U}$与$S_U \colon \pi^{-1}[x] \to G$互逆,即
    $$S_U \comp R_{\breve{p}_U} \colon G \to G \text{是恒等映射}$$
\end{note}

\begin{note}
    设$P(M, G)$是主丛,则不难证明如下的有用公式:
    $$R_g \comp R_p = R_{pg} \comp I_{g^{-1}}, ~ \forall p \in P, g \in G$$
    其中,$I_{g^{-1}} \colon G \to G$是由$g^{-1} \in G$构造的伴随同构。
\end{note}

\begin{example}
    对任给的李群$G$和流形$M$总可按如下三步构造一个主丛:
    \begin{enumerate}[(1)]
        \item 选$P \equiv M \times G$为丛流形,则$P$的任一点都可表为$p = (x, g)$,其中$x \in M, g \in G$。
        \item 定义自由右作用$R \colon P \times G \to P$为
        $$R_h(x, g) \coloneq (x, gh), ~ \forall x \in M, ~ h,g \in G$$
        (也可表为$(x, g)h \coloneq (x, gh)$)
        \item 定义投影映射为
        $$\pi(x, g) \coloneq x, ~ \forall x \in M, g \in G$$
    \end{enumerate}
    以上三步足以保证$P(M, G)$是个主丛,条件(c)自动满足,具体说,$\forall x \in M$都选$M$为条件中的开邻域$U$,从而$\pi^{-1}[U] = \pi^{-1}[M] = M \times G$,再把局域平凡$T_U \colon \pi^{-1}[U] \to U \times G$定义为恒等映射便可。
    这样构造的主丛显然是平凡的。
\end{example}

\section{主丛上的联络}

\section{与主丛相伴的纤维丛(伴丛)}

\section{物理场的整体规范不变性}

\section{物理场的局域规范不变性}

\section{截面的物理意义}

\section{规范势与联络}

\section{规范场强与曲率}

\section{矢丛上的联络和协变导数}
