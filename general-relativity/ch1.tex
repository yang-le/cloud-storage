\chapter{拓扑空间简介}
\section{集论初步}

\begin{definition}
若集$A$的每一元素都属于集$X$,就说$A$是$X$的\emph{子集},也说$A$含于$X$或$X$含$A$,记作$A \subset X$或$X \supset A$。
规定$\emptyset$是任一集合的子集。$A$称为$X$的\emph{真子集},若$A \subset X$且$A \neq X$。
集$X$和$Y$称为\emph{相等}的(记作$X = Y$),若$X \subset Y$且$Y \subset X$
\end{definition}

\begin{definition}
集合$A$,$B$的并集,交集,差集和补集定义为:

\emph{并集} $A \cup B := \{x \mid x \in A ~ \text{或} ~ x \in B\}$

\emph{交集} $A \cap B := \{x \mid x \in A, x \in B\}$

\emph{差集} $A - B := \{x \mid x \in A, x \notin B\}$

若$A$是$X$的子集,则$A$的\emph{补集}$-A$定义为$-A := X - A$

\end{definition}

\begin{theorem}
以上集运算符合如下规律:

\emph{交换律} $A \cup B = B \cup A$, $A \cap B = B \cap A$

\emph{结合律} $(A \cup B) \cup C = A \cup (B \cup C)$, $(A \cap B) \cap C = A \cap (B \cap C)$

\emph{分配律} $(A \cap B) \cup C = (A \cup C) \cap (B \cup C)$, $(A \cup B) \cap C = (A \cap C) \cup (B \cap C)$

\emph{De Morgan律} $A - (B \cup C) = (A - B) \cap (A - C)$, $A - (B \cap C) = (A - B) \cup (A - C)$

\end{theorem}

\begin{definition}
非空集合$X$,$Y$的卡氏积$X \times Y$定义为
$$X \times Y := \{(x, y) \mid x \in X, y \in Y\}$$
\end{definition}
多个(有限个)集合的卡氏积可以类似定义,而且规定卡氏积满足结合律,即$(X \times Y) \times Z = X \times (Y \times Z)$

\begin{example}
$\mathbb{R}^2 := \mathbb{R} \times \mathbb{R}$, $\mathbb{R}^n := \mathbb{R} \times \cdots \times \mathbb{R}$(共$n$个$\mathbb{R}$).
既然$\mathbb{R}^2$的元素是由两个实数构成的有序对,这两个实数就称为该元素的\emph{自然坐标}。类似地,$\mathbb{R}^n$的每一个元素有$n$个自然坐标。
可见$\mathbb{R}^n$天生就是有坐标的,但其他集合则未必。利用自然坐标可以给$\mathbb{R}^n$的任意两个元素定义距离的概念。
\end{example}

\begin{definition}
$\mathbb{R}^n$的任意两个元素$x = (x^1, \cdots, x^n)$, $y = (y^1, \cdots, y^n)$之间的\emph{距离}$|y - x|$定义为
$$|y - x| := \sqrt{\sum^n_{i = 1}(y^i - x^i)^2}$$
\end{definition}

\begin{definition}
设$X$,$Y$为非空集合。一个从$X$到$Y$的\emph{映射}(记作$f \colon X \to Y$)是一个法则,它给$X$的每一个元素指定$Y$的唯一的对应元素。
若$y \in Y$是$x \in X$的对应元素,就写$y = f(x)$,并称$y$为$x$在映射$f$下的\emph{像},称$x$为$y$的\emph{原像}(或\emph{逆像})。
$X$称为映射$f$的\emph{定义域},$X$的全体元素在映射$f$下的像的集合(记作$f[X]$)称为映射$f \colon X \to Y$的\emph{值域}。
映射$f \colon X \to Y$和$f' \colon X \to Y$称为\emph{相等}的,若$f(x) = f'(x) ~ \forall x \in X$。
\end{definition}

\begin{note}
从$\mathbb{R}^2$到$\mathbb{R}$的映射给出一个二元函数。同理,从$\mathbb{R}^n$到$\mathbb{R}^m$的映射给出$m$个$n$元函数。
\end{note}

\begin{definition}
映射$f \colon X \to Y$叫\emph{一一}的,若任一$y \in Y$有不多于一个逆像(可以没有)。$f \colon X \to Y$叫\emph{到上}的,若任一$y \in Y$都有逆像(可以多于一个)。
\end{definition}

\begin{definition}
$f \colon X \to Y$称为\emph{常值映射},若$f(x) = f(x') ~ \forall x, x' \in X$
\end{definition}

\begin{definition}
设$X$,$Y$,$Z$为集,$f \colon X \to Y$和$g \colon Y \to Z$为映射,则$f$和$g$的\emph{复合映射}$g \comp f$是从$X$到$Z$的映射,定义为$(g \comp f)(x) := g(f(x)) \in Z ~ \forall x \in X$
\end{definition}

\section{拓扑空间}

\begin{definition}
非空集合$X$的一个\emph{拓扑}$\mathscr{T}$是$X$的若干子集的集合,满足:
\begin{enumerate}[(a)]
\item $X,\emptyset \in \mathscr{T}$
\item 若$O_i \in \mathscr{T}, i = 1, 2, \cdots, n$,则$\bigcap\limits^n_{i = 1}O_i \in \mathscr{T}$
\item 若$O_\alpha \in \mathscr{T} ~ \forall \alpha$,则$\bigcup\limits_{\alpha}O_\alpha \in \mathscr{T}$
\end{enumerate}
\end{definition}

\begin{definition}
定义了拓扑$\mathscr{T}$的集合$X$称为\emph{拓扑空间}。拓扑空间$X$的子集$O$称为\emph{开子集}(简称\emph{开集}),若$O \in \mathscr{T}$。
\end{definition}

\begin{example}
设$X$为任意非空集合,令$\mathscr{T}$为$X$的全部子集的集合,则它满足拓扑的定义,故构成X的一个拓扑,叫\emph{离散拓扑}。
\end{example}

\begin{example}
设$X$为任意非空集合,令$\mathscr{T} = \{X, \emptyset\}$,则它满足拓扑的定义,故构成X的一个拓扑,叫\emph{凝聚拓扑}。
凝聚拓扑是元素最少的拓扑,而离散拓扑是元素最多的拓扑。
\end{example}

\begin{example}
\begin{enumerate}[(1)]
\item 设$X = \mathbb{R}$,则$\mathscr{T}_u := \{\text{空集或}\mathbb{R}\text{中能表为开区间之并的子集}\}$称为$\mathbb{R}$的\emph{通常拓扑}。
\item 设$X = \mathbb{R}^n$,则$\mathscr{T}_u := \{\text{空集或}\mathbb{R}^n\text{中能表为开球之并的子集}\}$称为$\mathbb{R}^n$的\emph{通常拓扑},
其中,开球的定义为$B(x_0, r) := \{x \in \mathbb{R}^n \mid |x - x_0| < r\}$,$x_0$称为球心,$r > 0$称为半径。
$\mathbb{R}^2$中的开球亦称开圆盘,$\mathbb{R}$中的开球就是开区间。
\end{enumerate}
根据上述定义,$\mathbb{R}$中任一开区间用$\mathscr{T}_u$衡量都是开集。今后在把$\mathbb{R}$看做拓扑空间时,如无声明就是指$(\mathbb{R}, \mathscr{T}_u)$
\end{example}
