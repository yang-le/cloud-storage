\chapter{拓扑空间简介}

\section{集论初步}

确切地指定了的若干事物的全体叫做一个集合,简称集。
集中的每一事物叫一个元素或点。
若$x$是集$X$的元素,则说``$x$属于$X$'',并记作$x \in X$。
符号$\notin$则代表``不属于''。
有两种表示集合的方法,一种是一一列出其元素,元素间用逗号隔开,全体元素用花括号括起来,如
$$X = \{1, 4, 5.6\}$$
表示由实数$1$,$4$以及$5.6$构成的集。
另一种表示方法是指出集中元素的共性,如
$$X = \{x \mid \text{$x$为实数}\}$$
表示$X$是全体实数的集合(这一特定集的通用记号为$\mathbb{R}$),而
$$X = \{x \in \mathbb{R} \mid x > 9\}$$
则表示全体大于$9$的实数的集合。

不含元素的集叫空集,记作$\emptyset$。

\begin{definition}
	若集$A$的每一元素都属于集$X$,就说$A$是$X$的\textbf{子集},也说$A$含于$X$或$X$含$A$,记作$A \subset X$或$X \supset A$。
	规定$\emptyset$是任一集合的子集。$A$称为$X$的\textbf{真子集},若$A \subset X$且$A \neq X$。
	集$X$和$Y$称为\textbf{相等}的(记作$X = Y$),若$X \subset Y$且$Y \subset X$
\end{definition}

\begin{note}
	子集定义的更确切表述本应是``集$A$叫集$X$的子集,当且仅当$A$的每一元素都属于$X$''。
	但为方便起见,凡在定义中的``若''或``当''都是``当且仅当''之意。
\end{note}

本书用$\coloneq$代表``定义为'',用$\equiv$代表``恒等''或``记作'',例如$C \equiv A - B$的含义是``把$A - B$记作$C$''。
采用这两个符号无非是为增加明确性,都换成等号也无妨。

\begin{definition}
	集合$A$,$B$的并集,交集,差集和补集定义为:

	\textbf{并集} $A \cup B \coloneq \{x \mid x \in A ~ \text{或} ~ x \in B\}$

	\textbf{交集} $A \cap B \coloneq \{x \mid x \in A, x \in B\}$ (条件``$x \in A, x \in B$''是``$x \in A \text{且} x \in B$''的简写,下同。)

	\textbf{差集} $A - B \coloneq \{x \mid x \in A, x \notin B\}$ (数学书常把差集记作$A \backslash B \text{或} A \sim B$,本书一律记作$A - B$。)

	若$A$是$X$的子集,则$A$的\textbf{补集}$-A$定义为$-A \coloneq X - A$

\end{definition}

\begin{theorem}
	以上集运算符合如下规律:

	\textbf{交换律} $A \cup B = B \cup A$, $A \cap B = B \cap A$

	\textbf{结合律} $(A \cup B) \cup C = A \cup (B \cup C)$, $(A \cap B) \cap C = A \cap (B \cap C)$

	\textbf{分配律} $(A \cap B) \cup C = (A \cup C) \cap (B \cup C)$, $(A \cup B) \cap C = (A \cap C) \cup (B \cap C)$

	\textbf{De Morgan律} $A - (B \cup C) = (A - B) \cap (A - C)$, $A - (B \cap C) = (A - B) \cup (A - C)$

\end{theorem}

\begin{proof}
	交换律和结合律根据定义显然。

	$A \subset A \cup C, B \subset B \cup C$,所以若$x \in A, x \in B$就有$x \in A \cup C, x \in B \cup C$,即
	$A \cap B \subset (A \cup C) \cap (B \cup C)$。同理,$C \subset (A \cup C) \cap (B \cup C)$。
	所以有$$(A \cap B) \cup C \subset (A \cup C) \cap (B \cup C)$$
	反过来,若$x \in (A \cup C) \cap (B \cup C)$,也即$x \in (A \cup C), x \in (B \cup C)$,而这可以有$4$种组合
	$x \in A, x \in B \text{或}x \in A, x \in C \text{或}x \in C, x \in B \text{或}x \in C, x \in C$
	即$x \in (A \cap B) \cup (A \cap C) \cup (B \cap C) \cup C$,注意到$(B \cap C) \cup C = C$,就有$x \in (A \cap B) \cup C$。这就说明
	$$(A \cup C) \cap (B \cup C) \subset (A \cap B) \cup C$$
	综上$(A \cap B) \cup C = (A \cup C) \cap (B \cup C)$。

	类似地,$A \cap C \subset A, B \cap C \subset B$,可以证明
	$$(A \cap C) \cup (B \cap C) \subset (A \cup B) \cap C$$
	反过来,若$x \in (A \cup B) \cap C$,也即$x \in (A \cup B), x \in C$,这可以有两种情况,$x \in A, x \in C$或$x \in B, x \in C$,因此
	$$(A \cup B) \cap C \subset (A \cap C) \cup (B \cap C)$$
	综上$(A \cup B) \cap C = (A \cap C) \cup (B \cap C)$

	对于De Morgan律,$x \in A - (B \cup C)$也即$x \in A, x \notin (B \cup C)$,即$x \in A, x \notin B, x \in A, x \notin C$。
	因此,$x \in (A - B) \cap (A - C)$,反之也成立,这说明
	$$A - (B \cup C) = (A - B) \cap (A - C)$$

	另一方面,$A - B \subset A, A - C \subset A$,因此$(A - B) \cup (A - C) \subset A$。
	又$x \in A - B$说明$x \notin B$,因此$x \in (A - B) \cup (A - C)$一方面说明$x \in A$,另一方面说明$x \notin B$或$x \notin C$即$x \notin B \cap C$。
	因此,$x \in A - (B \cap C)$,反之也成立,这说明
	$$(A - B) \cup (A - C) = A - (B \cap C)$$
\end{proof}

\begin{definition}
	非空集合$X$,$Y$的卡氏积$X \times Y$定义为
	$$X \times Y \coloneq \{(x, y) \mid x \in X, y \in Y\}$$
\end{definition}

就是说,$X \times Y$是这样一个集合,它的每一个元素是由$X$的一个元素$x$和$Y$的一个元素$y$组成的一个有序对$(x, y)$。
多个(有限个\footnote{无限多个集合的卡氏积也可定义,但已超出本书范围。})集合的卡氏积可以类似定义,例如
$$X \times Y \times Z \coloneq \{(x, y, z) \mid x \in X, y \in Y, z \in Z\}$$
而且规定卡氏积满足结合律,即$(X \times Y) \times Z = X \times (Y \times Z)$

\begin{example}
	$\mathbb{R}^2 \coloneq \mathbb{R} \times \mathbb{R}$, $\mathbb{R}^n \coloneq \mathbb{R} \times \cdots \times \mathbb{R}$(共$n$个$\mathbb{R}$).
	既然$\mathbb{R}^2$的元素是由两个实数构成的有序对,这两个实数就称为该元素的\textbf{自然坐标}。类似地,$\mathbb{R}^n$的每一个元素有$n$个自然坐标。
	可见$\mathbb{R}^n$天生就是有坐标的,但其他集合则未必。利用自然坐标可以给$\mathbb{R}^n$的任意两个元素定义距离的概念。
\end{example}

\begin{definition}
	$\mathbb{R}^n$的任意两个元素$x = (x^1, \cdots, x^n)$, $y = (y^1, \cdots, y^n)$之间的\textbf{距离}$|y - x|$定义为
	$$|y - x| \coloneq \sqrt{\sum^n_{i = 1}(y^i - x^i)^2}$$
\end{definition}

\begin{definition}
	设$X$,$Y$为非空集合。一个从$X$到$Y$的\textbf{映射}(记作$f \colon X \to Y$)是一个法则,它给$X$的每一个元素指定$Y$的唯一的对应元素。
	若$y \in Y$是$x \in X$的对应元素,就写$y = f(x)$,并称$y$为$x$在映射$f$下的\textbf{像},称$x$为$y$的\textbf{原像}(或\textbf{逆像})。
	$X$称为映射$f$的\textbf{定义域},$X$的全体元素在映射$f$下的像的集合(记作$f[X]$)称为映射$f \colon X \to Y$的\textbf{值域}。
	映射$f \colon X \to Y$和$f' \colon X \to Y$称为\textbf{相等}的,若$f(x) = f'(x) ~ \forall x \in X$。
\end{definition}

\begin{note}
	通常也把$y = f(x)$写成$f \colon x \mapsto y$。请注意$\mapsto$和$\to$的区别:
	$f \colon X \to Y$中的$\to$表示$f$是从$X$到$Y$(集合到集合)的映射;
	而$f \colon x \mapsto y$中的$\mapsto$则表示$x \in X$在映射$f$下的像是$y$(元素到元素)。
\end{note}

\begin{note}
	设$A \subset X$,则$A$的元素在$f$下的像组成的子集记作$f[A]$,即
	$$f[A] \equiv \{y \in Y \mid \exists x \in A \text{使得} y = f(x)\} \subset Y$$
\end{note}

\begin{example}
	普通微积分中的单值函数$y = f(x)$就是一个由$\mathbb{R}$(或其子集)到$\mathbb{R}$的映射。
\end{example}

\begin{note}
	从$\mathbb{R}^2$到$\mathbb{R}$的映射给出一个二元函数。同理,从$\mathbb{R}^n$到$\mathbb{R}^m$的映射给出$m$个$n$元函数。
\end{note}

\begin{definition}
	映射$f \colon X \to Y$叫\textbf{一一}的,若任一$y \in Y$有不多于一个逆像(可以没有)。$f \colon X \to Y$叫\textbf{到上}的,若任一$y \in Y$都有逆像(可以多于一个)。\footnote{
		不少数学书把本书的一一和到上映射分别叫\textbf{单射}和\textbf{满射},把既是单射又是满射的映射叫\textbf{一一映射}(又称\textbf{双射})。于是它们的一一映射强于本书的一一映射。
	}
\end{definition}

\begin{note}
	\textcircled{1} $f$为到上映射的充要条件是值域$f[X] = Y$。
	\textcircled{2} 若$f$为一一映射,则存在逆映射$f^{-1} \colon f[X] \to X$。
	然而,不论$f \colon X \to Y$是否有逆,都可以定义任一子集$B \subset Y$在$f$下的``逆像''$f^{-1}[B]$为
	$$f^{-1}[B] \coloneq \{x \in X \mid f(x) \in B\} \subset X$$
	注意,这里的``逆像''是$X$的子集而不是$X$的元素。
	例如,如果$X$有(且仅有)两个元素$x$和$x'$在$f$作用(即映射)下的像都是$y \in Y$,虽然逆映射$f^{-1} \colon Y \to X$不存在,
	但把$y$看作$Y$的独点子集(即$\{y\}$)时$f^{-1}[\{y\}]$(简记作$f^{-1}[y]$)仍有意义,含义为$f^{-1}[y] = \{x, x'\} \subset X$。
\end{note}

\begin{definition}
	$f \colon X \to Y$称为\textbf{常值映射},若$f(x) = f(x') ~ \forall x, x' \in X$
\end{definition}

\begin{definition}
	设$X$,$Y$,$Z$为集,$f \colon X \to Y$和$g \colon Y \to Z$为映射,则$f$和$g$的\textbf{复合映射}$g \comp f$是从$X$到$Z$的映射,定义为$(g \comp f)(x) \coloneq g(f(x)) \in Z ~ \forall x \in X$
\end{definition}

\begin{note}
	若$X = Y = Z = \mathbb{R}$,则复合映射$g \comp f$就是熟知的一元复合函数。
\end{note}

若$X$和$Y$是一般的集合,对$X$与$Y$之间的映射就只能提出``一一''和``到上''这两个要求;
但若$X$和$Y$还指定了某种结构,则往往可对$f \colon X \to Y$提出更多要求,例如可要求$f \colon \mathbb{R} \to \mathbb{R}$是连续的甚至光滑的。
一元函数$f \colon \mathbb{R} \to \mathbb{R}$的连续性在微积分中早有定义(``$\varepsilon$ -- $\delta$定义''),重述如下:
\textcircled{1} 称$f$在点$x$连续,若$\forall \varepsilon > 0 \exists \delta > 0$使得当$|x' - x| < \delta$时有$|f(x') - f(x)| < \varepsilon$;
\textcircled{2} 称$f$在$\mathbb{R}$上连续,若它在$\mathbb{R}$的任一点连续。
这一定义依赖于$\mathbb{R}$中任二元素的距离概念(对$\mathbb{R}$而言距离就是坐标之差),似乎无法推广到没有距离定义的两个集合之间的映射。
然而细想发现,$\varepsilon$ -- $\delta$定义可用开区间概念(而无需距离概念)重新表述如下:
设$X = Y = \mathbb{R}$,映射$f \colon X \to Y$叫做连续的,若$Y$中任一开区间的``逆像''都是$X$的开区间之并(或是空集)。
以上讨论从一个侧面说明``开区间之并''这一概念的用处:可以定义映射$f \colon \mathbb{R} \to \mathbb{R}$的连续性。
其实这一概念还有很多用处,因此往往有必要推广到除$\mathbb{R}$外的集合$X$。
为方便起见,把$\mathbb{R}$的任一可以表为开区间之并的子集(连同空集$\emptyset$)称为开子集。
为把开子集概念推广到任意集合$X$,应先找出$\mathbb{R}$的开子集的本质的、抽象的(因而可以推广的)性质。
它们是:
\begin{enumerate}[(a)]
	\item $\mathbb{R}$和空集$\emptyset$都是开子集
	\item 有限个开子集之交仍是开子集
	\item 任意个开子集之并仍是开子集
\end{enumerate}

把这三个性质推广,就可给任意集合$X$定义开子集概念。
定义了开子集的集合叫拓扑空间。
由开子集的概念出发又可定义许多概念并证明许多定理,从而发展为一门完整丰富的学科分支 --- 点集拓扑学。
下面将对拓扑空间的最基本内容做一介绍。

\section{拓扑空间}

如前所述,$\mathbb{R}$的子集分为开子集和非开子集两大类(任一子集要么是开的,要么是非开的。不要把非开子集称为闭子集。根据后面要讲的闭子集的定义,子集可以不开不闭,也可以既开又闭。)
开子集具有上述三个性质。对任意非空集合$X$也可用适当方式指定其中的某些子集是开的,其他为非开的。
为使这种指定有用,我们约定任何指定方式都要满足三个要求:
\begin{enumerate}[(a)]
	\item $X$本身和空集$\emptyset$为开子集
	\item 有限个开子集之交为开子集
	\item 任意个(可以有限个也可以无限个)开子集之并为开子集
\end{enumerate}

对同一集合,满足这三个要求的指定方式常常是很多的。
例如,设$X$为任意集合,可以指定$X$及$\emptyset$为开子集,其他子集都为非开。
这当然满足上述三要求,其特点是开子集最少,只有两个。
然而也可采用另一种极端的指定,即指定$X$的任意子集都是开子集。
不难看出这种指定也满足上述三要求。
上述两种指定虽然未必有太多用处,但它们至少能说明满足上述三要求的指定方式不止一种。
我们说,每种满足上述三要求的指定给集合$X$赋予了一种附加结构,称为拓扑结构。
对定义了拓扑结构的集合可以指着它的任一子集问:``这是开子集吗?''答案非``是''即``否'',泾渭分明。
反之,对没有定义拓扑结构的集合,这样的问题毫无意义。
定义了拓扑结构的集合$X$的全体开子集也组成一个集合,称为$X$的一个拓扑,记作$\mathscr{T}$(是topology的首字母的花体大写)。
用$\mathscr{P}$代表由$X$的全体子集组成的集合,则$X$的任一开子集$O$和任一非开子集$V$都是$\mathscr{P}$的元素。
$X$的全体开子集组成$\mathscr{P}$的一个子集$\mathscr{T}$(注意,它不是$X$的子集),它就是$X$的拓扑。
请注意符号$\subset$同$\in$的区别:$O \subset X$只表明$O$是$X$的子集,而$O \in \mathscr{T}$则表明$O$是$X$的开子集。
以上铺垫有助于理解如下用数学语言表述的定义。

\begin{definition}
	非空集合$X$的一个\textbf{拓扑}$\mathscr{T}$是$X$的若干子集的集合,满足:
	\begin{enumerate}[(a)]
		\item $X,\emptyset \in \mathscr{T}$
		\item 若$O_i \in \mathscr{T}, i = 1, 2, \cdots, n$,则$\bigcap\limits^n_{i = 1}O_i \in \mathscr{T}$
		\item 若$O_\alpha \in \mathscr{T} ~ \forall \alpha$,则$\bigcup\limits_{\alpha}O_\alpha \in \mathscr{T}$
	\end{enumerate}
\end{definition}

\begin{definition}
	定义了拓扑$\mathscr{T}$的集合$X$称为\textbf{拓扑空间}。拓扑空间$X$的子集$O$称为\textbf{开子集}(简称\textbf{开集}),若$O \in \mathscr{T}$。
\end{definition}

对同一集合$X$可以定义不同的拓扑$\mathscr{T}$(满足定义的$\mathscr{T}$可以很多)。
设$\mathscr{T}_1$和$\mathscr{T}_2$都是$X$的拓扑,则$X$的子集$A$可能满足$A \in \mathscr{T}_1, A \notin \mathscr{T}_2$,即$A$对$\mathscr{T}_1$而言(用$\mathscr{T}_1$衡量)是开集而对$\mathscr{T}_2$而言不是开集。
可见$\mathscr{T}_1$和$\mathscr{T}_2$把$X$定义为两个不同的拓扑空间。
为明确所选拓扑起见,可用$(X, \mathscr{T})$代表拓扑空间。
于是$(X, \mathscr{T}_1)$和$(X, \mathscr{T}_2)$代表不同的拓扑空间,虽然它们的``底集''都是$X$。
在明确选定拓扑后也可只用$X$代表拓扑空间。

对给定的具体集合$X$,应该选哪个拓扑使之成为一个拓扑空间?
这取决于$X$自身的性质以及我们关心哪些方面的问题。
例如,对集合$\mathbb{R}^n$,在通常关心的大多数问题中都选所谓的通常拓扑为拓扑。

\begin{example}
	设$X$为任意非空集合,令$\mathscr{T}$为$X$的全部子集的集合,则它满足拓扑的定义,故构成X的一个拓扑,叫\textbf{离散拓扑}。
\end{example}

\begin{example}
	设$X$为任意非空集合,令$\mathscr{T} = \{X, \emptyset\}$,则它满足拓扑的定义,故构成X的一个拓扑,叫\textbf{凝聚拓扑}。
	凝聚拓扑是元素最少的拓扑,而离散拓扑是元素最多的拓扑。
\end{example}

\begin{example}
	\begin{enumerate}[(1)]
		\item 设$X = \mathbb{R}$,则$\mathscr{T}_u \coloneq \{\text{空集或}\mathbb{R}\text{中能表为开区间之并的子集}\}$称为$\mathbb{R}$的\textbf{通常拓扑}。
		\item 设$X = \mathbb{R}^n$,则$\mathscr{T}_u \coloneq \{\text{空集或}\mathbb{R}^n\text{中能表为开球之并的子集}\}$称为$\mathbb{R}^n$的\textbf{通常拓扑},
		      其中,\textbf{开球}的定义为$B(x_0, r) \coloneq \{x \in \mathbb{R}^n \mid |x - x_0| < r\}$,$x_0$称为球心,$r > 0$称为半径。
		      $\mathbb{R}^2$中的开球亦称\textbf{开圆盘},$\mathbb{R}$中的开球就是开区间。
	\end{enumerate}
\end{example}

根据上述定义,$\mathbb{R}$中任一开区间用$\mathscr{T}_u$衡量都是开集。
然而,原则上也可以选其他拓扑使$\mathbb{R}$成为不同于$(\mathbb{R}, \mathscr{T}_u)$的拓扑空间。
例如,若以凝聚拓扑衡量,则除$\mathbb{R}$及$\emptyset$之外都不是开集;
反之,若以离散拓扑衡量,则$\mathbb{R}$中任一子集(包括闭区间和半闭区间)都是开集。
今后在把$\mathbb{R}$看做拓扑空间时,如无声明就是指$(\mathbb{R}, \mathscr{T}_u)$

\begin{example}
	设$(X_1, \mathscr{T}_1)$, $(X_2, \mathscr{T}_2)$为拓扑空间,$X = X_1 \times X_2$,定义$X$的拓扑为
	$$\mathscr{T} \coloneq \{O \in X \mid O\text{可表示为形如}O_1 \times O_2\text{的集合之并}, O_1 \in \mathscr{T}_1, O_2 \in \mathscr{T}_2\}$$
	则$\mathscr{T}$称为$X$的\textbf{乘积拓扑}。
\end{example}

\begin{example}
	设$(X, \mathscr{T})$为拓扑空间,$A$为$X$的任一非空子集。
	把$A$看作集合,当然也可指定拓扑(记作$\mathscr{S}$,是S的花体)使$A$成为拓扑空间,记作$(A, \mathscr{S})$。
	由于$A$是$X$的子集,我们希望$\mathscr{S}$与$\mathscr{T}$有尽量密切的联系。
	如果$A \in \mathscr{T}$,问题很简单,只须定义$\mathscr{S} \coloneq \{V \subset A \mid V \subset \mathscr{T}\}$。
	然而,如果$A \notin \mathscr{T}$,按上述定义就有$A \notin \mathscr{S}$,违背定义的条件。
	因此$\mathscr{S}$的上述定义不合法。
	一个巧妙的定义是
	$$\mathscr{S} \coloneq \{V \subset A \mid \exists O \in \mathscr{T} \text{使} V = A \cap O\}$$
	由上式可以证明即使$A \notin \mathscr{T}$也有$A \in \mathscr{S}$,而且$\mathscr{S}$满足定义的其他条件。
	这样定义的$\mathscr{S}$叫做$A$的、由$\mathscr{T}$导出的\textbf{诱导拓扑}。以后在把$(X, \mathscr{T})$的子集$A$看作拓扑空间时,如无声明都指$(A, \mathscr{S})$,
	其中$\mathscr{S}$是由$\mathscr{T}$诱导的拓扑。$(A, \mathscr{S})$称为$(X, \mathscr{T})$的\textbf{拓扑子空间}。
\end{example}

下面的例子有助于加深对诱导拓扑的理解。
$\mathbb{R}^2$中以$x_0$为心的单位圆周$S^1$定义为$S^1 \coloneq \{x \in \mathbb{R}^2 \mid |x - x_0| = 1\}$。
设$A \subset \mathbb{R}^2$是$S^1$,由于它不能表为$\mathbb{R}^2$中的开球之并(一条线窄到装不下任何开圆盘),$A$用$\mathbb{R}^2$的$\mathscr{T}_u$衡量不是开的。
用前述定义给$A$定义诱导拓扑$\mathscr{S}$,则不但$A$用$\mathscr{S}$衡量是开的,而且,设$V$是$A$中的任意一段(不含首末两点),则虽然$V$用$\mathscr{T}_u$衡量不是开集,用$\mathscr{S}$衡量却是开的,因为存在开圆$O \in \mathscr{T}_u$使$V = A \cap O$。

利用开集概念可对拓扑空间之间的映射定义连续性。
下面给出两个等价的连续定义

\begin{definition}
	\begin{enumerate}[(a)]
		\item 设$(X, \mathscr{T})$和$(Y, \mathscr{S})$为拓扑空间。映射$f \colon X \to Y$称为\textbf{连续}的,若$f^{-1}[O] \in \mathscr{T} ~ \forall O \in \mathscr{S}$。
		\item 设$(X, \mathscr{T})$和$(Y, \mathscr{S})$为拓扑空间。映射$f \colon X \to Y$称为\textbf{在点$x \in X$处连续},若对所有满足$f(x) \in G'$的$G' \in \mathscr{S}$, 存在$G \in \mathscr{T}$使$x \in G$且$f[G] \subset G'$。
		      $f \colon X \to Y$称为连续,若它在所有点$x \in X$上连续。
	\end{enumerate}
\end{definition}

\begin{note}
	不难看出,若$X = Y = \mathbb{R}$,$\mathscr{T} = \mathscr{S} = \mathscr{T}_u$,上述定义就回到$\varepsilon$ -- $\delta$定义。
\end{note}

\begin{definition}
	拓扑空间$(X, \mathscr{T})$和$(Y, \mathscr{S})$称为\textbf{互相同胚},若存在映射$f \colon X \to Y$,满足(a)$f$是一一到上的;(b)$f$及$f^{-1}$都连续。\footnote{
		的确存在其逆不连续的一一到上的连续映射(用离散和凝聚拓扑)。
	}
	这样的$f$称为从$(X, \mathscr{T})$到$(Y, \mathscr{S})$的\textbf{同胚映射},简称\textbf{同胚}。
\end{definition}

普通函数$y = f(x)$的连续性和可微性用$C^r$表示,其中$r$为非负整数,$C^0$代表连续,$C^r$代表$r$阶导数存在并连续,$C^\infty$代表任意阶导数存在并连续(称为\textbf{光滑})。
虽然用开集概念可以巧妙地把$C^0$性推广到拓扑空间之间的映射,但$r > 0$的$C^r$性则不能。
事实上,对拓扑空间之间的映射的最高要求已体现在同胚的定义中。
同胚映射$f \colon X \to Y$不仅在$X$和$Y$的点之间建立了一一对应的关系,而且还在$X$的开子集和$Y$的开子集之间建立了一一对应的关系,因而一切由拓扑决定的性质都可``全息''地被$f$``携带''到$Y$中。
因此,从纯拓扑学角度看,两个互相同胚的拓扑空间就``像得不能再像'',可以视作相等。

\begin{example}
	任一开区间$(a, b) \subset \mathbb{R}$与$\mathbb{R}$同胚。
\end{example}

\begin{example}
	圆周$S^1 \subset \mathbb{R}^2$配以诱导拓扑(由$\mathbb{R}^2$的$\mathscr{T}_u$诱导)可看做拓扑空间。
	它与$\mathbb{R}$不同胚。但挖去一点的圆周与$\mathbb{R}$同胚。
\end{example}

\begin{example}
	考虑欧氏平面上的一个圆和一个椭圆(均指圆周)。从拓扑学的角度看,$(\mathbb{R}^2, \mathscr{T}_u)$是拓扑空间,圆$S^1$和椭圆$E$是$\mathbb{R}^2$的两个子集:
	$S^1, E \in \mathbb{R}^2$。可用$\mathscr{T}_u$给$S^1$和$E$分别定义诱导拓扑使成两个拓扑空间$(S^1, \mathscr{S}_{S^1})$及$(E, \mathscr{S}_E)$。
	可以证明存在同胚映射$f \colon (S^1, \mathscr{S}_{S^1}) \to (E, \mathscr{S}_E)$,所以从纯拓扑眼光看两者完全一样。
\end{example}

\begin{definition}
	$N \subset X$称为$x \in X$的一个\textbf{邻域},若存在$O \in \mathscr{T}$使$x \in O \subset N$。自身是邻域的开集称为\textbf{开邻域}。
\end{definition}

\begin{note}
	设$X = \mathbb{R}$,$N = [a, b]$,则$N$按定义是$x$的邻域,当且仅当$a < x < b$。
	请特别注意``擦边''情况:若$x = a$,则$N$并非$x$的邻域,因为$\mathbb{R}$不存在开集$O$使$x \in O \subset N$。
	直观地说,要使$[a, b]$是$x$的邻域,$x$在$[a, b]$中应有``左邻右舍''。
	而$x = a$的任何``左邻''都不属于$[a, b]$,故$[a, b]$不应是$x = a$的邻域。
	另请注意如下微妙的例子:在拓扑空间$[0, \infty) \subset \mathbb{R}$中,$[0, 1)$是$0$的开邻域,$[0, 1]$是$0$的邻域\footnote{
	将$[0, \infty)$记作$A$。视其为拓扑空间即为其配备由$\mathscr{T}_u$定义的诱导拓扑$\mathscr{T}_A = \{V \subset A \mid \exists O \in \mathscr{T}_u \text{使} V = A \cap O\}$。
	现在只要取$O = (-1, 1)$,则$V = A \cap O = [0, 1)$是$A$中的开集,且满足$x \in V \subset [0, 1]$。
	}。
\end{note}

\begin{definition}
	$N \subset X$称为$A \subset X$的一个\textbf{邻域},若存在$O \in \mathscr{T}$使$A \subset O \subset N$。
\end{definition}

\begin{theorem}
	$A \subset X$是开集,当且仅当$A$是$x$的邻域$\forall x \in A$。
\end{theorem}

\begin{proof}
	\begin{enumerate}[(A)]
		\item 设$A$为开,则$\forall x \in A, \exists A \in \mathscr{T}$使$x \in A \subset A$,故由前述定义知$A$是$x$的邻域。
		\item 设$A$是$x$的邻域$\forall x \in A$,令$O = \bigcup\limits_{x \in A}O_x$($O_x \in \mathscr{T}$且满足$x \in O_x \subset A$),则$O = A$。
		又根据定义知$O \in \mathscr{T}$,故$A \in \mathscr{T}$,即$A$为开集。
	\end{enumerate}
\end{proof}

\begin{definition}
	$C \subset X$叫\textbf{闭集},若$-C \in \mathscr{T}$。
\end{definition}

\begin{theorem}
	闭集有以下性质:
	\begin{enumerate}[(a)]
		\item 任意个闭集的交集是闭集
		\item 有限个闭集的并集是闭集
		\item $X$及$\emptyset$是闭集
	\end{enumerate}
\end{theorem}

\begin{proof}
	由定义和De Morgan律易证。
\end{proof}

可见任何拓扑空间$(X, \mathscr{T})$都有两个既开又闭的子集,即$X$和$\emptyset$

\begin{definition}
	拓扑空间$(X, \mathscr{T})$称为\textbf{连通}的,若它除$X$和$\emptyset$外没有既开又闭的子集。
\end{definition}

\begin{example}
	设$A$和$B$是$\mathbb{R}$的开区间,$A \cap B = \emptyset$,以$\mathscr{T}$代表由$\mathbb{R}$的通常拓扑在子集$X \equiv A \cup B$上的诱导拓扑,
	则拓扑空间$(X, \mathscr{T})$的既开又闭的子集除$X$和$\emptyset$外还有$A$和$B$($A$和$B$在诱导拓扑下自然是开的,又因为$A$和$B$互为补集,故又都是闭的),
	所以$(X, \mathscr{T})$是不连通的。这同$A$和$B$在直观上互不连通相吻合\footnote{
		与直观感觉更一致的是称为``弧连通''的概念。它与连通概念有微妙差别。
	}。
\end{example}

设$(X, \mathscr{T})$为拓扑空间,$A \subset X$。分别定义$A$的闭包、内部和边界如下:

\begin{definition}
	$A$的\textbf{闭包}$\bar{A}$是所有含$A$的闭集的交集,即
	$$\bar{A} \coloneq \bigcap_\alpha C_\alpha, ~ A \subset C_\alpha, ~ -C_\alpha \in \mathscr{T}$$
\end{definition}

\begin{definition}
	$A$的\textbf{内部}$\operatorname{i}(A)$是所有含于$A$的开集的并集,即
	$$\operatorname{i}(A) \coloneq \bigcup_\alpha O_\alpha, ~ O_\alpha \subset A, ~ O_\alpha \in \mathscr{T}$$
\end{definition}

\begin{definition}
	$A$的\textbf{边界}$\dot{A} \coloneq \bar{A} - \operatorname{i}(A)$,$x \in \dot{A}$称为\textbf{边界点}。$\dot{A}$也记作$\partial A$
\end{definition}

\begin{theorem}
	$\bar{A}$,$\operatorname{i}(A)$及$\dot{A}$有以下性质:
	\begin{enumerate}[(a)]
		\item \textcircled{1}$\bar{A}$为闭集,\textcircled{2}$A \subset \bar{A}$,\textcircled{3}$A = \bar{A}$当且仅当$A$为闭集。
		\item \textcircled{1}$\operatorname{i}(A)$为开集,\textcircled{2}$\operatorname{i}(A) \subset A$,\textcircled{3}$\operatorname{i}(A) = A$当且仅当$A$为开集。
		\item $\dot{A}$为闭集
	\end{enumerate}
\end{theorem}

\begin{proof}
	(a)、(b)易证。(c)的证明如下:
	$X - \dot{A} = X - [\bar{A} - \operatorname{i}(A)] = (X - \bar{A}) \cup \operatorname{i}(A)$。
	因为$\bar{A}$为闭,故$X - \bar{A}$为开,加之$\operatorname{i}(A)$为开,故$X - \dot{A}$为开,因而$\dot{A}$为闭。
\end{proof}

\begin{definition}
	$X$的开子集的集合$\{O_\alpha\}$叫$A \subset X$的一个\textbf{开覆盖},若$A \subset \bigcup\limits_\alpha O_\alpha$。
	也可以说$\{O_\alpha\}$覆盖$A$。
\end{definition}

\section{紧致性}

\begin{definition}
	设$\{O_\alpha\}$是$A \subset X$的开覆盖。
	若$\{O_\alpha\}$的有限个元素构成的子集$\{O_{\alpha_1}, \cdots, O_{\alpha_n}\}$也覆盖$A$,就说$\{O_\alpha\}$有\textbf{有限子覆盖}。
\end{definition}

\begin{definition}
	$A \subset X$叫\textbf{紧致的},若它的任一开覆盖都有有限子覆盖。
\end{definition}

\begin{example}
	设$x \in X$,则独点子集$A \equiv \{x\}$必紧致。
\end{example}

\begin{proof}
	设$\{O_\alpha\}$是$A \subset X$的任一开覆盖,则$\{O_\alpha\}$中至少存在一个元素(记作$O_{\alpha_1}$)满足$x \in O_{\alpha_1}$。
	于是$\{O_{\alpha_1}\}$(作为$\{O_\alpha\}$的子集)是$A \equiv \{x\}$的开覆盖,故$\{O_\alpha\}$有有限子覆盖。
\end{proof}

\begin{example}
	$A \equiv (0, 1] \subset \mathbb{R}$不是紧致的。
\end{example}

\begin{proof}
	以$\mathbb{N}$代表自然数集,则$\{(1/n, 2) \mid n \in \mathbb{N}\}$是$A$的开覆盖,它没有有限子覆盖。
\end{proof}

类似地,$\mathbb{R}$中任一开区间或半开区间都非紧致。

\begin{example}
	$\mathbb{R}$不是紧致的。
\end{example}

\begin{theorem}
	$\mathbb{R}$的任一闭区间都紧致。
\end{theorem}

\begin{note}
	不要以为闭集一定紧致(就连$\mathbb{R}$中也有非紧致闭集)。
	紧性与闭性有密切联系,但不等价,其关系体现在以下两定理中。
\end{note}

为讲述以下定理,先补充以下定义。

\begin{definition}
	拓扑空间$(X, \mathscr{T})$叫\textbf{$T_2$空间}或\textbf{豪斯多夫空间},若
	$$\forall x, y \in X, ~ x \neq y, ~ \exists O_1, O_2 \in \mathscr{T} \text{使} x \in O_1, ~ y \in O_2 \text{且} O_1 \cap O_2 = \emptyset$$
\end{definition}

\begin{note}
	常见的拓扑空间(如$\mathbb{R}^n$)都是$T_2$空间。
	凝聚拓扑空间是非$T_2$空间的一例。
\end{note}

\begin{theorem}
	若$(X, \mathscr{T})$为$T_2$空间,$A \subset X$为紧致,则$A$为闭集。
\end{theorem}

\begin{theorem}
	若$(X, \mathscr{T})$为紧致且$A \subset X$为闭集,则$A$为紧致。
\end{theorem}

\begin{definition}
	$A \subset \mathbb{R}^n$叫\textbf{有界的},若存在开球$B \subset \mathbb{R}^n$使$A \subset B$。
\end{definition}

\begin{theorem}
	$A \subset \mathbb{R}$为紧致,当且仅当$A$为有界闭集。
\end{theorem}

\begin{theorem}
	设$A \subset X$紧致,$f \colon X \to Y$连续,则$f[A] \subset Y$紧致。
\end{theorem}

由以上定理可得推论:同胚映射保持子集的紧致性。

\begin{definition}
	在同胚映射下保持不变的性质称为\textbf{拓扑性质}或\textbf{拓扑不变性}。
\end{definition}

\begin{example}
	紧致性、连通性和$T_2$性都是拓扑性质。
	有界性不是拓扑性质,例如开区间$(a, b)$同胚于$\mathbb{R}$,但前者有界而后者无界。
	由此还可看出长度也不是拓扑性质。
\end{example}

数学分析中有个熟知定理:闭区间上的连续函数必在该区间上取得其最大值和最小值。
下述定理是这一定理的推广。

\begin{theorem}
	设$X$紧致,$f \colon X \to \mathbb{R}$连续,则$f[X] \subset \mathbb{R}$有界并取得其最大值和最小值。
\end{theorem}

\begin{theorem}
	设$(X_1, \mathscr{T}_1), (X_2, \mathscr{T}_2)$紧致,则$(X_1 \times X_2, \mathscr{T})$紧致($\mathscr{T}$为$\mathscr{T}_1$和$\mathscr{T}_2$的乘积拓扑)。
\end{theorem}

\begin{theorem}
	$A \subset \mathbb{R}^n$紧致,当且仅当它是有界闭集。
\end{theorem}

简单应用举例 ~ 考虑$(\mathbb{R}^2, \mathscr{T}_u)$。
设$S^1$是$\mathbb{R}^2$中的任一圆周,易见它是有界闭集,于是它为紧致。
因为连续映射保紧致性,而$\mathbb{R}$及其任一开区间都不紧致,可见$S^1$不可能与$\mathbb{R}$或其任一开区间同胚。
类似地,$\mathbb{R}$中任一闭区间都不可能与$\mathbb{R}$或其任一开区间同胚。

\begin{definition}
	映射$S \colon \mathbb{N} \to X$叫$X$中的\textbf{序列}。
\end{definition}

\begin{note}
	通常把序列记作$\{x_n\}$,其中$x_n \equiv S(n) \in X, ~ n \in \mathbb{N}$。
	$\{x_n\}$其实就是$X$中编了次序的一串点。
\end{note}

\begin{definition}
	$x \in X$叫序列$\{x_n\}$的\textbf{极限},若对$x$的任一开邻域$O$存在$N \in \mathbb{N}$使$x_n \in O ~ \forall n > N$。
	若$x$是$\{x_n\}$的极限,就说$\{x_n\}$\textbf{收敛}于$x$。
\end{definition}

\begin{definition}
	$x \in X$叫序列$\{x_n\}$的\textbf{聚点},若$x$的任一开邻域都含$\{x_n\}$的无限多点。
\end{definition}

\begin{note}
	$x$为$\{x_n\}$的极限$\Rightarrow x$为$\{x_n\}$的聚点,但反之不一定。
\end{note}

下述定理中有一条件涉及``第二可数''概念。
元素个数有限的集成为\textbf{有限集},否则称为\textbf{无限集}。
对有限集总可将其元素编号以便一个一个地数,所以有限集一定是可数集。
但无限集也不一定不可数,例如$\mathbb{N}$就是可数的无限集。
有限集比无限集简单,可数无限集比不可数无限集简单。
拓扑空间$(X, \mathscr{T})$称为\textbf{第二可数的},若$\mathscr{T}$存在可数子集$\{O_1, \cdots, O_k\} \subset \mathscr{T}$(或$\{O_1, \cdots\} \subset \mathscr{T}$)使得任一$O \in \mathscr{T}$可被表示为$\{O_1, \cdots, O_k\}$(或$\{O_1, \cdots\}$)的元素之并。
例如,$(\mathbb{R}^n, \mathscr{T}_u)$是第二可数的,因为$\mathscr{T}_u$有这样的子集(它的每个元素$O_i$是一个开球,球心的每个自然坐标都是有理数,半径也是有理数),使得任一$O \in \mathscr{T}_u$可被表为该子集的元素之并。

\begin{theorem}
	若$A \subset X$紧致,则$A$中任一序列都有在$A$内的聚点。
	反之,若$X$为第二可数且$A \subset X$中任一序列都有在$A$内的聚点,则$A$紧致。
\end{theorem}
