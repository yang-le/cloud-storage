\chapter{拓扑空间简介}
\section{集论初步}

\begin{definition}
若集$A$的每一元素都属于集$X$,就说$A$是$X$的\textbf{子集},也说$A$含于$X$或$X$含$A$,记作$A \subset X$或$X \supset A$。
规定$\emptyset$是任一集合的子集。$A$称为$X$的\textbf{真子集},若$A \subset X$且$A \neq X$。
集$X$和$Y$称为\textbf{相等}的(记作$X = Y$),若$X \subset Y$且$Y \subset X$
\end{definition}

\begin{definition}
集合$A$,$B$的并集,交集,差集和补集定义为:

\textbf{并集} $A \cup B := \{x \mid x \in A ~ \text{或} ~ x \in B\}$

\textbf{交集} $A \cap B := \{x \mid x \in A, x \in B\}$

\textbf{差集} $A - B := \{x \mid x \in A, x \notin B\}$

若$A$是$X$的子集,则$A$的\textbf{补集}$-A$定义为$-A := X - A$

\end{definition}

\begin{theorem}
以上集运算符合如下规律:

\textbf{交换律} $A \cup B = B \cup A$, $A \cap B = B \cap A$

\textbf{结合律} $(A \cup B) \cup C = A \cup (B \cup C)$, $(A \cap B) \cap C = A \cap (B \cap C)$

\textbf{分配律} $(A \cap B) \cup C = (A \cup C) \cap (B \cup C)$, $(A \cup B) \cap C = (A \cap C) \cup (B \cap C)$

\textbf{De Morgan律} $A - (B \cup C) = (A - B) \cap (A - C)$, $A - (B \cap C) = (A - B) \cup (A - C)$

\end{theorem}

\begin{definition}
非空集合$X$,$Y$的卡氏积$X \times Y$定义为
$$X \times Y := \{(x, y) \mid x \in X, y \in Y\}$$
\end{definition}
多个(有限个\footnote{无限多个集合的卡氏积也可定义,但已超出本书范围。})集合的卡氏积可以类似定义,而且规定卡氏积满足结合律,即$(X \times Y) \times Z = X \times (Y \times Z)$

\begin{example}
$\mathbb{R}^2 := \mathbb{R} \times \mathbb{R}$, $\mathbb{R}^n := \mathbb{R} \times \cdots \times \mathbb{R}$(共$n$个$\mathbb{R}$).
既然$\mathbb{R}^2$的元素是由两个实数构成的有序对,这两个实数就称为该元素的\textbf{自然坐标}。类似地,$\mathbb{R}^n$的每一个元素有$n$个自然坐标。
可见$\mathbb{R}^n$天生就是有坐标的,但其他集合则未必。利用自然坐标可以给$\mathbb{R}^n$的任意两个元素定义距离的概念。
\end{example}

\begin{definition}
$\mathbb{R}^n$的任意两个元素$x = (x^1, \cdots, x^n)$, $y = (y^1, \cdots, y^n)$之间的\textbf{距离}$|y - x|$定义为
$$|y - x| := \sqrt{\sum^n_{i = 1}(y^i - x^i)^2}$$
\end{definition}

\begin{definition}
设$X$,$Y$为非空集合。一个从$X$到$Y$的\textbf{映射}(记作$f \colon X \to Y$)是一个法则,它给$X$的每一个元素指定$Y$的唯一的对应元素。
若$y \in Y$是$x \in X$的对应元素,就写$y = f(x)$,并称$y$为$x$在映射$f$下的\textbf{像},称$x$为$y$的\textbf{原像}(或\textbf{逆像})。
$X$称为映射$f$的\textbf{定义域},$X$的全体元素在映射$f$下的像的集合(记作$f[X]$)称为映射$f \colon X \to Y$的\textbf{值域}。
映射$f \colon X \to Y$和$f' \colon X \to Y$称为\textbf{相等}的,若$f(x) = f'(x) ~ \forall x \in X$。
\end{definition}

\begin{note}
从$\mathbb{R}^2$到$\mathbb{R}$的映射给出一个二元函数。同理,从$\mathbb{R}^n$到$\mathbb{R}^m$的映射给出$m$个$n$元函数。
\end{note}

\begin{definition}
映射$f \colon X \to Y$叫\textbf{一一}的,若任一$y \in Y$有不多于一个逆像(可以没有)。$f \colon X \to Y$叫\textbf{到上}的,若任一$y \in Y$都有逆像(可以多于一个)。\footnote{
不少数学书把本书的一一和到上映射分别叫\textbf{单射}和\textbf{满射},把既是单射又是满射的映射叫\textbf{一一映射}(又称\textbf{双射})。于是它们的一一映射强于本书的一一映射。
}
\end{definition}

\begin{definition}
$f \colon X \to Y$称为\textbf{常值映射},若$f(x) = f(x') ~ \forall x, x' \in X$
\end{definition}

\begin{definition}
设$X$,$Y$,$Z$为集,$f \colon X \to Y$和$g \colon Y \to Z$为映射,则$f$和$g$的\textbf{复合映射}$g \comp f$是从$X$到$Z$的映射,定义为$(g \comp f)(x) := g(f(x)) \in Z ~ \forall x \in X$
\end{definition}

\section{拓扑空间}

\begin{definition}
非空集合$X$的一个\textbf{拓扑}$\mathscr{T}$是$X$的若干子集的集合,满足:
\begin{enumerate}[(a)]
\item $X,\emptyset \in \mathscr{T}$
\item 若$O_i \in \mathscr{T}, i = 1, 2, \cdots, n$,则$\bigcap\limits^n_{i = 1}O_i \in \mathscr{T}$
\item 若$O_\alpha \in \mathscr{T} ~ \forall \alpha$,则$\bigcup\limits_{\alpha}O_\alpha \in \mathscr{T}$
\end{enumerate}
\end{definition}

\begin{definition}
定义了拓扑$\mathscr{T}$的集合$X$称为\textbf{拓扑空间}。拓扑空间$X$的子集$O$称为\textbf{开子集}(简称\textbf{开集}),若$O \in \mathscr{T}$。
\end{definition}

\begin{example}
设$X$为任意非空集合,令$\mathscr{T}$为$X$的全部子集的集合,则它满足拓扑的定义,故构成X的一个拓扑,叫\textbf{离散拓扑}。
\end{example}

\begin{example}
设$X$为任意非空集合,令$\mathscr{T} = \{X, \emptyset\}$,则它满足拓扑的定义,故构成X的一个拓扑,叫\textbf{凝聚拓扑}。
凝聚拓扑是元素最少的拓扑,而离散拓扑是元素最多的拓扑。
\end{example}

\begin{example}
\begin{enumerate}[(1)]
\item 设$X = \mathbb{R}$,则$\mathscr{T}_u := \{\text{空集或}\mathbb{R}\text{中能表为开区间之并的子集}\}$称为$\mathbb{R}$的\textbf{通常拓扑}。
\item 设$X = \mathbb{R}^n$,则$\mathscr{T}_u := \{\text{空集或}\mathbb{R}^n\text{中能表为开球之并的子集}\}$称为$\mathbb{R}^n$的\textbf{通常拓扑},
其中,\textbf{开球}的定义为$B(x_0, r) := \{x \in \mathbb{R}^n \mid |x - x_0| < r\}$,$x_0$称为球心,$r > 0$称为半径。
$\mathbb{R}^2$中的开球亦称\textbf{开圆盘},$\mathbb{R}$中的开球就是开区间。
\end{enumerate}
根据上述定义,$\mathbb{R}$中任一开区间用$\mathscr{T}_u$衡量都是开集。今后在把$\mathbb{R}$看做拓扑空间时,如无声明就是指$(\mathbb{R}, \mathscr{T}_u)$
\end{example}

\begin{example}
设$(X_1, \mathscr{T}_1)$, $(X_2, \mathscr{T}_2)$为拓扑空间,$X = X_1 \times X_2$,定义$X$的拓扑为

$\mathscr{T} := \{O \in X \mid O\text{可表示为形如}O_1 \times O_2\text{的集合之并}, O_1 \in \mathscr{T}_1, O_2 \in \mathscr{T}_2\}$

则$\mathscr{T}$称为$X$的\textbf{乘积拓扑}。
\end{example}

\begin{example}
设$(X, \mathscr{T})$为拓扑空间,$A$为$X$的任一非空子集。为使$A$成为拓扑空间,为其定义拓扑$\mathscr{S}$为

$\mathscr{S} := \{V \subset A \mid \exists O \in \mathscr{T} \text{使} V = A \cap O\}$

这样定义的$\mathscr{S}$叫做$A$的、由$\mathscr{T}$导出的\textbf{诱导拓扑}。以后在把$(X, \mathscr{T})$的子集$A$看作拓扑空间时,如无声明都指$(A, \mathscr{S})$,
其中$\mathscr{S}$是由$\mathscr{T}$诱导的拓扑。$(A, \mathscr{S})$称为$(X, \mathscr{T})$的\textbf{拓扑子空间}。
\end{example}

\begin{definition}
\begin{enumerate}[(a)]
\item 设$(X, \mathscr{T})$和$(Y, \mathscr{S})$为拓扑空间。映射$f \colon X \to Y$称为\textbf{连续}的,若$f^{-1}[O] \in \mathscr{T} ~ \forall O \in \mathscr{S}$。
\item 设$(X, \mathscr{T})$和$(Y, \mathscr{S})$为拓扑空间。映射$f \colon X \to Y$称为\textbf{在点$x \in X$处连续},若对所有满足$f(x) \in G'$的$G' \in \mathscr{S}$, 存在$G \in \mathscr{T}$使$x \in G$且$f[G] \subset G'$。
$f \colon X \to Y$称为连续,若它在所有点$x \in X$上连续。
\end{enumerate}
\end{definition}

\begin{note}
不难看出,若$X = Y = \mathbb{R}$,$\mathscr{T} = \mathscr{S} = \mathscr{T}_u$,上述定义就回到$\varepsilon$ -- $\delta$定义。
\end{note}

\begin{definition}
拓扑空间$(X, \mathscr{T})$和$(Y, \mathscr{S})$称为\textbf{互相同胚},若存在映射$f \colon X \to Y$,满足(a)$f$是一一到上的;(b)$f$及$f^{-1}$都连续。\footnote{
的确存在其逆不连续的一一到上的连续映射(用离散和凝聚拓扑)。
}
这样的$f$称为从$(X, \mathscr{T})$到$(Y, \mathscr{S})$的\textbf{同胚映射},简称\textbf{同胚}。   
\end{definition}

普通函数$y = f(x)$的连续性和可微性用$C^r$表示,其中$r$为非负整数,$C^0$代表连续,$C^r$代表$r$阶导数存在并连续,$C^\infty$代表任意阶导数存在并连续[称为\textbf{光滑}]。

\begin{example}
任一开区间$(a, b) \subset \mathbb{R}$与$\mathbb{R}$同胚。
\end{example}

\begin{example}
圆周$S^1 \subset \mathbb{R}^2$配以诱导拓扑(由$\mathbb{R}^2$的$\mathscr{T}_u$诱导)可看做拓扑空间。
它与$\mathbb{R}$不同胚。但挖去一点的圆周与$\mathbb{R}$同胚。
\end{example}

\begin{example}
考虑欧氏平面上的一个圆和一个椭圆(均指圆周)。从拓扑学的角度看,$(\mathbb{R}^2, \mathscr{T}_u)$是拓扑空间,圆$S^1$和椭圆$E$是$\mathbb{R}^2$的两个子集:
$S^1, E \in \mathbb{R}^2$。可用$\mathscr{T}_u$给$S^1$和$E$分别定义诱导拓扑使成两个拓扑空间$(S^1, \mathscr{S}_{S^1})$及$(E, \mathscr{S}_E)$。
可以证明存在同胚映射$f \colon (S^1, \mathscr{S}_{S^1}) \to (E, \mathscr{S}_E)$,所以从纯拓扑眼光看两者完全一样。
\end{example}

\begin{definition}
$N \subset X$称为$x \in X$的一个\textbf{邻域},若存在$O \in \mathscr{T}$使$x \in O \subset N$。自身是邻域的开集称为\textbf{开邻域}。  
\end{definition}

\begin{note}
设$X = \mathbb{R}$,$N = [a, b]$,则$N$按定义是$x$的邻域,当且仅当$a < x < b$。
请特别注意``擦边''情况:若$x = a$,则$N$并非$x$的邻域,因为$\mathbb{R}$不存在开集$O$使$x \in O \subset N$。
直观地说,要使$[a, b]$是$x$的邻域,$x$在$[a, b]$中应有``左邻右舍''。
而$x = a$的任何``左邻''都不属于$[a, b]$,故$[a, b]$不应是$x = a$的邻域。
另请注意如下微妙的例子:在拓扑空间$[0, \infty) \subset \mathbb{R}$中,$[0, 1)$是$0$的开邻域,$[0, 1]$是$0$的邻域\footnote{
将$[0, \infty)$记作$A$。视其为拓扑空间即为其配备由$\mathscr{T}_u$定义的诱导拓扑$\mathscr{T}_A = \{V \subset A \mid \exists O \in \mathscr{T}_u \text{使} V = A \cap O\}$。
现在只要取$O = (-1, 1)$,则$V = A \cap O = [0, 1)$是$A$中的开集,且满足$x \in V \subset [0, 1]$。
}。
\end{note}

\begin{definition}
$N \subset X$称为$A \subset X$的一个\textbf{邻域},若存在$O \in \mathscr{T}$使$A \subset O \subset N$。
\end{definition}

\begin{theorem}
$A \subset X$是开集,当且仅当$A$是$x$的邻域$\forall x \in A$。
\end{theorem}

\begin{definition}
$C \subset X$叫\textbf{闭集},若$-C \in \mathscr{T}$。
\end{definition}

\begin{theorem}
闭集有以下性质:
\begin{enumerate}[(a)]
\item 任意个闭集的交集是闭集
\item 有限个闭集的并集是闭集
\item $X$及$\emptyset$是闭集
\end{enumerate}
\end{theorem}

可见任何拓扑空间$(X, \mathscr{T})$都有两个既开又闭的子集,即$X$和$\emptyset$

\begin{definition}
拓扑空间$(X, \mathscr{T})$称为\textbf{连通}的,若它除$X$和$\emptyset$外没有既开又闭的子集。
\end{definition}

\begin{example}
设$A$和$B$是$\mathbb{R}$的开区间,$A \cap B = \emptyset$,以$\mathscr{T}$代表由$\mathbb{R}$的通常拓扑在子集$X \equiv A \cup B$上的诱导拓扑,
则拓扑空间$(X, \mathscr{T})$的既开又闭的子集除$X$和$\emptyset$外还有$A$和$B$($A$和$B$在诱导拓扑下自然是开的,又因为$A$和$B$互为补集,故又都是闭的),
所以$(X, \mathscr{T})$是不连通的。这同$A$和$B$在直观上互不连通相吻合\footnote{
与直观感觉更一致的是称为``弧连通''的概念。它与连通概念有微妙差别。
}。
\end{example}

设$(X, \mathscr{T})$为拓扑空间,$A \subset X$。分别定义$A$的闭包、内部和边界如下:

\begin{definition}
$A$的\textbf{闭包}$\bar{A}$是所有含$A$的闭集的交集,即
$$\bar{A} := \bigcap_\alpha C_\alpha, ~ A \subset C_\alpha, ~ -C_\alpha \in \mathscr{T}$$
\end{definition}

\begin{definition}
$A$的\textbf{内部}$\interior(A)$是所有含于$A$的开集的并集,即
$$\interior(A) := \bigcup_\alpha O_\alpha, ~ O_\alpha \subset A, ~ O_\alpha \in \mathscr{T}$$
\end{definition}

\begin{definition}
$A$的\textbf{边界}$\dot{A} := \bar{A} - \interior(A)$,$x \in \dot{A}$称为\textbf{边界点}。$\dot{A}$也记作$\partial A$
\end{definition}

\begin{theorem}
$\bar{A}$,$\interior(A)$及$\dot{A}$有以下性质:
\begin{enumerate}[(a)]
\item \textcircled{1}$\bar{A}$为闭集,\textcircled{2}$A \subset \bar{A}$,\textcircled{3}$A = \bar{A}$当且仅当$A$为闭集。
\item \textcircled{1}$\interior(A)$为开集,\textcircled{2}$\interior(A) \subset A$,\textcircled{3}$\interior(A) = A$当且仅当$A$为开集。
\item $\dot{A}$为闭集
\end{enumerate}
\end{theorem}

\begin{definition}
$X$的开子集的集合$\{O_\alpha\}$叫$A \subset X$的一个\textbf{开覆盖},若$A \subset \bigcup\limits_\alpha O_\alpha$。
也可以说$\{O_\alpha\}$覆盖$A$。
\end{definition}

\section{紧致性}
