\chapter[李导数、Killing场和超曲面]{\\李导数、Killing场和超曲面}

\section{流形间的映射}

设$M$、$N$为流形(维数可不同),$\phi \colon M \to N$为光滑映射。
以$\mathscr{F}_M$和$\mathscr{F}_N$分别代表$M$和$N$上光滑函数的集合,$\mathscr{F}_M(k, l)$和$\mathscr{F}_N(k, l)$分别代表$M$和$N$上光滑$(k,l)$型张量场的集合。
$\phi$自然诱导出一系列映射如下。

\begin{definition}
	\textbf{拉回映射}$\phi^* \colon \mathscr{F}_N \to \mathscr{F}_M$定义为
	$$(\phi^*f)|_p \coloneq f|_{\phi(p)}, ~ \forall f \in \mathscr{F}_N, p \in M$$
	即$\phi^*f = f \comp \phi$。
\end{definition}

由定义不难证明:
\begin{enumerate}[(1)]
	\item $\phi^* \colon \mathscr{F}_N \to \mathscr{F}_M$是线性映射,即
	      $$\phi^*(\alpha f + \beta g) = \alpha\phi^*(f) + \beta\phi^*(g), ~ \forall f,g \in \mathscr{F}_N, ~ \alpha,\beta \in \mathbb{R}$$
	\item $\phi^*(fg) = \phi^*(f)\phi^*(g), ~ \forall f,g \in \mathscr{F}_N$
\end{enumerate}

\begin{definition}
	对$M$中任一点$p$可定义\textbf{推前映射}$\phi_* \colon V_p \to V_{\phi(p)}$如下:$\forall v^a \in V_p$,定义其像$\phi_*v^a \in V_{\phi(p)}$为
	$$(\phi_*v)(f) \coloneq v(\phi*f), ~ \forall f \in \mathscr{F}_N$$
\end{definition}

还应证明这样定义的$\phi_*v^a$满足矢量定义的两个要求,从而确是$\phi(p)$点的矢量。许多文献也把$\phi_*$称为$\phi$的\textbf{切映射}。

\begin{theorem}
	$\phi_* \colon V_p \to V_{\phi(p)}$是线性映射,即
	$$\phi_*(\alpha u^a + \beta v^a) = \alpha\phi_*u^a + \beta\phi_*v^a, ~ \forall u^a,v^a \in V_p, ~ \alpha,\beta \in \mathbb{R}$$
\end{theorem}

\begin{theorem}
	设$C(t)$是$M$中的曲线,$T^a$为曲线在$C(t_0)$点的切矢,则$\phi_*T^a \in V_{\phi(C(t_0))}$是曲线$\phi(C(t))$在$\phi(C(t_0))$点的切矢(曲线的切矢的像是曲线像的切矢)。
\end{theorem}

\begin{proof}
	根据推前映射、曲线切矢以及拉回映射的定义
	$$\phi_*T^a(f)
		= T^a(\phi^*f)
		= \left.\frac{\mathrm{d}(\phi^*f(C))}{\mathrm{d}t}\right|_{t_0}
		= \left.\frac{\mathrm{d}f(\phi(C))}{\mathrm{d}t}\right|_{t_0}$$
	而这就是曲线$\phi(C(t))$在$\phi(C(t_0))$点的切矢。
\end{proof}

\begin{definition}
	拉回映射可按如下方式延拓至$\phi^* \colon \mathscr{F}_N(0, l) \to \mathscr{F}_M(0, l)$:
	$\forall T \in \mathscr{F}_N(0, l)$,定义$\phi^*(T) \in \mathscr{F}_M(0, l)$为
	$$(\phi^*T)_{a_1 \cdots a_l}|_p(v_1)^{a_1}\cdots(v_l)^{a_l} \coloneq T_{a_1 \cdots a_l}|_{\phi(p)}(\phi_*v_1)^{a_1}\cdots(\phi_*v_l)^{a_l}, ~ \forall p \in M, v_1, \cdots, v_l \in V_p$$
\end{definition}

\begin{definition}
	$\forall p \in M$,推前映射$\phi_*$可按如下方式延拓至$\phi_* \colon \mathscr{T}_{V_p}(k, 0) \to \mathscr{T}_{V_{\phi(p)}}(k, 0)$(即$\phi_*$是把$p$点的$(k, 0)$型张量变为$\phi(p)$点的同型张量的映射):
	$\forall T \in \mathscr{T}_{V_p}(k, 0)$,其像$\phi_*(T) \in \mathscr{T}_{V_{\phi(p)}}(k, 0)$由下式定义:
	$$(\phi_*T)^{a_1 \cdots a_l}(\omega^1)_{a_1}\cdots(\omega^k)_{a_k} \coloneq T^{a_1 \cdots a_l}(\phi^*\omega^1)_{a_1}\cdots(\phi^*\omega^k)_{a_k}, ~ \forall \omega^1, \cdots, \omega^k \in V^*_{\phi(p)}$$
	其中$(\phi^*\omega)_a$定义为$(\phi^*\omega)_av^a \coloneq \omega_a(\phi_*v)^a, ~ \forall v^a \in V_p$。
\end{definition}

\begin{note}
	拉回映射$\phi^*$能把$N$上的$(0, l)$型张量场变为$M$上的同型张量场,是\CJKunderdot{场}变为\CJKunderdot{场}的映射;
	推前映射$\phi_*$只能把$M$中一点$p$的$(k, 0)$型张量变为其像点$\phi(p)$的同型张量。
	可否将$\phi_*$延拓为把$M$上的$(k, 0)$型张量\CJKunderdot{场}变为$N$上的同型张量\CJKunderdot{场}的映射?
	在一般情况下不能。以矢量场为例。关键在于,给定$M$上一个矢量场$v$后,要定义$N$上的像矢场$\phi_*v$就要对$N$的任一点$q$定义一个矢量,而这势必涉及$q$点的逆像$\phi^{-1}(q)$。
	如果$\phi$不是到上映射,则$\phi^{-1}(q)$可能不存在,从而无法用$\phi^{-1}(q)$点的$v$作为右边的$v$;
	如果$\phi$不是一一映射,则逆像$\phi^{-1}(q)$可能多于一点,从而无法确定该用哪一逆像点的$v$作为右边的$v$。
	这暗示,如果$\phi$只是光滑映射,则$\phi_*$未必能把场推前为场。
	然而,如果$\phi \colon M \to N$是微分同胚映射,则上述困难不复存在,推前映射$\phi_*$可看作把$M$上的$(k, 0)$型张量\CJKunderdot{场}变为$N$上同型张量\CJKunderdot{场}的映射,即$\phi_* \colon \mathscr{F}_M(k, 0) \to \mathscr{F}_N(k, 0)$。
	再者,由于$\phi^{-1}$存在而且光滑,其拉回映射$\phi^{-1*}$把$\mathscr{F}_M(0, l)$映到$\mathscr{F}_N(0, l)$,这可看作$\phi$的推前映射$\phi_*$,于是$\phi_*$又可进一步推广为$\phi_* \colon \mathscr{F}_M(k, l) \to \mathscr{F}_N(k, l)$。
	例如,设$T^a{}_b \in \mathscr{F}_M(1, 1)$,则$(\phi_*T)^a{}_b \in \mathscr{F}_N(1, 1)$定义为
	$$(\phi_*T)^a{}_b|_q\omega_av^b \coloneq T^a{}_b|_{\phi^{-1}(q)}(\phi^*\omega)_a(\phi^*v)^b, ~ \forall q \in N, \omega_a \in V^*_q, v^b \in V_q$$
	其中,$(\phi^*v)^b$应理解为$(\phi^{-1}_*v)^b$。
	同理,拉回映射也可推广为$\phi^* \colon \mathscr{F}_N(k, l) \to \mathscr{F}_M(k, l)$。
	推广后的$\phi_*$和$\phi^*$仍为线性映射,而且互逆。
\end{note}

设$\phi \colon M \to N$是微分同胚,$p \in M$,$\{x^\mu\}$和$\{y^\mu\}$分别是$M$和$N$的局域坐标系,坐标域$O_1$和$O_2$满足$p \in O_1, \phi(p) \in O_2$。
于是$p \in \phi^{-1}[O_2]$。$\phi$为微分同胚保证$M$和$N$的维数相等,故$\{x^\mu\}$和$\{y^\mu\}$的$\mu$都是从$1$到$n$。
微分同胚本是点的变换,但也可等价地看作坐标变换,因为可用$\phi \colon M \to N$在$\phi^{-1}[O_2]$上定义一组新坐标$\{x'^\mu\}$如下:
$\forall q \in \phi^{-1}[O_2]$,定义$x'^\mu(q) \coloneq y^\mu(\phi(q))$。
可见微分同胚映射$\phi$在$p$的邻域$O_1 \cap \phi^{-1}[O_2]$上自动诱导出一个坐标变换$x^\mu \mapsto x'^\mu$。
由前述定理不难证明$\forall q \in O_1 \cap \phi^{-1}[O_2]$有
$$\phi_*[(\partial / \partial x'^\mu)^a|_q] = (\partial / \partial y^\mu)^a|_{\phi(q)}$$
由此又可证明
$$\phi_*[(\mathrm{d}x'^\mu)^a|_q] = (\mathrm{d}y^\mu)^a|_{\phi(q)}$$
于是对微分同胚映射$\phi \colon M \to N$就存在两种观点:
\textcircled{1} \textbf{主动观点},它如实地认为$\phi$是点的变换(把$p$变为$\phi(p)$)以及由此导致的张量变换(把$p$点的张量$T$变为$\phi(p)$点的张量$\phi_*T$);
\textcircled{2} \textbf{被动观点},它认为点$p$及其上的所有张量$T$都没变,$\phi \colon M \to N$的后果是坐标系有了变换(从$\{x^\mu\}$变为$\{x'^\mu\}$)。
这两种观点虽然似乎相去甚远,但在实用上是等价的。
下面的定理可以看作等价性的某种表现。

\begin{theorem}
	$(\phi_*T)^{\mu_1 \cdots \mu_k}{}_{\nu_1 \cdots \nu_l}|_{\phi(p)} = T'^{\mu_1 \cdots \mu_k}{}_{\nu_1 \cdots \nu_l}|_p, ~ \forall T \in \mathscr{F}_M(k, l)$

	式中左边是新点$\phi(p)$的新张量$\phi_*T$在老坐标系$\{y^\mu\}$的分量,右边是老点$p$的老张量$T$在新坐标系$\{x'^\mu\}$的分量。
\end{theorem}

\begin{note}
	上式是实数等式,左边是由主动观点(认为点和张量变了而坐标系没变)所得的数,右边是由被动观点(认为点和张量没变但坐标系变了)所得的数。
	两边相等就表明两种观点在实用上等价。
\end{note}

\begin{example}
	设$v^a \in V_p$,令$u^a \equiv \phi_*v^a \in V_{\phi(p)}$,则由上述定理不难证明
	$$u^\mu = v^\nu(\partial x'^\mu / \partial x^\nu)|_p$$
\end{example}

\section{李导数}

根据前面的知识我们知道$M$上的一个光滑矢量场$v^a$给出一个单参微分同胚群$\phi$。\footnote{
	若$v^a$不完备,则只能给出单参微分同胚局部群。本节只涉及局部性质,无须明确区分局部和整体。
}设$T^{\cdots}{}_{\cdots}$是$M$上的光滑张量场,则$\phi^*_tT^{\cdots}{}_{\cdots}$也是同型光滑张量场,其中$\phi_t$是单参微分同胚群$\phi$的一个群元。
这两个张量场在点$p \in M$的值之差$\phi^*_tT^{\cdots}{}_{\cdots}|_p - T^{\cdots}{}_{\cdots}|_p$是$p$点的张量,它与$t$之商$(\phi^*_tT^{\cdots}{}_{\cdots}|_p - T^{\cdots}{}_{\cdots}|_p) / t$在$t$趋于零时的极限可看作张量场$T^{\cdots}{}_{\cdots}$在$p$点的某种导数,于是有以下定义:

\begin{definition}
	$\mathscr{L}_vT^{a_1 \cdots a_k}{}_{b_1 \cdots b_l} \coloneq \lim\limits_{t \to 0}\frac{1}{t}(\phi^*_tT^{a_1 \cdots a_k}{}_{b_1 \cdots b_l} - T^{a_1 \cdots a_k}{}_{b_1 \cdots b_l})$

	称为张量场$T^{a_1 \cdots a_k}{}_{b_1 \cdots b_l}$沿矢量场$v^a$的\textbf{李导数}($\mathscr{L}_v$中的$v$不写为$v^a$,以免误解。)。
\end{definition}

\begin{note}
	因$\phi^*_t$为线性映射,故李导数是由$\mathscr{F}_M(k, l)$到$\mathscr{F}_M(k, l)$的线性映射。还可证明$\mathscr{L}_v$同缩并可交换顺序。
\end{note}

\begin{theorem}
	$\mathscr{L}_vf = v(f), ~ \forall f \in \mathscr{F}$
\end{theorem}

\begin{proof}
	$\forall p \in M$,设$C(t)$是$\phi$过$p$点的轨道,$p = C(0)$,则$\phi_t(p) = C(t)$且$v^a|_p \equiv (\partial / \partial t)^a|_p$是$C(t)$在$p$点的切矢,故
	$$\begin{aligned}
			\mathscr{L}_vf|_p & = \lim_{t \to 0}\frac{1}{t}(\phi_t^*f - f)|_p = \lim_{t \to 0}\frac{1}{t}[f(\phi_t(p)) - f(p)]                \\
			                  & = \lim_{t \to 0}\frac{1}{t}[f(C(t)) - f(C(0))] = \frac{\mathrm{d}}{\mathrm{d}t}(f \comp C)|_{t = 0} = v(f)|_p
		\end{aligned}$$
\end{proof}

下面以$n = 2$为例介绍一种对计算李导数很有用的坐标系。
设$\{x^1, x^2\}$为坐标系,则$x^1$坐标线和$x^2$坐标线组成坐标``网格'',欲知坐标域中某点的坐标,只须看它位于网格的哪两条坐标线的交点。
求李导数时总要给定矢量场$v^a$,可以选定它的积分曲线为$x^1$坐标线($t$充当$x^1$),再相当任意地选定另一组与这组曲线横截(即交点上两线切矢不平行)的曲线作为$x^2$坐标线,这样得到的坐标系称为矢量场$v^a$的\textbf{适配坐标系}。
换句话说,矢量场$v^a$就是其适配坐标系的第一坐标基矢场,即$v^a = (\partial / \partial x^1)^a$。
以上讨论可推广至任意维流形。

\begin{theorem}
	张量场$T^{a_1 \cdots a_k}{}_{b_1 \cdots b_l}$沿$v^a$的李导数在$v^a$的适配坐标系的分量
	$$(\mathscr{L}_vT)^{\mu_1 \cdots \mu_k}{}_{\nu_1 \cdots \nu_l} = \frac{\partial T^{\mu_1 \cdots \mu_k}{}_{\nu_1 \cdots \nu_l}}{\partial x^1}$$
\end{theorem}

\begin{note}
	上式左边在坐标变换时满足张量变换律而右边则否,故不能改写为张量等式。
\end{note}

\begin{proof}
	仅以$n = 2, k = l = 1$为例(容易推广至一般情况)。
	因$\phi_t^* = (\phi_t^{-1})_* = \phi_{-t*}$,李导数定义式在任一坐标系(现在取适配坐标系)的分量式为
	$$(\mathscr{L}_vT)^{\mu}{}_{\nu}|_p = \lim\limits_{t \to 0}\frac{1}{t}[(\phi_{-t*}T)^{\mu}{}_{\nu}|_p - T^{\mu}{}_{\nu}|_p], ~ \forall p \in M$$
	令$q \equiv \phi_t(p)$,因上式只涉及$p$点附近的情况,总可以认为$p, q$点都在同一适配坐标域内。
	对$\phi_{-t}$而言,$q$为老点,$p$为新点,故由坐标变换``主动观点''和``被动观点''的等价性得
	$$(\phi_{-t*}T)^{\mu}{}_{\nu}|_p = T'^{\mu}{}_{\nu}|_q = \left[\frac{\partial x'^{\mu}}{\partial x^{\rho}}\frac{\partial x^{\sigma}}{\partial x'^{\nu}}T^{\rho}{}_{\sigma}\right]_q$$
	其中第二步用到张量的分量变换律。式中$x^{\sigma}$为适配(老)坐标,$x'^{\mu}$是由$\phi_{-t}$诱导的新坐标。
	上式右边涉及新老坐标间的偏导数在$q$点的值,要计算就必须找出$q$点的一个小邻域$N$内的坐标变换。
	$\forall \bar{q} \in N$,记$\bar{p} \equiv \phi_{-t}(\bar{q})$。
	由适配坐标的定义知$x^1(\bar{q}) = x^1(\bar{p}) + t, x^2(\bar{q}) = x^2(\bar{p})$,而按定义,$\phi_{-t}$在$\bar{q}$诱导的新坐标则为$x'^1(\bar{q}) \equiv x^1(\bar{p}), x'^2(\bar{q}) \equiv x^2(\bar{p})$,故$x'^1(\bar{q}) = x^1(\bar{q}) - t, x'^2(\bar{q}) = x^2(\bar{q})$。
	因为$\bar{q}$为$N$内任一点,故对$N$有$x'^1 = x^1 - t, x'^2 = x^2$,求导得$(\partial x'^{\mu} / \partial x^{\rho})|_q = \delta^{\mu}{}_{\rho}, (\partial x^{\sigma} / \partial x'^{\nu})|_q = \delta^{\sigma}{}_{\nu}$,于是上式成为$(\phi_{-t*}T)^{\mu}{}_{\nu}|_p = T^{\mu}{}_{\nu}|_q$,带入前式便得$(\mathscr{L}_vT)^{\mu}{}_{\nu}|_p = \partial T^{\mu}{}_{\nu} / \partial x^1|_p$。
\end{proof}

由上述定理可知$\mathscr{L}_v$满足莱布尼茨律。

\begin{theorem}
	$\mathscr{L}_vu^a = [v, u]^a, ~ \forall u^a, v^a \in \mathscr{F}(1, 0)$

	或者,借助于对易子的表达式,有
	$$\mathscr{L}_vu^a = v^b\nabla_bu^a - u^b\nabla_bv^a$$
	其中$\nabla_a$为任一无挠导数算符。
\end{theorem}

\begin{proof}
	待证命题是矢量等式,只须证明它在某一坐标系的分量等式$(\mathscr{L}_vu)^{\mu} = [v, u]^{\mu}$成立。
	最方便的当然是适配坐标系。设$v^a$的适配坐标系$\{x^\mu\}$的普通导数算符是$\partial_a$,则
	$$[v, u]^\mu = (\mathrm{d}x^\mu)_a[v, u]^a = (\mathrm{d}x^\mu)_a(v^b\partial_bu^a - u^b\partial_bv^a) = v^b\partial_bu^\mu = v(u^\mu) = \partial u^\mu / \partial x^1 = (\mathscr{L}_vu)^\mu$$
	其中第三步是因为$v^a = (\partial / \partial x^1)^a$,而普通导数算符作用于任一坐标基矢为零,第四步用到导数算符的定义条件,最后一步用到前面的定理。
\end{proof}

\begin{theorem}
	$\mathscr{L}_v\omega_a = v^b\nabla_b\omega_a + \omega_b\nabla_av^b, ~ \forall \omega^a \in \mathscr{F}(0, 1), v^a \in \mathscr{F}(1, 0)$

	其中$\nabla_a$为任一无挠导数算符。
\end{theorem}

\begin{proof}
	一方面$\mathscr{L}_v(\omega_au^a) = v^b\nabla_b(\omega_au^a) = v^b\nabla_b\omega_au^a + v^b\omega_a\nabla_bu^a$。
	另一方面$\mathscr{L}_v(\omega_au^a) = \mathscr{L}_v\omega_au^a + \omega_a\mathscr{L}_vu^a = \mathscr{L}_v\omega_au^a + \omega_av^b\nabla_bu^a - \omega_au^b\nabla_bv^a$。
	因此,$\mathscr{L}_v\omega_au^a = v^b\nabla_b\omega_au^a + \omega_au^b\nabla_bv^a$。该式对任意的$u^a$都成立,因此命题得证。
\end{proof}

\begin{theorem}
	$$\mathscr{L}_vT^{a_1 \cdots a_k}{}_{b_1 \cdots b_l} = v^c\nabla_cT^{a_1 \cdots a_k}{}_{b_1 \cdots b_l} - \sum^k_{i = 1}T^{a_1 \cdots c \cdots a_k}{}_{b_1 \cdots b_l}\nabla_cv^{a_i} + \sum^l_{j = 1}T^{a_1 \cdots a_k}{}_{b_1 \cdots c \cdots b_l}\nabla_{b_j}v^c$$
	$\forall T \in \mathscr{F}(k, l), v \in \mathscr{F}(1, 0)$,$\nabla_a$为任一导数算符。
\end{theorem}

\section{Killing矢量场}

本章至此未涉及度规及与之适配的导数算符,李导数的定义不要求流形$M$有附加结构。
但若$M$上选定了度规场$g_{ab}$,则对微分同胚映射$\phi \colon M \to M$还可提出更高的要求,即$\phi^*g_{ab} = g_{ab}$。
于是有如下定义:

\begin{definition}
	微分同胚$\phi \colon M \to M$称为\textbf{等度规映射},简称\textbf{等度规},若$\phi^*g_{ab} = g_{ab}$。
\end{definition}

\begin{note}
	\textcircled{1} 等度规映射是特殊的微分同胚映射,其特殊性在于``保度规''性,即$\phi^*g_{ab} = g_{ab}$。
	注意这是张量场的等式,其含义是每点$p$的两个张量$g_{ab}|_p$和$\phi^*g_{ab}|_p$相等。
	\textcircled{2} 由$\phi^{-1*} \comp \phi^* = (\phi \comp \phi^{-1})^* = \text{恒等映射}$易见$\phi \colon M \to M$为等度规映射当且仅当$\phi^{-1} \colon M \to M$为等度规映射。
\end{note}

流形$M$上众多矢量场中有一类特殊矢量场,即光滑矢量场。
每一光滑矢量场给出一个单参微分同胚群。
如果$M$上指定了度规场$g_{ab}$,则众多光滑矢量场中还可挑出特殊的一个子类,其中每个矢量场给出的单参微分同胚群是单参等度规群,即每个群元$\phi_t \colon M \to M$是$M$上的一个等度规映射。
于是有以下定义:

\begin{definition}
	$(M, g_{ab})$上的矢量场$\xi^a$称为\textbf{Killing矢量场},若它给出的单参微分同胚(局部)群是单参等度规(局部)群。
	等价地(还可以证明),$\xi^a$称为Killing矢量场,若$\mathscr{L}_{\xi}g_{ab} = 0$。
\end{definition}

\begin{theorem}
	$\xi^a$为Killing矢量场的充要条件是$\xi^a$满足如下的\textbf{Killing方程}:
	$$\nabla_a\xi_b + \nabla_b\xi_a = 0, ~ \text{或} \nabla_{(a}\xi_{b)} = 0, ~ \text{或} \nabla_a\xi_b = \nabla_{[a}\xi_{b]}$$
	(其中$\nabla_a$满足$\nabla_ag_{bc} = 0$)
\end{theorem}

\begin{proof}
	$0 = \mathscr{L}_{\xi}g_{ab} = \xi^c\nabla_cg_{ab} + g_{cb}\nabla_a\xi^c + g_{ac}\nabla_b\xi^c = \nabla_a\xi_b + \nabla_b\xi_a$
\end{proof}

\begin{theorem}
	若存在坐标系$\{x^\mu\}$使$g_{ab}$的全部分量满足$\partial g_{\mu\nu} / \partial x^1 = 0$,则$(\partial / \partial x^1)^a$是坐标域上的Killing矢量场。
\end{theorem}

\begin{proof}
	$\{x^\mu\}$是$(\partial / \partial x^1)^a$的适配坐标系。
	因此,$(\mathscr{L}_{\partial / \partial x^1}g)_{\mu\nu} = \partial g_{\mu\nu} / \partial x^1 = 0$,故$\mathscr{L}_{\partial / \partial x^1}g_{ab} = 0$,即$(\partial / \partial x^1)^a$为Killing矢量场。
\end{proof}

\begin{theorem}
	设$\xi^a$为Killing矢量场,$T^a$为测地线的切矢,则$T^a\nabla_a(T^b\xi_b) = 0$,即$T^b\xi_b$在测地线上为常数。
\end{theorem}

\begin{proof}
	$$T^a\nabla_a(T^b\xi_b) = T^a\nabla_aT^b\xi_b + T^aT^b\nabla_a\xi_b = T^{(a}T^{b)}\nabla_{a}\xi_{b} = T^{a}T^{b}\nabla_{(a}\xi_{b)} = 0$$
	其中用到测地线方程和Killing方程。
\end{proof}

设$\xi^a, \eta^a$是Killing矢量场,$\alpha, \beta$是常实数,则由Killing方程的线性性知$\alpha\xi^a + \beta\eta^a$也是Killing矢量场。
不难看出$M$上所有Killing矢量场的集合是个矢量空间。还可证明对易子$[\xi, \eta]^a$也是Killing矢量场。

\begin{theorem}
	$(M, g_{ab})$上最多有$n(n + 1) / 2$个独立的Killing矢量场($n \equiv \dim M$),即$M$上所有Killing矢量场的集合(作为矢量空间)的维数小于等于$n(n + 1) / 2$。
\end{theorem}

\begin{note}
	\textcircled{1} 等度规映射可看作一种``保度规''的对称变换,所以一个Killing矢量场代表$(M, g_{ab})$的一个对称性,具有$n(n + 1) / 2$个独立Killing矢量场的广义黎曼空间$(M, g_{ab})$称为最高对称性空间。
	\textcircled{2} 寻找$(M, g_{ab})$的全体Killing矢量场的一般方法是求Killing方程的通解。然而对某些简单的$(M, g_{ab})$还存在容易得多的方法。下面仅举数例。
\end{note}

\begin{example}
	找出下列广义黎曼空间的全体独立的Killing矢量场。
	\begin{enumerate}[(1)]
		\item $2$维欧氏空间$(\mathbb{R}^2, \delta_{ab})$。
		      设$\{x, y\}$为笛卡尔坐标系,则$\mathrm{d}s^2 = \mathrm{d}x^2 + \mathrm{d}y^2$,即欧氏度规$\delta_{ab}$在此系中的全部分量为常数,故由前述定理知$(\partial / \partial x)^a$和$(\partial / \partial y)^a$为Killing矢量场。
		      我们相信欧氏空间有最高对称性,$n = 2$时应有$3$个独立的Killing矢量场。
		      果然,若改用极坐标系,便有$\mathrm{d}s^2 = \mathrm{d}r^2 + r^2\mathrm{d}\varphi^2$,可见欧氏度规$\delta_{ab}$在此系中的全部分量与$\varphi$无关,所以$(\partial / \partial \varphi)^a$为Killing矢量场,它在笛卡尔系的坐标基底的展开式为$(\partial / \partial \varphi)^a = -y(\partial / \partial x)^a + x(\partial / \partial y)^a$。
		      展开系数与坐标有关,由此不难证明$(\partial / \partial \varphi)^a$独立于前两个Killing场。
		      $(\partial / \partial x)^a$和$(\partial / \partial y)^a$的Killing性反映$2$维欧氏度规沿$x$和$y$轴的平移不变性,$(\partial / \partial \varphi)^a$的Killing性表明它有旋转不变性。
		\item $3$维欧氏空间$(\mathbb{R}^3, \delta_{ab})$。
		      因为$n = 3$,故有$6$个独立的Killing矢量场,即$(\partial / \partial x)^a, (\partial / \partial y)^a, (\partial / \partial z)^a, -y(\partial / \partial x)^a + x(\partial / \partial y)^a, -z(\partial / \partial y)^a + y(\partial / \partial z)^a$和$-x(\partial / \partial z)^a + z(\partial / \partial x)^a$。
		      前$3$个反映$3$维欧氏度规沿$x, y, z$轴的平移不变性;后$3$个反映它绕$z, x, y$轴的旋转不变性。
		\item $2$维闵氏空间$(\mathbb{R}^2, \eta_{ab})$。
		      在洛伦兹坐标系$\{t, x\}$中有$\mathrm{d}s^2 = -\mathrm{d}t^2 + \mathrm{d}x^2$,故知$(\partial / \partial t)^a$和$(\partial / \partial x)^a$为Killing场。
		      为求第三个,用下式定义新坐标$\psi, \eta$:
		      $$x = \psi\operatorname{ch}\eta, ~ t = \psi\operatorname{sh}\eta, ~ 0 < \psi < \infty, ~ -\infty < \eta < \infty$$
		      闵氏线元可用新坐标表为$\mathrm{d}s^2 = \mathrm{d}\psi^2 - \psi^2\mathrm{d}\eta^2$。
		      上式表明$\eta_{ab}$在新坐标系的全体分量与坐标$\eta$无关,故$(\partial / \partial \eta)^a$也是Killing矢量场(其积分曲线是双曲线),它在洛伦兹坐标基底的展开式为
		      $$(\partial / \partial \eta)^a = t(\partial / \partial x)^a + x(\partial / \partial t)^a$$
		      由展开系数与坐标有关可知$(\partial / \partial \eta)^a$与前两个Killing场独立。
		      不难验证$(\partial / \partial \eta)^a$是$\mathbb{R}^2$上的Killing矢量场,叫做\textbf{伪转动}(boost)Killing矢量场,表明闵氏度规具有伪转动下的不变性,对应于洛伦兹变换。
		\item $4$维闵氏空间$(\mathbb{R}^4, \eta_{ab})$。
		      因$n = 4$,故独立的Killing场共$10$个,分三组:
		      \begin{enumerate}[(a)]
			      \item $4$个平移$(\partial / \partial t)^a, (\partial / \partial x)^a, (\partial / \partial y)^a, (\partial / \partial z)^a$;
			      \item $3$个空间转动$-y(\partial / \partial x)^a + x(\partial / \partial y)^a, -z(\partial / \partial y)^a + y(\partial / \partial z)^a, -x(\partial / \partial z)^a + z(\partial / \partial x)^a$;
			      \item $3$个伪转动$t(\partial / \partial x)^a + x(\partial / \partial t)^a, t(\partial / \partial y)^a + y(\partial / \partial t)^a, t(\partial / \partial z)^a + z(\partial / \partial t)^a$
		      \end{enumerate}
		      组(a)反映闵氏度规沿$t, x, y, z$轴的平移不变性,组(b)反映它绕$z, y, x$轴的空间旋转不变性,组(c)反映它在$tx, ty, tz$面内的伪转动下的不变性。
	\end{enumerate}
\end{example}

\begin{theorem}
	设$\{x, t\}$是$2$维闵氏空间$(\mathbb{R}^2, \eta_{ab})$的洛伦兹坐标系,$\phi_\lambda \colon \mathbb{R}^2 \to \mathbb{R}^2$是伪转动Killing场$\xi^a \equiv t(\partial / \partial x)^a + x(\partial / \partial t)^a$对应的单参等度规群的一个群元(即以参数$\lambda \in \mathbb{R}$刻画的那个等度规映射),则由$\phi_\lambda$诱导的坐标变换$\{x, t\} \mapsto \{x', t'\}$是洛伦兹变换。
\end{theorem}

\begin{note}
	本定理表明伪转动和洛伦兹变换是同一变换的两种(主动与被动)提法。
	类似地,欧氏空间的转动Killing场$-y(\partial / \partial x)^a + x(\partial / \partial y)^a$与坐标变换
	$$x' = x\cos\alpha - y\sin\alpha, ~ y' = x\sin\alpha + y\cos\alpha$$
	也是同一变换的两种提法。
\end{note}

\begin{proof}
	矢量场$\xi^a \equiv (\partial / \partial\eta)^a$的积分曲线的参数方程为$\mathrm{d}x^\mu(\eta) / \mathrm{d}\eta = \xi^\mu$($\mu = 0, 1$)。
	注意到$\xi^a \equiv t(\partial / \partial x)^a + x(\partial / \partial t)^a$,便得
	$$\frac{\mathrm{d}x(\eta)}{\mathrm{d}\eta} = t(\eta), ~ \frac{\mathrm{d}t(\eta)}{\mathrm{d}\eta} = x(\eta)$$
	$\forall p \in \mathbb{R}^2$,设$C(\eta)$是满足$p = C(0)$的积分曲线,即$x(0) = x_p, t(0) = t_p$,则不难证明上述方程的特解(即该线的参数式)为
	$$x(\eta) = x_p\operatorname{ch}\eta + t_p\operatorname{sh}\eta, ~ t(\eta) = x_p\operatorname{sh}\eta + t_p\operatorname{ch}\eta$$
	设$q \equiv \phi_\lambda(p)$,则$q$就是$C(\eta)$上参数值$\eta = \lambda$的点,即$q = C(\lambda)$,故由$\phi_\lambda$诱导的新坐标$t'$和$x'$满足
	$$x'_p \equiv x_q = x_p\operatorname{ch}\lambda + t_p\operatorname{sh}\lambda, ~ t'_p \equiv t_q = x_p\operatorname{sh}\lambda + t_p\operatorname{ch}\lambda$$
	因$p$点任意,故可去掉下标$p$而写成
	$$x' = x\operatorname{ch}\lambda + t\operatorname{sh}\lambda = \operatorname{ch}\lambda(x + t\operatorname{th}\lambda), ~ t' = t\operatorname{ch}\lambda + x\operatorname{sh}\lambda = \operatorname{ch}\lambda(t + x\operatorname{th}\lambda)$$
	令$v \equiv \operatorname{th}\lambda, \gamma \equiv (1 - v^2)^{-1/2} = \operatorname{ch}\lambda$,则
	$$x' = \gamma(x + vt), ~ t'= \gamma(t + vx)$$
	这便是熟知的洛伦兹变换(注意,我们用几何单位制,其中光速$c = 1$。)。
\end{proof}

由$\mathrm{d}s^2 = -\mathrm{d}t^2 + \mathrm{d}x^2$和上述定理易得$\mathrm{d}s^2 = -\mathrm{d}t'^2 + \mathrm{d}x'^2$,可见伪转动对应的等度规映射诱导的坐标变换把洛伦兹系$\{t, x\}$变为洛伦兹系$\{t', x'\}$。
此结果可推广为如下定理:

\begin{theorem}
	设$\{x^\mu\}$是$(\mathbb{R}^n, \eta_{ab})$的洛伦兹坐标系,则$\{x'^\mu\}$也是洛伦兹坐标系的充要条件是它由$\{x^\mu\}$通过等度规映射$\phi \colon \mathbb{R}^n \to \mathbb{R}^n$诱导而得。
\end{theorem}

\begin{proof}
	把$\eta_{ab}$记作$g_{ab}$,其在$\{x^\mu\}$和$\{x'^\mu\}$系的分量分别记作$g_{\mu\nu}$和$g'_{\mu\nu}$。
	\begin{enumerate}[(A)]
		\item 设$\phi \colon \mathbb{R}^n \to \mathbb{R}^n$是等度规映射(即$\phi^*g_{ab} = g_{ab}$),$\{x'^\mu\}$是由洛伦兹系$\{x^\mu\}$通过$\phi$诱导而得的坐标系,则$\forall p \in \mathbb{R}^n$有
		      $g'_{\mu\nu}|_p = (\phi_*g)_{\mu\nu}|_{\phi(p)} = (\phi^{-1*}g)_{\mu\nu}|_{\phi(p)} = g_{\mu\nu}|_{\phi(p)} = \eta_{\mu\nu}$。
		      其中第一步是坐标变换的主被动观点的等价性,最后一步用到$\{x^\mu\}$的洛伦兹性。上式说明$p$点的$g_{ab}$在$\{x'^\mu\}$系的分量为$\eta_{\mu\nu}$,故$\{x'^\mu\}$为洛伦兹系。
		\item 设$\{x^\mu\}$和$\{x'^\mu\}$都是洛伦兹系,$\phi \colon \mathbb{R}^n \to \mathbb{R}^n$是与坐标变换$\{x^\mu\} \mapsto \{x'^\mu\}$对应的微分同胚映射,则$\forall p \in \mathbb{R}^n$有
		      $(\phi^{-1*}g)_{\mu\nu}|_p = (\phi_*g)_{\mu\nu}|_p = g'_{\mu\nu}|_{\phi^{-1}(p)} = \eta_{\mu\nu} = g_{\mu\nu}|_p$,
		      其中,最后两步用到$\{x^\mu\}$和$\{x'^\mu\}$的洛伦兹性。上式表明$\phi^{-1*}g_{ab} = g_{ab}$,故$\phi^{-1}$(因而$\phi$)是等度规映射。
	\end{enumerate}
\end{proof}

\begin{note}
	本定理也适用于欧氏空间,只须把洛伦兹系改为笛卡尔系。可以说等度规映射保持坐标系的洛伦兹(笛卡尔)性。
\end{note}

\section{超曲面}

\begin{definition}
	设$M$,$S$为流形,$\dim S \le \dim M \equiv n$。映射$\phi \colon S \to M$称为\textbf{嵌入},若$\phi$是一一和$C^\infty$的,而且$\forall p \in S$,推前映射$\phi_* \colon V_p \to V_{\phi(p)}$非退化($V_{\phi(p)}$是指$\phi(p)$作为$M$的一点的切空间),即$\phi_*v^a = 0 \Rightarrow v^a = 0$.
\end{definition}

\begin{note}
	嵌入的上述条件使$S$的拓扑和流形结构可自然地被带到$\phi[S]$上去,从而使$\phi \colon S \to \phi[S]$成为微分同胚。
\end{note}

\begin{definition}
	嵌入$\phi \colon S \to M$称为$M$的一个嵌入子流形,简称子流形。也常把映射的像$\phi[S]$称为\textbf{嵌入子流形}。若$\dim S = n - 1$,则$\phi[S] \subset M$称为$M$的一张\textbf{超曲面}。
\end{definition}


\begin{example}
	设$U$是$M$的开子集,把$M$的流形结构限制在$U$上,$U$便成为与$M$同维的流形。把$U$看作定义中的$S$,令$\phi \colon U \to M$为恒等映射,则$U \equiv \phi[U]$便是$M$的一个嵌入子流形(同维嵌入)。
\end{example}

\begin{example}
	设$S$是$\mathbb{R}^3$(看作$M$)中的单位球面$S^2$,则恒等映射$\phi \colon S^2 \to \mathbb{R}^3$给出$\mathbb{R}^3$的一个嵌入子流形。注意到$S^2$比$\mathbb{R}^3$低一维,可知$S^2$是$\mathbb{R}^3$的一个超曲面。
\end{example}

设$\phi[S]$是$M$的超曲面,$q \in \phi[S] \subset M$。作为$M$的一点,$q$有$n$维切空间$V_q$。
若$w^a \in V_q$是过$q$且躺在$\phi[S]$上的某曲线的切矢(``躺在''是指曲线每点都在$\phi[S]$上),则说$w^a$切于$\phi[S]$。
$V_q$中全体切于$\phi[S]$的元素构成的子集记作$W_q$。超曲面的定义保证$W_q$是$V_q$的$n - 1$维子空间。
谈到超曲面时自然想到它的法矢。设$\phi[S]$是超曲面,$q \in \phi[S]$,则过$q$点的法矢$n^a$应定义为与$q$点所有切于$\phi[S]$的矢量正交的矢量。
然而正交性只有在指定度规后才有意义。当$M$没有度规时,不能定义法矢$n^a$,但可定义``法余矢''$n_a$。\textbf{余矢}是对偶矢量的别名。由于对偶矢量作用于矢量给出实数(无需度规),可定义法余矢如下:

\begin{definition}
	设$\phi[S]$是超曲面,$q \in \phi[S]$。非零对偶矢量$n_a \in V_q^*$称为$\phi[S]$在$q$点的\textbf{法余矢},若$n_aw^a = 0 ~ \forall w^a \in W_q$。
\end{definition}

\begin{theorem}
	超曲面$\phi[S]$上任一点$q$必有法余矢$n_a$。法余矢不唯一,但$q$点的任意两个法余矢之间只能差一实数因子。
\end{theorem}

\begin{proof}
	设$\{(e_2)^a, \cdots , (e_n)^a\}$为$W_q$任一基底,因$dim V_q = n$,$V_q$必有与$\{(e_2)^a, \cdots , (e_n)^a\}$线性无关的元素,任取其一并记作$(e_1)^a$,则$\{(e_\mu)^a \mid \mu = 1, \cdots, n\}$为$V_q$的基底,其对偶基底记作$\{(e^\mu)_a\}$。
	令$n_a = (e^1)_a$,则$n_a(e_\tau)^a = \delta^1{}_\tau = 0 ~ (\tau = 2, \cdots, n)$,故$n_aw^a = 0 ~ \forall w^a \in W_q$,可见$n_a$为法余矢。若存在$m_a$满足$m_a(e_\tau)^a = 0 ~ (\tau = 2, \cdots, n)$,则其在对偶基底$\{(e^\mu)_a\}$的分量$m_\tau = m_a(e_\tau)^a = 0 ~ (\tau = 2, \cdots, n)$,因而$m_a = m_1(e^1)_a = m_1n_a$,即$m_a$与$n_a$只差一因子$m_1$。
\end{proof}

\begin{note}
	非超曲面嵌入子流形(如$3$维流形中的曲线)的法余矢没有这样好的唯一性。
\end{note}

\begin{theorem}
	设$\phi[S]$是由$f = \text{常数}$给出的超曲面,则面上的$\nabla_af$是超曲面的法余矢。
\end{theorem}

\begin{proof}
	只须对任一$q \in \phi[S]$证明$w^a\nabla_af = 0 ~ \forall w^a \in W_q$。
	因$w^a$总切于过$q$并躺在$\phi[S]$上的某曲线$C(t)$,故$w^a\nabla_af = \frac{\partial}{\partial t}(f) = 0 ~ \forall w^a \in W_q$,最后一步是因$f$在$C(t)$上为常数。
\end{proof}

若$M$上有度规$g_{ab}$,则$n^a \equiv g^{ab}n_b \in V_q$与$\phi[S]$的所有矢量正交(因$g_{ab}n^aw^b = n_bw^b = 0 ~ \forall w^a \in W_q$),故$n^a$叫超曲面$\phi[S]$在$q$点的\textbf{法矢}。
若$g_{ab}$为正定度规(例如$\mathbb{R}^2$嵌入$3$维欧氏空间),$n^a$自然不属于$W_q$,即$n^a \in V_q - W_q$;然而,若$g_{ab}$为洛伦兹度规,$n^a$却有可能属于$W_q$。以下就$g_{ab}$为洛伦兹度规的情况进行讨论。

\begin{theorem}
	$n^a \in W_q$的充要条件为$n^an_a = 0$。
\end{theorem}

\begin{proof}
	\begin{enumerate}[(A)]
		\item 设$n^a \in W_q$,则$n^a$可看作$n_aw^a = 0$中的$w^a$,故$n_an^a = 0$。
		\item 由前面定理的证明知道对任一法余矢$n_a$存在基底$\{(e_\mu)^a\}$使$(e_2)^a, \cdots, (e_n)^a \in W_q$且$n_a = (e^1)_a$,故$n^a$在该基底的第一分量$n^1 = n^a(e^1)_a = n^an_a$。
		      因此$n_an^a = 0 \Rightarrow n^1 = 0 \Rightarrow n^a = \sum^n_{\tau = 2}n^\tau(e_\tau)^a \in W_q$。
	\end{enumerate}
\end{proof}

\begin{example}
	设$S = \mathbb{R}, M = \mathbb{R}^2, M$上度规$g_{ab} = \eta_{ab}, \phi \colon \mathbb{R} \to \mathbb{R}^2$为嵌入,则$\phi[\mathbb{R}]$是$2$维闵氏时空中的超曲面。
	设$t, x$为洛伦兹坐标,讨论以下三种有代表性的情况:
	\begin{enumerate}[(1)]
		\item $\phi[\mathbb{R}]$与$x$轴平行。$\forall q \in \phi[\mathbb{R}]$,令$(e_2)^a = (\partial / \partial x)^a$,选
		$$(e_1)^a = \alpha(\partial / \partial t)^a + \beta(\partial / \partial x)^a, \text{($\alpha, \beta$可为任意实数,但$\alpha \neq 0$。)}$$
		则不难验证$(e^1)_a = \alpha^{-1}(\mathrm{d}t)_a$。根据前面定理的证明过程可知$(e^1)_a$为法余矢$n_a$,相应的法矢为$n^a = \alpha^{-1}g^{ab}(\mathrm{d}t)_b = -\alpha^{-1}(\partial / \partial t)^a$,满足$n^a \notin W_q$且$n_an^a < 0$(即$n^a$为类时)。
		\item $\phi[\mathbb{R}]$与$t$轴平行。$\forall q \in \phi[\mathbb{R}]$,令$(e_2)^a = (\partial / \partial t)^a$,选
		$$(e_1)^a = \alpha(\partial / \partial t)^a + \beta(\partial / \partial x)^a, \text{($\alpha, \beta$可为任意实数,但$\beta \neq 0$。)}$$
		则$(e^1)_a = \beta^{-1}(\mathrm{d}x)_a$。取$(e^1)_a$为法余矢$n_a$,相应的法矢为$n^a = \beta^{-1}(\partial / \partial x)^a$,满足$n^a \notin W_q$且$n_an^a > 0$(即$n^a$为类空)。
		\item $\phi[\mathbb{R}]$与$x$轴夹$45\degree$角(按欧氏)。$\forall q \in \phi[\mathbb{R}]$,令$(e_2)^a = (\partial / \partial t)^a + (\partial / \partial x)^a$,选
		$$(e_1)^a = \alpha(\partial / \partial t)^a + \beta(\partial / \partial x)^a, \alpha \neq \beta$$
		则$(e^1)_a = (\alpha - \beta)^{-1}[(\mathrm{d}t)_a - (\mathrm{d}x)_a]$。取$(e^1)_a$为法余矢$n_a$,相应的法矢为$n^a = (\alpha - \beta)^{-1}g^{ab}[(\mathrm{d}t)_b - (\mathrm{d}x)_b] = -(\alpha - \beta)^{-1}[(\partial / \partial t)^a + (\partial / \partial x)^a] = -(\alpha - \beta)^{-1}(e_2)^a$,满足$n^a \in W_q$且$n_an^a = 0$(即$n^a$为类光)。
		这种情况下,超曲面的法矢既与面上所有切矢垂直(法矢定义),本身又是切矢之一!
	\end{enumerate}
\end{example}

\begin{definition}
	超曲面叫\textbf{类空}的,若其法矢处处类时($n^an_a < 0$);
	超曲面叫\textbf{类时}的,若其法矢处处类空($n^an_a > 0$);
	超曲面叫\textbf{类光}的,若其法矢处处类光($n^an_a = 0$)。
\end{definition}

若$n^an_a \neq 0$,今后谈法矢时都指归一化法矢,即$n^an_a = \pm 1$。

\begin{definition}
	设$\phi[S]$是流形$M$中的嵌入子流形(不一定是超曲面),$q \in \phi[S], W_q$是$q$点切于$\phi[S]$的切空间。
	$W_q$的度规$h_{ab}$叫做由$V_q$的度规$g_{ab}$生出的诱导度规,若
	$$h_{ab}w_1^aw_2^b = g_{ab}w_1^aw_2^b, ~ \forall w_1^a, w_2^b \in W_q$$
\end{definition}

诱导度规$h_{ab}$实质上是把$V_q$上度规$g_{ab}$的作用对象限制于$W_q$的结果。
当$\phi[S]$为类时或类空超曲面时,诱导度规$h_{ab}$可用其归一化法矢($n^an_a = 1$)方便地表示为
$$h_{ab} \equiv g_{ab} \mp n_an_b, \text{($n^an_a = +1$时取$-$,$n^an_a = -1$时取$+$)}$$
因为$\forall w_1^a, w_2^b \in W_q$有$h_{ab}w_1^aw_2^b = g_{ab}w_1^aw_2^b \mp n_aw_1^an_bw_2^b = g_{ab}w_1^aw_2^b$,即满足定义。

设$\phi[S]$为类时或类空超曲面,$q \in \phi[S]$,$h_{ab}$满足前式。令
$$h^a{}_b \equiv g^{ac}h_{cb} = \delta^a{}_b \mp n^an_b$$
则$\forall v^a \in V_q$有$h^a{}_bv^b = v^a \mp n^a(n_bv^b)$,或
$$v^a = h^a{}_bv^b \pm n^a(n_bv^b)$$
上式代表矢量$v^a$的一种分解,其中$\pm n^a(n_bv^b)$与法矢$n^a$平行,称为法向分量,
$h^a{}_bv^b$与法矢$n^a$垂直(因为$n_a(h^a{}_bv^b) = 0$),称为切向分量(切于$\phi[S]$的分量)。
$h^a{}_b$称为从$V_q$到$W_q$的\textbf{投影映射}。

\begin{theorem}
	类光超曲面上的诱导``度规''是退化的(因而没有诱导度规)。
\end{theorem}

\begin{proof}
	以$h_{ab}$代表诱导``度规''。超曲面的类光性导致$n^a \in W_q$,故$W_q$有非零元素$n^a$使$h_{ab}n^aw^b = g_{ab}n^aw^b = 0, ~ \forall w^a \in W_q$。
	可见$h_{ab}$是$W_q$上的退化张量。
\end{proof}

\begin{example}
	设$t, x, y, z$是$4$维闵氏空间$(\mathbb{R}^4, \eta_{ab})$的洛伦兹坐标,则$\eta_{ab}$可用对偶坐标基矢表为
	$$\eta_{ab} = \eta_{\mu\nu}(\mathrm{d}x^\mu)_a(\mathrm{d}x^\nu)_b = -(\mathrm{d}t)_a(\mathrm{d}t)_b + (\mathrm{d}x)_a(\mathrm{d}x)_b + (\mathrm{d}y)_a(\mathrm{d}y)_b + (\mathrm{d}z)_a(\mathrm{d}z)_b$$
	方程$t - (x^2 + y^2 + z^2)^{1/2} = 0$定义了一个类光超曲面$\mathscr{S}$,它是以原点为锥顶的圆锥面。
	$\forall q \in \mathscr{S} \subset \mathbb{R}^4$,有$4$维切空间$V_q$和$3$维切空间(切于$\mathscr{S}$)$W_q \subset V_q$。令
	$$n^a|_q \equiv (\partial / \partial t)^a|_q + (\partial / \partial z)^a|_q \text{(以下略去下标$q$)}$$
	则$n^a$是$q$点的类光法矢,故$n^a \in W_q$,因而$\{(\partial / \partial x)^a, (\partial / \partial y)^a, n^a\}$是$W_q$的基底。
	现在计算$\eta_{ab}$在$W_q$上的诱导``度规''$h_{ab}$在此基底的分量$h_{\mu\nu}$。
	$$h_{11} = h_{ab}(\partial / \partial x)^a(\partial / \partial x)^b = \eta_{ab}(\partial / \partial x)^a(\partial / \partial x)^b = 1$$
	类似地,有$h_{22} = h_{ab}(\partial / \partial y)^a(\partial / \partial x)^b = 1$。而$h_{\mu\nu}$的第三个对角元(记作$h_{nn}$)则为
	$$h_{nn} = h_{ab}n^an^b = \eta_{ab}[(\partial / \partial t)^a + (\partial / \partial z)^a][(\partial / \partial t)^b + (\partial / \partial z)^b] = 1 - 1 = 0$$
	而且容易验证$h_{\mu\nu}$的所有非对角元为零,故
	$$(h_{\mu\nu}) = \begin{bmatrix}
		1 & 0 & 0 \\
		0 & 1 & 0 \\
		0 & 0 & 0
	\end{bmatrix}$$
	因而$h_{ab}$退化(也说其``号差''为$(+, +, 0)$)。
	可见$\eta_{ab}$在类光超曲面$\mathscr{S}$上无诱导度规。
	然而,令$S$为$\mathscr{S}$与任一等$t$面($t > 0$)之交(是$2$维球面),以$\hat W_q \subset W_q$代表$W_q$中所有切于$S$的元素组成的子空间,则$\eta_{ab}$在$\hat W_q$却有诱导度规,记作$\hat h_{ab}$,而且不难验证
	$$\hat h_{ab} = (\mathrm{d}x)_a(\mathrm{d}x)_b + (\mathrm{d}y)_a(\mathrm{d}y)_b$$
	读者不难对$(\mathbb{R}^4, \eta_{ab})$中由$t \text{--} z = 0$定义的类光超曲面做类似讨论。
\end{example}
