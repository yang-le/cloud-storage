\chapter[李导数、Killing场和超曲面]{\\李导数、Killing场和超曲面}

\section{流形间的映射}

设$M$、$N$为流形(维数可不同),$\phi \colon M \to N$为光滑映射。
以$\mathscr{F}_M$和$\mathscr{F}_N$分别代表$M$和$N$上光滑函数的集合,$\mathscr{F}_M(k, l)$和$\mathscr{F}_N(k, l)$分别代表$M$和$N$上光滑$(k,l)$型张量场的集合。
$\phi$自然诱导出一系列映射如下。

\begin{definition}
拉回映射$\phi^* \colon \mathscr{F}_N \to \mathscr{F}_M$定义为
$$(\phi^*f)|_p := f|_{\phi(p)}, ~ \forall f \in \mathscr{F}_N, p \in M$$
即$\phi^*f = f \comp \phi$。
\end{definition}

由定义不难证明:
\begin{enumerate}[(1)]
\item $\phi^* \colon \mathscr{F}_N \to \mathscr{F}_M$是线性映射,即
$$\phi^*(\alpha f + \beta g) = \alpha\phi^*(f) + \beta\phi^*(g), ~ \forall f,g \in \mathscr{F}_N, ~ \alpha,\beta \in \mathbb{R}$$
\item $\phi^*(fg) = \phi^*(f)\phi^*(g), ~ \forall f,g \in \mathscr{F}_N$
\end{enumerate}

\begin{definition}
对$M$中任一点$p$可定义推前映射$\phi_* \colon V_p \to V_{\phi(p)}$如下:$\forall v^a \in V_p$,定义其像$\phi_*v^a \in V_{\phi(p)}$为
$$(\phi_*v)(f) := v(\phi*f), ~ \forall f \in \mathscr{F}_N$$
\end{definition}

还应证明这样定义的$\phi_*v^a$满足矢量定义的两个要求,从而确是$\phi(p)$点的矢量。许多文献也把$\phi_*$称为$\phi$的切映射。

\begin{theorem}
$\phi_* \colon V_p \to V_{\phi(p)}$是线性映射,即
$$\phi_*(\alpha u^a + \beta v^a) = \alpha\phi_*u^a + \beta\phi_*v^a, ~ \forall u^a,v^a \in V_p, ~ \alpha,\beta \in \mathbb{R}$$
\end{theorem}

\begin{theorem}
设$C(t)$是$M$中的曲线,$T^a$为曲线在$C(t_0)$点的切矢,则$\phi_*T^a \in V_{\phi(C(t_0))}$是曲线$\phi(C(t))$在$\phi(C(t_0))$点的切矢(曲线的切矢的像是曲线像的切矢)。
\end{theorem}

\begin{proof}
根据推前映射、曲线切矢以及拉回映射的定义
$$\phi_*T^a(f)
= T^a(\phi^*f)
= \left.\frac{\mathrm{d}(\phi^*f(C))}{\mathrm{d}t}\right|_{t_0}
= \left.\frac{\mathrm{d}f(\phi(C))}{\mathrm{d}t}\right|_{t_0}$$
而这就是曲线$\phi(C(t))$在$\phi(C(t_0))$点的切矢。
\end{proof}

\begin{definition}
拉回映射可按如下方式延拓至$\phi^* \colon \mathscr{F}_N(0, l) \to \mathscr{F}_M(0, l)$:
$\forall T \in \mathscr{F}_N(0, l)$,定义$\phi^*(T) \in \mathscr{F}_M(0, l)$为
$$(\phi^*T)_{a_1 \cdots a_l}|_p(v_1)^{a_1}\cdots(v_l)^{a_l} := T_{a_1 \cdots a_l}|_{\phi(p)}(\phi_*v_1)^{a_1}\cdots(\phi_*v_l)^{a_l}, ~ \forall p \in M, v_1, \cdots, v_l \in V_p$$
\end{definition}

\begin{definition}
$\forall p \in M$,推前映射$\phi_*$可按如下方式延拓至$\phi_* \colon \mathscr{T}_{V_p}(k, 0) \to \mathscr{T}_{V_{\phi(p)}}(k, 0)$(即$\phi_*$是把$p$点的$(k, 0)$型张量变为$\phi(p)$点的同型张量的映射):
$\forall T \in \mathscr{T}_{V_p}(k, 0)$,其像$\phi_*(T) \in \mathscr{T}_{V_{\phi(p)}}(k, 0)$由下式定义:
$$(\phi_*T)^{a_1 \cdots a_l}(\omega^1)_{a_1}\cdots(\omega^k)_{a_k} := T^{a_1 \cdots a_l}(\phi^*\omega^1)_{a_1}\cdots(\phi^*\omega^k)_{a_k}, ~ \forall \omega^1, \cdots, \omega^k \in V^*_{\phi(p)}$$
其中$(\phi^*\omega)_a$定义为$(\phi^*\omega)_av^a := \omega_a(\phi_*v)^a, ~ \forall v^a \in V_p$。
\end{definition}

\begin{note}
拉回映射$\phi^*$能把$N$上的$(0, l)$型张量场变为$M$上的同型张量场,是\CJKunderdot{场}变为\CJKunderdot{场}的映射;
推前映射$\phi_*$只能把$M$中一点$p$的$(k, 0)$型张量变为其像点$\phi(p)$的同型张量。
可否将$\phi_*$延拓为把$M$上的$(k, 0)$型张量\CJKunderdot{场}变为$N$上的同型张量\CJKunderdot{场}的映射?
在一般情况下不能。以矢量场为例。关键在于,给定$M$上一个矢量场$v$后,要定义$N$上的像矢场$\phi_*v$就要对$N$的任一点$q$定义一个矢量,而这势必涉及$q$点的逆像$\phi^{-1}(q)$。
如果$\phi$不是到上映射,则$\phi^{-1}(q)$可能不存在,从而无法用$\phi^{-1}(q)$点的$v$作为右边的$v$;
如果$\phi$不是一一映射,则逆像$\phi^{-1}(q)$可能多于一点,从而无法确定该用哪一逆像点的$v$作为右边的$v$。
这暗示,如果$\phi$只是光滑映射,则$\phi_*$未必能把场推前为场。
然而,如果$\phi \colon M \to N$是微分同胚映射,则上述困难不复存在,推前映射$\phi_*$可看作把$M$上的$(k, 0)$型张量\CJKunderdot{场}变为$N$上同型张量\CJKunderdot{场}的映射,即$\phi_* \colon \mathscr{F}_M(k, 0) \to \mathscr{F}_N(k, 0)$。
再者,由于$\phi^{-1}$存在而且光滑,其拉回映射$\phi^{-1*}$把$\mathscr{F}_M(0, l)$映到$\mathscr{F}_N(0, l)$,这可看作$\phi$的推前映射$\phi_*$,于是$\phi_*$又可进一步推广为$\phi_* \colon \mathscr{F}_M(k, l) \to \mathscr{F}_N(k, l)$。
例如,设$T^a{}_b \in \mathscr{F}_M(1, 1)$,则$(\phi_*T)^a{}_b \in \mathscr{F}_N(1, 1)$定义为
$$(\phi_*T)^a{}_b|_q\omega_av^b := T^a{}_b|_{\phi^{-1}(q)}(\phi^*\omega)_a(\phi^*v)^b, ~ \forall q \in N, \omega_a \in V^*_q, v^b \in V_q$$
其中,$(\phi^*v)^b$应理解为$(\phi^{-1}_*v)^b$。
同理,拉回映射也可推广为$\phi^* \colon \mathscr{F}_N(k, l) \to \mathscr{F}_M(k, l)$。
推广后的$\phi_*$和$\phi^*$仍为线性映射,而且互逆。
\end{note}

设$\phi \colon M \to N$是微分同胚,$p \in M$,$\{x^\mu\}$和$\{y^\mu\}$分别是$M$和$N$的局域坐标系,坐标域$O_1$和$O_2$满足$p \in O_1, \phi(p) \in O_2$。
于是$p \in \phi^{-1}[O_2]$。$\phi$为微分同胚保证$M$和$N$的维数相等,故$\{x^\mu\}$和$\{y^\mu\}$的$\mu$都是从$1$到$n$。
微分同胚本是点的变换,但也可等价地看作坐标变换,因为可用$\phi \colon M \to N$在$\phi^{-1}[O_2]$上定义一组新坐标$\{x'^\mu\}$如下:
$\forall q \in \phi^{-1}[O_2]$,定义$x'^\mu(q) := y^\mu(\phi(q))$。
可见微分同胚映射$\phi$在$p$的邻域$O_1 \cap \phi^{-1}[O_2]$上自动诱导出一个坐标变换$x^\mu \mapsto x'^\mu$。
由前述定理不难证明$\forall q \in O_1 \cap \phi^{-1}[O_2]$有
$$\phi_*[(\partial / \partial x'^\mu)^a|_q] = (\partial / \partial y^\mu)^a|_{\phi(q)}$$
由此又可证明
$$\phi_*[(\mathrm{d}x'^\mu)^a|_q] = (\mathrm{d}y^\mu)^a|_{\phi(q)}$$
于是对微分同胚映射$\phi \colon M \to N$就存在两种观点:
\textcircled{1} 主动观点,它如实地认为$\phi$是点的变换(把$p$变为$\phi(p)$)以及由此导致的张量变换(把$p$点的张量$T$变为$\phi(p)$点的张量$\phi_*T$);
\textcircled{2} 被动观点,它认为点$p$及其上的所有张量$T$都没变,$\phi \colon M \to N$的后果是坐标系有了变换(从$\{x^\mu\}$变为$\{x'^\mu\}$)。
这两种观点虽然似乎相去甚远,但在实用上是等价的。
下面的定理可以看作等价性的某种表现。

\begin{theorem}
$(\phi_*T)^{\mu_1 \cdots \mu_k}{}_{\nu_1 \cdots \nu_l}|_{\phi(p)} = T'^{\mu_1 \cdots \mu_k}{}_{\nu_1 \cdots \nu_l}|_p, ~ \forall T \in \mathscr{F}_M(k, l)$

式中左边是新点$\phi(p)$的新张量$\phi_*T$在老坐标系$\{y^\mu\}$的分量,右边是老点$p$的老张量$T$在新坐标系$\{x'^\mu\}$的分量。
\end{theorem}

\begin{note}
上式是实数等式,左边是由主动观点(认为点和张量变了而坐标系没变)所得的数,右边是由被动观点(认为点和张量没变但坐标系变了)所得的数。
两边相等就表明两种观点在实用上等价。
\end{note}

\begin{example}
设$v^a \in V_p$,令$u^a \equiv \phi_*v^a \in V_{\phi(p)}$,则由上述定理不难证明
$$u^\mu = v^\nu(\partial x'^\mu / \partial x^\nu)|_p$$
\end{example}

\section{李导数}

根据前面的知识我们知道$M$上的一个光滑矢量场$v^a$给出一个单参微分同胚群$\phi$。\footnote{
若$v^a$不完备,则只能给出单参微分同胚局部群。本节只涉及局部性质,无须明确区分局部和整体。
}设$T^{\cdots}{}_{\cdots}$是$M$上的光滑张量场,则$\phi^*_tT^{\cdots}{}_{\cdots}$也是同型光滑张量场,其中$\phi_t$是单参微分同胚群$\phi$的一个群元。
这两个张量场在点$p \in M$的值之差$\phi^*_tT^{\cdots}{}_{\cdots}|_p - T^{\cdots}{}_{\cdots}|_p$是$p$点的张量,它与$t$之商$(\phi^*_tT^{\cdots}{}_{\cdots}|_p - T^{\cdots}{}_{\cdots}|_p) / t$在$t$趋于零时的极限可看作张量场$T^{\cdots}{}_{\cdots}$在$p$点的某种导数,于是有以下定义:

\begin{definition}
$\mathscr{L}_vT^{a_1 \cdots a_k}{}_{b_1 \cdots b_l} := \lim\limits_{t \to 0}\frac{1}{t}(\phi^*_tT^{a_1 \cdots a_k}{}_{b_1 \cdots b_l} - T^{a_1 \cdots a_k}{}_{b_1 \cdots b_l})$

称为张量场$T^{a_1 \cdots a_k}{}_{b_1 \cdots b_l}$沿矢量场$v^a$的李导数($\mathscr{L}_v$中的$v$不写为$v^a$,以免误解。)。
\end{definition}

\begin{note}
因$\phi^*_t$为线性映射,故李导数是由$\mathscr{F}_M(k, l)$到$\mathscr{F}_M(k, l)$的线性映射。还可证明$\mathscr{L}_v$同缩并可交换顺序。
\end{note}

\begin{theorem}
$\mathscr{L}_vf = v(f), ~ \forall f \in \mathscr{F}$
\end{theorem}

\begin{proof}
$\forall p \in M$,设$C(t)$是$\phi$过$p$点的轨道,$p = C(0)$,则$\phi_t(p) = C(t)$且$v^a|_p \equiv (\partial / \partial t)^a|_p$是$C(t)$在$p$点的切矢,故
$$\begin{aligned}
\mathscr{L}_vf|_p & = \lim_{t \to 0}\frac{1}{t}(\phi_t^*f - f)|_p = \lim_{t \to 0}\frac{1}{t}[f(\phi_t(p)) - f(p)] \\
& = \lim_{t \to 0}\frac{1}{t}[f(C(t)) - f(C(0))] = \frac{\mathrm{d}}{\mathrm{d}t}(f \comp C)|_{t = 0} = v(f)|_p
\end{aligned}$$
\end{proof}

下面以$n = 2$为例介绍一种对计算李导数很有用的坐标系。
设$\{x^1, x^2\}$为坐标系,则$x^1$坐标线和$x^2$坐标线组成坐标``网格'',欲知坐标域中某点的坐标,只须看它位于网格的哪两条坐标线的交点。
求李导数时总要给定矢量场$v^a$,可以选定它的积分曲线为$x^1$坐标线($t$充当$x^1$),再相当任意地选定另一组与这组曲线横截(即交点上两线切矢不平行)的曲线作为$x^2$坐标线,这样得到的坐标系称为矢量场$v^a$的适配坐标系。
换句话说,矢量场$v^a$就是其适配坐标系的第一坐标基矢场,即$v^a = (\partial / \partial x^1)^a$。
以上讨论可推广至任意维流形。

\section{Killing矢量场}

\section{超曲面}
