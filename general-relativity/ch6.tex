\chapter{量子力学数学基础简介}

量子力学数学基础的主要内容是泛函分析,特别是希尔伯特空间及其线性算符的理论。
读过本书前两章的读者对集合、映射、拓扑、矢量空间及其对偶空间、张量、度规等概念比较熟悉,这为学习泛函分析提供了一定的方便。
学过上述概念的物理系学生往往跃跃欲试地把量子力学的内积、左右矢、线性算符以及波函数用正交归一基底展开等一系列问题同上述概念联系起来思考,力图求得更为深入和清晰的理解。
他们希望有一份简明读物作参考。
本章主要为满足这种需要而产生。
讲解中尽量与本书前两章以及量子力学的有关内容(特别是Dirac的左右矢记号)相联系或对比。
为了尽量节省读者的时间,我们只介绍最为必需的概念和定理,而且略去其中少数概念的定义和某些定理的证明。
想深入学习泛函分析的读者则应阅读有关教程或专著。

\section{Hilbert空间初步}

\subsection{Hilbert空间及其对偶空间}

第二章中定义的矢量空间称为实矢量空间。
把定义中的$\mathbb{R}$改为全体复数的集合$\mathbb{C}$便得到复矢量空间。
定义了内积的复矢量空间$V$称为(复)内积空间。
内积是一个从$V \times V$到$\mathbb{C}$的映射。
以$\operatorname{i}$代表这一映射,即$\operatorname{i} \colon V \times V \to \mathbb{C}$,则$\operatorname{i}$可看作有两个槽的机器,在两槽中分别输入$f, g \in V$,便得一个复数。
因此,$\operatorname{i}$也可形象地记作$\operatorname{i}(\bullet,\bullet)$,括号中的两个圆点代表两个槽。
为简单起见,索性把$\operatorname{i}(\bullet,\bullet)$简写为$(\bullet,\bullet)$,确切定义如下:

\begin{definition}
    复矢量空间$V$称为内积空间,若存在内积映射$\operatorname{i} \colon V \times V \to \mathbb{C}$,对任意$f, g, h \in V$和任意$c \in \mathbb{C}$满足
    \begin{enumerate}[(a)]
        \item $(f, g + h) = (f, g) + (f, h)$;
        \item $(f, cg) = c(f, g)$;
        \item $(f, g) = \widebar{(g, f)}$ (其中$\widebar{(g, f)}$代表复数$(g, f)$的共轭复数);
        \item $(f, f) \geq 0$,且$(f, f) = 0 \Leftrightarrow f = 0$。\footnote{
            只含零元的空间也可看作内积空间,但本章在不加声明时只讨论维数大于零的内积空间。
        }
    \end{enumerate}
\end{definition}

\begin{note}
    上述定义的前两条件表明内积映射$(\bullet,\bullet)$对第二槽是线性的。
    第三条件(配以前两条件)则表明$(f + g, h) = (f, h) + (g, h)$及$(cf, g) = \bar{c}(f, g)$。
    由于具有这种性质,我们说映射$(\bullet,\bullet)$对第一槽是反线性(或共轭线性)的。
    最后一个条件表明$(f, g) = 0 ~ \forall g \in V \Rightarrow f = 0$,因可取$g = f$。
\end{note}

\begin{note}
    上述定义对实矢量空间也适用,只须把$\mathbb{C}$改为$\mathbb{R}$。
    实空间的内积就是第二章定义的正定度规,但是,由于非正定度规不满足上述定义的最后一个条件,所以度规未必是内积。
\end{note}

\begin{example}
    $$C[a, b] \equiv \{[a, b] \subset \mathbb{R} \text{上连续复值函数} f(x)\}$$
    是无限维复矢量空间。定义内积
    $$(f, g) \coloneq \int^a_b \widebar{f(x)}g(x)\mathrm{d}x, ~ \forall f, g \in C[a, b]$$
    则$C[a, b]$是内积空间。
\end{example}

利用内积可自然定义内积空间中任意两点的距离,进而定义空间的拓扑。

\begin{definition}
    内积空间$V$中任意两元素$f$和$g$的距离定义为
    $$d(f, g) \coloneq \sqrt{(f - g, f - g)}$$
    易见$d(f, g) = d(g, f)$。
    设$f \in V, r > 0$,则以$f$为心、以$r$为半径的开球定义为
    $$B(f, r) \coloneq \{g \in V \mid d(g, f) < r\}$$
    用开球可给$V$定义拓扑(类似于第一章中的``通常拓扑''):
    $$\mathscr{T} \coloneq \{\text{空集或$V$中能表为开球之并的子集}\}$$
    可见内积空间$V$可自然地被定义为一个拓扑空间。
    上式定义的拓扑$\mathscr{T}$叫做$V$的自然拓扑。
    今后如无特别声明,凡涉及$V$的拓扑时一律指这一拓扑。
\end{definition}

第二章讲过(有限维)实矢量空间$V$的对偶空间$V^*$,它是由$V$到$\mathbb{R}$的全体线性映射的集合。
对(有限或无限维)复矢量空间$V$,对偶矢量可定义为由$V$到$\mathbb{C}$的线性映射。
然而内积空间除了是复矢量空间外还是拓扑空间,因此对映射$\eta \colon V \to \mathbb{C}$还可问及是否连续的问题。
(这时还涉及$\mathbb{C}$的拓扑,其定义也很自然:设$z_1, z_2 \in \mathbb{C}$且$z_1 = a_1 + \mathrm{i}b_1, z_2 = a_2 + \mathrm{i}b_2$(其中$a_1, a_2, b_1, b_2 \in \mathbb{R}$),则$z_1, z_2$的距离可定义为$d(z_1, z_2) \coloneq [(a_1 - a_2)^2 + (b_1 - b_2)^2]^{1 / 2}$,用此距离便可定义开球,从而定义$\mathbb{C}$的拓扑。)
在泛函分析中,每个连续的线性映射$\eta \colon V \to \mathbb{C}$称为$V$上的一个连续线性泛函。
我们关心$V$上全体连续线性泛函的集合(它有许多好的性质),并称它为$V$的对偶空间。\footnote{
    若$V$为有限维,则$V$上的线性泛函必定连续。
    因此只当$V$为无限维时连续性才是对线性泛函有实质意义的要求。
}

\begin{definition}
    内积空间$V$的对偶空间(又称共轭空间)定义为
    $$V^* \coloneq \{\eta \colon V \to \mathbb{C} \mid \eta \text{为连续的线性映射}\}$$
    $V^*$也可看作复矢量空间,为此只须用如下的自然方法定义加法和数乘:
    \begin{enumerate}[]
        \item 加法 ~ $(\eta_1 + \eta_2)(f) \coloneq \eta_1(f) + \eta_2(f), ~ \forall \eta_1 \eta_2 \in V^*, f \in V$;
        \item 数乘 ~ $(c\eta)(f) \coloneq c \cdot \eta(f), ~ \forall \eta \in V^*, c \in \mathbb{C}$
    \end{enumerate}
    (由此可知零元是$V^*$中这样的元素,它作用于任意$f \in V$都得零。)
\end{definition}

读者至此自然会联想到量子力学的右矢和左矢空间,并猜想右矢空间是内积空间$V$,而左矢空间则是$V^*$。
然而事情比此略为复杂。
要保证Dirac的左右矢记号得心应手,左右逢源,左、右矢空间应该``完全对等''。
这似乎不难:根据第二章,有限维实矢量空间$V$上的度规自然诱导一个从$V$到$V^*$的一一到上的线性映射。
然而,由于复空间的内积与实空间的度规的少许不同,复空间$V$上的内积自然诱导的从$V$到$V^*$的映射不是线性而是反线性的。
不过这不构成什么问题,真正构成问题的是量子力学中用到的内积空间多数是无限维的,而这导致上述映射未必到上,即$V$与$V^*$``不一样大''。
先看如下命题:

\begin{theorem}
    内积映射$(\bullet,\bullet)$自然诱导出一个一一的、反线性的映射$\nu \colon V \to V^*$。
\end{theorem}

\begin{proof}
    设$f \in V$,则$\eta_f \equiv (f, \bullet)$是从$V$到$\mathbb{C}$的映射。
    由内积定义可知它是线性的。
    还可证明(略)它是连续的,因此$\eta_f \in V^*$。
    具体说,$\eta_f$是$V^*$的这样一个元素,它作用于$g \in V$的结果为$\eta_f(g) \coloneq (f, g)$。
    可见$(\bullet,\bullet)$自然诱导出一个映射$\nu \colon V \to V^*$,定义为$\nu(f) \coloneq \eta_f$。
    根据映射$\nu$的定义,其反线性性是显然的,其一一性证明如下:
    设$\exists f_1, f_2 \in V$使得$\nu(f_1)(g) - \nu(f_2)(g) = 0 ~ \forall g \in V$,则有$(f_1 - f_2, g) = 0 ~ \forall g \in V$,因此就有$f_1 = f_2$。
\end{proof}

若$V$是有限维空间,则$V^*$与$V$有相同的维数(请参考第二章中对实矢量空间的证明),因而上述反线性映射$\nu \colon V \to V^*$一定到上。
若$V$是无限维的,则$V^*$也是无限维的,这时$\nu \colon V \to V^*$不一定到上。
直观地说,$V^*$有可能``比$V$大'',即$\nu[V] \subset V^*$但$\nu[V] \neq V^*$。
然而,为保证Dirac的左右矢运算给出正确结果,我们需要$V^*$同$V$``一样大'',即$\nu[V] = V^*$。
不久将看到,其实$V^*$比$V$最多``只多一层皮'',只要给$V$适当地``补上一层皮'',$\nu \colon V \to V^*$就可到上,$V$与$V^*$就``一样大''。
为了用准确语言表述这些直观思考,我们先介绍下列数学概念。

\begin{definition}
    设$V$为内积空间,$f \in V$,$\{f_n\}$是$V$中的一个序列(见第一章最后一节中的定义)。
    我们说$\{f_n\}$收敛于$f$(记作$f_n \to f$),若$\displaystyle\lim_{n \to \infty}d(f, f_n) = 0$。
    $f$称为序列$\{f_n\}$的极限,\footnote{
        内积空间可自然看作拓扑空间,而拓扑空间中的点序列的极限已有定义(见第一章最后一节)。
        不难证明该定义同本页定义等价。
    }记作$\displaystyle f = \lim_{n \to \infty}f_n$。
\end{definition}

\begin{definition}
    $V$中的序列$\{f_n\}$称为柯西序列,若$\forall \varepsilon > 0, \exists N > 0$使当$n, m \geq N$时有$d(f_n, f_m) < \varepsilon$。
\end{definition}

可以证明,收敛(于任一$f \in V$)的序列一定是柯西序列,但反之不然。

\begin{definition}
    内积空间$V$称为完备的,若其中任一柯西序列收敛。
\end{definition}

前述例子中的$C[a, b]$是不完备的内积空间。因为存在这样的柯西序列,其收敛于一个不连续函数(因此它在$C[a, b]$中不收敛)。
柯西序列$\{f_n(x)\}$在$C[a, b]$内没有极限表明$C[a, b]$不完备。
为使之完备化,可把$[a, b]$上虽不连续却平方可积的复值函数(例如上述柯西序列的极限)包含进去,扩大后的空间记作$L^2[a, b]$,即
$$L^2[a, b] \equiv \{f \colon [a, b] \to \mathbb{C} \mid \int^b_a |f(x)|^2 \mathrm{d}x < \infty\}\footnote{
    式中的积分是指Lebesgue积分。
    严格地说,函数$f(x)$还应具有``可测性''。
    可以这样说:物理上遇到的函数都满足这一要求。
    此外,两个函数如果只在测度为零的集上不同就应被视为$L^2[a, b]$的同一元素。
    本脚注同样适用于$L^2(\mathbb{R}^n)$。
}$$
仍用前面例子中的内积定义,则可证$L^2[a, b]$是完备的内积空间。

内积空间$C[a, b]$的这一完备化程序很有启发性。
事实上,任何不完备的内积空间$V$都可以完备化,为此只须把它略加扩大---把所有柯西序列``应有的极限点''都补进$V$中。
可以证明,对任何不完备内积空间$V$,总可找到完备的内积空间$\tilde V$,使得$V \subset \tilde V$而且$\bar V = \tilde V$,其中$\bar V$是把$\tilde V$看作拓扑空间(用其内积定义拓扑)时子集$V$的闭包。
直观地可以说$\tilde V$比$V$``最多只多一层皮''。

\begin{definition}
    完备的内积空间叫希尔伯特空间,记作$\mathscr{H}$。
\end{definition}

\begin{note}
    有限维的内积空间一定完备,因此一定是Hilbert空间。
    然而量子力学中用到的Hilbert空间多数是无限维的。
    无限维是许多问题变得复杂的根源。
\end{note}

同$L^2[a, b]$相仿,
$$L^2(\mathbb{R}^n) \equiv \{f \colon \mathbb{R}^n \to \mathbb{C} \mid \int|f|^2\mathrm{d}^nx < \infty\} ~ \text{(内积定义仿照前述例子)}$$
也是Hilbert空间,其中$L^2(\mathbb{R}^3)$是量子力学常用的波函数空间。

由于具有完备性,Hilbert空间有许多很好的性质,其中对我们特别有用的就是$\mathscr{H}$与其对偶空间$\mathscr{H}^*$``一样大'',即$\nu[\mathscr{H}] = \mathscr{H}^*$,见如下命题:

\begin{theorem}
    设$\mathscr{H}$是Hilbert空间,$\mathscr{H}^*$是其对偶空间,则$\forall \eta \in \mathscr{H}^*$,有唯一的$f_\eta \in \mathscr{H}$使$\eta(g) = (f_\eta, g) ~ \forall g \in \mathscr{H}$。
\end{theorem}

\begin{proof}
    见任一泛函分析教程中关于Riesz表现定理的证明。
\end{proof}

\begin{note}
    上述定理表明,对Hilbert空间,前述命题中的$\nu \colon \mathscr{H} \to \mathscr{H}^*$是到上的,即$\nu$是一一到上的反线性映射。
    这个命题的重要性在于保证$\mathscr{H}$同$\mathscr{H}^*$``一样大'',请注意不完备内积空间没有这样好的结论。
    可见物理学不但需要内积空间,而且需要完备的内积空间---Hilbert空间。
\end{note}

利用映射$\nu \colon \mathscr{H} \to \mathscr{H}^*$还可把$\mathscr{H}^*$定义为Hilbert空间:
$\forall \eta, \xi \in \mathscr{H}^*$,由上述定理可知有唯一的$f_\eta, f_\xi \in \mathscr{H}$使$\eta = \nu(f_\eta), \xi = \nu(f_\xi)$。
定义$\eta$和$\xi$的内积为
$$(\eta, \xi) \coloneq (f_\xi, f_\eta)$$
则不难验证$(\eta, \xi)$满足内积定义,故$\mathscr{H}^*$是内积空间。
还可证明$\mathscr{H}^*$是完备的,因而也是Hilbert空间。
可见$\mathscr{H}^*$与$\mathscr{H}$实在非常``像'':
它们之间不但存在一一到上的反线性映射,而且这一映射还在上式的意义上保内积。
$\mathscr{H}$和$\mathscr{H}^*$的这种相像性使我们可以把它们分别用作量子力学中的右矢和左矢空间(详见第四小节)。

\subsection{Hilbert空间的正交归一基}

$N$维矢量空间$V$的一个基底无非是由$N$个元素组成的、满足如下两个要求的一个序列$\{e_1, \cdots, e_N\}$:
\textcircled{1} $\{e_1, \cdots, e_N\}$线性独立;
\textcircled{2} $V$的任一元素$f$可由$\{e_1, \cdots, e_N\}$线性表出。
我们想把基底概念推广至无限维的Hilbert空间$\mathscr{H}$。

\begin{definition}
    $\mathscr{H}$的有限子集$\{f_1, \cdots, f_N\}$称为线性独立的,若
    $$\sum^N_{n = 1}c_nf_n = 0 \Rightarrow c_n = 0, n = 1, \cdots, N$$
    $\mathscr{H}$的任一子集$\{f_\alpha\}$称为线性独立的,若$\{f_\alpha\}$的任一非空有限子集线性独立。
\end{definition}
