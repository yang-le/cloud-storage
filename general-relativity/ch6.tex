\chapter{李群和李代数}

\section{群论初步}

\begin{definition}
    集合$G$配以满足以下条件的映射$G \times G \to G$(叫群乘法)称为群:
    \begin{enumerate}[(a)]
        \item $(g_1g_2)g_3 = g_1(g_2g_3) ~ \forall g_1, g_2, g_3 \in G$
        \item 存在恒等元$e \in G$,使$eg = ge = g, ~ \forall g \in G$
        \item $\forall g \in G$, 存在逆元$g^{-1} \in G$,使$g^{-1}g = gg^{-1} = e$
    \end{enumerate}
\end{definition}

\begin{note}
    恒等元是唯一的\footnote{
        考虑有两个恒等元$e_1, e_2$,有$e_1e_2 = e_1 = e_2$
    },任一群元的逆元也是唯一的\footnote{
        考虑任一群元$g$有两个逆元$g_1^{-1}, g_2^{-1}$,有$g_1^{-1} = (g_2^{-1}g)g_1^{-1} = g_2^{-1}(gg_1^{-1}) = g_2^{-1}$
    }。
\end{note}

\begin{definition}
    乘法满足交换律的群(即$gh = hg ~ \forall g, h \in G$)称为阿贝尔群。
    只含有限个元素的群叫有限群,否则叫无限群。
    群$G$的子集$H$称为$G$的子群,若$H$用$G$的乘法为乘法也构成群。
\end{definition}

\begin{definition}
    设$G$和$G'$是群。
    映射$\mu \colon G \to G'$叫同态,若
    $$\mu(g_1g_2) = \mu(g_1)\mu(g_2), \forall g_1, g_2 \in G$$
\end{definition}

\begin{theorem}
    同态映射$\mu \colon G \to G'$有以下性质:
    \begin{enumerate}[(a)]
        \item 若$e, e'$各为$G, G'$的恒等元,则$\mu(e) = e'$\footnote{
            $\mu(g) = \mu(ge) = \mu(g)\mu(e), ~ \forall \mu(g) \in G'$,考虑到恒等元的唯一性,则必有$\mu(e) = e'$。
        }
        \item $\mu(g^{-1}) = \mu(g)^{-1}, \forall g \in G$\footnote{
            $\mu(g^{-1}) = \mu(g^{-1})e' = \mu(g^{-1})\mu(g)\mu(g)^{-1} = \mu(e)\mu(g)^{-1} = \mu(g)^{-1}$
        }
        \item $\mu[G]$是$G'$的子群\footnote{
            显然$\mu[G]$是$G'$的子集(否则$\mu$就不构成$G \to G'$的映射)。
            按$\mu$的定义,容易验证$\mu[G]$上的乘法是结合的,单位元是$\mu(e)$,$\mu(g)$逆元为$\mu(g^{-1})$
        },当$G$是阿贝尔群时$\mu[G]$是$G'$的阿贝尔子群\footnote{
            若$G$上的乘法可交换,则$\mu$的定义保证$\mu[G]$上的乘法也可交换。
            注意这并不说明$G'$上的乘法可交换。
        }。
    \end{enumerate}
\end{theorem}

\begin{definition}
    一一到上的同态映射称为同构。
    当有可能与矢量空间之间的同构混淆时又明确地把群之间的同构称为群同构。
    同构$\mu \colon G \to G$称为群$G$上的自同构。
\end{definition}

\begin{example}
    $\forall g \in G$,可构造一个称为伴随同构的自同构映射,又称内自同构,记作$I_g \colon G \to G$,定义\footnote{
        当然应该验证这一定义符合同构映射的要求。
        首先$I_g(h_1h_2) = I_g(h_1)I_g(h_2)$说明这是同态映射。
        其次,$I_g(h_1) = I_g(h2) \Rightarrow h_1 = h_2, \forall h_1, h_2 \in G$说明该映射是一一的。
        最后,$I_g(g^{-1}hg) = h, \forall h \in G$说明该映射是到上的。
    }为$$I_g(h) \coloneq ghg^{-1}, \forall h \in G$$
\end{example}

\begin{note}
    今后常把两个同构的群视作一样,并用等号表示。
\end{note}

\begin{definition}
    群$G$和$G'$(看作两个集合)的卡氏积$G \times G'$按下列乘法
    $$(g_1, g_1')(g_2, g_2') \coloneq (g_1g_2, g_1'g_2'), ~ \forall g_1, g_2 \in G, ~ g_1',g_2' \in G'$$
    构成的群称为$G$和$G'$的直积群。
\end{definition}

\begin{example}
    以加法为群乘法,则$\mathbb{R}$是群。
    $\mathbb{R}^2 \equiv \mathbb{R} \times \mathbb{R}$配以由上式定义的乘法就构成直积群,而且此乘法正好是$\mathbb{R}^2$上(自然定义)的加法。
\end{example}

\begin{definition}
    设$H$是群$G$的子群,$g \in G$,则$gH \equiv \{gh \mid h \in H\}$称为$H$的含$g$的左陪集。
    类似地可定义右陪集。
\end{definition}

\begin{note}
    若子群$H$的两个左陪集有交,则两者必相等\footnote{
        设$p = g_1h_1 = g_2h_2$是$g_1H \cap g_2H$中的一个元素。
        $\forall x = g_1h_x \in g_1H$,有$x = g_1h_1h_1^{-1}h_x = g_2h_2h_1^{-1}h_x \in g_2H$,反之亦然。
    }。
\end{note}

\begin{definition}
    群$G$的子群$H$称为正规子群或不变子群,若
    $$ghg^{-1} \in H, ~ \forall g \in G, h \in H$$
\end{definition}

\begin{definition}
    设$G$是群,则$A(G) \equiv \{\mu \colon G \to G \mid \mu \text{为自同构映射}\}$以映射的复合为群乘法构成群,称为群$G$的自同构群。
    ``以映射的复合为乘法''是指$\forall \mu, \nu \in A(G)$,群乘积$\mu\nu \in A(G)$定义为$(\mu\nu)(g) \equiv \mu(\nu(g)) ~ \forall g \in G$。
\end{definition}

\begin{theorem}
    以$A_I(G)$代表$G$上全体内自同构映射的集合,即
    $$A_I(G) \equiv \{I_g \colon G \to G \mid g \in G\} \subset A(G)$$
    则$A_I(G)$是群$A(G)$的一个正规子群。
\end{theorem}

\begin{proof}
    $A_I(G)$显然是$A(G)$的子集,并且容易证明,以映射复合为群乘法,$A_I(G)$也构成群,所以$A_I(G)$是$A(G)$的子群。
    其恒等元为$I_e$,其相应于$I_g$的逆元为$I_{g^{-1}}$。
    为了验证它还是正规子群,考虑$(\mu I_g \mu^{-1})(h) = \mu(g(\mu^{-1}(h))g^{-1}) = \mu(g)\mu(\mu^{-1}(h))\mu(g^{-1}) = I_{\mu(g)}(h), \forall \mu \in A(G), I_g \in A_I(G), h \in G$。
    所以有$\mu I_g \mu^{-1} = I_{\mu(g)} \in A_I(G), \forall \mu \in A(G), I_g \in A_I(G)$。
\end{proof}

\begin{definition}
    设$H$和$K$是群,且存在同态映射$\mu \colon K \to A(H)$。$\forall k \in K$,把$\mu(k) \in A(H)$简记作$\mu_k$,则$G \equiv H \times K$配以下式定义的群乘法
    $$(h, k)(h', k') \coloneq (h\mu_k(h'), kk'), ~ \forall h,h' \in H, k, k' \in K$$
    所构成的群称为$H$和$K$的半直积群\footnote{
        真是一个巧妙的定义。
        
        结合律:$[(h, k)(h', k')](h'', k'') = (h\mu_k(h'), kk')(h'', k'') = (h\mu_k(h')\mu_{kk'}(h''), kk'k'')$
        而$(h, k)[(h', k')(h'', k'')] = (h, k)(h'\mu_{k'}(h''), k'k'') = (h\mu_k(h'\mu_{k'}(h'')), kk'k'') = (h\mu_k(h')\mu_k\mu_{k'}(h''), kk'k'') = (h\mu_k(h')\mu_{kk'}(h''), kk'k'')$
        最后两个等号是因为$A(H)$中的元素皆为同态映射,而且$\mu$本身也是同态映射(因此$\mu(ab) = \mu(a)\mu(b)$)。

        单位元:注意到同态映射保单位元,而$A(H)$中的单位元就是恒等映射,故$(e_h, e_k)(h, k) = (e_h\mu_{e_k}(h), e_kk) = (e_hh, e_kk)$
        而$(h, k)(e_h, e_k) = (h\mu_k(e_h), ke_k) = (he_h, ke_k)$

        逆元:$(\mu_{k^{-1}}(h^{-1}), k^{-1})(h, k) = (\mu_{k^{-1}}(h^{-1})\mu_{k^{-1}}(h), k^{-1}k) = (e_h, e_k)$
        而$(h, k)(\mu_{k^{-1}}(h^{-1}), k^{-1}) = (h\mu_k(\mu_{k^{-1}}(h^{-1})), kk^{-1}) = (e_h, e_k)$
    },记作$G \equiv H \otimes_S K$。
\end{definition}

\section{李群}

\begin{definition}
    若$G$既是$n$维(实)流形又是群,其群乘映射$G \times G \to G$(请注意$G \times G$也是流形)和求逆元映射$G \to G$都是$C^\infty$的,则$G$叫$n$维(实)李群\footnote{
        约定把有离散拓扑的可数群称为零维李群。
        因为有限群的默认拓扑是离散拓扑,所以有限群都可看作零维李群。
    }。
\end{definition}

\begin{example}
    以加法为群乘法,则$\mathbb{R}$是$1$维李群。
\end{example}

\begin{example}
    $\mathbb{R}$和$\mathbb{R}$的直积群$\mathbb{R}^2$是$2$维李群。
    推而广之,$\mathbb{R}^n$是$n$维李群。
\end{example}

\begin{example}
    设$\phi \colon \mathbb{R} \times M \to M$是流形$M$上的任一单参微分同胚群,则$\{\phi_t \mid t \in \mathbb{R}\}$是$1$维李群\footnote{
        $\phi(t, p) = p ~ \forall p \in M, t \in \mathbb{R}$的情况可视为例外。
        这是与$M$上的$0$矢量场对应的那个特殊的单参微分同胚群,是只含恒等元的独点群,可看作零维李群。
    },同构于$\mathbb{R}$。
\end{example}

\begin{example}
    易证广义黎曼空间$(M, g_{ab})$上的两个等度规映射的复合也是等度规映射,因此$(M, g_{ab})$上全体等度规映射的集合以复合映射为乘法构成群,称为$(M, g_{ab})$的等度规群。
    还可验证等度规群是李群。
    闵氏时空的等度规群是$10$维李群,施瓦西时空的等度规群是$4$维李群。
    一般地,$n$维广义黎曼空间$(M, g_{ab})$的等度规群的维数$m \leq n(n+1)/2$。
    但$(M, g_{ab})$上全体微分同胚的集合则``大''到不能构成有限群。
    事实上,它是一个无限维群。
\end{example}

\begin{note}
    本章只限于讨论有限维李群,虽然许多结论对无限维李群也适用。
\end{note}

以下如无特别声明,$G$一律代表李群。
李群的双重身份(即是群又是流形)使得用几何语言研究李群成为可能。
群乘映射和求逆元映射的光滑性则使李群具有一系列好性质。

\begin{definition}
    李群$G$和$G'$之间的$C^\infty$同态映射$\mu \colon G \to G'$称为李群同态。
    李群同态$\mu$称为李群同构,若$\mu$为微分同胚。
\end{definition}

\begin{definition}
    李群$G$的子集$H$称为$G$的李子群,若$H$既是$G$的子流形又是$G$的子群。
\end{definition}

\begin{definition}
    $\forall g \in G$,映射$L_g \colon h \mapsto gh ~ \forall h \in G$叫做由$g$生成的左平移。
\end{definition}

\begin{note}
    \textcircled{1} 由李群定义中关于群乘映射和求逆元映射的$C^\infty$性可知左平移$L_g \colon G \to G$是微分同胚映射。
    \textcircled{2} 易见$L_{gh} = L_g \comp L_h$。
\end{note}

以下的讨论经常涉及$G$的一点的矢量和$G$的一个子集上的矢量场,并要对两者作明确区分。
我们将用$A, B, \cdots$代表一点的矢量,用$\bar A, \bar B, \cdots$代表矢量场,用$\bar A_g$代表矢量场$\bar A$在点$g \in G$的值。
为简化表达式,本章中所有矢量(除少数情况外)都不加抽象指标。

\begin{definition}
    $G$上的矢量场$\bar A$叫左不变的,若
    $$L_{g*}\bar A = \bar A, ~ \forall g \in G$$
    其中$L_{g*}$是由左平移映射$L_g \colon G \to G$诱导的推前映射。
\end{definition}

\begin{note}
    \textcircled{1} 左不变矢量场必为$C^\infty$矢量场;
    \textcircled{2} 不难看出左不变矢量场的定义式等价于
    $$(L_{g*}\bar A)_{gh} = \bar A_{gh}$$
    推前映射一般只能把一点的张量映射为张量,但当$\phi \colon M \to N$是微分同胚时,$\phi_*$可把$M$上张量场$v$映为$N$上张量场$\phi_*v$,定义为$(\phi_*v)|_{\phi(p)} = \phi_*(v|_p) ~ \forall p \in M$。
    用于现在的情况,就是$(L_{g*}\bar A)_{gh} = L_{g*}(\bar A_h)$。
    于是上式又等价于$$\bar A_{gh} = L_{g*}(\bar A_h)$$
    上式可作为左不变矢量场$\bar A$的等价定义。
\end{note}

不难看出左不变矢量场之和以及左不变矢量场乘以常数仍为左不变矢量场,故$\mathscr{L} \equiv \{\bar A \mid \bar A \text{是$G$上的左不变矢量场}\}$是矢量空间。

\begin{theorem}
    $G$上全体左不变矢量场的集合$\mathscr{L}$与$G$的恒等元$e$的切空间$V_e$(作为两个矢量空间)同构。
\end{theorem}

\begin{proof}
    $\forall A \in V_e$,用下式定义$G$上的矢量场$\bar A$:
    $$\bar A_g \coloneq L_{g*}A, ~ \forall g \in G$$
    由此得$\bar A_e = A$。把上式的$g$改为$gh$得
    $$\bar A_{gh} = L_{gh*}A = (L_g \comp L_h)_*A = (L_{g*} \comp L_{h*})A = L_{g*}(L_{h*}A) = L_{g*}\bar A_h$$
    因此\footnote{
        关于复合映射的推前等于推前映射的复合,证明如下:
        $(L_g \comp L_h)_*A(f) \coloneq A((L_g \comp L_h)^*f) ~ \forall f \in \mathscr{F}$,
        而$(L_g \comp L_h)^*f|_p \coloneq f|_{(L_g \comp L_h)(p)} = f|_{L_g(L_h(p))} = L_g^*f|_{L_h(p)} = L_h^*(L_g^*f)|_p = (L_h^* \comp L_g^*)f|_p ~ \forall p$。
        因此$(L_g \comp L_h)_*A(f) = A((L_h^* \comp L_g^*)f) ~ \forall f \in \mathscr{F}$。另一方面,$[(L_{g*} \comp L_{h*})A](f) = [L_{g*}(L_{h*}A)](f) = (L_{h*}A)(L_g^*f) = A(L_h^*L_g^*f) = A((L_h^* \comp L_g^*)f) ~ \forall f \in \mathscr{F}$
    }$\bar A$是左不变矢量场。
    可见上式定义了一个映射$\eta \colon V_e \to \mathscr{L}$(把$A$映为$\bar A$)。
    $L_{g*}$的线性性保证$\eta$的线性性,由$\bar A_e = A$易见\footnote{
        假设有$A \neq B$,则$\eta(A) = \bar A, \eta(B) = \bar B$,因其在$e$点不等($\bar A_e = A \neq B = \bar B_e$),故$\bar A \neq \bar B$。
    }$\eta$是一一映射。
    于是欲证$\eta$为同构只须证$\eta$为到上映射。
    $\forall \bar A \in \mathscr{L}$,有$\bar A_e \in V_e$。
    注意到$\eta(\bar A_e)|_g = L_{g*}\bar A_e = \bar A_{ge} = \bar A_g, ~ \forall g \in G$。
    这说明$\eta(\bar A_e) = \bar A, ~ \forall \bar A \in \mathscr{L}$。
    即$\eta$为到上映射\footnote{
        就是说,任给一个$\bar A \in \mathscr{L}$,都可以找到其原像$\bar A_e \in V_e$。
    }。
\end{proof}

\section{李代数}

在矢量空间$\mathscr{V}$上定义某种称为``乘法''的映射就得到一个代数。
一种重要的乘法叫李括号,记作$[,] \colon \mathscr{V} \times \mathscr{V} \to \mathscr{V}$,它是满足以下两条件的双线性映射:
\begin{enumerate}[(a)]
    \item $[A, B] = -[B, A], ~ \forall A, B \in \mathscr{V}$
    \item $[A, [B, C]] + [C, [A, B]] + [B, [A, C]] = 0, ~ \forall A, B, C \in \mathscr{V}$
\end{enumerate}
第二个条件称为雅克比恒等式。

\begin{definition}
    定义了李括号的矢量空间称为李代数。
    任意两个元素的李括号都为零的李代数称为阿贝尔李代数。
\end{definition}

本章只讨论有限维实矢量空间上的李代数(实李代数),虽然许多结论对无限维(实或复)李代数也适用。

\begin{example}
    把$\mathbb{R}^3$看作$3$维矢量空间,用下式定义李括号
    $$[\vec v, \vec u] \coloneq \vec v \times \vec u, \forall \vec v, \vec u \in \mathbb{R}^3$$
    则$\mathbb{R}^3$成为李代数。
\end{example}

\begin{example}
    $\mathscr{M} \equiv \{m \times m \text{矩阵}\}$显然为$m^2$维矢量空间。
    用矩阵对易子定义李括号,即
    $$[A, B] \coloneq AB - BA, \forall A, B \in \mathscr{M}, \text{(其中$AB$是$A, B$的矩阵积)}$$
    则$\mathscr{M}$是李代数。
\end{example}

\begin{theorem}
    $G$上全体左不变矢量场的集合$\mathscr{L}$是李代数。
\end{theorem}

\begin{proof}
    以矢量场对易子为李括号。因为$\forall \bar A, \bar B \in \mathscr{L}$有
    $$L_{g*}[\bar A, \bar B] = [L_{g*}\bar A, L_{g*}\bar B] = [\bar A, \bar B]$$
    所以$[\bar A, \bar B] \in \mathscr{L} ~ \forall \bar A, \bar B \in \mathscr{L}$,可见对易子的确是从$\mathscr{L} \times \mathscr{L}$到$\mathscr{L}$的映射。
    对易子当然是双线性的和反称的。而且可以验证对易子也满足雅克比恒等式。
\end{proof}

\begin{definition}
    设$\mathscr{V}$和$\mathscr{W}$是李代数。
    线性映射$\beta \colon \mathscr{V} \to \mathscr{W}$称为李代数同态,若它保李括号,即$\beta([A, B]) = [\beta(A), \beta(B)] ~ \forall A, B \in \mathscr{V}$。
    李代数同态$\beta \colon \mathscr{V} \to \mathscr{W}$称为李代数同构,若$\beta$是一一到上映射。
\end{definition}

\begin{note}
    今后常把两个同构的李代数视作相同,并用等号表示。
\end{note}

对李群$G$的恒等元$e$的切空间$V_e$用下式定义李括号:
$$[A, B] \coloneq [\bar A, \bar B]_e, ~ \forall A, B \in V_e$$
(其中$\bar A, \bar B$分别是$A, B$对应的左不变矢量场。)则$V_e$成为李代数\footnote{
    这巧妙地利用了$V_e$线性同构于$\mathscr{L}$这一事实。
},称为李群$G$的李代数,记作$\mathscr{G}$。易证$\mathscr{G}$与$\mathscr{L}$有李代数同构关系,$\eta$可充当同构映射。
反之,给定一个李代数,是否也可找到一个李群,它的李代数是所给的李代数?
答案是:这样的李群一定存在,并且唯一到只差整体拓扑结构的程度。(例如,以流形$S^1$上的角坐标之和作为群乘法,则$S^1$是$1$维李群,它与$1$维李群$\mathbb{R}$不同---有不同的整体拓扑---但却有相同的李代数。第$6$节末还将给出有相同李代数的不同李群的另外两个重要例子。)
准确地说,给定一个李代数,总可找到唯一的单连通李群(其流形为单连通流形\footnote{
    任一闭曲线可通过连续变形缩为一点的连通流形称为单连通流形。
}的李群),它以所给李代数为李代数。
这是李群理论中的一个重要定理。李群和李代数的这一密切联系使李群的讨论大为简化,因为李代数比李群简单得多。

\begin{theorem}
    设$\mathscr{G}$和$\mathscr{\hat G}$分别是李群$G$和$\hat G$的李代数,$\rho \colon G \to \hat G$是同态映射,则$\rho$在点$e \in G$诱导的推前映射$\rho_* \colon \mathscr{G} \to \mathscr{\hat G}$是李代数同态。
\end{theorem}

\begin{definition}
    李代数$\mathscr{G}$的子空间$\mathscr{H}$称为$\mathscr{G}$的李子代数,若
    $$[A, B] \in \mathscr{H}, ~ \forall A, B \in \mathscr{H}$$
    其中$[A, B]$是把$A, B$看作$\mathscr{G}$的元素时的李括号,现在也称为子代数$\mathscr{H}$的李括号。
\end{definition}

\begin{theorem}
    设$H$是李群$G$的李子群,则$H$的李代数$\mathscr{H}$是$\mathscr{G}$的李子代数。
\end{theorem}

\section{单参子群和指数映射}

\section{常用李群及其李代数}

\section{李代数的结构常数}

\section{李变换群和Killing矢量场}

\section{伴随表示和Killing型}

\section{固有洛伦兹群和洛伦兹代数}

