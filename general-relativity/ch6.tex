\chapter{量子力学数学基础简介}

量子力学数学基础的主要内容是泛函分析,特别是希尔伯特空间及其线性算符的理论。
读过本书前两章的读者对集合、映射、拓扑、矢量空间及其对偶空间、张量、度规等概念比较熟悉,这为学习泛函分析提供了一定的方便。
学过上述概念的物理系学生往往跃跃欲试地把量子力学的内积、左右矢、线性算符以及波函数用正交归一基底展开等一系列问题同上述概念联系起来思考,力图求得更为深入和清晰的理解。
他们希望有一份简明读物作参考。
本章主要为满足这种需要而产生。
讲解中尽量与本书前两章以及量子力学的有关内容(特别是Dirac的左右矢记号)相联系或对比。
为了尽量节省读者的时间,我们只介绍最为必需的概念和定理,而且略去其中少数概念的定义和某些定理的证明。
想深入学习泛函分析的读者则应阅读有关教程或专著。

\section{Hilbert空间初步}

\subsection{Hilbert空间及其对偶空间}

第二章中定义的矢量空间称为实矢量空间。
把定义中的$\mathbb{R}$改为全体复数的集合$\mathbb{C}$便得到复矢量空间。
定义了内积的复矢量空间$V$称为(复)内积空间。
内积是一个从$V \times V$到$\mathbb{C}$的映射。
以$\operatorname{i}$代表这一映射,即$\operatorname{i} \colon V \times V \to \mathbb{C}$,则$\operatorname{i}$可看作有两个槽的机器,在两槽中分别输入$f, g \in V$,便得一个复数。
因此,$\operatorname{i}$也可形象地记作$\operatorname{i}(\bullet,\bullet)$,括号中的两个圆点代表两个槽。
为简单起见,索性把$\operatorname{i}(\bullet,\bullet)$简写为$(\bullet,\bullet)$,确切定义如下:

\begin{definition}
    复矢量空间$V$称为内积空间,若存在内积映射$\operatorname{i} \colon V \times V \to \mathbb{C}$,对任意$f, g, h \in V$和任意$c \in \mathbb{C}$满足
    \begin{enumerate}[(a)]
        \item $(f, g + h) = (f, g) + (f, h)$;
        \item $(f, cg) = c(f, g)$;
        \item $(f, g) = \widebar{(g, f)}$ (其中$\widebar{(g, f)}$代表复数$(g, f)$的共轭复数);
        \item $(f, f) \geq 0$,且$(f, f) = 0 \Leftrightarrow f = 0$。\footnote{
            只含零元的空间也可看作内积空间,但本章在不加声明时只讨论维数大于零的内积空间。
        }
    \end{enumerate}
\end{definition}

\begin{note}
    上述定义的前两条件表明内积映射$(\bullet,\bullet)$对第二槽是线性的。
    第三条件(配以前两条件)则表明$(f + g, h) = (f, h) + (g, h)$及$(cf, g) = \bar{c}(f, g)$。
    由于具有这种性质,我们说映射$(\bullet,\bullet)$对第一槽是反线性(或共轭线性)的。
    最后一个条件表明$(f, g) = 0 ~ \forall g \in V \Rightarrow f = 0$,因可取$g = f$。
\end{note}

\begin{note}
    上述定义对实矢量空间也适用,只须把$\mathbb{C}$改为$\mathbb{R}$。
    实空间的内积就是第二章定义的正定度规,但是,由于非正定度规不满足上述定义的最后一个条件,所以度规未必是内积。
\end{note}

\begin{example}
    $$C[a, b] \equiv \{[a, b] \subset \mathbb{R} \text{上连续复值函数} f(x)\}$$
    是无限维复矢量空间。定义内积
    $$(f, g) \coloneq \int^a_b \widebar{f(x)}g(x)\mathrm{d}x, ~ \forall f, g \in C[a, b]$$
    则$C[a, b]$是内积空间。
\end{example}

利用内积可自然定义内积空间中任意两点的距离,进而定义空间的拓扑。

\begin{definition}
    内积空间$V$中任意两元素$f$和$g$的距离定义为
    $$d(f, g) \coloneq \sqrt{(f - g, f - g)}$$
    易见$d(f, g) = d(g, f)$。
    设$f \in V, r > 0$,则以$f$为心、以$r$为半径的开球定义为
    $$B(f, r) \coloneq \{g \in V \mid d(g, f) < r\}$$
    用开球可给$V$定义拓扑(类似于第一章中的``通常拓扑''):
    $$\mathscr{T} \coloneq \{\text{空集或$V$中能表为开球之并的子集}\}$$
    可见内积空间$V$可自然地被定义为一个拓扑空间。
    上式定义的拓扑$\mathscr{T}$叫做$V$的自然拓扑。
    今后如无特别声明,凡涉及$V$的拓扑时一律指这一拓扑。
\end{definition}

第二章讲过(有限维)实矢量空间$V$的对偶空间$V^*$,它是由$V$到$\mathbb{R}$的全体线性映射的集合。
对(有限或无限维)复矢量空间$V$,对偶矢量可定义为由$V$到$\mathbb{C}$的线性映射。
然而内积空间除了是复矢量空间外还是拓扑空间,因此对映射$\eta \colon V \to \mathbb{C}$还可问及是否连续的问题。
(这时还涉及$\mathbb{C}$的拓扑,其定义也很自然:设$z_1, z_2 \in \mathbb{C}$且$z_1 = a_1 + \mathrm{i}b_1, z_2 = a_2 + \mathrm{i}b_2$(其中$a_1, a_2, b_1, b_2 \in \mathbb{R}$),则$z_1, z_2$的距离可定义为$d(z_1, z_2) \coloneq [(a_1 - a_2)^2 + (b_1 - b_2)^2]^{1 / 2}$,用此距离便可定义开球,从而定义$\mathbb{C}$的拓扑。)
在泛函分析中,每个连续的线性映射$\eta \colon V \to \mathbb{C}$称为$V$上的一个连续线性泛函。
我们关心$V$上全体连续线性泛函的集合(它有许多好的性质),并称它为$V$的对偶空间。\footnote{
    若$V$为有限维,则$V$上的线性泛函必定连续。
    因此只当$V$为无限维时连续性才是对线性泛函有实质意义的要求。
}

\begin{definition}
    内积空间$V$的对偶空间(又称共轭空间)定义为
    $$V^* \coloneq \{\eta \colon V \to \mathbb{C} \mid \eta \text{为连续的线性映射}\}$$
    $V^*$也可看作复矢量空间,为此只须用如下的自然方法定义加法和数乘:
    \begin{enumerate}[]
        \item 加法 ~ $(\eta_1 + \eta_2)(f) \coloneq \eta_1(f) + \eta_2(f), ~ \forall \eta_1 \eta_2 \in V^*, f \in V$;
        \item 数乘 ~ $(c\eta)(f) \coloneq c \cdot \eta(f), ~ \forall \eta \in V^*, c \in \mathbb{C}$
    \end{enumerate}
    (由此可知零元是$V^*$中这样的元素,它作用于任意$f \in V$都得零。)
\end{definition}

读者至此自然会联想到量子力学的右矢和左矢空间,并猜想右矢空间是内积空间$V$,而左矢空间则是$V^*$。
然而事情比此略为复杂。
要保证Dirac的左右矢记号得心应手,左右逢源,左、右矢空间应该``完全对等''。
这似乎不难:根据第二章,有限维实矢量空间$V$上的度规自然诱导一个从$V$到$V^*$的一一到上的线性映射。
然而,由于复空间的内积与实空间的度规的少许不同,复空间$V$上的内积自然诱导的从$V$到$V^*$的映射不是线性而是反线性的。
不过这不构成什么问题,真正构成问题的是量子力学中用到的内积空间多数是无限维的,而这导致上述映射未必到上,即$V$与$V^*$``不一样大''。
先看如下命题:

\begin{theorem}
    内积映射$(\bullet,\bullet)$自然诱导出一个一一的、反线性的映射$\nu \colon V \to V^*$。
\end{theorem}

\begin{proof}
    设$f \in V$,则$\eta_f \equiv (f, \bullet)$是从$V$到$\mathbb{C}$的映射。
    由内积定义可知它是线性的。
    还可证明(略)它是连续的,因此$\eta_f \in V^*$。
    具体说,$\eta_f$是$V^*$的这样一个元素,它作用于$g \in V$的结果为$\eta_f(g) \coloneq (f, g)$。
    可见$(\bullet,\bullet)$自然诱导出一个映射$\nu \colon V \to V^*$,定义为$\nu(f) \coloneq \eta_f$。
    根据映射$\nu$的定义,其反线性性是显然的,其一一性证明如下:
    设$\exists f_1, f_2 \in V$使得$\nu(f_1)(g) - \nu(f_2)(g) = 0 ~ \forall g \in V$,则有$(f_1 - f_2, g) = 0 ~ \forall g \in V$,因此就有$f_1 = f_2$。
\end{proof}

若$V$是有限维空间,则$V^*$与$V$有相同的维数(请参考第二章中对实矢量空间的证明),因而上述反线性映射$\nu \colon V \to V^*$一定到上。
若$V$是无限维的,则$V^*$也是无限维的,这时$\nu \colon V \to V^*$不一定到上。
直观地说,$V^*$有可能``比$V$大'',即$\nu[V] \subset V^*$但$\nu[V] \neq V^*$。
然而,为保证Dirac的左右矢运算给出正确结果,我们需要$V^*$同$V$``一样大'',即$\nu[V] = V^*$。
不久将看到,其实$V^*$比$V$最多``只多一层皮'',只要给$V$适当地``补上一层皮'',$\nu \colon V \to V^*$就可到上,$V$与$V^*$就``一样大''。
为了用准确语言表述这些直观思考,我们先介绍下列数学概念。

\begin{definition}
    设$V$为内积空间,$f \in V$,$\{f_n\}$是$V$中的一个序列(见第一章最后一节中的定义)。
    我们说$\{f_n\}$收敛于$f$(记作$f_n \to f$),若$\displaystyle\lim_{n \to \infty}d(f, f_n) = 0$。
    $f$称为序列$\{f_n\}$的极限,\footnote{
        内积空间可自然看作拓扑空间,而拓扑空间中的点序列的极限已有定义(见第一章最后一节)。
        不难证明该定义同本页定义等价。
    }记作$\displaystyle f = \lim_{n \to \infty}f_n$。
\end{definition}

\begin{definition}
    $V$中的序列$\{f_n\}$称为柯西序列,若$\forall \varepsilon > 0, \exists N > 0$使当$n, m \geq N$时有$d(f_n, f_m) < \varepsilon$。
\end{definition}

可以证明,收敛(于任一$f \in V$)的序列一定是柯西序列,但反之不然。

\begin{definition}
    内积空间$V$称为完备的,若其中任一柯西序列收敛。
\end{definition}

前述例子中的$C[a, b]$是不完备的内积空间。因为存在这样的柯西序列,其收敛于一个不连续函数(因此它在$C[a, b]$中不收敛)。
柯西序列$\{f_n(x)\}$在$C[a, b]$内没有极限表明$C[a, b]$不完备。
为使之完备化,可把$[a, b]$上虽不连续却平方可积的复值函数(例如上述柯西序列的极限)包含进去,扩大后的空间记作$L^2[a, b]$,即
$$L^2[a, b] \equiv \{f \colon [a, b] \to \mathbb{C} \mid \int^b_a |f(x)|^2 \mathrm{d}x < \infty\}\footnote{
    式中的积分是指Lebesgue积分。
    严格地说,函数$f(x)$还应具有``可测性''。
    可以这样说:物理上遇到的函数都满足这一要求。
    此外,两个函数如果只在测度为零的集上不同就应被视为$L^2[a, b]$的同一元素。
    本脚注同样适用于$L^2(\mathbb{R}^n)$。
}$$
仍用前面例子中的内积定义,则可证$L^2[a, b]$是完备的内积空间。

内积空间$C[a, b]$的这一完备化程序很有启发性。
事实上,任何不完备的内积空间$V$都可以完备化,为此只须把它略加扩大---把所有柯西序列``应有的极限点''都补进$V$中。
可以证明,对任何不完备内积空间$V$,总可找到完备的内积空间$\tilde V$,使得$V \subset \tilde V$而且$\bar V = \tilde V$,其中$\bar V$是把$\tilde V$看作拓扑空间(用其内积定义拓扑)时子集$V$的闭包。
直观地可以说$\tilde V$比$V$``最多只多一层皮''。

\begin{definition}
    完备的内积空间叫希尔伯特空间,记作$\mathscr{H}$。
\end{definition}

\begin{note}
    有限维的内积空间一定完备,因此一定是Hilbert空间。
    然而量子力学中用到的Hilbert空间多数是无限维的。
    无限维是许多问题变得复杂的根源。
\end{note}

同$L^2[a, b]$相仿,
$$L^2(\mathbb{R}^n) \equiv \{f \colon \mathbb{R}^n \to \mathbb{C} \mid \int|f|^2\mathrm{d}^nx < \infty\} ~ \text{(内积定义仿照前述例子)}$$
也是Hilbert空间,其中$L^2(\mathbb{R}^3)$是量子力学常用的波函数空间。

由于具有完备性,Hilbert空间有许多很好的性质,其中对我们特别有用的就是$\mathscr{H}$与其对偶空间$\mathscr{H}^*$``一样大'',即$\nu[\mathscr{H}] = \mathscr{H}^*$,见如下命题:

\begin{theorem}
    设$\mathscr{H}$是Hilbert空间,$\mathscr{H}^*$是其对偶空间,则$\forall \eta \in \mathscr{H}^*$,有唯一的$f_\eta \in \mathscr{H}$使$\eta(g) = (f_\eta, g) ~ \forall g \in \mathscr{H}$。
\end{theorem}

\begin{proof}
    见任一泛函分析教程中关于Riesz表现定理的证明。
\end{proof}

\begin{note}
    上述定理表明,对Hilbert空间,前述命题中的$\nu \colon \mathscr{H} \to \mathscr{H}^*$是到上的,即$\nu$是一一到上的反线性映射。
    这个命题的重要性在于保证$\mathscr{H}$同$\mathscr{H}^*$``一样大'',请注意不完备内积空间没有这样好的结论。
    可见物理学不但需要内积空间,而且需要完备的内积空间---Hilbert空间。
\end{note}

利用映射$\nu \colon \mathscr{H} \to \mathscr{H}^*$还可把$\mathscr{H}^*$定义为Hilbert空间:
$\forall \eta, \xi \in \mathscr{H}^*$,由上述定理可知有唯一的$f_\eta, f_\xi \in \mathscr{H}$使$\eta = \nu(f_\eta), \xi = \nu(f_\xi)$。
定义$\eta$和$\xi$的内积为
$$(\eta, \xi) \coloneq (f_\xi, f_\eta)$$
则不难验证$(\eta, \xi)$满足内积定义,故$\mathscr{H}^*$是内积空间。
还可证明$\mathscr{H}^*$是完备的,因而也是Hilbert空间。
可见$\mathscr{H}^*$与$\mathscr{H}$实在非常``像'':
它们之间不但存在一一到上的反线性映射,而且这一映射还在上式的意义上保内积。
$\mathscr{H}$和$\mathscr{H}^*$的这种相像性使我们可以把它们分别用作量子力学中的右矢和左矢空间(详见第四小节)。

\subsection{Hilbert空间的正交归一基}

$N$维矢量空间$V$的一个基底无非是由$N$个元素组成的、满足如下两个要求的一个序列$\{e_1, \cdots, e_N\}$:
\textcircled{1} $\{e_1, \cdots, e_N\}$线性独立;
\textcircled{2} $V$的任一元素$f$可由$\{e_1, \cdots, e_N\}$线性表出。
我们想把基底概念推广至无限维的Hilbert空间$\mathscr{H}$。

\begin{definition}
    $\mathscr{H}$的有限子集$\{f_1, \cdots, f_N\}$称为线性独立的,若
    $$\sum^N_{n = 1}c_nf_n = 0 \Rightarrow c_n = 0, n = 1, \cdots, N$$
    $\mathscr{H}$的任一子集$\{f_\alpha\}$称为线性独立的,若$\{f_\alpha\}$的任一非空有限子集线性独立。
\end{definition}

如果$\mathscr{H}$中存在满足以下两条件的无限序列$\{e_n\}$:
\textcircled{1} $\{e_n\}$线性独立;
\textcircled{2} $\mathscr{H}$的任一元素$f$可由$\{e_n\}$线性表出:
$$f = \sum^\infty_{n = 1}c_ne_n, ~ c_n \in \mathbb{C}$$
就说$\{e_n\}$构成$\mathscr{H}$的一个基底。
(上式中的$\displaystyle f = \sum^\infty_{n = 1}c_ne_n$是$\displaystyle f = \lim_{N \to \infty}\sum^N_{n = 1}c_ne_n$的简写。
请注意这里的极限涉及$\mathscr{H}$的拓扑,对未定义拓扑的矢量空间无意义。)

下面再讨论$\mathscr{H}$的正交归一基。

\begin{definition}
    Hilbert空间$\mathscr{H}$中的序列$\{f_n\}$叫正交归一序列,若
    $$(f_m, f_n) = \delta_{mn}$$
\end{definition}

不难证明$\mathscr{H}$中的任一正交归一序列都线性独立。
所以,若$\mathscr{H}$的维数有限,则当$\{f_n\}$的元素个数等于$\mathscr{H}$的维数时,$\{f_n\}$自然构成$\mathscr{H}$的一个基底,而且是正交归一基。
当$\mathscr{H}$是无限维时,要成为正交归一基,$\{f_n\}$的元素必须无限多个。
然而,并非由无限多个元素构成的正交归一序列$\{f_n\}$都是正交归一基,这里有一个是否已把元素``选够''的问题
(例如,设$\{f_n\}$是基底,则只取$n$为偶数的子集也含无限多个元素,但却不是基底),
只有满足下面定义的完备性条件的$\{f_n\}$才能成为正交归一基。

\begin{definition}
    $\mathscr{H}$中的正交归一序列$\{f_n\}$叫完备的,若$\mathscr{H}$中除零元外不存在与每个$f_n$都正交的元素
    (即不能通过给$\{f_n\}$添加新元素而得到``更大''的正交归一序列)。
\end{definition}

$\{f_n\}$的完备性保证$\mathscr{H}$的任一元素$f$都可用$\{f_n\}$线性表出,因此$\mathscr{H}$的任一完备的正交归一序列(如果存在)都是$\mathscr{H}$的正交归一基。
改用$\{e_n\}$代表完备的正交归一序列,则任一$f \in \mathscr{H}$都可用$\{e_n\}$线性表出。由前述两式得
$$c_n = (e_n, f)$$
故
$$f = \sum^\infty_{n = 1}(e_n, f)e_n$$

\subsection{Hilbert空间上的线性算符}

\begin{definition}
    映射$A \colon \mathscr{H} \to \mathscr{H}$称为$\mathscr{H}$上的算符,数学书一般译作算子。
    $A$作用于$f \in \mathscr{H}$的结果记作$Af$。
    $A$称为线性算符,若
    $$A(c_1f_1 + c_2f_2) = c_1Af_1 + c_2Af_2, ~ \forall f_1, f_2 \in \mathscr{H}, c_1, c_2 \in \mathbb{C}$$
\end{definition}

\begin{definition}
    算符$A \colon \mathscr{H} \to \mathscr{H}$和$B \colon \mathscr{H} \to \mathscr{H}$称为相等的,若$Af = Bf ~ \forall f \in \mathscr{H}$。
\end{definition}

今后如无特别声明,行文中的算符均指线性算符。
$\mathscr{H}$上全体线性算符的集合$\mathscr{L}(\mathscr{H})$也是个复矢量空间\footnote{
    注意和$\mathscr{H}^*$区分,$\mathscr{H}^*$中的元素皆是从$\mathscr{H}$到$\mathbb{C}$的连续线性映射。
},只要用如下的自然方式定义加法和数乘:
\begin{enumerate}[]
    \item 加法 ~ $(A_1 + A_2)f \coloneq A_1f + A_2f, ~ \forall A_1 A_2 \in \mathscr{L}(\mathscr{H}), f \in \mathscr{H}$;
    \item 数乘 ~ $(cA)f \coloneq c(Af), ~ \forall A \in \mathscr{L}(\mathscr{H}), c \in \mathbb{C}$
\end{enumerate}
($\mathscr{L}(\mathscr{H})$的零元(也叫零算符)是这样的算符,它作用于任意$f \in \mathscr{H}$都得$\mathscr{H}$的零元。)

算符分为有界算符和无界算符两大类(定义见下节)。
本节只讨论有界算符。

\begin{definition}
    $\mathscr{H}$上的一个线性算符$A \colon \mathscr{H} \to \mathscr{H}$自然诱导出$\mathscr{H}^*$上的一个线性算符$A^* \colon \mathscr{H}^* \to \mathscr{H}^*$,定义为
    $$(A^*\eta)(f) \coloneq \eta(Af), ~ \forall f \in \mathscr{H}, \eta \in \mathscr{H}^*$$
    易见$A^*\eta$(作为从$\mathscr{H}$到$\mathbb{C}$的映射)是线性的,还可证明它是连续的,因此$A^*\eta \in \mathscr{H}^*$,从而保证$A^*$是从$\mathscr{H}^*$到$\mathscr{H}^*$的映射。
    这样定义的$A^*$称为算符$A$的对偶算符。
\end{definition}

\begin{note}
    \textcircled{1} $A$和$A^*$是两个不同Hilbert空间上的算符,其中$A \colon \mathscr{H} \to \mathscr{H}$而$A^* \colon \mathscr{H}^* \to \mathscr{H}^*$。
    \textcircled{2} 不难证明:
    \begin{enumerate}[(a)]
        \item $A^*$的确是线性算符
        \item $A^*$与$A$的对应关系是线性的,即
        $$(A_1 + A_2)^* = A_1^* + A_2^*, ~ (cA)^* = cA^*$$
    \end{enumerate}
\end{note}

$\mathscr{H}$上的任一线性算符$A$的对偶算符$A^*$又可自然诱导出$\mathscr{H}$上的一个线性算符$A^\dagger \colon \mathscr{H} \to \mathscr{H}$。
设$\nu \colon \mathscr{H} \to \mathscr{H}^*$是前述一一、到上、反线性映射,则可定义$A^\dagger$为如下的复合映射:
$$A^\dagger \coloneq \nu^{-1} \comp A^* \comp \nu$$
$\nu$的反线性性导致$\nu^{-1}$的反线性性,加上$A^*$的线性性,便知$A^\dagger$是线性的。

\begin{definition}
    如上定义的$A^\dagger \colon \mathscr{H} \to \mathscr{H}$叫$A$的伴随算符。
\end{definition}

\begin{note}
    $A$和$A^\dagger$都是$\mathscr{H}$上的算符,但$A^*$是$\mathscr{H}^*$上的算符。
\end{note}

\begin{theorem}
    设$A^\dagger$是$A$的伴随算符,则
    $$(f, Ag) = (A^\dagger f, g), ~ \forall f, g \in \mathscr{H}$$
    反之,若$B \colon \mathscr{H} \to \mathscr{H}$满足
    $$(f, Ag) = (Bf, g), ~ \forall f, g \in \mathscr{H}$$
    则$B = A^\dagger$。
\end{theorem}

\begin{proof}
    $(f, Ag) = \eta_f(Ag) = (A^*\eta_f)(f) \equiv \eta_h(g) = (h, g) = (A^\dagger f, g)$,
    其中第一步用到$\eta_f$的定义,第二步用到$A^*$的定义,第三步无非是把$A^*\eta_f$记作$\eta_h$,最后一步用到$A^\dagger$的定义。
    反之,
    $$0 = (Bf, g) - (A^\dagger f, g) = (Bf - A^\dagger f, g), ~ \forall f, g \in \mathscr{H}$$
    故$0 = Bf - A^\dagger f = (B - A^\dagger)f ~ \forall f \in \mathscr{H}$,因而$B = A^\dagger$。
\end{proof}

\begin{note}
    可见上式可用作$A^\dagger$的等价定义。
\end{note}

\begin{theorem}
    $A^\dagger$与$A$的对应关系是反线性的,即
    \[\begin{split}
        (A_1 + A_2)^\dagger = A_1^\dagger + A_2^\dagger \\
        (cA)^\dagger = \bar{c}A^\dagger, ~ \forall c \in \mathbb{C}
    \end{split}\]
\end{theorem}

\begin{proof}
    $((A_1 + A_2)^\dagger)f, g) = (f, (A_1 + A_2)g) = (f, A_1g) + (f, A_2g) = (A_1^\dagger f, g) + (A_2^\dagger f, g) = ((A_1^\dagger + A_2^\dagger)f, g)$
    
    $((cA)^\dagger f, g) = (f, cAg) = c(f, Ag) = c(A^\dagger f, g) = (\bar{c}A^\dagger f, g)$
\end{proof}

\begin{theorem}
    设$A$为$\mathscr{H}$上的有界算符,则$A^{\dagger\dagger} = A$
\end{theorem}

\begin{proof}
    $(A^{\dagger\dagger}f, g) = (f, A^\dagger g) = \widebar{(A^\dagger g, f)} = \widebar{(g, Af)} = (Af, g)$
\end{proof}

\begin{definition}
    (有界)线性算符$A \colon \mathscr{H} \to \mathscr{H}$称为自伴的或厄米的,若$A = A^\dagger$,即
    $$(f, Ag) = (Af, g), ~ \forall f, g \in \mathscr{H}$$
\end{definition}

\begin{note}
    ``厄米算符就是自伴算符''的说法只对有界算符成立。
    对无界算符,自伴性强于厄米性,详见下节。
\end{note}

\subsection{Dirac的左右矢记号}

在Dirac的记号中,每一$f \in \mathscr{H}$记作$\ket{f}$,称为右矢;
每一$\eta \in \mathscr{H}^*$记作$\bra{\eta}$,称为左矢。
$\bra{\eta}$作用于$\ket{f}$所得复数记作$\braket{\eta | f}$,即$\braket{\eta | f} \equiv \eta(f)$。
物理学家常把$\braket{\eta | f}$称为$\bra{\eta}$与$\ket{f}$的内积,在泛函分析中$\braket{\eta | f}$则是$g_\eta$与$f$的内积,其中$g_\eta \equiv \nu^{-1}(\eta) \in \mathscr{H}$。
记号$\braket{\eta | f}$中的$\braket{}$可看作一个尖括号,``括号''在英文中是``bracket'',去掉字母c并拆为两半,自然把左半$\bra{}$称为bra,而右半$\ket{}$称为ket。
$\ket{f} \in \mathscr{H}$在$\nu \colon \mathscr{H} \to \mathscr{H}^*$映射下的像$\eta_f \in \mathscr{H}^*$本应记作$\bra{\eta_f}$,但可简记为$\bra{f}$,这不会与$\ket{f}$混淆,却可形象地表明$\bra{f}$就是$\ket{f}$在$\nu$映射下的对应物。
通常也把这种对应关系记作$\bra{f} \leftrightarrow \ket{f}$。
同样,$\bra{\eta} \in \mathscr{H}^*$的逆像$\nu^{-1}(\eta)$可简记为$\ket{\eta}$,即$\bra{\eta} \leftrightarrow \ket{\eta}$,原来的$(f, g) = \eta_f(g)$则可表为$(f, g) = \braket{f | g}$。
实际上,使用Dirac记号后无需再以拉丁字母$f, g, \cdots$和希腊字母$\eta, \xi, \cdots$分别代表$\mathscr{H}$和$\mathscr{H}^*$的元素。
设$\ket{\psi} \in \mathscr{H}, c \in \mathbb{C}$,则$c\ket{\psi} \in \mathscr{H}$,则可记作$\ket{c\psi}$,即$c\ket{\psi} \equiv \ket{c\psi}$。
由内积空间的定义有
\[\begin{split}
    \braket{\psi | \phi} = \widebar{\braket{\phi | \psi}}, \\
    \braket{\psi | c\phi} = c\braket{\psi | \phi}, \\
    \braket{c\psi | \phi} = \bar{c}\braket{\psi | \phi}
\end{split}\]
其中$\bra{c\psi}$代表$\ket{c\psi}$在映射$v$下的像,即$\bra{c\psi} \leftrightarrow \ket{c\psi}$。
注意到$\nu$的反线性性,得
$$\bra{c\psi} = \nu(c\psi) = \bar{c}\nu(\psi) = \bar{c}\bra{\psi}$$
故映射$\nu \colon \mathscr{H} \to \mathscr{H}^*$的反线性性体现为
$$\bra{c\psi} = \bar{c}\bra{\psi}$$
或
$$c\ket{\psi} \leftrightarrow \bar{c}\bra{\psi}$$
其中$\bar{c}\bra{\psi}$是复数$\bar{c}$乘左矢$\bra{\psi}$所得的左矢,也可记为$\bra{\psi}\bar{c}$

算符$A$作用于右矢$\ket{\psi} \in \mathscr{H}$所得的右矢记作$\ket{A\psi}$,即$A\ket{\psi} \equiv \ket{A\psi}$。
把用$A^*$作用于左矢$\bra{\eta} \in \mathscr{H}^*$的结果记作$A^*\bra{\eta}$,则$A^*$的定义式可表为
$$(A^*\bra{\eta})\ket{f} = \bra{\eta}(A\ket{f}), ~ \forall \ket{f} \in \mathscr{H}, \bra{\eta} \in \mathscr{H}^*$$
现在说明上式右边的圆括号可以去掉。
先回到不用Dirac记号的式子。
因为$A \colon \mathscr{H} \to \mathscr{H}$而$\eta \colon \mathscr{H} \to \mathbb{C}$,所以$\eta \comp A \colon \mathscr{H} \to \mathbb{C}$。
$\eta$是$\mathscr{H}$上的连续线性泛函保证$\eta \comp A$也是,故$\eta \comp A \in \mathscr{H}^*$。
进一步把$\eta \comp A$简记作$\eta A$,则有
$$(A^*\eta)(f) = (\eta \comp A)(f) = (\eta A)(f)$$
因而$A^*\eta = \eta A$($\in \mathscr{H}^*$),用Dirac记号则为$A^*\bra{\eta} = \ket{\eta}A$,于是前式左边等于$(\bra{\eta}A)\ket{f}$,前式就成为
$$(\bra{\eta}A)\ket{f} = \bra{\eta}(A\ket{f})$$
上式表明圆括号没有必要,$\bra{\eta}A\ket{f}$有明确含义,它既可理解为$(\bra{\eta}A)\ket{f}$,也可理解为$\bra{\eta}(A\ket{f})$。
有人把$A^\dagger$记作$A^*$,则他们的$\bra{\eta}A^*$是我们的$\bra{\eta}A^\dagger = A^{\dagger*}\bra{\eta}$。

\begin{theorem}
    $A\ket{\psi} \leftrightarrow \bra{\psi}A^\dagger$
\end{theorem}

\begin{proof}
    因$A\ket{\psi} \equiv \ket{A\psi} \leftrightarrow \bra{A\psi}$,故只须证$\braket{A\psi | \phi} = (\bra{\psi}A^\dagger)\ket{\phi} ~ \forall \ket{\phi} \in \mathscr{H}$。
    由上式得
    $$(\bra{\psi}A^\dagger)\ket{\phi} = \bra{\psi}(A^\dagger\ket{\phi}) = \braket{\psi | A^\dagger\phi} = \braket{A\psi | \phi}$$
    其中最后一步用到$A^{\dagger\dagger} = A$。
\end{proof}

由上可知,任一算符$A$的本征方程$A\ket{\psi} = c\ket{\psi}$的左矢形式为
$$\bra{\psi}A^\dagger = \bra{\psi}\bar{c}$$

设$\{\ket{e_n}\}$是$\mathscr{H}$的正交归一基,则可定义$\mathscr{H}$上的线性算符$\sum_n\ket{e_n}\bra{e_n}$为
$$(\sum_n\ket{e_n}\bra{e_n})\ket{\psi} \coloneq \sum_n\ket{e_n}\braket{e_n | \psi} \in \mathscr{H}, ~ \forall \ket{\psi} \in \mathscr{H}$$
注意到$\sum_n\ket{e_n}\braket{e_n | \psi} = \ket{\psi}$就有
$$\sum_n\ket{e_n}\bra{e_n} = I$$
其中$I$代表单位算符(恒等映射),其定义为$I\ket{\psi} \coloneq \ket{\psi} ~ \forall \ket{\psi} \in \mathscr{H}$。
上式便是量子力学中常用的完备性关系。

\subsection{态矢和射线}

量子系统每一时刻的态由Hilbert空间$\mathscr{H}$中的一个矢量(右矢$\ket{\psi}$)表示,因此右矢叫做态矢。
然而态矢与态的对应关系不是一一的。
Dirac说过(大意):设由$\ket{\psi}$代表的态与自己叠加,结果将对应于态矢$c_1\ket{\psi} + c_2\ket{\psi} = (c_1 + c_2)\ket{\psi}$,其中$c_1$和$c_2$是任意复数。
我们应该假定,除了$c_1 + c_2 = 0$的情况外,结果态$(c_1 + c_2)\ket{\psi}$与原始态$\ket{\psi}$相同,即态矢$(c_1 + c_2)\ket{\psi}$和$\ket{\psi}$应代表相同的态。
(``自己与自己叠加不会得出新态'',请注意这与经典物理非常不同。)
就是说,右矢$\ket{\psi}$和$c\ket{\psi}$($c$为任意非零复数)代表同一状态。
于是,若对$\mathscr{H}$的任意非零元素$\ket{\psi}$定义$\mathscr{H}$的子集$r_\psi \coloneq \{c\ket{\psi} \mid c \in \mathbb{C}, c \neq 0\}$,并称$r_\psi$为过$\ket{\psi}$的一条射线,则一条射线对应于量子系统的一个态。
以$\mathscr{H}$中的所有射线为元素的集合$\mathscr{R}$叫射线空间。

\section{无界算符及其自伴性}

前面一节讨论的是``$\mathscr{H}$上的''``有界''线性算符。
下面先解释这两个定语的含义。

``$\mathscr{H}$上的算符(operator on $\mathscr{H}$)''是指从$\mathscr{H}$到$\mathscr{H}$的映射,其定义域是整个$\mathscr{H}$。
然而量子力学中大多数算符$A$的定义域(记作$D_A$)都只能是$\mathscr{H}$的一个真子集,即$A$只能是从$D_A \subset \mathscr{H}$($D_A \neq \mathscr{H}$)到$\mathscr{H}$的映射。
这称为$\mathscr{H}$中的算符(operator in $\mathscr{H}$)。\footnote{
    这种用on和in形容算符定义域的约定只在部分文献中采用。
    在不采用这种约定的文献中,``operator on $\mathscr{H}$''不表明该算符的定义域是$\mathscr{H}$。
}以$1$维空间的波函数空间$\mathscr{H} = L^2(\mathbb{R})$为例。(如前所述,此乃无限维Hilbert空间。)
定义位置算符$X \colon D_X \to L^2(\mathbb{R})$为
$$(X\psi)(x) \coloneq x\psi(x), ~ \forall \psi \in D_X, x \in \mathbb{R}$$
上式的含义是:$X$作用于$D_X$的任一元素$\psi$的结果是这样一个波函数,其在点$x \in \mathbb{R}$的值等于$\psi$在点$x$的值$\psi(x)$乘以$x$。
算符$X$的定义域$D_X \neq L^2(\mathbb{R})$,因为$\exists \psi \in L^2(\mathbb{R})$使$X\psi \notin L^2(\mathbb{R})$。
例如,函数
\[\psi(x) = \begin{cases}
    1 / x, & x \geq 1 \\
    0, & x < 1
\end{cases}\]
是平方可积的,但函数
\[(X\psi)(x) = x\psi(x) = \begin{cases}
    1, & x \geq 1 \\
    0, & x < 1
\end{cases}\]
却非平方可积,说明$\psi \in L^2(\mathbb{R})$而$\psi \notin D_X$。
因而位置算符$X$只是$L^2(\mathbb{R})$中(而非$L^2(\mathbb{R})$上)的线性算符。
第二个例子是动量算符$P \colon D_P \to L^2(\mathbb{R})$,其定义为
$$(P\psi)(x) \coloneq -\mathrm{i}\hbar\frac{\mathrm{d}\psi(x)}{\mathrm{d}x}, ~ \forall \psi \in D_P$$
$D_P$是$L^2(\mathbb{R})$的这样的子集,其中每个元素$\psi(x)$几乎处处可微,\footnote{
    关于$D_P$的进一步限制可见后文。
}而且
$$\frac{\mathrm{d}\psi(x)}{\mathrm{d}x} \in L^2(\mathbb{R})$$
并非$L^2(\mathbb{R})$的每个元素都满足这一条件,所以$D_P \neq L^2(\mathbb{R})$,即动量算符$P$也只是$L^2(\mathbb{R})$中(而非$L^2(\mathbb{R})$上)的线性算符。

下面介绍有界算符的定义。

\begin{definition}
    线性算符$A \colon D_A(\subset \mathscr{H}) \to \mathscr{H}$称为有界算符,若$\exists M > 0$使
    $$\sqrt{(Af, Af)} \leq M\sqrt{(f, f)}, ~ \forall f \in D_A$$
    否则称为无界算符。
\end{definition}

\begin{theorem}
    设$A$是$\mathscr{H}$上的有界线性算符,则$A^\dagger$也是$\mathscr{H}$上的有界线性算符。
\end{theorem}

对有限维Hilbert空间,所有线性算符都是有界的。
反之,量子力学用到的无限维Hilbert空间中许多重要线性算符(例如位置算符$X$,动量算符$P$和哈氏算符$H$)都是无界算符。
$X$的无界性可证明如下:考虑波函数序列$\{\psi_1(x), \psi_2(x), \cdots, \psi_n(x), \cdots \}$,其中$\psi_n(x)$定义为
\[\psi_n(x) = \begin{cases}
    1, & x \in [n, n + 1) \\
    0, & x \notin [n, n + 1)
\end{cases}\]
则易见$\psi_n \in D_X$。有
$$(\psi_n, \psi_n) = \int^{+\infty}_{-\infty}\widebar{\psi_n(x)}\psi_n(x)\mathrm{d}x = \int^{n + 1}_n\mathrm{d}x = 1$$
另一方面,
$$(X\psi_n, X\psi_n) = \int^{+\infty}_{-\infty}x^2\widebar{\psi_n(x)}\psi_n(x)\mathrm{d}x = \int^{n + 1}_nx^2\mathrm{d}x > n^2$$
因为$n$可为任意大的自然数,所以不存在$M > 0$使
$$\sqrt{(X\psi_n, X\psi_n))} \leq M\sqrt{(\psi_n, \psi_n)}, ~ \forall \psi_n(x)$$
可见$X$无界。

可以证明,有界算符的定义域总可延拓至整个$\mathscr{H}$,因此讨论有界算符时可只关心$\mathscr{H}$上的算符。
然而对无界算符不能如此简化,所以涉及无界算符时要格外注意定义域问题。

设$A$是$\mathscr{H}$中的无界算符,定义域$D_A \neq \mathscr{H}$。
我们遇到的第一个问题是如何定义其伴随算符$A^\dagger$。
回忆上节对$\mathscr{H}$上的算符$A$的对偶算符$A^*$和伴随算符$A^\dagger$的定义。
先用下式定义$A^*$:
$$(A^*\eta)(f) \coloneq \eta(Af), ~ \forall f \in \mathscr{H}$$
再用$A^*$定义$A^\dagger$。
可以证明这样定义的$A^*$(和$A^\dagger$)是$\mathscr{H}^*$上(和$\mathscr{H}$上)的有界算符。
然而,$A^*$的上述定义对$\mathscr{H}$中的算符$A$不适用。
由于$D_A \neq \mathscr{H}$,$Af$对于不在$D_A$中的$f$无意义,因此上式中的``$\forall f \in \mathscr{H}$''应改为``$\forall f \in D_A$''。
然而这样改后的式子就不成其为$A^*$的定义,因为要定义$\mathscr{H}^*$的元素$A^*\eta$必须定义它对$\mathscr{H}$的每一元素的作用。
可见当$D_A \neq \mathscr{H}$时$A^\dagger$需要另给定义。
注意到其等价定义,可考虑用下法直接定义$A^\dagger$。
对给定的$f \in \mathscr{H}$,若存在唯一的$h_f \in \mathscr{H}$使
$$(f, Ag) = (h_f, g), ~ \forall g \in D_A$$
就把$A^\dagger f$定义为$h_f$,即
$$A^\dagger f \coloneq h_f$$
$A^\dagger$的定义域自然为
$$D_{A^\dagger} = \{f \in \mathscr{H} \mid \exists \text{唯一} h_f \in \mathscr{H} \text{使} (f, Ag) = (h_f, g) ~ \forall g \in D_A\}$$
用此法定义$A^\dagger$的可能性取决于满足式$(f, Ag) = (h_f, g), ~ \forall g \in D_A$的$h_f$的唯一性,其充要条件将由下述定理给出。
为讲此定理先介绍两个术语:拓扑空间$X$的子集$U \subset X$称为稠密子集,若$\bar U = X$($\bar U$代表$U$的闭包);Hilbert空间$\mathscr{H}$中的算符$A$称为稠定的,若$\widebar{D_A} = \mathscr{H}$。

\begin{theorem}
    设$A$是$\mathscr{H}$中的线性算符,$f \in \mathscr{H}$,则满足
    $$(f, Ag) = (h, g), ~ \forall g \in D_A$$
    的$h \in \mathscr{H}$是唯一的当且仅当$A$是稠定的。
\end{theorem}

由此可知,当且仅当$\mathscr{H}$中的算符$A$为稠定时,其伴随算符$A^\dagger$可由前式定义,其定义域亦如前所述。
我们有
$$(f, Ag) = (A^\dagger f, g), ~ \forall g \in D_A, f \in D_{A^\dagger}$$
不难证明这样定义的$A^\dagger$是线性的,而且若$A$是$\mathscr{H}$上的有界算符,则这样定义的$A^\dagger$与上节定义的$A^\dagger$相同(这时$A^\dagger$也是$\mathscr{H}$上的有界算符)。
考虑到上述定理,今后凡谈到算符$A$的伴随算符$A^\dagger$时都默认$A$是稠定的。
可以证明位置和动量算符都是稠定算符。
应该指出,$D_A$稠密并不保证$D_{A^\dagger}$稠密。
在最坏的情况下,$D_{A^\dagger}$可以``小''到只含零元的程度,即$D_{A^\dagger} = \{0\}$。
幸好这种情况很少出现。
定义域不稠密的无界算符在量子力学中没有什么用处。

算符的厄米性对量子力学的重要性是众所周知的。
上节已对$\mathscr{H}$上的算符$A$的厄米性下了如下定义:
$$(f, Ag) = (Af, g), ~ \forall f, g \in \mathscr{H}$$
$\mathscr{H}$中的算符$A \colon D_A \to \mathscr{H}$的厄米性仍可用上式定义,只须把式中的$\forall f, g \in \mathscr{H}$改为$\forall f, g \in D_A$。
然而,泛函分析有个Hellinger--Toeplitz定理,它断言Hilbert空间$\mathscr{H}$上满足上式的算符$A$必然有界。
这就导致如下严酷的结论:无界的厄米算符的定义域不可能是全$\mathscr{H}$。
因此涉及无界厄米算符时必须格外注意定义域问题。
下面将看到,无界算符的厄米性和自伴性的关键区别就在于定义域。

设$A, B$是$\mathscr{H}$上的算符,则其和、积及相等的定义十分简单:
\begin{enumerate}[]
    \item 和 ~ $(A + B)f \coloneq Af + Bf, ~ \forall f \in \mathscr{H}$,
    \item 积 ~ $(AB)f \coloneq A(Bf), ~ \forall f \in \mathscr{H}$,
    \item 相等 ~ $A = B \Leftrightarrow Af = Bf, ~ \forall f \in \mathscr{H}$。
\end{enumerate}
然而对$\mathscr{H}$中的算符就要麻烦一些。
例如,$A + B$的定义域只能是$D_A \cap D_B$,而$AB$的定义域则还涉及$B$的值域,因为只当$Bf \in D_A$时$ABf$才有意义。
$A$与$B$的相等性和相互包含性则由如下定义规定:

\begin{definition}
    设$A, B$是$\mathscr{H}$中的线性算符,其定义域分别为$D_A$和$D_B$,则
    \begin{enumerate}[(a)]
        \item $A = B$若$D_A = D_B$而且$Af = Bf, ~ \forall f \in D_A = D_B$;
        \item $A \subset B$若$D_A \subset D_B$而且$Af = Bf, ~ \forall f \in D_A$(把$B$称为$A$的延拓或扩张)。
    \end{enumerate}
\end{definition}

\begin{definition}
    $\mathscr{H}$中的任意(有界或无界)稠定线性算符$A$称为厄米的,若
    $$(f, Ag) = (Af, g), ~ \forall f, g \in D_A$$
\end{definition}

\begin{theorem}
    $\mathscr{H}$中的稠定算符$A$为厄米算符的充要条件是$A \subset A^\dagger$。\footnote{
        $A$为厄米不保证$A^\dagger$为厄米。
        事实上,对厄米算符$A$有$A \subset A^{\dagger\dagger} \subset A^\dagger$(却未必有$A^\dagger \subset A^{\dagger\dagger}$)。
    }
\end{theorem}

\begin{proof}
    \begin{enumerate}[(A)]
        \item 设$A$为厄米,则$\forall f \in D_A$,有$(f, Ag) = (h_f, g) ~ \forall g \in D_A$。
        对比$D_{A^\dagger}$的定义发现$f \in D_{A^\dagger}$(其中的$h_f$的唯一性由$A$的稠定性保证,$h_f$就是$Af$)。
        所以,$f \in D_A \Rightarrow f \in D_{A^\dagger}$,因而$A \subset A^\dagger$。
        \item 设$A \subset A^\dagger$,则$D_A \subset D_{A^\dagger}$且$\forall f \in D_A$有$A^\dagger f = Af$,故
        $$(f, Ag) = (Af, g), ~ \forall f, g \in D_A$$
        可见$A$是厄米的。
    \end{enumerate}
\end{proof}

\begin{note}
    讨论表明(此处只介绍结论),若厄米算符$A$有界,可把$A$唯一地延拓为$\mathscr{H}$上的有界算符,仍记作$A$(唯一性由$D_A$的稠密性保证)。
    同样,$A^\dagger$也可被唯一地延拓为$\mathscr{H}$上的有界算符,而且$(f, Ag) = (Af, g) = (A^\dagger f, g) ~ \forall f, g \in \mathscr{H}$。
    可见上述命题中的$A \subset A^\dagger$对有界厄米算符$A$可表为$A = A^\dagger$。
    然而确实存在$D_A \neq D_{A^\dagger}$(因而$A \neq A^\dagger$)的无界厄米算符$A$,对这种算符一般只能写$A \subset A^\dagger$。
    这说明对无界算符$A$来说,$A = A^\dagger$是比$A \subset A^\dagger$更高的要求。
    满足$A = A^\dagger$的算符非常重要,值得赋予专名:
\end{note}

\begin{definition}
    $\mathscr{H}$中的任意(有界或无界)稠定算符$A$称为自伴的,若$A = A^\dagger$。
\end{definition}

由此可知,无界算符的自伴性强于厄米性。
偏偏量子力学中多数算符是无界算符,因此区分厄米性和自伴性就成为重要问题。
厄米性和自伴性的关键差别在于定义域,所以对量子力学的算符应该特别注意定义域及其有关问题(例如定义域的延拓或压缩以及算符的性质在定义域延拓或压缩时的可能改变)。
讨论量子力学问题时出现的若干混淆正是不注意算符的定义域问题所致。

\begin{theorem}
    $L^2(\mathbb{R})$中的位置算符$X \colon D_X \to L^2(\mathbb{R})$是自伴算符。
\end{theorem}


