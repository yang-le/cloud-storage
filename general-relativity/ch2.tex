\chapter{流形和张量场}

\section{微分流形}

\begin{definition}
拓扑空间$(M, \mathscr{T})$称为\textbf{$n$维微分流形},简称$n$维流形,若$M$有开覆盖${O_\alpha}$,即$M = \bigcup\limits_\alpha O_\alpha$,满足
\begin{enumerate}[(a)]
\item 对每一$O_\alpha$,存在同胚$\psi_\alpha \colon O_\alpha \to V_\alpha$($V_\alpha$是$\mathbb{R}^n$用通常拓扑衡量的开子集);
\item 若$O_\alpha \cap O_\beta \neq \emptyset$,则复合映射$\psi_\beta \comp \psi_\alpha^{-1}$是$C^\infty$(光滑)的。
\end{enumerate}
\end{definition}

\begin{note}
\textcircled{1}$\psi_\beta \comp \psi_\alpha^{-1}$是从$\psi_\alpha[O_\alpha \cap O_\beta] \subset \mathbb{R}^n$到$\psi_\beta[O_\alpha \cap O_\beta] \subset \mathbb{R}^n$的映射。
因$\mathbb{R}^n$的每点都有$n$个自然坐标,故$\psi_\beta \comp \psi_\alpha^{-1}$提供了$n$个$n$元函数。所谓$\psi_\beta \comp \psi_\alpha^{-1}$是$C^\infty$的,就是指这每一个$n$元函数都是$C^\infty$的。
\textcircled{2}设$p \in O_\alpha$,则$\psi_\alpha(p) \in \mathbb{R}^n$,故$\psi_\alpha(p)$有$n$个自然坐标。很自然地把这$n$个数称为$p$点在映射$\psi_\alpha$下获得的\textbf{坐标}。
$M$作为拓扑空间,其元素本来一般没有坐标,但作为流形,$M$中位于$O_\alpha$内的元素(点)就可以通过映射$\psi_\alpha$获得坐标。若$O_\alpha \cap O_\beta \neq \emptyset$,
则$O_\alpha \cap O_\beta$内的点既可以通过$\psi_\alpha$又可以通过$\psi_\beta$获得坐标,这两组坐标一般不同。我们说$(O_\alpha, \psi_\alpha)$构成一个(局域)\textbf{坐标系},其\textbf{坐标域}为$O_\alpha$;
$(O_\beta, \psi_\beta)$构成另一坐标系,其坐标域为$O_\beta$。于是$O_\alpha \cap O_\beta$内的点至少有两组坐标,分别记作$\{x^\mu\}$和$\{x'^\nu\}$($\mu, \nu = 1, \cdots, n$)。
由映射$\psi_\beta \comp \psi_\alpha^{-1}$提供的、体现两组坐标之间关系的$n$个$n$元函数$$x'^1 = \phi^1(x^1, \cdots, x^n), \cdots, x'^n = \phi^n(x^1, \cdots, x^n)$$就称为一个\textbf{坐标变换}。
定义的条件(b)保证坐标变换中的函数关系$x'^\mu = \phi^\mu(x^1, \cdots, x^n)$都是$C^\infty$的。为方便起见也常称$\{x^\mu\}$为坐标系,虽然从$\{x^\mu\}$中看不出坐标域的范围。
物理学家也常把$x'^\mu = \phi^\mu(x^1, \cdots, x^n)$记作$x'^\mu = x'^\mu(x^1, \cdots, x^n)$。
\end{note}

\begin{definition}
坐标系$(O_\alpha, \psi_\alpha)$在数学上又叫\textbf{图},满足定义条件(a)、(b)的全体图的集合$\{(O_\alpha, \psi_\alpha)\}$叫\textbf{图册}。
条件(b)又称\textbf{相容性条件},因此说一个图册中的任意两个图都是相容的。
\end{definition}

\begin{example}
设$M = (\mathbb{R}^2, \mathscr{T}_u)$。选$O_1 = \mathbb{R}^2, \psi_1 =\text{恒等映射}$,则$\{(O_1, \psi_1)\}$便是只含一个图的图册,故$\mathbb{R}^2$是$2$维流形,而且是能用一个坐标域覆盖的流形,称为\textbf{平凡流形}。
根据这个图册,$\mathbb{R}^2$中每点的坐标就是它作为$\mathbb{R}^2$的元素天生就有的自然坐标。$\mathbb{R}^2$的点当然也可用其他坐标(如极坐标)描述。
其实这无非是选择与图$(O_1, \psi_1)$相容的另一个图$(O_2, \psi_2)$,其中$\psi_2$把$p \in O_2$映为$\psi_2(p) \in \mathbb{R}^2$,再把$\psi_2(p)$的自然坐标称为$p$点的新坐标而已。
但应注意坐标域$O_2$未必能包括$\mathbb{R}^2$的全体点(例如极坐标)。

同理可知$\mathbb{R}^n$是$n$维平凡流形。
\end{example}

\begin{example}
设$M = (S^1, \mathscr{S})$,其中$S^1 := \{x \in \mathbb{R}^2 \mid |x - o| = 1\}$是以原点$o$为心的单位圆周,$\mathscr{S}$是$\mathbb{R}^2$的$\mathscr{T}_u$在$S^1$上诱导的拓扑。则可证明$S^1$是$1$维流形。
\end{example}

\begin{example}
设$M = (S^2, \mathscr{S})$,其中$S^2 := \{x \in \mathbb{R}^3 \mid |x - o| = 1\}$是以原点$o$为心的单位球面,$\mathscr{S}$是$\mathbb{R}^3$的$\mathscr{T}_u$在$S^2$上诱导的拓扑。则可证明$S^2$是$2$维流形。
\end{example}

设图册$\{(O_\alpha, \psi_\alpha)\}$把拓扑空间$M$定义为一个流形,则此图册中的任意两个图自然是相容的。
但也可用另一图册$\{(O'_\beta, \psi'_\beta)\}$把同一$M$定义为流形,这时有两种可能:
\textcircled{1}这两个图册不相容,即存在$O_\alpha$和$O'_\beta$使$O_\alpha \cap O'_\beta \neq \emptyset$,且在$O_\alpha \cap O'_\beta$上$\psi_\alpha$与$\psi'_\beta$不满足定义条件(b),
这时就说这两个图册把$M$定义为两个不同的微分流形,并说这两个图册代表两种不同的\textbf{微分结构};
\textcircled{2}这两个图册是相容的,这时就说它们把$M$定义为同一个微分流形(只有一种微分结构)。为方便起见,不妨把$\{(O_\alpha, \psi_\alpha); (O'_\beta, \psi'_\beta)\}$看成一个图册。
更进一步,索性把所有与$(O_\alpha, \psi_\alpha)$相容的图都放到一起造出一个最大的图册。今后说到$M$是一个流形时,总是默认已选定某一最大的图册作为微分结构。这使我们可以进行任意坐标变换。

微分流形与拓扑空间的重要区别是前者除有拓扑结构外还有微分结构,因此两个流形之间的映射不但可谈及是否连续,还可谈及是否可微,乃至是否$C^\infty$。
设$M$和$M'$是两个流形,维数依次为$n$和$n'$,$\{(O_\alpha, \psi_\alpha)\}$和$\{(O'_\beta, \psi'_\beta)\}$依次为两者的图册,$f \colon M \to M'$是一个映射。
$\forall p \in M$,任取坐标系$(O_\alpha, \psi_\alpha)$使$p \in O_\alpha$以及坐标系$(O'_\beta, \psi'_\beta)$使$f(p) \in O'_\beta$,
则$\psi'_\beta \comp f \comp \psi_\alpha^{-1}$是从$V_\alpha \equiv \psi_\alpha[O_\alpha]$到$\mathbb{R}^{n'}$的映射,
因此相应于$n'$个$n$元函数,它们的$C^r$性可用以定义$f \colon M \to M'$的$C^r$性。

\begin{definition}
$f \colon M \to M'$称为\textbf{$C^r$类映射},如果$\forall p \in M$,映射$\psi'_\beta \comp f \comp \psi_\alpha^{-1}$对应的$n'$个$n$元函数是$C^r$类的。
\end{definition}

\begin{note}
由于同一图册中各图相容,上述定义与坐标系$\{(O_\alpha, \psi_\alpha)\}$及$\{(O'_\beta, \psi'_\beta)\}$的选择无关。
\end{note}

\begin{definition}
微分流形$M$和$M'$称为\textbf{互相微分同胚},若存在$f \colon M \to M'$,满足(a)$f$是一一到上的;(b)$f$及$f^{-1}$是$C^\infty$的。这样的$f$称为从$M$到$M'$的\textbf{微分同胚映射},简称\textbf{微分同胚}。
\end{definition}

\begin{note}
\textcircled{1} 微分同胚是对流形间的映射可以提出的最高要求(若流形上还有附加结构则另当别论),互相微分同胚的流形可视作相等。
\textcircled{2} 只有维数相等的流形才可能微分同胚。
\textcircled{3} 若把流形定义中的$O_\alpha$和$V_\alpha$看做流形,则$\psi_\alpha$也是微分同胚。
\end{note}

映射$f \colon M \to M'$的一个重要而简单的特例是$M' = \mathbb{R}$的情况。这时$M$的每点对应着一个实数,于是有如下定义:

\begin{definition}
$f \colon M \to \mathbb{R}$称为\textbf{$M$上的函数}或\textbf{$M$上的标量场}。若$f$为$C^\infty$的,则称为\textbf{$M$上的光滑函数}。
$M$上全体光滑函数的的集合记作$\mathscr{F}_M$,在不会混淆时简记为$\mathscr{F}$。今后在提到函数而不加声明时都是指光滑函数。
\end{definition}

\begin{example}
$\mathbb{R}^3$中位于$q$点的点电荷的电势是流形$M \equiv \mathbb{R}^3 - \{q\}$上的光滑函数。
\end{example}

\begin{example}
坐标系$(O, \psi)$的第$\mu$坐标$x^\mu$是定义在$O$上的光滑函数。
\end{example}

函数$f \colon M \to \mathbb{R}$与坐标系$(O, \psi)$结合可得一个$n$元函数$F(x^1, \cdots, x^n)$,因为$n$个坐标决定$O$中的唯一一点$p$,
而由$f \colon M \to \mathbb{R}$可得唯一的实数$f(p)$。然而$f$与另一坐标系$(O', \psi')$结合将给出另一$n$元函数$F'(x'^1, \cdots, x'^n)$,
函数关系$F$和$F'$不同,因为$F = f \comp \psi^{-1}$而$F' = f \comp \psi'^{-1}$。可见与函数$f \colon M \to \mathbb{R}$相应的多元函数(指函数关系)
是坐标系依赖的。应注意区分函数(标量场)$f$和它与坐标系结合而得的多元函数$F$。

设$M, N$为流形,则它们必为拓扑空间,故$M \times N$也是拓扑空间。不难利用$M, N$的流形结构把$M \times N$进一步定义为流形。
设$M, N$的维数分别为$m, n$,则$M \times N$的维数是$m + n$,即$\dim{(M \times N)} = \dim{M} + \dim{N}$。

\section{切矢和切矢场}

\subsection{切矢量}

\begin{definition}
实数域上的一个\textbf{矢量空间}是一个集合$V$配以两个映射,即$V \times V \to V$(叫\textbf{加法})及$\mathbb{R} \times V \to V$(叫\textbf{数乘}),满足如下条件:
\begin{enumerate}[(a)]
\item $v_1 + v_2 = v_2 + v_1, ~ \forall v_1, v_2 \in V$;
\item $(v_1 + v_2) + v_3 = v_1 + (v_2 + v_3), ~ \forall v_1, v_2, v_3 \in V$;
\item $\exists \underline{0} \in V, \text{使} \underline{0} + v = v, ~ \forall v \in V$;
\item $\alpha_1(\alpha_2v) = (\alpha_1\alpha_2)v, ~ \forall v \in V, ~ \alpha_1, \alpha_2 \in \mathbb{R}$;
\item $(\alpha_1 + \alpha_2)v = \alpha_1v + \alpha_2v, ~ \forall v \in V, ~ \alpha_1, \alpha_2 \in \mathbb{R}$;
\item $\alpha(v_1 + v_2) = \alpha v_1 + \alpha v_2, ~ \forall v_1, v_2 \in V, ~ \alpha \in \mathbb{R}$;
\item $1 \cdot v = v, ~ \forall v \in V$。
\end{enumerate}
\end{definition}

\begin{note}
由此$7$条可以推出:(1)$0 \cdot v = \underline{0}$,(2)$\forall v \in V, ~ \exists u \in V, \text{使} v + u = \underline{0}$。
约定把$u$记作$-v$。
\end{note}

今后也常把$V$的零元简写作$0$,即符号$0$既代表$0 \in \mathbb{R}$又代表$\underline{0} \in V$。

\begin{definition}
映射$v \colon \mathscr{F}_M \to \mathbb{R}$称为\textbf{点$p \in M$的一个矢量},若$\forall f, g \in \mathscr{F}_M, ~ \alpha, \beta \in \mathbb{R}$,有
\begin{enumerate}[(a)]
\item (线性性) $v(\alpha f + \beta g) = \alpha v(f) + \beta v(g)$;
\item (莱布尼茨律) $v(fg) = f|_p v(g) + g|_p v(f)$,其中$f|_p$代表函数$f$在$p$点的值,亦可记作$f(p)$。
\end{enumerate}
\end{definition}

\begin{note}
因$f$和$g$是$M$上的函数,故$fg$也是$M$上的函数,它在$M$的任一点$p$的值定义为$f(p)$与$g(p)$之积。
\end{note}

根据定义,要定义$p$点的一个矢量只须指定一个从$\mathscr{F}_M$到$\mathbb{R}$的、满足条件(a)、(b)的映射,
就是说,指定一个对应规律(法则),根据这一规律,每一$f \in \mathscr{F}_M$对应于一个确定的实数。
因为这种映射很多,所以$p$点有很多(无限多)矢量。
例如,设$(O, \psi)$是坐标系,其坐标为$x^\mu$,则$M$上任一光滑函数$f \in \mathscr{F}_M$与$(O, \psi)$结合得$n$元函数
$F(x^1, \cdots, x^n)$,借此可给$O$中任一点$p$定义$n$个矢量,记作$X_\mu$(其中$\mu = 1, \cdots, n$),
它(们)作用于任一$f \in \mathscr{F}_M$的结果$X_\mu(f)$定义为如下实数:
$$X_\mu(f) := \left.\frac{\partial F(x^1, \cdots, x^n)}{\partial x^\mu}\right|_p, ~ \forall f \in \mathscr{F}_M$$
其中$\partial F(x^1, \cdots, x^n)/\partial x^\mu|_p$是$\partial F/\partial x^\mu|_{(x^1(p), \cdots, x^n(p))}$的简写。
今后也把$\partial F(x^1, \cdots, x^n)/\partial x^\mu$简写为$\partial f(x^1, \cdots, x^n)/\partial x^\mu$或$\partial f(x)/\partial x^\mu$甚至$\partial f/\partial x^\mu$。
应认出$\partial f/\partial x^\mu$中的$f$代表$n$元函数$F(x^1, \cdots, x^n)$而非标量场$f$。于是上式可简写为
$$X_\mu(f) := \left.\frac{\partial f(x)}{\partial x^\mu}\right|_p, ~ \forall f \in \mathscr{F}_M$$

\begin{theorem}
以$V_p$代表$M$中点$p$所有矢量的集合,则$V_p$是$n$维矢量空间($n$是$M$的维数),即$\dim V_p = \dim M \equiv n$。
\end{theorem}

\begin{definition}
坐标域内任一点$p$的$\{X_1, \cdots, X_n\}$称为$V_p$的一个\textbf{坐标基底},
每个$X_\mu$称为一个\textbf{坐标基矢},$v \in V_p$用$\{X_\mu\}$线性表出的系数$v^{\mu}$称为$v$的\textbf{坐标分量}。
\end{definition}

\begin{theorem}
设$\{x^\mu\}$和$\{x'^\nu\}$为两个坐标系,其坐标域的交集非空,
$p$为交集中的一点,$v \in V_p$,$\{v^\mu\}$和$\{v'^\nu\}$是$v$在这两个系的坐标分量,则
$$v'^\nu = \left.\frac{\partial x'^\nu}{\partial x^\mu}\right|_pv^\mu$$
其中$x'^\nu$是两系间坐标变换函数$x'^\nu(x^\mu)$的简写。
\end{theorem}

\begin{proof}[证明]
根据坐标分量的定义,$v = v^\mu X_\mu = v'^\nu X'_\nu$。
因此可以通过$X_\mu$和$X'_\nu$之间的关系来找到$v^\mu$和$v'^\nu$之间的关系。
为此,设$q$是两坐标域交集内的任一点,则标量场$f$在$q$点的值$f|_q$满足$f|_q = f(x(q)) = f'(x'(q))$,简记为$f(x) = f'(x')$。
另一方面,每一$x'^\nu$又是$n$个$x^\mu$的函数(坐标变换关系),简记为$x'^\nu = x'^\nu(x)$,故$f(x) = f'(x'(x))$。于是
$$
X_\mu(f) = \left.\frac{\partial f(x)}{\partial x^\mu}\right|_p \\
= \left.\frac{\partial f'(x'(x))}{\partial x^\mu}\right|_p \\
= \left(\frac{\partial f'(x')}{\partial x'^\nu}\frac{\partial x'^\nu}{\partial x^\mu}\right)_p \\
= \left.\frac{\partial x'^\nu}{\partial x^\mu}\right|_pX'_\nu(f), ~ \forall f \in \mathscr{F}_M
$$
上式表明映射$X_\mu$和$\partial x'^\nu/\partial x^\mu|_pX'_\nu$相等,即
$$X_\mu = \left.\frac{\partial x'^\nu}{\partial x^\mu}\right|_pX'_\nu$$
所以$v = v^\mu X_\mu = v'^\nu X'_\nu$可表为
$$v^\mu\left.\frac{\partial x'^\nu}{\partial x^\mu}\right|_pX'_\nu = v'^\nu X'_\nu$$
因$\{X'_\nu\}$中的$n$个基矢彼此线性独立,故得证。
\end{proof}

上述定理称为\textbf{矢量}(的分量)\textbf{变换式},许多书籍采用此式作为矢量的定义。

\begin{definition}
设$I$为$\mathbb{R}$的一个区间,则$C^r$类映射$C \colon I \to M$称为$M$上的一条$C^r$类的\textbf{曲线}。
今后如无声明,``曲线''均指光滑($C^\infty$类)曲线。对任一$t \in I$,有唯一的点$C(t) \in M$与之对应。
$t$称为曲线的\textbf{参数}。
\end{definition}

\begin{note}
此处的曲线与直观的曲线概念有密切联系,但也有差别。
直观的曲线往往是指上述映射$C \colon I \to M$的像,即$M$的子集$C[I]$,并且不提及参数。
上述定义的曲线则是指映射本身,是``带参数的曲线''。
设映射$C \colon I \to M$和$C' \colon I' \to M$的像重合,则直观上往往认为它们是同一曲线,但只要$C$和$C'$是不同映射,根据定义,它们就是不同曲线。
不过在大多数情况下可以说$C$和$C'$是``同一曲线''的两种参数化,准确地说,曲线$C' \colon I' \to M$称为曲线$C \colon I \to M$的\textbf{重参数化},若存在到上映射$\alpha \colon I \to I'$,满足
(a)$C = C' \comp \alpha$;(b)由$\alpha$诱导的函数$t' = \alpha(t)$有处处非零的导数。
解释:由$C = C' \comp \alpha$得$$C(t) = C'(\alpha(t)) = C'(t'), ~ \forall t \in I$$
映射$\alpha$的到上性保证$C'[I'] = C[I]$,即两曲线映射有相同的像。
\end{note}

\begin{note}
\textcircled{1}曲线$C$的像也常记作$C(t)$(而不是$C[I]$),以表明曲线的参数为$t$。
应注意,若$t$为某一具体值(``死的''),则$C(t)$只代表曲线像中的一点;
只当把$t$理解为``可跑遍$I$''时(``活的''),$C(t)$才代表曲线的像。
往往也把曲线的像简称为曲线。
\textcircled{2}设$(O, \psi)$是坐标系,$C[I] \in O$,则$\psi \comp C$是从$I \subset \mathbb{R}$到$\mathbb{R}^n$的映射,相当于$n$个一元函数$x^\mu = x^\mu(t), \mu = 1, \cdots, n$。
这$n$个等式称为曲线的\textbf{参数方程}或\textbf{参数表达式}或\textbf{参数式}。一个简单的例子是$\mathbb{R}^n$中以原点为心的单位圆周,其在自然坐标系中的参数式为$x^1 = \cos t, x^2 = \sin t$。
\end{note}

\begin{definition}
设$(O, \psi)$为坐标系,$x^\mu$为坐标,则$O$的子集
$$\{p \in O \mid x^2(p) = \text{常数}, \cdots, x^n(p) = \text{常数}\}$$
可以看成以$x^1$为参数的一条曲线(的像)(改变$x^2, \cdots, x^n$的常数值则得另一曲线),叫做\textbf{$x^1$坐标线}。
\textbf{$x^\mu$坐标线}可仿此定义。
\end{definition}

\begin{example}
在$2$维欧氏空间中,笛卡儿系$\{x, y\}$的$x$及$y$坐标线是互相正交的两组平行直线,
极坐标系$\{r, \varphi\}$的$\varphi$坐标线是以原点为心的无数同心圆,$r$坐标线是从原点出发的无数半直线。
\end{example}

直观的想法认为曲线上一点有无限多个彼此平行的切矢。但若把曲线定义为映射(``带参数的曲线''),则一条曲线的一点只有一个切矢。定义如下:

\begin{definition}
设$C(t)$是流形$M$上的$C^1$曲线,则线上$C(t_0)$点切于$C(t)$的切矢$T$是$C(t_0)$点的矢量,它对$f \in \mathscr{F}_M$的作用定义为:
$$T(f) := \left.\frac{\diff (f \comp C)}{\diff t}\right|_{t_0}, ~ \forall f \in \mathscr{F}_M$$
\end{definition}

\begin{note}
\textcircled{1}$f \colon M \to \mathbb{R}$是$M$上的函数(标量场),不是什么一元函数,
但与曲线$C \colon I \to M$的结合$f \comp C$便是以$t$为自变数的一元函数(也可记作$f(C(t))$)。
在不会混淆的情况下,$\diff (f \comp C) / \diff t$也可简写成$\diff f / \diff t$。
\textcircled{2}$C(t_0)$点切于$C(t)$的切矢$T$也常记作$\partial / \partial t|_{C(t_0)}$,于是上式也可写成
$$\left.\frac{\partial}{\partial t}\right|_{C(t_0)}(f) := \left.\frac{\diff (f \comp C)}{\diff t}\right|_{t_0} := \left.\frac{\diff f(C(t))}{\diff t}\right|_{t_0}, ~ \forall f \in \mathscr{F}_M$$
\end{note}

\begin{example}
$x^\mu$坐标线是以$x^\mu$为参数的曲线,又$p$点的坐标基矢$X_\mu$就是过$p$的$x^\mu$坐标线的切矢,故也常记作$\partial / \partial x^\mu|_p$,于是坐标基矢$X_\mu$对$f$的作用又可表为
$$\left.\frac{\partial}{\partial x^\mu}\right|_p(f) := \left.\frac{\partial f(x)}{\partial x^\mu}\right|_p, ~ \forall f \in \mathscr{F}_M$$
可见符号$\partial f / \partial x^\mu$既可理解为$\partial F(x^1, \cdots, x^n) / \partial x^\mu$,又可理解为坐标线的切矢$\partial / \partial x^\mu$对标量场$f$的作用。
\end{example}

\begin{theorem}
设曲线$C(t)$在某坐标系中的参数式为$x^\mu = x^\mu(t)$,则线上任一点的切矢$\partial / \partial t$在该坐标基底的展开式为
$$\frac{\partial}{\partial t} = \frac{\diff x^\mu(t)}{\diff t}\frac{\partial}{\partial x^\mu}$$
就是说,曲线$C(t)$的切矢$\partial / \partial t$的坐标分量是$C(t)$在该系的参数式$x^\mu(t)$对$t$的导数。
\end{theorem}

\begin{proof}[证明]
注意到$f(C(t_0)) = f(x^\mu(t_0)) \equiv f(p), ~ \forall f \in \mathscr{F}_M$,因此
$$\left.\frac{\partial}{\partial t}\right|_{C(t_0)}(f) = \left.\frac{\diff f(C(t))}{\diff t}\right|_{t_0}
= \left.\frac{\diff f(x^\mu(t))}{\diff t}\right|_{t_0} = \left.\frac{\partial f}{\partial x^\mu}\frac{\diff x^\mu(t)}{\diff t}\right|_{t_0}, ~ \forall f \in \mathscr{F}_M$$
对比等式的两边,可以发现$\partial / \partial t$和$\frac{\diff x^\mu(t)}{\diff t}\partial / \partial x^\mu$在同一点$p = C(t_0)$上对任一$f$的作用是相等的,故定理得证。
\end{proof}

\begin{definition}
非零矢量$u, v \in V_p$称为\textbf{互相平行}的,若存在$\alpha \in \mathbb{R}$使$v = \alpha u$。
\end{definition}

由定义可知曲线的切矢依赖于曲线的参数化,一条曲线$C(t)$的一点$C(t_0)$只有一个切于$C(t)$的切矢。直观上之所以认为曲线上一点有无数(互相平行的)切矢,
是因为把曲线理解为映射的像而不是映射本身(把无数个有相同像的曲线映射``简并化''为一条曲线)。
下面的定理表明,若两条曲线$C$和$C'$的像相同,则它们在任一像点的切矢互相平行。

\begin{theorem}
设曲线$C' \colon I' \to M$是$C \colon I \to M$的重参数化,则两者在任一像点的切矢$\partial / \partial t$和$\partial / \partial t'$有如下关系:
$$\frac{\partial}{\partial t} = \frac{\diff t'(t)}{\diff t}\frac{\partial}{\partial t'}$$
其中$t'(t)$是由映射$\alpha \colon I \to I'$诱导而得的一元函数,即$\alpha(t)$。
\end{theorem}

根据定义,$\forall p \in M$,若指定任一曲线$C(t)$使$p = C(t_0)$,则$V_p$中必有一元素可被视为该曲线在$C(t_0)$点的切矢。
现在问:指定$V_p$中任一元素$v$,可否找到过$p$的曲线,其在$p$点的切矢是$v$。答案是肯定的:这种曲线不但存在,而且很多。
例如,任选坐标系$\{x^\mu\}$使$p$含于其坐标域内,则以$x^\mu(t) = x^\mu|_p + v^\mu(t)$为参数式的曲线便是所需曲线,
其中$v^\mu$是$v$在该系的坐标分量。

综上所述,$V_p$中任一元素可视为过$p$的某曲线的切矢,因此$p$点的矢量亦称\textbf{切矢量},$V_p$则称为$p$点的\textbf{切空间}。

\subsection{流形上的矢量场}

\begin{definition}
设$A$为$M$的子集。若给$A$中每点指定一个矢量,就得到一个定义在$A$上的\textbf{矢量场}。
\end{definition}

\begin{example}
非自相交曲线$C(t)$上每点的切矢构成$C(t)$(看作$M$的子集)上的一个矢量场。
\end{example}

设$v$是$M$上的矢量场,$f$是$M$上的函数,则$v$在$M$的任一点$p$的值$v|_p$将按定义把$f$映射为一个实数$v|_p(f)$,
它在$p$点跑遍$M$时构成$M$上的一个函数$v(f)$。因此,矢量场$v$可视为把函数$f$变为函数$v(f)$的映射。

\begin{definition}
$M$上的矢量场$v$称为\textbf{$C^\infty$类(光滑)}的,若$v$作用于$C^\infty$类函数的结果仍为$C^\infty$类函数,
即$v(f) \in \mathscr{F}_M, ~ \forall f \in \mathscr{F}_M$。$v$称为\textbf{$C^r$类}的,若$v$作用于$C^\infty$类函数得$C^r$类函数。
\end{definition}

今后如无声明,``矢量场''均指光滑($C^\infty$)矢量场。

\begin{example}
\begin{enumerate}[(1)]
\item 坐标基矢$\{X_\mu \equiv \partial / \partial x^\mu\}$构成坐标域上的$n$个光滑矢量场,叫\textbf{坐标基矢场}。
\item $\mathbb{R}^3$中位于$q$的点电荷的静电场强$\vec{E}$是流形$M \equiv \mathbb{R}^3 - {q}$上的光滑矢量场。
\end{enumerate}
\end{example}

\begin{theorem}
$M$上的矢量场$v$是$C^\infty$(或$C^r$)类的充要条件是它在任一坐标基底的分量$v^\mu$为$C^\infty$(或$C^r$)类函数。
\end{theorem}

设$v$为$M$上的光滑矢量场,则$v(f) \in \mathscr{F}_M, ~ \forall f \in \mathscr{F}_M$。
若$u$为$M$上另一光滑矢量场,则$u(v(f)) \in \mathscr{F}_M$。但函数$u(v(f))$未必等于$v(u(f))$,于是有如下定义

\begin{definition}
两个光滑矢量场$u$和$v$的\textbf{对易子}是一个光滑矢量场$[u, v]$,定义为
$$[u, v](f) := u(v(f)) - v(u(f)), ~ \forall f \in \mathscr{F}_M$$
\end{definition}

\begin{note}
上式是对易子$[u, v]$(作为矢量场)的定义式,它在每点$p \in M$的值$[u, v]|_p$(作为$p$点的矢量,即从$\mathscr{F}_M$到$\mathbb{R}$的映射)的定义应理解为
$$[u, v]|_p(f) := u|_p(v(f)) - v|_p(u(f)), ~ \forall f \in \mathscr{F}_M$$
要确信上式定义的$[u, v]|_p$是$p$点的矢量,还应证明它满足矢量定义的两个条件。
\end{note}

\begin{theorem}
设$\{x^\mu\}$为任一坐标系,则$[\partial / \partial x^\mu, \partial / \partial x^\nu] = 0, ~ \mu,\nu = 1, \cdots, n$。
\end{theorem}

上述定理表明任一坐标系的任意两个基矢场都互相对易。

\begin{definition}
曲线$C(t)$叫矢量场$v$的\textbf{积分曲线},若其上每点的切矢等于该点的$v$值。
\end{definition}

\begin{theorem}
设$v$是$M$上的光滑矢量场,则$M$的任一点$p$必有$v$的唯一的积分曲线$C(t)$经过(满足$C(0) = p$)。
\end{theorem}

下面要用到群论的初步知识,故补充如下定义:

\begin{definition}
一个\textbf{群}是一个集合$G$配以满足以下条件的映射$G \times G \to G$(叫\textbf{群乘法},元素$g_1$和$g_2$的乘积记作$g_1g_2$):
\begin{enumerate}[(a)]
\item $(g_1g_2)g_3 = g_1(g_2g_3), ~ \forall g_1, g_2, g_3 \in G$;
\item 存在\textbf{恒等元}$e \in G$使$eg = ge = g, ~ \forall g \in G$;
\item $\forall g \in G$,存在\textbf{逆元}$g^{-1} \in G$使$g^{-1}g = gg^{-1} = e$。
\end{enumerate}
\end{definition}

对称性在物理学中有重要意义,群论是研究对称性的有力工具。如果某一对象在某一变换下不变,就说它有该变换下的对称性。
考虑带电面上的一个动点,它沿$x$(或$y$)轴平移。由于动点的电荷面密度$\sigma$在平移时不变,我们说$\sigma$具有沿$x$(或$y$)轴的平移对称性。
更明确地说,$\sigma$沿$x$轴的平移对称性是指函数$\sigma(x, y, z)$满足
$$\sigma(x, y, z) = \sigma(x + a, y, z), ~ \forall a \in \mathbb{R},$$
其中$x \mapsto x + a, y \mapsto y, z \mapsto z$所代表的平移变换称为沿$x$轴的一个\textbf{平移}。
设$G$是沿$x$轴的所有平移的集合,则$G$中的元素由实数$a$表征,记作$\phi_a \in G$。
把$p \equiv (x, y, z)$和$q \equiv (x + a, y, z)$看做$\mathbb{R}^3$的点,则上述变换式相当于映射$\phi_a \colon \mathbb{R}^3 \to \mathbb{R}^3$(满足$\phi_a(p) = q$),而且是微分同胚映射。
再者,对$G$用下式定义群乘法
$$\phi_a\phi_b := \phi_{a + b}, ~ \forall \phi_a, \phi_b \in G$$
则$G$构成群($\phi_0$是恒等元,$\phi_{-a}$是$\phi_a$的逆元)。
这个群的无限多个元素可用实数$a$表征,因此称$a$为\textbf{参数},称$G$为\textbf{单参数群}。
又因每一群元素$\phi_a \in G$都是$\mathbb{R}^3$上的一个微分同胚,故又称$G$为$\mathbb{R}^3$上的\textbf{单参微分同胚群}。

设$M$是流形,则$\mathbb{R} \times M$是比$M$高一维的流形。
设有映射$\phi \colon \mathbb{R} \times M \to M$,则它能把一个实数$t \in \mathbb{R}$和一个点$p \in M$变为一个点$\phi(t, p) \in M$。
若给定$t$,记$\phi_t \equiv \phi(t, \bullet)$,则有映射$\phi_t \colon M \to M$。
同理,若给定$p$,记$\phi_p \equiv \phi(\bullet, p)$,则有映射$\phi_p \colon \mathbb{R} \to M$。

\begin{definition}
$C^\infty$映射$\phi \colon \mathbb{R} \times M \to M$称为$M$上的一个\textbf{单参微分同胚群},若
\begin{enumerate}[(a)]
\item $\phi_t \colon M \to M$是微分同胚$\forall t \in \mathbb{R}$;
\item $\phi_t \comp \phi_s = \phi_{t + s}, ~ \forall t, s \in \mathbb{R}$。
\end{enumerate}
\end{definition}

\begin{note}
集合$\{\phi_t \mid t \in \mathbb{R}\}$是以复合映射为乘法的群,各群元$\phi_t$是从$M$到$M$的微分同胚映射,$\phi_0$为恒等元(由定义知$\phi_t \comp \phi_0 = \phi_t$,故$\phi_0$是恒等映射)。
所谓$\phi \colon \mathbb{R} \times M \to M$是$M$上的一个单参微分同胚群,其实是指集合$\{\phi_t \mid t \in \mathbb{R}\}$是一个单参微分同胚群。
\end{note}

设$\phi \colon \mathbb{R} \times M \to M$是单参微分同胚群,则$\forall p \in M$,$\phi_p \colon \mathbb{R} \to M$是过$p$点的一条光滑曲线(满足$\phi_p(0) = p$),叫做这个单参微分同胚群过$p$点的\textbf{轨道}。
把这条曲线在点$\phi_p(0)$的切矢记作$v|_p$,便得$M$上的一个光滑矢量场$v$。可见$M$上的一个单参微分同胚群给出$M$上的一个光滑矢量场。
再看逆命题是否成立。设$v$是$M$上的光滑矢量场,看来$\forall t \in \mathbb{R}$可以借用其积分曲线定义从$M$到$M$的微分同胚映射$\phi_t$($\forall p \in M$,定义$\phi_t(p)$为这样一个点,它位于过$p$的积分曲线上,其参数值与$p$的参数值之差为$t$)。于是似乎可以得到一个单参微分同胚群。
然而可能出现如下问题:某条积分曲线当参数取某些值时像点不存在(人为挖去$M$的某一区域就可造出这种情况),因此只能说$M$上的一个光滑矢量场给出一个\textbf{单参微分同胚局部群}。

\section{对偶矢量场}

\begin{definition}
设$V$是$\mathbb{R}$上的有限维矢量空间。线性映射$\omega \colon V \to \mathbb{R}$称为$V$上的\textbf{对偶矢量}。
$V$上全体对偶矢量的集合称为$V$的\textbf{对偶空间},记作$V^*$。
\end{definition}

\begin{note}
由于$V$上有加法和数乘,对映射$\omega$的线性要求有确切含义,即
$$\omega(\alpha v + \beta u) = \alpha\omega(v) + \beta\omega(u), ~ \forall v, u \in V, ~ \alpha, \beta \in \mathbb{R}$$
\end{note}

\begin{example}
设$V$为全体$2 \times 1$实矩阵的集合,则它在矩阵加法和数乘规则下构成$2$维矢量空间。
以$\omega$代表任一$1 \times 2$实矩阵$(c, d)$,其对$V$的任一元素
$v = \begin{pmatrix}a \\ b\end{pmatrix}$
的作用可用矩阵乘法定义:$\omega(v) := (c, d)\begin{pmatrix}a \\ b\end{pmatrix} = (ac + bd)$,
结果是一个$1 \times 1$矩阵,可认同为一个实数$ac + bd$。这样定义的映射$\omega \colon V \to \mathbb{R}$
显然是线性的,可见任一$1 \times 2$实矩阵都是$V$上的对偶矢量。推广可知:若把列矩阵($n \times 1$矩阵)看做矢量,
则行矩阵($1 \times n$矩阵)就是对偶矢量。
\end{example}

\begin{theorem}
$V^*$是矢量空间,且$\dim V^* = \dim V$。
\end{theorem}

\begin{proof}
易证$V^*$是矢量空间,只需定义合适的加法、数乘和零元即可。

设$\{e_\mu\}$是$V$的一组基矢,今定义$e^{\mu*}$为$e^{\mu*}(e_\nu) := \delta^\mu{}_\nu, ~ \mu, \nu = 1, \cdots, n$。
下面证明$\{e^{\mu*}\}$是$V^*$的一组基矢。

易证$e^{1*}, \cdots, e^{n*}$彼此线性独立。

$\forall \omega \in V^*$,令$\omega_\mu \equiv \omega(e_\mu), ~ \mu = 1, \cdots, n$,则易证$\omega = \omega_\mu e^{\mu*}$。

因此$V^*$中任一元素可用$\{e^{\mu*}\}$线性表出,故$\{e^{\mu*}\}$是$V^*$的一个基底,叫做$\{e_\mu\}$的\textbf{对偶基底}。于是可知$\dim V^* = \dim V$。
\end{proof}

两个矢量空间叫\textbf{同构}的,若两者之间存在一一到上的线性映射(这种映射称为\textbf{同构映射})。两矢量空间同构的充要条件是维数相同。

由于$\dim V^* = \dim V$,$V^*$与$V$当然同构。同构映射不难找到。
例如,设$\{e_\mu\}$是$V$的一个基底,$\{e^{\mu*}\}$是其对偶基底,则由$e_\mu \mapsto e^{\mu*}$定义的线性映射就是一个同构映射。
但$\{e_\mu\}$的选择十分任意,而基底改变后按上述方式定义的同构映射随之而变,故$V^*$与$V$之间不存在一个特殊的(与众不同的)同构映射,除非在$V$上另加结构。

$V^*$既然是矢量空间,自然也有对偶空间,记作$V^{**}$。有别于$V$与$V^*$的关系,$V$与$V^{**}$间存在一个自然的、与众不同的同构映射,定义如下:
$\forall v \in V$,欲给它自然地定义一个像$v^{**} \in V^{**}$。因为$V^{**}$是$V^*$的对偶空间,$v^{**}$应该是从$V^*$到$\mathbb{R}$的线性映射。
对它下定义无非是给出一个规律,按照这一规律,每一$\omega \in V^*$对应于唯一的实数$v^{**}(\omega)$。
因为$v^{**}$有待定义为$v$的像,所以$v^{**}(\omega)$与$v$和$\omega$都应有关,而由$v$和$\omega$构造的最简单的实数就是$\omega(v)$,自然把$v^{**}$定义为
$$v^{**}(\omega) := \omega(v), ~ \forall \omega \in V^*$$
这一映射$V \to V^{**}$是同构映射。这一自然同构关系表明$V$和$V^{**}$可视为同一空间(把每一$v \in V$与其像$v^{**} \in V^{**}$认同)。
所以,真正有用的是$V$和$V^*$,再取对偶(不论多少次)也得不到更多有用的空间。

\begin{theorem}
若矢量空间$V$中有一基底变换$e'_\mu = A^\nu{}_\mu e_\nu$($A^\nu{}_\mu$无非是新基矢$e'_\mu$用原基底展开的第$\nu$分量),
以$A^\nu{}_\mu$为元素排成的(非退化)方阵记作$A$,则相应的对偶基底变换为
$$e'^{\mu*} = (\tilde{A}^{-1})_\nu{}^\mu e^{\nu*}$$
其中$\tilde{A}$是$A$的转置矩阵,$\tilde{A}^{-1}$是$\tilde{A}$之逆。
\end{theorem}

\begin{proof}
只须证明等式两边作用于$e'_\alpha$所得结果相同,证明如下
$$\begin{aligned}
(\tilde{A}^{-1})_\nu{}^\mu e^{\nu*}(e'_\alpha) & = (\tilde{A}^{-1})_\nu{}^\mu e^{\nu*}(A^\beta{}_\alpha e_\beta) \\
& = (\tilde{A}^{-1})_\nu{}^\mu A^\beta{}_\alpha e^{\nu*}(e_\beta) \\
& = (A^{-1})^\mu{}_\nu A^\beta{}_\alpha \delta^{\nu}{}_\beta \\
& = (A^{-1})^\mu{}_\nu A^\nu{}_\alpha \\
& = \delta^\mu{}_\alpha \\
& = e'^{\mu*}(e'_\alpha)
\end{aligned}$$
\end{proof}

以上属代数范畴,下面回到流形$M$。因$p \in M$有矢量空间$V_p$,故也有${V_p}^*$。
若在$M$(或$A \in M$)上每点指定一个对偶矢量,就得到$M$(或$A$)上的一个\textbf{对偶矢量场}。
$M$上的对偶矢量场$\omega$叫做\textbf{光滑}的,若对所有光滑矢量场$v$有$\omega(v) \in \mathscr{F}_M$。

设$f \in \mathscr{F}_M$,我们来说明$f$自然诱导出$M$上的一个对偶矢量场,记作$\diff f$。
要定义$\diff f$只须说明它在$M$的任一点$p$的值$\diff f|_p \in {V_p}^*$的定义,
而要定义$\diff f|_p$只须给出它对$p$点任一矢量$v \in V$作用所得的实数,这个实数应与$f$和$v$都有关,
而由$f$和$v$能构造的最自然(最简单)的实数便是$v(f)$,因此定义$\diff f|_p$为
$$\diff f|_p(v) := v(f), ~ \forall v \in V_p$$
由此易证
$$\diff (fg)|_p = f|_p(\diff g)|_p + g|_p(\diff f)|_p$$
这正是微分算符$\diff$所满足的莱布尼茨律。

设$(O, \psi)$是一坐标系,则第$\mu$个坐标$x^\mu$可以看作$O$上的函数,
于是$\diff x^\mu$(看作特殊的$\diff f$)是定义在$O$上的对偶矢量场。
设$p \in O$,$\partial / \partial x^\nu$是$V_p$的第$\nu$个坐标基矢,
则在$p$点有$$\diff x^\mu\left(\frac{\partial}{\partial x^\nu}\right) = \frac{\partial}{\partial x^\nu}(x^\mu) = \delta^\mu{}_\nu$$
由此可见$\{\diff x^\mu\}$正是与坐标基底$\{\partial / \partial x^\nu\}$对应的\textbf{对偶坐标基底}。
上式对$O$的任一点成立,因此,同$\partial / \partial x^\nu$是$O$上的第$\nu$个坐标基矢场类似,
$\diff x^\mu$是$O$上的第$\mu$个对偶坐标基矢场,$\{\diff x^\mu\}$则是$O$上的第$\mu$个对偶坐标基底场。
$O$上任一对偶矢量场$\omega$可借$\{\diff x^\mu\}$展开:$$\omega = \omega_\mu\diff x^\mu$$
其中$\omega_\mu$称为$\omega$在该系的坐标分量,其表达式为:$$\omega_\mu = \omega(\partial / \partial x^\mu)$$

\begin{theorem}
设$(O, \psi)$是一坐标系,$f$是$O$上的光滑函数,$f(x)$是$f \comp \psi^{-1}$对应的$n$元函数$f(x^1, \cdots, x^n)$的简写,
则$\diff f$可用对偶坐标基底$\{\diff x^\mu\}$展开如下:
$$\diff f = \frac{\partial f(x)}{\partial x^\mu}\diff x^\mu, ~ \forall f \in \mathscr{F}_O$$
\end{theorem}

\begin{proof}
只须验证两边作用于任一坐标基矢$\partial / \partial x^\nu$得相同结果,甚易。
\end{proof}

\begin{theorem}
设坐标系$\{x^\mu\}$和$\{x'^\nu\}$的坐标域有交,则交域中任一点$p$的对偶矢量$\omega$在两坐标系中的分量$\omega_\mu$和$\omega'_\nu$的变换关系为
$$\omega'_\nu = \left.\frac{\partial x^\mu}{\partial x'^\nu}\right|_p\omega_\mu$$
\end{theorem}

\section{张量场}

\begin{definition}
矢量空间$V$上的一个\textbf{$(k, l)$型张量}是一个多重线性映射
$$T \colon \underbrace{V^* \times \cdots \times V^*}_\text{$k$个} \times \underbrace{V \times \cdots \times V}_\text{$l$个} \to \mathbb{R}$$
\end{definition}

\begin{note}
$T$可比喻为一部机器,有$k$个``上槽''和$l$个``下槽'',只要在上、下槽分别输入$k$个对偶矢量和$l$个矢量,便生产出一个实数,且此实数对每个输入量都线性依赖(多重线性映射的含义)。
\end{note}

\begin{example}
\begin{enumerate}[(1)]
\item $V$上的对偶矢量是$V$上的$(0, 1)$型张量。
\item $V$的元素$v$可看作$V$上的$(1, 0)$型张量(因$v$可被认同为$v^{**}$,而$v^{**}$是从$V^*$到$\mathbb{R}$的线性映射)。
\end{enumerate}
\end{example}

今后用$\mathscr{T}_V(k, l)$表示$V$上全体$(k, l)$型张量的集合,于是$V = \mathscr{T}_V(1, 0)$,$V^* = \mathscr{T}_V(0, 1)$。

设$T \in \mathscr{T}_V(1, 1)$,则$T \colon V^* \times V \to \mathbb{R}$。但$T$也可看成另一映射。
因为$\forall \omega \in V^*, v \in V$有$T(\omega; v) \in \mathbb{R}$,
所以$T(\omega; \bullet)$是一部只有下槽的机器,能把一个矢量线性地变为实数,这表明$T(\omega; \bullet)$是$V$上的对偶矢量,即$T(\omega; \bullet) \in V^*$。
给定$T$后,再给一个$\omega \in V^*$便能造出$T(\omega; \bullet) \in V^*$,故$T$也可看作把对偶矢量$\omega$变为对偶矢量$T(\omega; \bullet)$的映射(而且是线性映射),
即$T \colon V^* \xrightarrow[\text{线性地}]{} V^*$。类似地还可以把$T$看成$T \colon V \xrightarrow[\text{线性地}]{} V$。
对同一$T \in \mathscr{T}_V(1, 1)$的这$3$种看法是等价的。为便于陈述,我们称这种把同一张量看成不同映射的做法为``张量面面观''。
能够用``面面观''想问题是用映射定义张量的重要好处之一。今后将常用到。

\begin{definition}
$V$上的$(k, l)$和$(k', l')$型张量$T$和$T'$的\textbf{张量积}$T \otimes T'$是一个$(k + k', l + l')$型张量,定义如下:
$$\begin{aligned}
& T \otimes T'(\omega^1, \cdots, \omega^k, \omega^{k + 1}, \cdots, \omega^{k + k'}; v_1, \cdots, v_l, v_{l + 1}, \cdots, v_{l + l'}) \\
:= & T(\omega^1, \cdots, \omega^k; v_1, \cdots, v_l)T'(\omega^{k + 1}, \cdots, \omega^{k + k'}; v_{l + 1}, \cdots, v_{l + l'})
\end{aligned}$$
\end{definition}

欧氏空间矢量场论中的并矢$\vec{v}\vec{u}$其实就是矢量$\vec{v}$和$\vec{u}$的张量积,只不过略去$\otimes$号。\footnote{
类似地,量子力学的$\ket{\psi}\ket{\phi}$也是$\ket{\psi}$和$\ket{\phi}$的张量积,只不过略去$\otimes$号。
但量子力学中$\ket{\psi}$所在的矢量空间是复数域上的无限维矢量空间,比现在讨论的实数域上的有限维矢量空间复杂。
}

张量积是否满足交换律?设$\omega \in V^*, v \in V \equiv V^{**}$,则$v \otimes \omega \in \mathscr{T}_V(1, 1), \omega \otimes v \in \mathscr{T}_V(1, 1)$。
由定义知$\forall \mu \in V^*, u \in V$有$v \otimes \omega(\mu; u) = v(\mu)\omega(u) = \omega(u)v(\mu) = \omega \otimes v(\mu; u)$(其中$v(\mu)$应理解为$v^{**}(\mu)$),
故$v \otimes \omega = \omega \otimes v$。但两个矢量(或两个对偶矢量)的张量积在交换顺序后一般成为另一张量,即$v \otimes u \neq u \otimes v, \omega \otimes \mu \neq \mu \otimes \omega$。
例如,欧氏空间的并矢就不满足交换律。

\begin{theorem}
$\mathscr{T}_V(k, l)$是矢量空间,$\dim\mathscr{T}_V(k, l) = n^{k + l}$。
\end{theorem}

\begin{proof}
\begin{enumerate}[(A)]
\item 用自然的方法定义加法、数乘和零元使$\mathscr{T}_V(k, l)$成为矢量空间。
\item 证明其基矢共$n^{k + l}$个。以$n = 2, k = 2, l = 1$为例(不难推广至一般情况)。
设$\{e_1, e_2\}$为$V$的一个基底,$\{e^{1*}, e^{2*}\}$为其对偶基底。
只须证明以下$8$个元素构成$\mathscr{T}_V(2, 1)$的一个基底:
$$\begin{aligned}
e_1 \otimes e_1 \otimes e^{1*}, ~ e_1 \otimes e_1 \otimes e^{2*}, ~ e_1 \otimes e_2 \otimes e^{1*}, ~ e_1 \otimes e_2 \otimes e^{2*}, \\
e_2 \otimes e_1 \otimes e^{1*}, ~ e_2 \otimes e_1 \otimes e^{2*}, ~ e_2 \otimes e_2 \otimes e^{1*}, ~ e_2 \otimes e_2 \otimes e^{2*}.
\end{aligned}$$
先证它们线性独立,再证任意$T \in \mathscr{T}_V(2, 1)$可表为:
$$T = T^{\mu\nu}{}_\sigma e_\mu \otimes e_\nu \otimes e^{\sigma*},$$
其中
$$T^{\mu\nu}{}_\sigma = T(e^{\mu*}, e^{\nu*}; e_\sigma).$$
\end{enumerate}
\end{proof}

\begin{note}
$T^{\mu\nu}{}_\sigma$是张量$T$在基底$\{e_\mu \otimes e_\nu \otimes e^{\sigma*}\}$的分量,简称为$T$在基底$\{e_\mu\}$的分量。
\end{note}

下面介绍张量的另一重要运算,即缩并。如前所述,$(1, 1)$型张量$T$可看作从$V$到$V$的线性映射,其实它就是线性代数所讲的线性变换。
$T$在任一基底${e_\mu \otimes e^{\nu*}}$的分量排成的矩阵($T^\mu{}_\nu$)显然与基底有关,不难证明同一$T$在任意两个基底的分量对应的两个矩阵($T^\mu{}_\nu$)和($T'^\mu{}_\nu$)互为相似矩阵,证明如下:
$$\begin{aligned}
T'^\mu{}_\nu & = T(e'^{\mu*}; e'_\nu) = T((\tilde{A}^{-1})_\rho{}^\mu e^{\rho*};A^\sigma{}_\nu e_\sigma) = (\tilde{A}^{-1})_\rho{}^\mu A^\sigma{}_\nu T(e^{\rho*}; e_\sigma) \\
& = (\tilde{A}^{-1})_\rho{}^\mu A^\sigma{}_\nu T^\rho{}_\sigma = (A^{-1})^\mu{}_\rho T^\rho{}_\sigma A^\sigma{}_\nu = (A^{-1}TA)^\mu{}_\nu
\end{aligned}$$
于是有矩阵等式$T' = A^{-1}TA$(其中$T'$,$A$,$T$都代表矩阵。$T$有时代表张量,有时代表矩阵,读者应能根据上下文识别。)。
可见$T'$与$T$互为相似矩阵。以$T'^\mu{}_\mu$(是$\sum_{\mu = 1}^n T'^\mu{}_\mu$的简写)和$T^\rho{}_\rho$分别代表矩阵$T'$和$T$的迹,则由上式易得:
$$T'^\mu{}_\mu = (A^{-1})^\mu{}_\rho T^\rho{}_\sigma A^\sigma{}_\mu = A^\sigma{}_\mu(A^{-1})^\mu{}_\rho T^\rho{}_\sigma = \delta^\sigma{}_\rho T^\rho{}_\sigma = T^\rho{}_\rho$$
这就证明了同一$(1, 1)$型张量在不同基底的矩阵有相同的迹。在关心张量时,应该抓住其与基底无关的性质,$(1, 1)$型张量$T$的迹$T^\mu{}_\mu$就是这样一种性质,通常把它称为$T$的\textbf{缩并}或\textbf{收缩},暂记作$\contraction T$,即
$$\contraction T := T^\mu{}_\mu = T(e^{\mu*}; e_\mu)$$

再讨论$(2, 1)$型张量$T$的缩并。$T$可记作$T(\bullet, \bullet; \bullet)$,它有两个上槽和一个下槽,故有两种可能的缩并:
\textcircled{1}第一上槽与下槽的缩并$\contraction^1_1T := T(e^{\mu*}, \bullet; e_\mu)$;
\textcircled{2}第二上槽与下槽的缩并$\contraction^2_1T := T(\bullet, e^{\mu*}; e_\mu)$。
若改用另一基底${e'_\rho}$定义这两种缩并,分别记作$(\contraction^1_1T)'$和$(\contraction^2_1T)'$,则易证$\contraction^1_1T = (\contraction^1_1T)'$,$\contraction^2_1T = (\contraction^2_1T)'$。
由``张量面面观''可知$\contraction^1_1T$和$\contraction^2_1T$都是$(1, 0)$型张量,它们在任一基底的分量可用$T$在该基底的分量表为$(\contraction^1_1T)^\nu = T(e^{\mu*}, e^{\nu*}; e_\mu) = T^{\mu\nu}{}_\mu$,$(\contraction^2_1T)^\nu = T^{\nu\mu}{}_\mu$(已略去求和号)。
不难推广上述讨论而得出$(k, l)$型张量的缩并定义如下:

\begin{definition}
$T \in \mathscr{T}_V(k, l)$的第$i$上标($i \leq k$)与第$j$下标($j \leq l$)的\textbf{缩并}定义为:
$$\begin{aligned}
\contraction^i_jT := T(\bullet, \cdots, & e^{\mu*}, \bullet, \cdots; & \bullet, \cdots, e&_\mu, \bullet, \cdots) \in \mathscr{T}_V(k - 1, l - 1) ~ \text{(要对$\mu$求和)} \\
& \uparrow & \uparrow & \\
& \text{第$i$上槽} & \text{第$j$下槽} &
\end{aligned}$$
\end{definition}

\begin{note}
\textcircled{1}$\contraction^i_jT$与基底选择无关。
\textcircled{2}由上式易见$(k, l)$型张量的每一缩并都是一个$(k - 1, l - 1)$型张量。
\textcircled{3}联合使用张量积和缩并运算可从原有张量得到各种类型的新张量。
例如,设$v \in V$,$\omega \in V^*$,则$v \otimes \omega$是$(1, 1)$型张量,而$\contraction(v \otimes \omega)$则是$(0, 0)$型张量(标量)。
\end{note}

后面经常遇到先求张量积再做缩并的运算,其结果可看做张量对矢量(或对偶矢量)的作用。
作为例子,请考虑如下$3$个等式。

\begin{enumerate}[(a)]
\item $\contraction(v \otimes \omega) = \omega_\mu v^\mu = \omega(v) = v(\omega), ~ \forall v \in V, \omega \in V^*$。
\item $\contraction^1_2(T \otimes v) = T(\bullet, v), ~ \forall v \in V, T \in \mathscr{T}_V(0, 2)$。
\item $\contraction^2_2(T \otimes \omega) = T(\bullet, \omega; \bullet), ~ \forall \omega \in V^*, T \in \mathscr{T}_V(2, 1)$。
\end{enumerate}

我们只给出(b)的证明。待证等式左边的$T \otimes v$是$(1, 2)$型张量,是一部有$1$个上槽、$2$个下槽的机器,可表为$T \otimes v(\bullet; \bullet, \bullet)$,故
$$\contraction^1_2(T \otimes v) = T \otimes v(e^{\mu*}; \bullet, e_\mu)$$
所以只须证明下式
$$T \otimes v(e^{\mu*}; \bullet, e_\mu) = T(\bullet, v)$$
而此式是对偶矢量的等式,只须证明两边作用于任一$(u \in V)$给出相同实数。
$$T \otimes v(e^{\mu*}; u, e_\mu) = T(u, e_\mu)v(e^{\mu*}) = T(u, e_\mu)v^\mu = T(u, v)$$

除以上三式外还有许多类似等式。这些等式是如下规律的表现:``$T$对$\omega$(或$v$)的作用就是先求$T$与$\omega$(或$v$)的张量积再缩并'',或者粗略地说,``作用就是先积后并''。
对两个张量先求张量积再缩并的操作常又简称为对它们做缩并,因此上述粗略提法还可简化为``作用就是缩并''。

下面回到流形$M$。$M$中任一点$p$的切空间$V_p$的全体$(k, l)$型张量的集合自然记作$\mathscr{T}_{V_p}(k, l)$。
设$\{e_\mu\}$和$\{e^{\mu*}\}$是$V_p$的任一基底及对偶基底,则$T$同样可写成展开式。
若选坐标系$\{x^\mu\}$使坐标域含$p$,则可用坐标基矢$\partial / \partial x^\mu$和对偶坐标基矢$\diff x^\mu$充当$e_\mu$和$e^{\mu*}$,即
$$T = T^{\mu\nu}{}_\sigma\frac{\partial}{\partial x^\mu}\otimes\frac{\partial}{\partial x^\nu}\otimes\diff x^\sigma$$
其中坐标分量$T^{\mu\nu}{}_\sigma$可表为
$$T^{\mu\nu}{}_\sigma = T(\diff x^\mu, \diff x^\nu; \partial / \partial x^\sigma)$$

若在流形$M$(或$A \subset M$)上每点指定一个$(k, l)$型张量,就得到$M$(或$A$)上的一个$(k, l)$型\textbf{张量场}。
$M$上张量场$T$称为\textbf{光滑的},若对所有光滑对偶矢量场$\omega^1, \cdots, \omega^k$及光滑矢量场$v_1, \cdots, v_l$有$T(\omega^1, \cdots, \omega^k; v_1, \cdots, v_l) \in \mathscr{F}_M$。
今后如无声明,``张量场''均指光滑($C^\infty$)张量场。

\begin{theorem}
$(k, l)$型张量在两个坐标系中的分量的变换关系为(简称\textbf{张量变换律})
$$T'^{\mu_1\cdots\mu_k}{}_{\nu_1\cdots\nu_l} = \frac{\partial x'^{\mu_1}}{\partial x^{\rho_1}}\cdots\frac{\partial x'^{\mu_k}}{\partial x^{\rho_k}}\frac{\partial x^{\sigma_1}}{\partial x'^{\nu_1}}\cdots\frac{\partial x^{\sigma_l}}{\partial x'^{\nu_l}}T^{\rho_1\cdots\rho_k}{}_{\sigma_1\cdots\sigma_l}$$
\end{theorem}

\begin{note}
许多教科书采用上式作为张量定义。
\end{note}

\section{度规张量场}

\begin{definition}
矢量空间$V$上的一个\textbf{度规}$g$是$V$上的一个对称、非退化的$(0, 2)$型张量。
对称是指$g(v, u) = g(u, v) ~ \forall v, u \in V$,非退化是指$g(v, u) = 0 ~ \forall u \in V \Rightarrow v = 0 \in V$。
\end{definition}

\begin{note}
这一抽象的非退化性定义与矩阵的非退化性(行列式非零)有密切联系。
可以证明,若$g$非退化,则它在$V$的任一基底$\{e_\mu\}$的分量$g_{\mu\nu} \equiv g(e_\mu, e_\nu)$排成的矩阵也非退化。
反之,若$V$有基底使$g$的分量矩阵非退化,则$g$非退化。
\end{note}

度规很像内积。但上述度规$g$与一般内积的区别在于$g(v, v)$可以为负,且$g(v, v) = 0$不意味着$v = 0$。
今后也常把$g(v, u)$称为$v$和$u$在度规$g$下的内积。矢量空间$V$一旦定义了度规$g$,其元素的长度及元素间的正交性就可以定义如下:

\begin{definition}
$v \in V$的\textbf{长度}或\textbf{大小}定义为$|v| := \sqrt{|g(v, v)|}$。矢量$v, u \in V$叫\textbf{相互正交}的,若$g(v, u) = 0$。
$V$的基底$\{e_\mu\}$叫\textbf{正交归一}的,若任二基矢正交且每一基矢$e_\mu$满足$g(e_\mu, e_\mu) = \pm 1$(不对$\mu$求和)。
\end{definition}

\begin{note}
上述定义表明度规$g$在正交归一基底的分量满足
$$
g_{\mu\nu} =
\begin{cases}
0, & \mu \neq \nu \\
\pm 1, & \mu = \nu
\end{cases}
$$
因此,度规在正交归一基底的分量排成的矩阵是对角矩阵,且对角元为$+1$或$-1$。
\end{note}

\begin{theorem}
任何带度规的矢量空间都有正交归一基底。
度规写成对角矩阵时对角元中$+1$和$-1$的个数与所选正交归一基底无关。
\end{theorem}

\begin{definition}
用正交归一基底写成对角矩阵后,对角元全为$+1$的度规叫\textbf{正定}的或\textbf{黎曼}的,
对角元全为$-1$的度规叫\textbf{负定}的,其他度规叫\textbf{不定}的,只有一个对角元为$-1$的不定度规叫\textbf{洛伦兹}的。
对角元之和叫度规的\textbf{号差}。相对论中用得最多的是洛伦兹度规和正定度规。
\end{definition}

\begin{definition}
带洛伦兹度规$g$的矢量空间$V$的元素可分为三类:
\textcircled{1}满足$g(v, v) > 0$的$v$称为\textbf{类空矢量};
\textcircled{2}满足$g(v, v) < 0$的$v$称为\textbf{类时矢量};
\textcircled{3}满足$g(v, v) = 0$的$v$称为\textbf{类光矢量}。
\end{definition}

\begin{note}
若度规是洛伦兹的,则$g(v, v) = 0$未必导致$v = \underline{0}$。
非零的$4$维类光矢量在相对论中有重要地位,例如便于描写电磁波及引力波在$4$维时空中的传播。
\end{note}

度规$g$是$(0, 2)$型张量,即由$V \times V$到$\mathbb{R}$的双重线性映射,所以$\forall v, u \in V$有$g(v, u) \in \mathbb{R}$,因而$g(v, \bullet) \in V^*$。
给定$g$后,再给一个$v \in V$便可造出$g(v, \bullet) \in V^*$,故$g$可看作由$V$到$V^*$的线性映射,即$g \colon V \xrightarrow[\text{线性地}]{} V^*$,这是一个同构映射。
因此,在$V$选定度规后就有了一个自然的、与众不同的从$V$到$V^*$的同构映射,我们很自然地用这一映射把$V$与$V^*$认同。
小结:无论有无度规,$V$都与$V^{**}$自然认同;如果有度规,则$V$与$V^*$也自然认同。

下面回到流形$M$上来。

\begin{definition}
$M$上的对称的、处处非退化的$(0, 2)$型张量场称为\textbf{度规张量场}。
\end{definition}

\begin{note}
这里我们只关心号差处处一样的度规场。
\end{note}

度规场的一大用处就是定义曲线长度。先讨论$2$维欧氏空间。
设曲线$C(t)$在自然坐标系$\{x, y\}$的参数式为$x = x(t), y = y(t)$,则曲线元段线长的平方$\diff l^2$(是$(\diff l)^2$的简写)为

$\diff l^2 = \diff x^2 + \diff y^2 = [(\diff x / \diff t)^2 + (\diff y / \diff t)^2]\diff t^2 = [(T^1)^2 + (T^2)^2]\diff t^2 = |T|^2\diff t^2$,

其中$T$是$C(t)$的切矢。由上式得
$$\diff l = |T|\diff t$$
于是$C(t)$的线长为
$$l = \int |T| \diff t$$
上式可推广至带有正定度规场$g$的任意流形$M$上。
设$C(t)$是$M$上任一$C^1$曲线,$T$是其切矢,即$T \equiv \partial / \partial t$,则$|T| = \sqrt{g(T, T)}$,故$C(t)$的线长自然定义为
$$l := \int \sqrt{g(T, T)}\diff t$$

对有洛伦兹度规场$g$的流形$M$,在定义线长前应注意曲线的类型。
若$C^1$曲线$C(t)$各点的切矢都类空,则$C(t)$叫\textbf{类空曲线}。
类似地可定义\textbf{类时曲线}和\textbf{类光曲线}。
类空和类光曲线的线长仍由上式定义(因此类光曲线的线长恒为零)。
注意到类时曲线有$g(T, T) < 0$,其元线长应定义为$\diff l := \sqrt{-g(T, T)}\diff t$。
于是有如下定义:

\begin{definition}
设流形$M$上有洛伦兹度规场$g$,则$M$上的类空、类光及类时曲线$C(t)$的线长定义为
$$l := \int\sqrt{|g(T, T)|}\diff t, ~ \text{其中}T \equiv \partial / \partial t$$
\end{definition}

对于从类时转向类空(或相反)的曲线(``不伦不类''的曲线),线长没有定义。
下面对线长的讨论虽是就洛伦兹度规而言的,但对正定度规也适用(把所有曲线看作类空曲线)。

不难证明曲线的线长与其参数化无关,就是说,曲线重参数化(保持映射的像不变而适当改变参数)不改变线长。
此外,由于线长的定义不涉及坐标系,线长当然与坐标系无关。但是,如果曲线位于坐标系$\{x^\mu\}$的坐标域内,线长也可借助于坐标系计算。因为

$g(T, T) = g(T^\mu\partial / \partial x^\mu, T^\nu\partial / \partial x^\nu) = T^\mu T^\nu g(\partial / \partial x^\mu, \partial / \partial x^\nu) = (\diff x^\mu / \diff t)(\diff x^\nu / \diff t)g_{\mu\nu}$

(最后一步用到``曲线切矢的坐标分量等于曲线在该系的参数式对参数的导数'',即$T^\mu = \diff x^\mu / \diff t$。)所以元线长
$$dl = \sqrt{|g_{\mu\nu}\diff x^\mu\diff x^\nu|}$$
引入记号$\diff s^2 \equiv g_{\mu\nu}\diff x^\mu\diff x^\nu$,则线长
$$l = \int\sqrt{\diff s^2} ~ \text{(对类空曲线)} $$
$$l = \int\sqrt{-\diff s^2} ~ \text{(对类时曲线)} $$
记号$\diff s^2$在微分几何中经常出现,通常称为\textbf{线元}。
对类空曲线,$\diff s^2$等于元段长$\diff l$的平方$\diff l^2$;
对类时曲线,$\diff s^2$等于$-\diff l^2$,因而不是任何实数的平方。
实际上,$\diff s^2$只是一个记号,对类时曲线它根本不是任何实数的平方,
但因$\diff s^2 \equiv g_{\mu\nu}\diff x^\mu \diff x^\nu$右边含有度规$g$在所涉及的坐标系的全部分量$g_{\mu\nu}$,
从线元表达式可以直接``读出''度规的全体坐标分量。例如,设$2$维流形上的度规$g$在某坐标系$\{t, x\}$的线元表达式为
$$\diff s^2 = -x\diff t^2 + \diff x^2 + 4\diff t\diff x$$
便可读出$g$在该系的分量为$g_{tt} = -x, g_{xx} = 1, g_{tx} = g_{xt} = 2$。可见给定线元(表达式)相当于给定度规场。

设$C \colon I \to M$是类空或类时曲线,则线上任一点$C(t)$的切矢$T$的长度$|T|$是$t$的函数,可记作$|T|(t)$。
任意指定线上一点$C(t_0)$作为测量线长的起点,则介于$C(t_0)$点和$C(t)$点的曲线段的线长$l(t) = \int^t_{t_0}|T|(t')\diff t'$是$t$的函数。
$l$也可充当该线的参数,称为\textbf{线长参数}。由$dl \equiv \sqrt{|g(T, T)|}\diff t$可知,以线长为参数的曲线切矢满足$|g(T, T)| = 1$,即有单位长。

\begin{definition}
设流形$M$上给定度规场$g$,则$(M, g)$叫\textbf{广义黎曼空间}(若$g$为正定,叫\textbf{黎曼空间};若$g$为洛伦兹,叫\textbf{伪黎曼空间},物理上叫\textbf{时空}。)
\end{definition}

下面介绍广义黎曼空间的两个简单而重要的例子,即欧氏空间和闵氏空间。

\begin{definition}
设$\{x^\mu\}$是$\mathbb{R}^n$的自然坐标,在$\mathbb{R}^n$上定义度规张量场$\delta$为
$$\delta := \delta_{\mu\nu}\diff x^\mu\otimes\diff x^\nu$$
则$(\mathbb{R}^n, \delta)$称为\textbf{$n$维欧氏空间},$\delta$称为\textbf{欧氏度规}。
\end{definition}

上式表明$\delta$在自然坐标系的对偶坐标基底$\{\diff x^\mu\otimes\diff x^\nu\}$的分量为
$
\delta_{\mu\nu} \equiv
\begin{cases}
0, & \mu \neq \nu, \\
+1, & \mu = \nu,
\end{cases}
$
因此,欧氏度规在自然坐标系的线元表达式应为$\diff s^2 = \delta_{\mu\nu}\diff x^\mu\diff x^\nu$。
若$n = 2$,便有$\diff s^2 = (\diff x^1)^2 + (\diff x^2)^2$。这正是熟知的$2$维欧氏空间的线元表达式。
由定义可知自然坐标基底用欧氏度规衡量是正交归一的,因为由

$\delta(\partial / \partial x^\alpha, \partial / \partial x^\beta) = \delta_{\mu\nu}\diff x^\mu\otimes\diff x^\nu(\partial / \partial  x^\alpha, \partial / \partial x^\beta) = \delta_{\mu\nu}\diff x^\mu(\partial / \partial x^\alpha)\diff x^\nu(\partial / \partial x^\beta)$

易见$$\delta(\partial / \partial x^\alpha, \partial / \partial x^\beta) = \delta_{\alpha\beta}$$
但满足上式的坐标系未必是自然坐标系。例如,对$2$维欧氏空间,由自然坐标系按下式定义的坐标系
$$x' = x + a, ~ y' = y +b ~ \text{($a, b$为常数)}$$
的基底$\{\partial / \partial x', \partial / \partial y'\}$也满足上式(因而也正交归一)。
进一步,不难验证由以下三式分别定义的$\{x', y'\}$的坐标基底$\{\partial / \partial x', \partial / \partial y'\}$也满足上式:
\begin{gather*}
x' = x\cos\alpha + y\sin\alpha, ~ y' = -x\sin\alpha + y\cos\alpha ~ \text{($\alpha$为常数)}, \\
x' = -x, ~ y' = y, \\
x' = x, ~ y' = -y.
\end{gather*}

\begin{definition}
$n$维欧氏空间中满足
$\delta(\partial / \partial x^\alpha, \partial / \partial x^\beta) = \delta_{\alpha\beta}$
的坐标系叫\textbf{笛卡尔坐标系}或\textbf{直角坐标系}。换句话说,一个坐标系叫笛卡尔系,若其坐标基底用欧氏度规$\delta$衡量为正交归一。
\end{definition}

\begin{note}
\textcircled{1}因$\delta := \delta_{\mu\nu}\diff x^\mu\otimes\diff x^\nu$与上述定义等价,也可说满足此式的坐标系是笛卡尔系。
\textcircled{2}自然坐标系当然是笛卡尔系。
\textcircled{3}$2$维欧氏空间中任意两个笛卡尔系之间的关系只能取上述四式中的一种形式(或它们的复合)。
前二种分别称为\textbf{平移}和\textbf{转动},后二种的每一种称为\textbf{反射}。
\textcircled{4}要分清符号$\delta$和$\delta_{\mu\nu}$。
$\delta$代表欧氏度规,是张量场;而$\delta_{\mu\nu}$则是$\delta$在笛卡尔系的分量。
还要注意$\delta$在非笛卡尔系的分量不是$\delta_{\mu\nu}$。
\end{note}

极坐标系$\{r, \varphi\}$是$2$维欧氏空间中非笛卡尔系的一例。
物理书中使用极坐标系时,相应的基底常用$\{\hat{e}_r, \hat{e}_\varphi\}$(顶上加$\hat{}$表示单位矢),它是正交归一的,但却不是极坐标系的坐标基底$\{\partial / \partial r, \partial / \partial\varphi\}$,
关键在于$\partial / \partial\varphi$不归一,因$\delta(\partial / \partial\varphi, \partial / \partial\varphi) = r^2 \neq 1$\footnote{
按照矢量的分量变换律,有$\partial / \partial\varphi = (\diff x^\mu / \diff\varphi)(\partial / \partial x^\mu)$。
再根据极坐标与直角坐标的关系$x^1 = r\cos\varphi, ~ x^2 = r\sin\varphi$,带入即可求出。
}。
实际上,$\hat{e}_\varphi$是对$\partial / \partial\varphi$归一化的产物,即$\hat{e}_\varphi := r^{-1}\partial / \partial\varphi$。
可见物理书中常用的$\{\hat{e}_r, \hat{e}_\varphi\}$不是极坐标系的坐标基底而是与极坐标系相应的正交归一基底。

欧氏空间是最简单的黎曼空间。下面介绍最简单的伪黎曼空间---闵氏空间。
$4$维洛伦兹度规在对角化后的对角元为$(-1, 1, 1, 1)$,为了突出这个唯一的$-1$,我们把它所在的行、列记为$0$行$0$列,三个$+1$所在行、列分别记为$1, 2, 3$行和$1, 2, 3$列。
这种对角矩阵的元素记作$\eta_{\mu\nu}$(以区别于$\delta_{\mu\nu}$),即$\eta_{00} \equiv -1, \eta_{11} \equiv \eta_{22} \equiv \eta_{33} \equiv 1$。
推广到$n$维则有
$$
\eta_{\mu\nu} \equiv
\begin{cases}
0, ~ \mu \neq \nu, \\
-1, ~ \mu = \nu = 0, \\
+1, ~ \mu = \nu = 1, \cdots, n -1.
\end{cases}
$$
下面给出闵氏空间的定义。

\begin{definition}
设$\{x^\mu\}$是$\mathbb{R}^n$的自然坐标,在$\mathbb{R}^n$上定义度规张量场$\eta$为
$$\eta := \eta_{\mu\nu}\diff x^\mu\otimes\diff x^\nu$$
则$(\mathbb{R}^n, \eta)$称为\textbf{$n$维闵氏空间}(物理上称为\textbf{$n$维闵氏时空}),$\eta$称为\textbf{闵氏度规}。
\end{definition}

由定义可知闵氏度规在自然坐标系的线元表达式为$\diff s^2 = \eta_{\mu\nu}\diff x^\mu\diff x^\nu$。
以$n = 4$为例,有$\diff s^2 = -(\diff x^0)^2 + (\diff x^1)^2 + (\diff x^2)^2 + (\diff x^3)^2$。
这正是熟知的$4$维闵氏时空的线元(狭义相对论中的元间隔)表达式。不难证明
$$\eta(\partial / \partial x^\alpha, \partial / \partial x^\beta) = \eta_{\alpha\beta}$$
可见自然坐标基底$\{\partial / \partial x^\mu\}$用闵氏度规衡量也是正交归一的(第$0$坐标基矢归$-1$,其他归$+1$)。
但满足上式的却不一定是自然坐标系。例如,以$2$维闵氏空间为例,设$\{t, x\}$是自然坐标,则
$$t' = t + a, ~ x' = x + b ~ \text{($a, b$为常数)}$$
的坐标基底$\{\partial / \partial t', \partial / \partial x'\}$也满足上式。
不难验证,由以下三式分别定义的$\{t', x'\}$系的坐标基底$\{\partial / \partial t', \partial / \partial x'\}$也满足上式:
\begin{gather*}
t' = t\ch\lambda + x\sh\lambda, ~ x' = t\sh\lambda + x\ch\lambda ~ \text{($\lambda$为常数)}, \\
t' = -t, ~ x' = x, \\
t' = t, ~ x' = -x.
\end{gather*}

\begin{definition}
$n$维闵氏空间中满足$\eta(\partial / \partial x^\alpha, \partial / \partial x^\beta) = \eta_{\alpha\beta}$的坐标系叫\textbf{洛伦兹坐标系}或\textbf{伪笛卡尔坐标系},也有文献称之为笛卡尔坐标系。
\end{definition}

\begin{note}
\textcircled{1}闵氏空间的自然坐标当然是洛伦兹坐标。
\textcircled{2}$2$维闵氏空间中的任意两个洛伦兹坐标系之间的关系只能取上述四式中的一种形式(或它们的复合)。
第一种称为\textbf{平移},第二种称为\textbf{伪转动},后两种的每一种称为\textbf{反射}。
\textcircled{3}闵氏度规张量$\eta$在非洛伦兹坐标基底的分量不等于$\eta_{\mu\nu}$
\end{note}

\section{抽象指标记号}
表示张量的常用方法有两种。第一种是用不带指标的字母(如$T$)代表张量,这有两个缺点:
\textcircled{1} 看不出张量类型;
\textcircled{2} 不易表明哪一上槽与哪一下槽做缩并(前面所用的记号$C^i_jT$是暂时的,在运算中有诸多不便。)
第二种表示法是用分量(如$T^{\mu\nu}{}_\sigma$)代表张量,用分量服从的等式代表张量服从的等式。
分量等式是数量等式,因此在采用这种表示法的文献中所有等式都是数量等式。这种表示法可以克服第一种表示法的两个困难,但自身却有一严重缺点:
有时由于选用某一(或某类)特殊基底而得到较为简单的分量等式,它只对特殊基底成立,因而不代表张量等式。
我们希望知道哪些等式能够代表张量等式而哪些不能,然而在分量表示法中难以区分。
为了克服这一缺点(同时保留分量表示法的所有优点),Penrose首创``抽象指标记号'',要点如下:
\begin{enumerate}[1.]
\item $(k, l)$型张量用带有$k$个上标和$l$个下标的字母表示,上下标为小写拉丁字母,只表示张量类型,故称\textbf{抽象指标}。
例如,$v^a$代表矢量,上标$a$与$\vec{v}$中的$\vec{}$作用一样(故不能谈及$a = 1$或$a = 2$的问题),$\omega_a$代表对偶矢量,$T^{ab}{}_c$代表$(2, 1)$型张量,等等。
$v^a$和$v^b$代表相同的矢量(即矢量$\vec{v}$),但写等式时要注意``指标平衡'',例如可以写$\alpha u^a + v^a = w^a$或$\alpha u^b + v^b = w^b$而不可以写$\alpha u^a + v^b = w^a$。
\item 重复上下抽象指标表示对这两个指标求缩并。例如
$$T^a{}_a = T(e^{\mu*}; e_\mu) = T^\mu{}_\mu, T^{ab}{}_a = T(e^{\mu*}, \bullet; e_\mu), T^{ab}{}_b = T(\bullet, e^{\mu*}; e_\mu)$$
\item 张量积记号省略。例如,设$T \in \mathscr{T}_V(2, 1), S \in \mathscr{T}_V(1, 1)$,则$T \otimes S$写成$T^{ab}{}_cS^d{}_e$。
在不用指标的张量表示法中,一般来说$\omega \otimes \mu \neq \mu \otimes \omega$,因为当作用于对象$(v, u)$时,$\omega$作用于$v$还是$u$的问题由字母顺序决定($\omega \otimes \mu$的第一字母$\omega$作用于$(v, u)$的第一字母$v$)。
在抽象指标记号中,由于重复上下指标代表缩并,$\omega \otimes \mu(v, u)$既可写成$\omega_a\mu_bv^au^b$又可写成$\mu_b\omega_av^au^b$(都代表$\omega(v)\mu(u)$)。
既然在这种写法中$\omega_a\mu_b$和$\mu_b\omega_a$的作用对象都是$v^au^b$,便有$\omega_a\mu_b = \mu_b\omega_a$。
就是说,代表张量的字母\CJKunderdot{带着}自己的\CJKunderdot{抽象指标}可以交换。
张量积顺序的不可交换性体现为$\omega_a\mu_b \neq \omega_b\mu_a$。
\item 涉及张量分量时,相应指标用小写希腊字母$\mu$,$\nu$,$\alpha$,$\beta$等(正如前面一直用的),这种指标称为\textbf{具体指标},可以问及$\mu = 1$还是$\mu = 2$的问题。
张量在基矢上的展开式$T = T^{\mu\nu}{}_\sigma e_\mu \otimes e_\nu \otimes e^{\sigma*}$现在写成
$$T^{ab}{}_c = T^{\mu\nu}{}_\sigma(e_\mu)^a(e_\nu)^b(e^\sigma)_c$$($(e^\sigma)_c$的抽象下标$c$已表明它是对偶基矢,无须写为$(e^{\sigma*})_c$。)
而$T^{\mu\nu}{}_\sigma = T(e^{\mu*}, e^{\nu*}; e_\sigma)$现在写成
$$T^{\mu\nu}{}_\sigma = T^{ab}{}_c(e^\mu)_a(e^\nu)_b(e_\sigma)^c$$
注意,上两式的指标(无论抽象的还是具体的)都是``平衡''的。
设$T \in \mathscr{T}_V(0, 2)$,则$T$应记作$T_{ab}$。
令$e_\mu$为某基底的第$\mu$基矢,则可知$T(\bullet, e_\mu) = C^1_2(T \otimes e_\mu)$,而$T \otimes e_\mu$用抽象指标应记为$T_{ab}(e_\mu)^c$,故$T(\bullet, e_\mu)$应记作$T_{ab}(e_\mu)^b$,也可简记作$T_{a\mu}$,即
$$T(\bullet, e_\mu) \equiv T_{ab}(e_\mu)^b \equiv T_{a\mu}$$
这是既有抽象指标又有具体指标的张量的表达方式,不妨认为$T_{a1}, \cdots, T_{an}$代表$n$个对偶矢量,其中$T_{a\mu}$代表``第$\mu$个对偶矢量''。
\item 由``张量面面观''可知,$V$上的$(1, 1)$型张量$T^a{}_b$既可看作从$V$到$V$的线性映射又可看作从$V^*$到$V^*$的线性映射。
就是说,$T^a{}_b$作用于矢量$v^b \in V$仍为矢量,记作$u^a \equiv T^a{}_bv^b \in V$;
$T^a{}_b$作用于对偶矢量$\omega_a \in V^*$仍为对偶矢量,记作$\mu_b \equiv T^a{}_b\omega_a \in V^*$。
其实,由抽象指标也可一望而知$T^a{}_bv^b$和$T^a{}_b\omega_a$分别是矢量和对偶矢量,可见抽象指标记号是``张量面面观''的一种简单而直观的体现。
以$\delta^a{}_b$代表从$V$到$V$的恒等映射,即$\delta^a{}_bv^b := v^a ~ \forall v^b \in V$,则易见它也是从$V^*$到$V^*$的恒等映射,即$\delta^a{}_b\omega_a = \omega_b ~ \forall \omega_a \in V^*$。
进一步不难证明$\delta^a{}_b$与任一张量缩并的结果是把该张量的上标$b$换为$a$(或把下标$a$换为$b$),例如$\delta^a{}_bT_{ac} = T_{bc}, \delta^a{}_bT^{cb}{}_e = T^{ca}{}_e$。
设$\{(e_\mu)^a\}$是$V$的基底,$\{(e^\mu)_a\}$是其对偶基底,则
$$(e^\mu)_a(e_\mu)^b = \delta^b{}_a$$
这是$(1, 1)$型张量等式,证明时只须验证两边作用于任一矢量$v^a$得相同结果。
设$\{(e_\mu)^a\}$是$V$的基底,$\{(e^\mu)_a\}$是其对偶基底,则$\delta^a{}_b$在此基底的分量$\delta^\mu{}_\nu \equiv \delta^a{}_b(e^\mu)_a(e_\nu)^b$满足$\delta^\mu{}_\nu = \begin{cases}+1, ~ (\mu = \nu) \\0, ~ (\mu \neq \nu)\end{cases}$。
证明很简单,以$\delta^1{}_1$为例,$\delta^1{}_1 = \delta^a{}_b(e^1)_a(e_1)^b = (e^1)_a(e_1)^a = 1$。
请注意,即使是洛伦兹号差的情况下也有$\delta^0{}_0 = +1$。
\item 因度规$g \in \mathscr{T}_V(0, 2)$,故应记为$g_{ab}$。设$v \in V$,则$g(\bullet, v) \in V^*$。
把$g$看作$T$,便得$g(\bullet, v) = C^1_2(g \otimes v) = C^1_2(g_{ab}v^c) = g_{ab}v^b$,故$g(\bullet, v)$应记作$g_{ab}v^b$。
又因有度规$g$时$V$与$V^*$在同构映射$g \colon V \to V^*$下自然认同,而$g_{ab}v^b \equiv g(\bullet, v)$正是$v^a$在这一映射下的像,故应与$v^a$认同,索性就把$g_{ab}v^b$记作$v_a$(可看作$v_a$的定义)。
就是说,虽然在数学上$v^a$与$v_a$是两种不同性质的量(矢量和对偶矢量),但在应用上两者代表的是同一事物(故都用$v$表示)。
于是常写
$$v_a = g_{ab}v^b$$
又由于$g \colon V \to V^*$是同构映射,其逆映射$g^{-1}$自然存在。
不难论证$g^{-1}$是$(2, 0)$型张量,本应记作$(g^{-1})^{ab}$,但通常都简记为$g^{ab}$(有上指标就不会与$g_{ab}$混淆)。
根据类似推理,任一$\omega_b \in V^*$在$g^{ab}$映射下的像为$g^{ab}\omega_b$,索性记作$\omega^a$,以表示与$\omega_a$代表同一事物,于是
$$\omega^a = g^{ab}\omega_b$$
上述两式表明可用$g_{ab}$及$g^{ab}$对上、下指标分别做``下降''和``上升''处理。
这种升降指标操作适用于任何张量中的任何抽象指标。
例如,$(1, 1)$型张量$T$在抽象指标记号中可表为$T^a{}_b$,所谓用度规对它降指标,其实是用$g$和$T$通过张量积及缩并运算求得一个$(0, 2)$型张量$g(\bullet, e_\mu) \otimes T(e^{\mu*}; \bullet)$,在抽象指标记号中就把它记作$T_{ab}$,即$T_{ab} \equiv g_{ac}T^c{}_b$。
\end{enumerate}

依次使用上述两式得
$$\omega^a = g^{ab}\omega_b = g^{ab}(g_{bc}\omega^c), ~ \forall \omega^a \in V$$
故
$$g^{ab}g_{bc} = \delta^a{}_c$$
其实这是$g^{ab}$作为$g_{ab}$的逆映射的必然结果。

设$\{(e_\mu)^a\}$是$V$的任一基底,$\{(e^\mu)_a\}$是其对偶基底,以$g_{\mu\nu}$和$g^{\mu\nu}$分别代表$g_{ab}$和$g^{ab}$在这一基底的分量,则
$$g^{\mu\nu}g_{\nu\sigma} = g^{ab}(e^\mu)_a(e^\nu)_bg_{cd}(e_\nu)^c(e_\sigma)^d = g^{ab}(e^\mu)_ag_{bd}(e_\sigma)^d = (e^\mu)_a(e_\sigma)^a = \delta^\mu{}_\sigma$$
其中第二、三步分别用到$(e^\mu)_a(e_\mu)^b = \delta^b{}_a$和$g^{ab}g_{bc} = \delta^a{}_c$。
上式表明度规$g_{ab}$在任一基底的分量$g_{\mu\nu}$的矩阵有逆(逆矩阵就是度规$g_{ab}$之逆$g^{ab}$在同一基底的分量$g^{\mu\nu}$的矩阵),因而非退化。
可见$g_{ab}$的非退化性保证它在任一基底的矩阵$(g_{\mu\nu})$的非退化性。
反之,设存在基底$\{(e_\mu)^a\}$及其对偶基底$\{(e^\mu)_a\}$使$(g_{\mu\nu})$为非退化,则$(g_{\mu\nu})$有逆矩阵$(g^{\mu\nu})$,令$g^{ab} \equiv g^{\mu\nu}(e_\mu)^a(e_\nu)^b$,则由$g^{\mu\nu}g_{\nu\sigma} = \delta^\mu{}_\sigma$易证$g^{ab}g_{bc} = \delta^a{}_c$,
可见$g_{ab} \colon V \to V^*$有逆映射$g^{ab}$,因而非退化($g_{ab} \colon V \to V^*$有逆表明它是一一映射,而如果$g_{ab}$退化,则$V$中除零元外还有$v^a \neq 0$,其像也为$\underline{0} \in V^*$,与一一性矛盾。)。

不难看出,用度规及其逆的分量$g_{\mu\nu}$及$g^{\mu\nu}$可对张量分量的上、下具体指标做下降和上升处理。
例如,可把$g_{\mu\nu}v^\nu$写成$v_\mu$,因为
$$g_{\mu\nu}v^\nu = g_{ab}(e_\mu)^a(e_\nu)^bv^\nu = g_{ab}(e_\mu)^av^b = v_a(e_\mu)^a = v_\mu$$

作为抽象指标记号的例子,此处介绍$4$维闵氏度规$\eta_{ab}$的抽象指标表达式。
闵氏度规的定义式在抽象指标记号中应表为
$$\eta_{ab} := \eta_{\mu\nu}(\diff x^\mu)_a(\diff x^\nu)_b$$
其中$\{(\diff x^\mu)_a\}$是洛伦兹坐标系的对偶基底。以$\{t, x, y, z\}$代表$\{x^0, x^1, x^2, x^3\}$,则因为非零的$\eta_{\mu\nu}$只有$\eta_{00} = -1, \eta_{11} = \eta_{22} = \eta_{33} = 1$,上式可表为
$$\eta_{ab} = -(\diff t)_a(\diff t)_b + (\diff x)_a(\diff x)_b + (\diff y)_a(\diff y)_b + (\diff z)_a(\diff z)_b$$
与线元表达式$\diff s^2 = -\diff t^2 + \diff x^2 + \diff y^2 + \diff z^2$相应。如果改用球坐标系$\{t, r, \theta, \varphi\}$,则由
$$x = r\sin\theta\cos\varphi, y = r\sin\theta\sin\varphi, z = r\cos\theta$$
不难从前式导出
$$\eta_{ab} = -(\diff t)_a(\diff t)_b + (\diff r)_a(\diff r)_b + r^2(\diff \theta)_a(\diff \theta)_b + r^2\sin^2\theta(\diff \varphi)_a(\diff \varphi)_b$$
与线元表达式$\diff s^2 = -\diff t^2 + \diff r^2 + r^2(\diff \theta^2 + \sin^2\theta\diff\varphi^2)$相应。

在许多不用抽象指标记号的文献中,$4$维时空和$3$维黎曼空间的分量指标分别用希腊字母$\mu, \nu, \cdots$(都从$0$取到$3$)和拉丁字母$i, j, k, \cdots$(都从$1$取到$3$)。
拉丁字母在本书中本应代表抽象指标,然而为区分$4$维和$3$维的分量指标,我们允许一个例外,即凡涉及$3$维黎曼空间时用拉丁字母中从$i$起的若干字母$i, j, k, \cdots$充当具体指标(都从$1$取到$3$),其他拉丁字母(如$a, b, c$等)仍作为抽象指标。
例如$3$维矢量$\vec{v}$可表为$v^a = v^i(\partial / \partial x^i)^a$($i$从$1$到$3$取和)。

在抽象指标记号中,坐标基矢记作$(\partial / \partial x^\mu)^a$,对偶坐标基矢记作$(\diff x^\mu)_a$。
用度规$g_{ab}$和$g^{ab}$对前者和后者分别降、升指标,得对偶矢量$g_{ab}(\partial / \partial x^\mu)^b$和矢量$g^{ab}(\diff x^\mu)_b$。
以$\omega_a$简记$g_{ab}(\partial / \partial x^\mu)^b$并用对偶坐标基矢展开为$\omega_a = \omega_\nu(\diff x^\nu)_a$,两边作用于$(\partial / \partial x^\sigma)^a$后得$g_{\sigma\mu} = \omega_\sigma$\footnote{
左边 = $g_{ab}(\partial / \partial x^\mu)^b(\partial / \partial x^\sigma)^a = g_{a\mu}(\partial / \partial x^\sigma)^a = g_{\sigma\mu}$,
右边 = $\omega_\nu(\diff x^\nu)_a(\partial / \partial x^\sigma)^a = \omega_\nu\delta^\nu{}_\sigma = \omega_\sigma$
},故
$$g_{ab}(\partial / \partial x^\mu)^b = g_{\mu\nu}(\diff x^\nu)_a$$
可见$g_{ab}(\partial / \partial x^\mu)^b$一般不等于$(\diff x^\mu)_a$。类似可得
$$g^{ab}(\diff x^\mu)_b = g^{\mu\nu}(\partial / \partial x^\nu)^a$$
当$g_{ab} = \delta_{ab}$(欧氏度规)且$\{x^\mu\}$为笛卡尔系时上二式简化为
$$\delta_{ab}(\partial / \partial x^\mu)^b = (\diff x^\mu)_a, ~ \delta^{ab}(\diff x^\mu)_b = (\partial / \partial x^\mu)^a$$
当$g_{ab} = \eta_{ab}$(以$4$维闵氏度规为例)且$\{x^\mu\}$为洛伦兹系时则有
$$\begin{aligned}
\eta_{ab}(\partial / \partial x^0)^b = -(\diff x^0)_a, ~ & \eta_{ab}(\partial / \partial x^i)^b = (\diff x^i)_a; \\
\eta^{ab}(\diff x^0)_b = -(\partial / \partial x^0)^a, ~ & \eta^{ab}(\diff x^i)_b = (\partial / \partial x^i)^a.
\end{aligned}$$
其中的$i = 1, 2, 3$,此时的$i$不是抽象指标。

张量的上指标和下指标在文献中又常分别称为逆变指标和协变指标。相应地,矢量$v^a$和对偶矢量$\omega_a$也分别称为逆变矢量和协变矢量。

张量的对称性可方便地用抽象指标表述如下:

\begin{definition}
$T \in \mathscr{T}_V(0, 2)$称为对称的,若$T(u, v) = T(v, u), ~ \forall u, v \in V$。
\end{definition}

由于$T(u, v) = T_{ab}u^av^b, T(v, u) = T_{ab}v^au^b = T_{ba}u^av^b$,故在抽象指标记号中$T$为对称的充要条件是$T_{ab} = T_{ba}$。
$(0, 2)$型张量在抽象指标记号中本来既可记作$T_{ab}$又可记作$T_{ba}$,两者代表同一张量。
然而只当$T$为对称张量时才允许写为$T_{ab} = T_{ba}$,可见用抽象指标写等式时比用它单独表示一个张量时要更为小心。
同理,$(1, 1)$型张量既可表为$T^a{}_b$,也可表为$T_b{}^a$,用度规降指标后分别为$g_{ca}T^a{}_b = T_{cb}$和$g_{ca}T_b{}^a = T_{bc}$。
虽然两者代表同一张量,但只有降指标后为对称张量的$(1, 1)$型张量才允许写为$T^a{}_b = T_b{}^a$。
在不用度规升降指标时,$(k, l)$型张量的上、下指标各排各的序,两个上下指标之间没有顺序问题。
因此,如果愿意,$(1, 1)$型张量可写为$T^a_b$,$(2, 1)$型张量可写为$T^{ab}_c$等。
然而这种写法在用度规升降指标时出现不确定性。由于经常要升降指标,本书从一开始就把上、下两排指标错开,例如写成$T^{ab}{}_c$。

以上讨论表明抽象指标记号在形式上与具体指标记号极为相似,这正是抽象指标记号的一大好处:它既可表示张量等式,又保留了具体指标记号的许多优点。

\begin{definition}
$(0, 2)$型张量$T_{ab}$的对称部分(记作$T_{(ab)}$)和反称部分(记作$T_{[ab]}$)分别定义为
$$T_{(ab)} := \frac{1}{2}(T_{ab} + T_{ba}), ~ T_{[ab]} := \frac{1}{2}(T_{ab} - T_{ba})$$
\end{definition}

一般地,$(0, l)$型张量$T_{a_1 \cdots a_l}$的对称和反称部分定义为
\begin{gather*}
T_{(a_1 \cdots a_l)} := \frac{1}{l!}\sum_\pi T_{a_{\pi(1)} \cdots a_{\pi(l)}}, \\
T_{[a_1 \cdots a_l]} := \frac{1}{l!}\sum_\pi \delta_\pi T_{a_{\pi(1)} \cdots a_{\pi(l)}},
\end{gather*}
其中$\pi$代表$(1, \cdots, l)$的一种排列,$\pi(1)$是指$\pi$所代表的那种排列中的第$1$个数字,$\sum\limits_\pi$代表对各种排列取和,$\delta_\pi \equiv \pm 1$(偶排列取$+$,奇排列取$-$。)。例如
\begin{gather*}
T_{(a_1 a_2 a_3)} := \frac{1}{6}(T_{a_1a_2a_3} + T_{a_3a_1a_2} + T_{a_2a_3a_1} + T_{a_1a_3a_2} + T_{a_3a_2a_1} + T_{a_2a_1a_3}), \\
T_{[a_1 a_2 a_3]} := \frac{1}{6}(T_{a_1a_2a_3} + T_{a_3a_1a_2} + T_{a_2a_3a_1} - T_{a_1a_3a_2} - T_{a_3a_2a_1} - T_{a_2a_1a_3}).
\end{gather*}

\begin{definition}
$T \in \mathscr{T}_V(0, l)$称为全对称的,若$T_{a_1 \cdots a_l} = T_{(a_1 \cdots a_l)}$;
$T$称为全反称的,若$T_{a_1 \cdots a_l} = T_{[a_1 \cdots a_l]}$。
\end{definition}

以上三个定义也适用于$(k, 0)$型张量。例如,$T$叫全对称的,若$T^{a_1 \cdots a_l} = T^{(a_1 \cdots a_l)}$。

\begin{note}
任一$(0, 2)$型张量可表为其对称和反称部分之和,即$T_{ab} = T_{(ab)} + T_{[ab]}$,但对$l > 2$的$(0, l)$型张量不成立。
例如,$T_{abc} \neq T_{(abc)} + T_{[abc]}$,然而$T_{abc} = T_{(abc)} \Rightarrow T_{[abc]} = 0$(见下文定理)。
\end{note}

\begin{theorem}
\begin{enumerate}[(a)]
\item 设$T_{a_1 \cdots a_l} = T_{(a_1 \cdots a_l)}$,则
$$T_{a_1 \cdots a_l} = T_{a_\pi(1) \cdots a_\pi(l)}\text{($\pi$代表任一种排列)}$$
即$T_{(a_1 \cdots a_l)}$展开式(共$l!$项)中每一项都等于$T_{a_1 \cdots a_l}$,例如
$$T_{abc} = T_{(abc)} \Rightarrow T_{abc} = T_{acb} = T_{cab} = T_{cba} = T_{bca} = T_{bac}$$
\item 设$T_{a_1 \cdots a_l} = T_{[a_1 \cdots a_l]}$,则
$$T_{a_1 \cdots a_l} = \delta_\pi T_{a_\pi(1) \cdots a_\pi(l)}$$
即$T_{[a_1 \cdots a_l]}$展开式中的偶排列项等于$T_{a_1 \cdots a_l}$,奇排列项等于$-T_{a_1 \cdots a_l}$,例如
$$T_{abc} = T_{[abc]} \Rightarrow T_{abc} = -T_{acb} = T_{cab} = -T_{cba} = T_{bca} = -T_{bac}$$
\end{enumerate}
对$(k, 0)$型(上指标)全对称和全反称张量也有类似结论。
\end{theorem}

今后会经常遇到对带有圆、方括号的指标的运算,下面的定理对许多运算将带来很大方便。

\begin{theorem}
\begin{enumerate}[(a)]
\item 缩并时括号有``传染性'',即
$$T_{[a_1 \cdots a_l]}S^{a_1 \cdots a_l} = T_{[a_1 \cdots a_l]}S^{[a_1 \cdots a_l]} = T_{a_1 \cdots a_l}S^{[a_1 \cdots a_l]}$$
对圆括号亦然。
\item 括号内的同种子括号可随意增删,例如
$$T_{[[ab]c]} = T_{[abc]}, ~ \text{其中} ~ T_{[[ab]c]} \equiv \frac{1}{2}(T_{[abc]} - T_{[bac]})$$
\item 括号内加异种子括号得零,例如
$$T_{[(ab)c]} = 0, ~ T_{(a[bcd])} = 0$$
\item 异种子括号缩并得零,例如
$$T^{(abc)}S_{[abc]} = 0$$
\item \begin{gather*}
T_{a_1 \cdots a_l} = T_{(a_1 \cdots a_l)} \Rightarrow T_{[a_1 \cdots a_l]} = 0 \\
T_{a_1 \cdots a_l} = T_{[a_1 \cdots a_l]} \Rightarrow T_{(a_1 \cdots a_l)} = 0
\end{gather*}
\end{enumerate}
对$(k, 0)$型(上指标)全对称和全反称张量也有类似结论。
\end{theorem}
