\chapter{流形和张量场}

\section{微分流形}

\begin{definition}
拓扑空间$(M, \mathscr{T})$称为\textbf{$n$维微分流形},简称$n$维流形,若$M$有开覆盖${O_\alpha}$,即$M = \bigcup\limits_\alpha O_\alpha$,满足
\begin{enumerate}[(a)]
\item 对每一$O_\alpha$,存在同胚$\psi_\alpha \colon O_\alpha \to V_\alpha$($V_\alpha$是$\mathbb{R}^n$用通常拓扑衡量的开子集);
\item 若$O_\alpha \cap O_\beta \neq \emptyset$,则复合映射$\psi_\beta \comp \psi_\alpha^{-1}$是$C^\infty$(光滑)的。
\end{enumerate}
\end{definition}

\begin{note}
\textcircled{1}$\psi_\beta \comp \psi_\alpha^{-1}$是从$\psi_\alpha[O_\alpha \cap O_\beta] \subset \mathbb{R}^n$到$\psi_\beta[O_\alpha \cap O_\beta] \subset \mathbb{R}^n$的映射。
因$\mathbb{R}^n$的每点都有$n$个自然坐标,故$\psi_\beta \comp \psi_\alpha^{-1}$提供了$n$个$n$元函数。所谓$\psi_\beta \comp \psi_\alpha^{-1}$是$C^\infty$的,就是指这每一个$n$元函数都是$C^\infty$的。
\textcircled{2}设$p \in O_\alpha$,则$\psi_\alpha(p) \in \mathbb{R}^n$,故$\psi_\alpha(p)$有$n$个自然坐标。很自然地把这$n$个数称为$p$点在映射$\psi_\alpha$下获得的\textbf{坐标}。
$M$作为拓扑空间,其元素本来一般没有坐标,但作为流形,$M$中位于$O_\alpha$内的元素(点)就可以通过映射$\psi_\alpha$获得坐标。若$O_\alpha \cap O_\beta \neq \emptyset$,
则$O_\alpha \cap O_\beta$内的点既可以通过$\psi_\alpha$又可以通过$\psi_\beta$获得坐标,这两组坐标一般不同。我们说$(O_\alpha, \psi_\alpha)$构成一个(局域)\textbf{坐标系},其\textbf{坐标域}为$O_\alpha$;
$(O_\beta, \psi_\beta)$构成另一坐标系,其坐标域为$O_\beta$。于是$O_\alpha \cap O_\beta$内的点至少有两组坐标,分别记作$\{x^\mu\}$和$\{x'^\nu\}$($\mu, \nu = 1, \cdots, n$)。
由映射$\psi_\beta \comp \psi_\alpha^{-1}$提供的、体现两组坐标之间关系的$n$个$n$元函数$$x'^1 = \phi^1(x^1, \cdots, x^n), \cdots, x'^n = \phi^n(x^1, \cdots, x^n)$$就称为一个\textbf{坐标变换}。
定义的条件(b)保证坐标变换中的函数关系$x'^\mu = \phi^\mu(x^1, \cdots, x^n)$都是$C^\infty$的。为方便起见也常称$\{x^\mu\}$为坐标系,虽然从$\{x^\mu\}$中看不出坐标域的范围。
物理学家也常把$x'^\mu = \phi^\mu(x^1, \cdots, x^n)$记作$x'^\mu = x'^\mu(x^1, \cdots, x^n)$。
\end{note}

\begin{definition}
坐标系$(O_\alpha, \psi_\alpha)$在数学上又叫\textbf{图},满足定义条件(a)、(b)的全体图的集合$\{(O_\alpha, \psi_\alpha)\}$叫\textbf{图册}。
条件(b)又称\textbf{相容性条件},因此说一个图册中的任意两个图都是相容的。
\end{definition}

\begin{example}
设$M = (\mathbb{R}^2, \mathscr{T}_u)$。选$O_1 = \mathbb{R}^2, \psi_1 =\text{恒等映射}$,则$\{(O_1, \psi_1)\}$便是只含一个图的图册,故$\mathbb{R}^2$是$2$维流形,而且是能用一个坐标域覆盖的流形,称为\textbf{平凡流形}。
根据这个图册,$\mathbb{R}^2$中每点的坐标就是它作为$\mathbb{R}^2$的元素天生就有的自然坐标。$\mathbb{R}^2$的点当然也可用其他坐标(如极坐标)描述。
其实这无非是选择与图$(O_1, \psi_1)$相容的另一个图$(O_2, \psi_2)$,其中$\psi_2$把$p \in O_2$映为$\psi_2(p) \in \mathbb{R}^2$,再把$\psi_2(p)$的自然坐标称为$p$点的新坐标而已。
但应注意坐标域$O_2$未必能包括$\mathbb{R}^2$的全体点(例如极坐标)。

同理可知$\mathbb{R}^n$是$n$维平凡流形。
\end{example}

\begin{example}
设$M = (S^1, \mathscr{S})$,其中$S^1 := \{x \in \mathbb{R}^2 \mid |x - o| = 1\}$是以原点$o$为心的单位圆周,$\mathscr{S}$是$\mathbb{R}^2$的$\mathscr{T}_u$在$S^1$上诱导的拓扑。则可证明$S^1$是$1$维流形。
\end{example}

\begin{example}
设$M = (S^2, \mathscr{S})$,其中$S^2 := \{x \in \mathbb{R}^3 \mid |x - o| = 1\}$是以原点$o$为心的单位球面,$\mathscr{S}$是$\mathbb{R}^3$的$\mathscr{T}_u$在$S^2$上诱导的拓扑。则可证明$S^2$是$2$维流形。
\end{example}

设图册$\{(O_\alpha, \psi_\alpha)\}$把拓扑空间$M$定义为一个流形,则此图册中的任意两个图自然是相容的。
但也可用另一图册$\{(O'_\beta, \psi'_\beta)\}$把同一$M$定义为流形,这时有两种可能:
\textcircled{1}这两个图册不相容,即存在$O_\alpha$和$O'_\beta$使$O_\alpha \cap O'_\beta \neq \emptyset$,且在$O_\alpha \cap O'_\beta$上$\psi_\alpha$与$\psi'_\beta$不满足定义条件(b),
这时就说这两个图册把$M$定义为两个不同的微分流形,并说这两个图册代表两种不同的\textbf{微分结构};
\textcircled{2}这两个图册是相容的,这时就说它们把$M$定义为同一个微分流形(只有一种微分结构)。为方便起见,不妨把$\{(O_\alpha, \psi_\alpha); (O'_\beta, \psi'_\beta)\}$看成一个图册。
更进一步,索性把所有与$(O_\alpha, \psi_\alpha)$相容的图都放到一起造出一个最大的图册。今后说到$M$是一个流形时,总是默认已选定某一最大的图册作为微分结构。这使我们可以进行任意坐标变换。

微分流形与拓扑空间的重要区别是前者除有拓扑结构外还有微分结构,因此两个流形之间的映射不但可谈及是否连续,还可谈及是否可微,乃至是否$C^\infty$。
设$M$和$M'$是两个流形,维数依次为$n$和$n'$,$\{(O_\alpha, \psi_\alpha)\}$和$\{(O'_\beta, \psi'_\beta)\}$依次为两者的图册,$f \colon M \to M'$是一个映射。
$\forall p \in M$,任取坐标系$(O_\alpha, \psi_\alpha)$使$p \in O_\alpha$以及坐标系$(O'_\beta, \psi'_\beta)$使$f(p) \in O'_\beta$,
则$\psi'_\beta \comp f \comp \psi_\alpha^{-1}$是从$V_\alpha \equiv \psi_\alpha[O_\alpha]$到$\mathbb{R}^{n'}$的映射,
因此相应于$n'$个$n$元函数,它们的$C^r$性可用以定义$f \colon M \to M'$的$C^r$性。

\begin{definition}
$f \colon M \to M'$称为\textbf{$C^r$类映射},如果$\forall p \in M$,映射$\psi'_\beta \comp f \comp \psi_\alpha^{-1}$对应的$n'$个$n$元函数是$C^r$类的。
\end{definition}

\begin{note}
由于同一图册中各图相容,上述定义与坐标系$\{(O_\alpha, \psi_\alpha)\}$及$\{(O'_\beta, \psi'_\beta)\}$的选择无关。
\end{note}

\begin{definition}
微分流形$M$和$M'$称为\textbf{互相微分同胚},若存在$f \colon M \to M'$,满足(a)$f$是一一到上的;(b)$f$及$f^{-1}$是$C^\infty$的。这样的$f$称为从$M$到$M'$的\textbf{微分同胚映射},简称\textbf{微分同胚}。
\end{definition}

\begin{note}
\textcircled{1} 微分同胚是对流形间的映射可以提出的最高要求(若流形上还有附加结构则另当别论),互相微分同胚的流形可视作相等。
\textcircled{2} 只有维数相等的流形才可能微分同胚。
\textcircled{3} 若把流形定义中的$O_\alpha$和$V_\alpha$看做流形,则$\psi_\alpha$也是微分同胚。
\end{note}

映射$f \colon M \to M'$的一个重要而简单的特例是$M' = \mathbb{R}$的情况。这时$M$的每点对应着一个实数,于是有如下定义:

\begin{definition}
$f \colon M \to \mathbb{R}$称为\textbf{$M$上的函数}或\textbf{$M$上的标量场}。若$f$为$C^\infty$的,则称为\textbf{$M$上的光滑函数}。
$M$上全体光滑函数的的集合记作$\mathscr{F}_M$,在不会混淆时简记为$\mathscr{F}$。今后在提到函数而不加声明时都是指光滑函数。
\end{definition}

\begin{example}
$\mathbb{R}^3$中位于$q$点的点电荷的电势是流形$M \equiv \mathbb{R}^3 - \{q\}$上的光滑函数。
\end{example}

\begin{example}
坐标系$(O, \psi)$的第$\mu$坐标$x^\mu$是定义在$O$上的光滑函数。
\end{example}

函数$f \colon M \to \mathbb{R}$与坐标系$(O, \psi)$结合可得一个$n$元函数$F(x^1, \cdots, x^n)$,因为$n$个坐标决定$O$中的唯一一点$p$,
而由$f \colon M \to \mathbb{R}$可得唯一的实数$f(p)$。然而$f$与另一坐标系$(O', \psi')$结合将给出另一$n$元函数$F'(x'^1, \cdots, x'^n)$,
函数关系$F$和$F'$不同,因为$F = f \comp \psi^{-1}$而$F' = f \comp \psi'^{-1}$。可见与函数$f \colon M \to \mathbb{R}$相应的多元函数(指函数关系)
是坐标系依赖的。应注意区分函数(标量场)$f$和它与坐标系结合而得的多元函数$F$。

设$M, N$为流形,则它们必为拓扑空间,故$M \times N$也是拓扑空间。不难利用$M, N$的流形结构把$M \times N$进一步定义为流形。
设$M, N$的维数分别为$m, n$,则$M \times N$的维数是$m + n$,即$\dim{(M \times N)} = \dim{M} + \dim{N}$。

\section{切矢和切矢场}

\subsection{切矢量}

\begin{definition}
实数域上的一个\textbf{矢量空间}是一个集合$V$配以两个映射,即$V \times V \to V$(叫\textbf{加法})及$\mathbb{R} \times V \to V$(叫\textbf{数乘}),满足如下条件:
\begin{enumerate}[(a)]
\item $v_1 + v_2 = v_2 + v_1, ~ \forall v_1, v_2 \in V$;
\item $(v_1 + v_2) + v_3 = v_1 + (v_2 + v_3), ~ \forall v_1, v_2, v_3 \in V$;
\item $\exists \underline{0} \in V, \text{使} \underline{0} + v = v, ~ \forall v \in V$;
\item $\alpha_1(\alpha_2v) = (\alpha_1\alpha_2)v, ~ \forall v \in V, ~ \alpha_1, \alpha_2 \in \mathbb{R}$;
\item $(\alpha_1 + \alpha_2)v = \alpha_1v + \alpha_2v, ~ \forall v \in V, ~ \alpha_1, \alpha_2 \in \mathbb{R}$;
\item $\alpha(v_1 + v_2) = \alpha v_1 + \alpha v_2, ~ \forall v_1, v_2 \in V, ~ \alpha \in \mathbb{R}$;
\item $1 \cdot v = v, ~ \forall v \in V$。
\end{enumerate}
\end{definition}

\begin{note}
由此$7$条可以推出:(1)$0 \cdot v = \underline{0}$,(2)$\forall v \in V, ~ \exists u \in V, \text{使} v + u = \underline{0}$。
约定把$u$记作$-v$。
\end{note}

今后也常把$V$的零元简写作$0$,即符号$0$既代表$0 \in \mathbb{R}$又代表$\underline{0} \in V$。

\begin{definition}
映射$v \colon \mathscr{F}_M \to \mathbb{R}$称为\textbf{点$p \in M$的一个矢量},若$\forall f, g \in \mathscr{F}_M, ~ \alpha, \beta \in \mathbb{R}$,有
\begin{enumerate}[(a)]
\item (线性性) $v(\alpha f + \beta g) = \alpha v(f) + \beta v(g)$;
\item (莱布尼茨律) $v(fg) = f|_p v(g) + g|_p v(f)$,其中$f|_p$代表函数$f$在$p$点的值,亦可记作$f(p)$。
\end{enumerate}
\end{definition}

\begin{note}
因$f$和$g$是$M$上的函数,故$fg$也是$M$上的函数,它在$M$的任一点$p$的值定义为$f(p)$与$g(p)$之积。
\end{note}

根据定义,要定义$p$点的一个矢量只须指定一个从$\mathscr{F}_M$到$\mathbb{R}$的、满足条件(a)、(b)的映射,
就是说,指定一个对应规律(法则),根据这一规律,每一$f \in \mathscr{F}_M$对应于一个确定的实数。
因为这种映射很多,所以$p$点有很多(无限多)矢量。
例如,设$(O, \psi)$是坐标系,其坐标为$x^\mu$,则$M$上任一光滑函数$f \in \mathscr{F}_M$与$(O, \psi)$结合得$n$元函数
$F(x^1, \cdots, x^n)$,借此可给$O$中任一点$p$定义$n$个矢量,记作$X_\mu$(其中$\mu = 1, \cdots, n$),
它(们)作用于任一$f \in \mathscr{F}_M$的结果$X_\mu(f)$定义为如下实数:
$$X_\mu(f) := \left.\frac{\partial F(x^1, \cdots, x^n)}{\partial x^\mu}\right|_p, ~ \forall f \in \mathscr{F}_M$$
其中$\partial F(x^1, \cdots, x^n)/\partial x^\mu|_p$是$\partial F/\partial x^\mu|_{(x^1(p), \cdots, x^n(p))}$的简写。
今后也把$\partial F(x^1, \cdots, x^n)/\partial x^\mu$简写为$\partial f(x^1, \cdots, x^n)/\partial x^\mu$或$\partial f(x)/\partial x^\mu$甚至$\partial f/\partial x^\mu$。
应认出$\partial f/\partial x^\mu$中的$f$代表$n$元函数$F(x^1, \cdots, x^n)$而非标量场$f$。于是上式可简写为
$$X_\mu(f) := \left.\frac{\partial f(x)}{\partial x^\mu}\right|_p, ~ \forall f \in \mathscr{F}_M$$

\begin{theorem}
以$V_p$代表$M$中点$p$所有矢量的集合,则$V_p$是$n$维矢量空间($n$是$M$的维数),即$\dim V_p = \dim M \equiv n$。
\end{theorem}

\begin{definition}
坐标域内任一点$p$的$\{X_1, \cdots, X_n\}$称为$V_p$的一个\textbf{坐标基底},
每个$X_\mu$称为一个\textbf{坐标基矢},$v \in V_p$用$\{X_\mu\}$线性表出的系数$v^{\mu}$称为$v$的\textbf{坐标分量}。
\end{definition}

\begin{theorem}
设$\{x^\mu\}$和$\{x'^\nu\}$为两个坐标系,其坐标域的交集非空,
$p$为交集中的一点,$v \in V_p$,$\{v^\mu\}$和$\{v'^\nu\}$是$v$在这两个系的坐标分量,则
$$v'^\nu = \left.\frac{\partial x'^\nu}{\partial x^\mu}\right|_pv^\mu$$
其中$x'^\nu$是两系间坐标变换函数$x'^\nu(x^\mu)$的简写。
\end{theorem}

\begin{proof}[证明]
根据坐标分量的定义,$v = v^\mu X_\mu = v'^\nu X'_\nu$。
因此可以通过$X_\mu$和$X'_\nu$之间的关系来找到$v^\mu$和$v'^\nu$之间的关系。
为此,设$q$是两坐标域交集内的任一点,则标量场$f$在$q$点的值$f|_q$满足$f|_q = f(x(q)) = f'(x'(q))$,简记为$f(x) = f'(x')$。
另一方面,每一$x'^\nu$又是$n$个$x^\mu$的函数(坐标变换关系),简记为$x'^\nu = x'^\nu(x)$,故$f(x) = f'(x'(x))$。于是
$$
X_\mu(f) = \left.\frac{\partial f(x)}{\partial x^\mu}\right|_p \\
= \left.\frac{\partial f'(x'(x))}{\partial x^\mu}\right|_p \\
= \left(\frac{\partial f'(x')}{\partial x'^\nu}\frac{\partial x'^\nu}{\partial x^\mu}\right)_p \\
= \left.\frac{\partial x'^\nu}{\partial x^\mu}\right|_pX'_\nu(f), ~ \forall f \in \mathscr{F}_M
$$
上式表明映射$X_\mu$和$\partial x'^\nu/\partial x^\mu|_pX'_\nu$相等,即
$$X_\mu = \left.\frac{\partial x'^\nu}{\partial x^\mu}\right|_pX'_\nu$$
所以$v = v^\mu X_\mu = v'^\nu X'_\nu$可表为
$$v^\mu\left.\frac{\partial x'^\nu}{\partial x^\mu}\right|_pX'_\nu = v'^\nu X'_\nu$$
因$\{X'_\nu\}$中的$n$个基矢彼此线性独立,故得证。
\end{proof}

上述定理称为\textbf{矢量}(的分量)\textbf{变换式},许多书籍采用此式作为矢量的定义。

\begin{definition}
设$I$为$\mathbb{R}$的一个区间,则$C^r$类映射$C \colon I \to M$称为$M$上的一条$C^r$类的\textbf{曲线}。
今后如无声明,``曲线''均指光滑($C^\infty$类)曲线。对任一$t \in I$,有唯一的点$C(t) \in M$与之对应。
$t$称为曲线的\textbf{参数}。
\end{definition}

\begin{note}
此处的曲线与直观的曲线概念有密切联系,但也有差别。
直观的曲线往往是指上述映射$C \colon I \to M$的像,即$M$的子集$C[I]$,并且不提及参数。
上述定义的曲线则是指映射本身,是``带参数的曲线''。
设映射$C \colon I \to M$和$C' \colon I' \to M$的像重合,则直观上往往认为它们是同一曲线,但只要$C$和$C'$是不同映射,根据定义,它们就是不同曲线。
不过在大多数情况下可以说$C$和$C'$是``同一曲线''的两种参数化,准确地说,曲线$C' \colon I' \to M$称为曲线$C \colon I \to M$的\textbf{重参数化},若存在到上映射$\alpha \colon I \to I'$,满足
(a)$C = C' \comp \alpha$;(b)由$\alpha$诱导的函数$t' = \alpha(t)$有处处非零的导数。
解释:由$C = C' \comp \alpha$得$$C(t) = C'(\alpha(t)) = C'(t'), ~ \forall t \in I$$
映射$\alpha$的到上性保证$C'[I'] = C[I]$,即两曲线映射有相同的像。
\end{note}

\begin{note}
\textcircled{1}曲线$C$的像也常记作$C(t)$(而不是$C[I]$),以表明曲线的参数为$t$。
应注意,若$t$为某一具体值(``死的''),则$C(t)$只代表曲线像中的一点;
只当把$t$理解为``可跑遍$I$''时(``活的''),$C(t)$才代表曲线的像。
往往也把曲线的像简称为曲线。
\textcircled{2}设$(O, \psi)$是坐标系,$C[I] \in O$,则$\psi \comp C$是从$I \subset \mathbb{R}$到$\mathbb{R}^n$的映射,相当于$n$个一元函数$x^\mu = x^\mu(t), \mu = 1, \cdots, n$。
这$n$个等式称为曲线的\textbf{参数方程}或\textbf{参数表达式}或\textbf{参数式}。一个简单的例子是$\mathbb{R}^n$中以原点为心的单位圆周,其在自然坐标系中的参数式为$x^1 = \cos t, x^2 = \sin t$。
\end{note}

\begin{definition}
设$(O, \psi)$为坐标系,$x^\mu$为坐标,则$O$的子集
$$\{p \in O \mid x^2(p) = \text{常数}, \cdots, x^n(p) = \text{常数}\}$$
可以看成以$x^1$为参数的一条曲线(的像)(改变$x^2, \cdots, x^n$的常数值则得另一曲线),叫做\textbf{$x^1$坐标线}。
\textbf{$x^\mu$坐标线}可仿此定义。
\end{definition}

\begin{example}
在$2$维欧氏空间中,笛卡儿系$\{x, y\}$的$x$及$y$坐标线是互相正交的两组平行直线,
极坐标系$\{r, \varphi\}$的$\varphi$坐标线是以原点为心的无数同心圆,$r$坐标线是从原点出发的无数半直线。
\end{example}

直观的想法认为曲线上一点有无限多个彼此平行的切矢。但若把曲线定义为映射(``带参数的曲线''),则一条曲线的一点只有一个切矢。定义如下:

\begin{definition}
设$C(t)$是流形$M$上的$C^1$曲线,则线上$C(t_0)$点切于$C(t)$的切矢$T$是$C(t_0)$点的矢量,它对$f \in \mathscr{F}_M$的作用定义为:
$$T(f) := \left.\frac{\diff (f \comp C)}{\diff t}\right|_{t_0}, ~ \forall f \in \mathscr{F}_M$$
\end{definition}

\begin{note}
\textcircled{1}$f \colon M \to \mathbb{R}$是$M$上的函数(标量场),不是什么一元函数,
但与曲线$C \colon I \to M$的结合$f \comp C$便是以$t$为自变数的一元函数(也可记作$f(C(t))$)。
在不会混淆的情况下,$\diff (f \comp C) / \diff t$也可简写成$\diff f / \diff t$。
\textcircled{2}$C(t_0)$点切于$C(t)$的切矢$T$也常记作$\partial / \partial t|_{C(t_0)}$,于是上式也可写成
$$\left.\frac{\partial}{\partial t}\right|_{C(t_0)}(f) := \left.\frac{\diff (f \comp C)}{\diff t}\right|_{t_0} := \left.\frac{\diff f(C(t))}{\diff t}\right|_{t_0}, ~ \forall f \in \mathscr{F}_M$$
\end{note}

\begin{example}
$x^\mu$坐标线是以$x^\mu$为参数的曲线,又$p$点的坐标基矢$X_\mu$就是过$p$的$x^\mu$坐标线的切矢,故也常记作$\partial / \partial x^\mu|_p$,于是坐标基矢$X_\mu$对$f$的作用又可表为
$$\left.\frac{\partial}{\partial x^\mu}\right|_p(f) := \left.\frac{\partial f(x)}{\partial x^\mu}\right|_p, ~ \forall f \in \mathscr{F}_M$$
可见符号$\partial f / \partial x^\mu$既可理解为$\partial F(x^1, \cdots, x^n) / \partial x^\mu$,又可理解为坐标线的切矢$\partial / \partial x^\mu$对标量场$f$的作用。
\end{example}

\begin{theorem}
设曲线$C(t)$在某坐标系中的参数式为$x^\mu = x^\mu(t)$,则线上任一点的切矢$\partial / \partial t$在该坐标基底的展开式为
$$\frac{\partial}{\partial t} = \frac{\diff x^\mu(t)}{\diff t}\frac{\partial}{\partial x^\mu}$$
就是说,曲线$C(t)$的切矢$\partial / \partial t$的坐标分量是$C(t)$在该系的参数式$x^\mu(t)$对$t$的导数。
\end{theorem}

\begin{proof}[证明]
注意到$f(C(t_0)) = f(x^\mu(t_0)) \equiv f(p), ~ \forall f \in \mathscr{F}_M$,因此
$$\left.\frac{\partial}{\partial t}\right|_{C(t_0)}(f) = \left.\frac{\diff f(C(t))}{\diff t}\right|_{t_0}
= \left.\frac{\diff f(x^\mu(t))}{\diff t}\right|_{t_0} = \left.\frac{\partial f}{\partial x^\mu}\frac{\diff x^\mu(t)}{\diff t}\right|_{t_0}, ~ \forall f \in \mathscr{F}_M$$
对比等式的两边,可以发现$\partial / \partial t$和$\frac{\diff x^\mu(t)}{\diff t}\partial / \partial x^\mu$在同一点$p = C(t_0)$上对任一$f$的作用是相等的,故定理得证。
\end{proof}

\begin{definition}
非零矢量$u, v \in V_p$称为\textbf{互相平行}的,若存在$\alpha \in \mathbb{R}$使$v = \alpha u$。
\end{definition}

由定义可知曲线的切矢依赖于曲线的参数化,一条曲线$C(t)$的一点$C(t_0)$只有一个切于$C(t)$的切矢。直观上之所以认为曲线上一点有无数(互相平行的)切矢,
是因为把曲线理解为映射的像而不是映射本身(把无数个有相同像的曲线映射``简并化''为一条曲线)。
下面的定理表明,若两条曲线$C$和$C'$的像相同,则它们在任一像点的切矢互相平行。

\begin{theorem}
设曲线$C' \colon I' \to M$是$C \colon I \to M$的重参数化,则两者在任一像点的切矢$\partial / \partial t$和$\partial / \partial t'$有如下关系:
$$\frac{\partial}{\partial t} = \frac{\diff t'(t)}{\diff t}\frac{\partial}{\partial t'}$$
其中$t'(t)$是由映射$\alpha \colon I \to I'$诱导而得的一元函数,即$\alpha(t)$。
\end{theorem}

根据定义,$\forall p \in M$,若指定任一曲线$C(t)$使$p = C(t_0)$,则$V_p$中必有一元素可被视为该曲线在$C(t_0)$点的切矢。
现在问:指定$V_p$中任一元素$v$,可否找到过$p$的曲线,其在$p$点的切矢是$v$。答案是肯定的:这种曲线不但存在,而且很多。
例如,任选坐标系$\{x^\mu\}$使$p$含于其坐标域内,则以$x^\mu(t) = x^\mu|_p + v^\mu(t)$为参数式的曲线便是所需曲线,
其中$v^\mu$是$v$在该系的坐标分量。

综上所述,$V_p$中任一元素可视为过$p$的某曲线的切矢,因此$p$点的矢量亦称\textbf{切矢量},$V_p$则称为$p$点的\textbf{切空间}。

\subsection{流形上的矢量场}

\begin{definition}
设$A$为$M$的子集。若给$A$中每点指定一个矢量,就得到一个定义在$A$上的\textbf{矢量场}。
\end{definition}

\begin{example}
非自相交曲线$C(t)$上每点的切矢构成$C(t)$(看作$M$的子集)上的一个矢量场。
\end{example}

设$v$是$M$上的矢量场,$f$是$M$上的函数,则$v$在$M$的任一点$p$的值$v|_p$将按定义把$f$映射为一个实数$v|_p(f)$,
它在$p$点跑遍$M$时构成$M$上的一个函数$v(f)$。因此,矢量场$v$可视为把函数$f$变为函数$v(f)$的映射。

\begin{definition}
$M$上的矢量场$v$称为\textbf{$C^\infty$类(光滑)}的,若$v$作用于$C^\infty$类函数的结果仍为$C^\infty$类函数,
即$v(f) \in \mathscr{F}_M, ~ \forall f \in \mathscr{F}_M$。$v$称为\textbf{$C^r$类}的,若$v$作用于$C^\infty$类函数得$C^r$类函数。
\end{definition}

今后如无声明,``矢量场''均指光滑($C^\infty$)矢量场。

\begin{example}
\begin{enumerate}[(1)]
\item 坐标基矢$\{X_\mu \equiv \partial / \partial x^\mu\}$构成坐标域上的$n$个光滑矢量场,叫\textbf{坐标基矢场}。
\item $\mathbb{R}^3$中位于$q$的点电荷的静电场强$\vec{E}$是流形$M \equiv \mathbb{R}^3 - {q}$上的光滑矢量场。
\end{enumerate}
\end{example}

\begin{theorem}
$M$上的矢量场$v$是$C^\infty$(或$C^r$)类的充要条件是它在任一坐标基底的分量$v^\mu$为$C^\infty$(或$C^r$)类函数。
\end{theorem}

设$v$为$M$上的光滑矢量场,则$v(f) \in \mathscr{F}_M, ~ \forall f \in \mathscr{F}_M$。
若$u$为$M$上另一光滑矢量场,则$u(v(f)) \in \mathscr{F}_M$。但函数$u(v(f))$未必等于$v(u(f))$,于是有如下定义

\begin{definition}
两个光滑矢量场$u$和$v$的\textbf{对易子}是一个光滑矢量场$[u, v]$,定义为
$$[u, v](f) := u(v(f)) - v(u(f)), ~ \forall f \in \mathscr{F}_M$$
\end{definition}

\begin{note}
上式是对易子$[u, v]$(作为矢量场)的定义式,它在每点$p \in M$的值$[u, v]|_p$(作为$p$点的矢量,即从$\mathscr{F}_M$到$\mathbb{R}$的映射)的定义应理解为
$$[u, v]|_p(f) := u|_p(v(f)) - v|_p(u(f)), ~ \forall f \in \mathscr{F}_M$$
要确信上式定义的$[u, v]|_p$是$p$点的矢量,还应证明它满足矢量定义的两个条件。
\end{note}
