\chapter{黎曼(内禀)曲率张量}

\section{导数算符}

欧氏空间有熟知的导数算符$\vec{\nabla}$,它作用于函数(标量场)$f$得矢量场$\vec{\nabla}f$(梯度),作用于矢量场$\vec{v}$(再求缩并)得标量场$\vec{\nabla}\cdot\vec{v}$(散度)等。
由于存在欧氏度规$\delta_{ab}$,欧氏空间的矢量$v^a$与对偶矢量$v_a = \delta_{ab}v^b$自然认同。
现在要把$\vec{\nabla}$推广到任意流形,其上可以没有度规,所以要分清矢量和对偶矢量。
研究发现在推广时$\vec{\nabla}$更像对偶矢量,故应记作$\nabla_a$。
其实$\nabla$本身是算符,既非矢量也非对偶矢量,所谓把$\nabla$看作对偶矢量是指它作用于函数$f$的结果$\nabla_af$是对偶矢量。
推而广之,$\nabla$作用于任一$(k, l)$型张量场的结果是$(k, l + 1)$型张量场。
于是有如下定义:

\begin{definition}
以$\mathscr{F}_M(k, l)$代表流形$M$上全体$C^\infty$的$(k, l)$型张量场的集合(函数$f$可看作$(0, 0)$型张量场(标量场),故$\mathscr{F}_M(0, 0) \equiv \mathscr{F}_M$)。
映射$\nabla \colon \mathscr{F}_M(k, l) \to \mathscr{F}_M(k, l + 1)$称为$M$上的(无挠)导数算符\footnote{
$\mathscr{F}(k, l)$可放宽为全体$C^1$类$(k, l)$型张量场的集合。就是说,$\nabla_a$可作用于任一$C^1$类张量场。
},若它满足如下条件:
\begin{enumerate}[(a)]
\item 具有线性性:
\end{enumerate}
\end{definition}

\section{矢量场沿曲线的导数和平移}

\subsection{矢量场沿曲线的平移}

\subsection{与度规相适配的导数算符}

\subsection{矢量场沿曲线的导数与沿曲线的平移的关系}

\section{测地线}

\section{黎曼曲率张量}

\subsection{黎曼曲率的定义和性质}

\subsection{由度规计算黎曼曲率}

\section{内禀曲率和外曲率}

