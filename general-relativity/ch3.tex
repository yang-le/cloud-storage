\chapter{黎曼(内禀)曲率张量}

\section{导数算符}

欧氏空间有熟知的导数算符$\vec{\nabla}$,它作用于函数(标量场)$f$得矢量场$\vec{\nabla}f$(梯度),作用于矢量场$\vec{v}$(再求缩并)得标量场$\vec{\nabla}\cdot\vec{v}$(散度)等。
由于存在欧氏度规$\delta_{ab}$,欧氏空间的矢量$v^a$与对偶矢量$v_a = \delta_{ab}v^b$自然认同。
现在要把$\vec{\nabla}$推广到任意流形,其上可以没有度规,所以要分清矢量和对偶矢量。
研究发现在推广时$\vec{\nabla}$更像对偶矢量,故应记作$\nabla_a$。
其实$\nabla$本身是算符,既非矢量也非对偶矢量,所谓把$\nabla$看作对偶矢量是指它作用于函数$f$的结果$\nabla_af$是对偶矢量。
推而广之,$\nabla$作用于任一$(k, l)$型张量场的结果是$(k, l + 1)$型张量场。
于是有如下定义:

\begin{definition}
以$\mathscr{F}_M(k, l)$代表流形$M$上全体$C^\infty$的$(k, l)$型张量场的集合(函数$f$可看作$(0, 0)$型张量场(标量场),故$\mathscr{F}_M(0, 0) \equiv \mathscr{F}_M$)。
映射$\nabla \colon \mathscr{F}_M(k, l) \to \mathscr{F}_M(k, l + 1)$称为$M$上的(无挠)导数算符\footnote{
$\mathscr{F}(k, l)$可放宽为全体$C^1$类$(k, l)$型张量场的集合。就是说,$\nabla_a$可作用于任一$C^1$类张量场。
},若它满足如下条件:
\begin{enumerate}[(a)]
\item 具有线性性:
$$\begin{aligned}
& \nabla_a(\alpha T^{b_1 \cdots b_k}{}_{c_1 \cdots c_l} + \beta S^{b_1 \cdots b_k}{}_{c_1 \cdots c_l}) = \alpha\nabla_aT^{b_1 \cdots b_k}{}_{c_1 \cdots c_l} + \beta\nabla_aS^{b_1 \cdots b_k}{}_{c_1 \cdots c_l} \\
& \forall T^{b_1 \cdots b_k}{}_{c_1 \cdots c_l}, S^{b_1 \cdots b_k}{}_{c_1 \cdots c_l} \in \mathscr{F}_M(k, l), ~ \alpha, \beta \in \mathbb{R}
\end{aligned}$$
\item 满足莱布尼茨律:
$$\begin{aligned}
& \nabla_a(T^{b_1 \cdots b_k}{}_{c_1 \cdots c_l}S^{d_1 \cdots d_{k'}}{}_{e_1 \cdots e_{l'}}) = T^{b_1 \cdots b_k}{}_{c_1 \cdots c_l} \nabla_a S^{d_1 \cdots d_{k'}}{}_{e_1 \cdots e_{l'}} + S^{d_1 \cdots d_{k'}}{}_{e_1 \cdots e_{l'}} \nabla_a T^{b_1 \cdots b_k}{}_{c_1 \cdots c_l} \\
& \forall T^{b_1 \cdots b_k}{}_{c_1 \cdots c_l} \in \mathscr{F}_M(k, l), ~ S^{d_1 \cdots d_{k'}}{}_{e_1 \cdots e_{l'}} \in \mathscr{F}_M(k', l')
\end{aligned}$$
\item 与缩并可交换顺序
\item $v(f) = v^a\nabla_af, ~ \forall f \in \mathscr{F}_M, v \in \mathscr{F}_M(1, 0)$
\item 具有无挠性:$\nabla_a\nabla_bf = \nabla_b\nabla_af, ~ \forall f \in \mathscr{F}_M$
\end{enumerate}
\end{definition}

\begin{note}
\begin{enumerate}[(1)]
\item 条件(c)又可表为$\nabla \comp C = C \comp \nabla$,其中$C$代表缩并。今后将常写
$$\nabla_a(v^b\omega_b) = v^b\nabla_a\omega_b + \omega_b\nabla_av^b$$
一类的式子,这就要用到条件(c),因为上式的导出过程为
$$\begin{aligned}
\nabla_a(v^b\omega_b) & = \nabla_a[C(v^b\omega_c)] = C^1_2[\nabla_a(v^b\omega_c)] \\
& = C^1_2(v^b\nabla_a\omega_c) + C^1_2[(\nabla_av^b)\omega_c] = v^b\nabla_a\omega_b + \omega_b\nabla_av^b
\end{aligned}$$
\item 条件(d)左边的函数$v(f)$不宜记作$v^a(f)$,因$v^a(f)$易被误以为矢量场。
这是应写而不写抽象指标的少数例子之一。对条件(d)可用欧氏空间的$\vec{\nabla}$为例理解。
设$v^a$为欧氏空间中任一矢量场,其在笛卡尔系坐标基底的展开式为
$$v^a = v^1(\partial / \partial x)^a + v^2(\partial / \partial y)^a + v^3(\partial / \partial z)^a$$
则它对函数$f$的作用可表为
$$v(f) = v^1(\partial f / \partial x)^a + v^2(\partial f / \partial y)^a + v^3(\partial f / \partial z)^a = \vec{v}\cdot\vec{\nabla}f = v^a\nabla_af$$
可见条件(d)是这一性质对任意流形的推广。
\item 设$\nabla_a$是任一导数算符,则由条件(d)易证
$$\nabla_af = (\diff f)_a, ~ \forall f \in \mathscr{F}_M$$
其中,$(\diff f)_a$是函数$f$生成的对偶矢量场$\diff f$的抽象指标表示。
\item 条件(e)实质上是下式的抽象指标表述:
$$(\nabla\nabla f)(u, v) = (\nabla\nabla f)(v, u), ~ \forall u, v \in \mathscr{F}_M(1, 0)$$
亦即$\nabla\nabla f$是个对称的$(0, 2)$型张量。
\item 满足条件(a)~(d)而不满足条件(e)的导数算符叫有挠导数算符。广义相对论中只用无挠导数算符。
本书的$\nabla_a$在不加声明时一律代表无挠导数算符。
\end{enumerate}
\end{note}

\section{矢量场沿曲线的导数和平移}

\subsection{矢量场沿曲线的平移}

\subsection{与度规相适配的导数算符}

\subsection{矢量场沿曲线的导数与沿曲线的平移的关系}

\section{测地线}

\section{黎曼曲率张量}

\subsection{黎曼曲率的定义和性质}

\subsection{由度规计算黎曼曲率}

\section{内禀曲率和外曲率}

