\chapter{黎曼(内禀)曲率张量}

\section{导数算符}

欧氏空间有熟知的导数算符$\vec{\nabla}$,它作用于函数(标量场)$f$得矢量场$\vec{\nabla}f$(梯度),作用于矢量场$\vec{v}$(再求缩并)得标量场$\vec{\nabla}\cdot\vec{v}$(散度)等。
由于存在欧氏度规$\delta_{ab}$,欧氏空间的矢量$v^a$与对偶矢量$v_a = \delta_{ab}v^b$自然认同。
现在要把$\vec{\nabla}$推广到任意流形,其上可以没有度规,所以要分清矢量和对偶矢量。
研究发现在推广时$\vec{\nabla}$更像对偶矢量,故应记作$\nabla_a$。
其实$\nabla$本身是算符,既非矢量也非对偶矢量,所谓把$\nabla$看作对偶矢量是指它作用于函数$f$的结果$\nabla_af$是对偶矢量。
推而广之,$\nabla$作用于任一$(k, l)$型张量场的结果是$(k, l + 1)$型张量场。
于是有如下定义:

\begin{definition}
以$\mathscr{F}_M(k, l)$代表流形$M$上全体$C^\infty$的$(k, l)$型张量场的集合(函数$f$可看作$(0, 0)$型张量场(标量场),故$\mathscr{F}_M(0, 0) \equiv \mathscr{F}_M$)。
映射$\nabla \colon \mathscr{F}_M(k, l) \to \mathscr{F}_M(k, l + 1)$称为$M$上的(无挠)导数算符\footnote{
$\mathscr{F}(k, l)$可放宽为全体$C^1$类$(k, l)$型张量场的集合。就是说,$\nabla_a$可作用于任一$C^1$类张量场。
},若它满足如下条件:
\begin{enumerate}[(a)]
\item 具有线性性:
$$\begin{aligned}
& \nabla_a(\alpha T^{b_1 \cdots b_k}{}_{c_1 \cdots c_l} + \beta S^{b_1 \cdots b_k}{}_{c_1 \cdots c_l}) = \alpha\nabla_aT^{b_1 \cdots b_k}{}_{c_1 \cdots c_l} + \beta\nabla_aS^{b_1 \cdots b_k}{}_{c_1 \cdots c_l} \\
& \forall T^{b_1 \cdots b_k}{}_{c_1 \cdots c_l}, S^{b_1 \cdots b_k}{}_{c_1 \cdots c_l} \in \mathscr{F}_M(k, l), ~ \alpha, \beta \in \mathbb{R}
\end{aligned}$$
\item 满足莱布尼茨律:
$$\begin{aligned}
& \nabla_a(T^{b_1 \cdots b_k}{}_{c_1 \cdots c_l}S^{d_1 \cdots d_{k'}}{}_{e_1 \cdots e_{l'}}) = T^{b_1 \cdots b_k}{}_{c_1 \cdots c_l} \nabla_a S^{d_1 \cdots d_{k'}}{}_{e_1 \cdots e_{l'}} + S^{d_1 \cdots d_{k'}}{}_{e_1 \cdots e_{l'}} \nabla_a T^{b_1 \cdots b_k}{}_{c_1 \cdots c_l} \\
& \forall T^{b_1 \cdots b_k}{}_{c_1 \cdots c_l} \in \mathscr{F}_M(k, l), ~ S^{d_1 \cdots d_{k'}}{}_{e_1 \cdots e_{l'}} \in \mathscr{F}_M(k', l')
\end{aligned}$$
\item 与缩并可交换顺序
\item $v(f) = v^a\nabla_af, ~ \forall f \in \mathscr{F}_M, v \in \mathscr{F}_M(1, 0)$
\item 具有无挠性:$\nabla_a\nabla_bf = \nabla_b\nabla_af, ~ \forall f \in \mathscr{F}_M$
\end{enumerate}
\end{definition}

\begin{note}
\begin{enumerate}[(1)]
\item 条件(c)又可表为$\nabla \comp C = C \comp \nabla$,其中$C$代表缩并。今后将常写
$$\nabla_a(v^b\omega_b) = v^b\nabla_a\omega_b + \omega_b\nabla_av^b$$
一类的式子,这就要用到条件(c),因为上式的导出过程为
$$\begin{aligned}
\nabla_a(v^b\omega_b) & = \nabla_a[C(v^b\omega_c)] = C^1_2[\nabla_a(v^b\omega_c)] \\
& = C^1_2(v^b\nabla_a\omega_c) + C^1_2[(\nabla_av^b)\omega_c] = v^b\nabla_a\omega_b + \omega_b\nabla_av^b
\end{aligned}$$
\item 条件(d)左边的函数$v(f)$不宜记作$v^a(f)$,因$v^a(f)$易被误以为矢量场。
这是应写而不写抽象指标的少数例子之一。对条件(d)可用欧氏空间的$\vec{\nabla}$为例理解。
设$v^a$为欧氏空间中任一矢量场,其在笛卡尔系坐标基底的展开式为
$$v^a = v^1(\partial / \partial x)^a + v^2(\partial / \partial y)^a + v^3(\partial / \partial z)^a$$
则它对函数$f$的作用可表为
$$v(f) = v^1(\partial f / \partial x)^a + v^2(\partial f / \partial y)^a + v^3(\partial f / \partial z)^a = \vec{v}\cdot\vec{\nabla}f = v^a\nabla_af$$
可见条件(d)是这一性质对任意流形的推广。
\item 设$\nabla_a$是任一导数算符,则由条件(d)易证
$$\nabla_af = (\diff f)_a, ~ \forall f \in \mathscr{F}_M$$
其中,$(\diff f)_a$是函数$f$生成的对偶矢量场$\diff f$的抽象指标表示。
\item 条件(e)实质上是下式的抽象指标表述:
$$(\nabla\nabla f)(u, v) = (\nabla\nabla f)(v, u), ~ \forall u, v \in \mathscr{F}_M(1, 0)$$
亦即$\nabla\nabla f$是个对称的$(0, 2)$型张量。
\item 满足条件(a)~(d)而不满足条件(e)的导数算符叫有挠导数算符。广义相对论中只用无挠导数算符。
本书的$\nabla_a$在不加声明时一律代表无挠导数算符。
\end{enumerate}
\end{note}

任何流形必定存在满足上述定义的导数算符。事实上,导数算符不但存在,而且很多。下面讨论多到什么程度。
由$\nabla_af = (\diff f)_a$可知任意两个导数算符$\nabla_a$和$\tilde\nabla_a$作用于同一函数的结果相同,即
$$\nabla_af = \tilde\nabla_af = (\diff f)_a, ~ \forall f \in \mathscr{F}_M$$
可见$\nabla_a$与$\tilde\nabla_a$的不同只能体现在对非$(0, 0)$型张量场的作用上。先讨论$(0, 1)$型张量场(对偶矢量场)。
设在点$p \in M$给定一个对偶矢量$\mu_b \in V_p^*$,考虑$M$上的任意两个对偶矢量场$\omega_b, \omega'_b \in \mathscr{F}_M(0, 1)$,满足$\omega'_b|_p = \omega_b|_p = \mu_b$($\omega_b$和$\omega'_b$称为$\mu_b$在$M$上的两个延拓)。
设$\nabla_a$为导数算符,则$\nabla_a\omega'_b|_p$与$\nabla_a\omega_b|_p$一般并不相同。这类似于以下事实:
两个一元函数$f(x)$和$f'(x)$在$x_0$点取值相同($f'(x_0) = f(x_0)$)并不保证$(\diff f' /\diff x)|_{x_0} = (\diff f /\diff x)|_{x_0}$。
然而下面要证明,对$M$上任意两个导数算符$\nabla_a$和$\tilde\nabla_a$,只要$\omega'_b|_p = \omega_b|_p$就有
$$[(\tilde\nabla_a - \nabla_a)\omega'_b]_p = [(\tilde\nabla_a - \nabla_a)\omega_b]_p$$
其中$(\tilde\nabla_a - \nabla_a)\omega_b$是$\tilde\nabla_a\omega_b - \nabla_a\omega_b$的简写。

\begin{theorem}
设$p \in M, ~ \omega_b, \omega'_b \in \mathscr{F}(0, 1)$满足$\omega'_b|_p = \omega_b|_p$,则
$$[(\tilde\nabla_a - \nabla_a)\omega'_b]_p = [(\tilde\nabla_a - \nabla_a)\omega_b]_p$$
\end{theorem}

\begin{proof}
只须证明
$$[\nabla_a(\omega'_b - \omega_b)]_p = [\tilde\nabla_a(\omega'_b - \omega_b)]_p$$
设$\Omega_b \equiv \omega'_b - \omega_b$,选坐标系$\{x^\mu\}$使其坐标域含$p$,则$\omega'_b|_p = \omega_b|_p$导致$\Omega_\mu(p) = 0$,其中$\Omega_\mu$是$\Omega_b$的坐标分量。
于是对$p$点有
$$\begin{aligned}[]
[\nabla_a(\omega'_b - \omega_b)]_p & = [\nabla_a\Omega_b]_p = \{\nabla_a[\Omega_\mu(\diff x^\mu)_b]\}|_p \\
& = \Omega_\mu(p)[\nabla_a(\diff x^\mu)_b]_p + [(\diff x^\mu)_b\nabla_a\Omega_\mu]_p = [(\diff x^\mu)_b\nabla_a\Omega_\mu]_p
\end{aligned}$$
同理有$[\tilde\nabla_a(\omega'_b - \omega_b)]_p = [(\diff x^\mu)_b\tilde\nabla_a\Omega_\mu]_p$。又$[\nabla_a\Omega_\mu]_p = [\tilde\nabla_a\Omega_\mu]_p$,得证。
\end{proof}

虽然导数$[\nabla_a\omega_b]_p$和$[\tilde\nabla_a\omega_b]_p$依赖于$\omega_b$在$p$点的一个邻域内的值,然而上述定理表明$[(\tilde\nabla_a - \nabla_a)\omega_b]_p$只依赖于$\omega_b$在$p$点的值,这说明$(\tilde\nabla_a - \nabla_a)$是把$p$点的对偶矢量$\omega_b|_p$变为$p$点的$(0, 2)$型张量($[(\tilde\nabla_a - \nabla_a)\omega_b]_p$)的线性映射
(给定$p$点的任一对偶矢量$\mu_b$,任选对偶矢量场$\omega_b$使它在$p$点的值$\omega_b|_p = \mu_b$,则$[(\tilde\nabla_a - \nabla_a)\omega_b]_p$便是$\mu_b$在该映射下的像。)。
所以$(\tilde\nabla_a - \nabla_a)$在$p$点对应于一个$(1, 2)$型张量$C^c{}_{ab}$,满足
$$[(\tilde\nabla_a - \nabla_a)\omega_b]_p = C^c{}_{ab}\omega_c|_p$$

因为$p$点可任选,所以$M$上的两个导数算符$\nabla_a$和$\tilde\nabla_a$在对$\omega_b$的作用上的差别体现为$M$上的一个$(1, 2)$型张量场$C^c{}_{ab}$,即
\begin{theorem}
$\nabla_a\omega_b = \tilde\nabla_a\omega_b - C^c{}_{ab}\omega_c, ~ \forall \omega_b \in \mathscr{F}(0, 1)$
\end{theorem}

$\nabla_a$的无挠性导致张量场$C^c{}_{ab}$的如下对称性:
\begin{theorem}
$C^c{}_{ab} = C^c{}_{ba}$
\end{theorem}

\begin{proof}
令$\omega_b = \nabla_bf = \tilde\nabla_bf$,其中$f \in \mathscr{F}_M$,则上式给出$\nabla_a\nabla_bf = \tilde\nabla_a\tilde\nabla_bf - C^c{}_{ab}\nabla_cf$。
交换指标$a$,$b$得$\nabla_b\nabla_af = \tilde\nabla_b\tilde\nabla_af - C^c{}_{ba}\nabla_cf$。
两式相减,注意到无挠性条件,便有$C^c{}_{ab}\nabla_cf = C^c{}_{ba}\nabla_cf$。
令$T^c{}_{ab} \equiv C^c{}_{ab} - C^c{}_{ba}$,则对所有的$f \in \mathscr{F}_M$有$T^c{}_{ab}\nabla_cf = 0$,于是$T^c{}_{ab}$在任一坐标基底的分量$T^\sigma{}_{\mu\nu} = T^c{}_{ab}(\diff x^\sigma)_c(\partial / \partial x^\mu)_a(\partial / \partial x^\nu)_b = 0$(其中第二步是因为$T^c{}_{ab}(\diff x^\sigma)_c = T^c{}_{ab}\nabla_cx^\sigma = 0$(把$x^\sigma$看作$f$)),因而$T^c{}_{ab} = 0$。
\end{proof}

\begin{theorem}
$\nabla_av^b = \tilde\nabla_av^b + C^b{}_{ac}v^c, ~ \forall v^b \in \mathscr{F}_M(1, 0)$
\end{theorem}

\begin{proof}
设$\omega_b$为$M$上任一对偶矢量场,则
$$\nabla_a(\omega_bv^b) = \omega_b\nabla_av^b + v^b\nabla_a\omega_b = \omega_b\nabla_av^b + v^b(\tilde\nabla_a\omega_b - C^c{}_{ab}\omega_c)$$
另一方面,$\tilde\nabla_a(\omega_bv^b) = \omega_b\tilde\nabla_av^b + v^b\tilde\nabla_a\omega_b$。
而$\omega_bv^b$为标量场,故$\nabla_a(\omega_bv^b) = \tilde\nabla_a(\omega_bv^b)$,故以上两式右边相等,因而得
$$\omega_b\nabla_av^b = \omega_b\tilde\nabla_av^b + C^c{}_{ab}v^b\omega_c = \omega_b\tilde\nabla_av^b + C^b{}_{ac}v^c\omega_b, ~ \forall \omega_b \in \mathscr{F}_M(0, 1)$$
于是有定理结论。
\end{proof}

用类似方法可以证明$\nabla_a$与$\tilde\nabla_a$作用于任一$(k, l)$型张量场$T^{a_1 \cdots a_k}{}_{b_1 \cdots b_l}$所得结果之差$\nabla_aT^{a_1 \cdots a_k}{}_{b_1 \cdots b_l} - \tilde\nabla_aT^{a_1 \cdots a_k}{}_{b_1 \cdots b_l}$可表为$k + l$项,每项都含$C^c{}_{ab}$,与$T$的某一上指标缩并的$k$项前面为$+$号,与$T$的某一下指标缩并的$l$项前面为$-$号,例如
$$\nabla_aT^b{}_{c} = \tilde\nabla_aT^b{}_{c} + C^b{}_{ad}T^d{}_{c} - C^d{}_{ac}T^b{}_{d}$$
一般形式见下面的定理:
\begin{theorem}
$$\nabla_aT^{b_1 \cdots b_k}{}_{c_1 \cdots c_l} = \tilde\nabla_aT^{b_1 \cdots b_k}{}_{c_1 \cdots c_l} + \sum_iC^{b_i}{}_{ad}T^{b_1 \cdots d \cdots b_k}{}_{c_1 \cdots c_l} - \sum_jC^d{}_{ac_j}T^{b_1 \cdots b_k}{}_{c_1 \cdots d \cdots c_l}, ~ \forall T \in \mathscr{F}_M(k, l)$$
\end{theorem}

上述定理表明任意两个导数算符的差别仅体现在一个张量场$C^c{}_{ab}$上。
反之也不难验证,任给一个导数算符$\tilde\nabla_a$和一个下标对称的光滑张量场$C^c{}_{ab}$,由上式定义的$\nabla_a$必满足导数算符的全部条件,因而也是一个导数算符。
可见流形上只要有一个导数算符就会有许多导数算符。
选定导数算符$\nabla_a$后的流形$M$可记作$(M, \nabla_a)$,它比$M$本身有更多结构($\nabla_a$提供附加结构),例如可谈及矢量沿曲线的平移及$(M, \nabla_a)$的曲率。

设$\{x^\mu\}$是$M$的一个坐标系,其坐标基底和对偶坐标基底分别为$\{(\partial / \partial x^\mu)^a\}$和$\{(\diff x^\mu)_a\}$。
在坐标域$O$上定义映射$\partial_a \colon \mathscr{F}_O(k, l) \to \mathscr{F}_O(k, l + 1)$如下(仅以$T^b{}_c \in \mathscr{F}_O(1, 1)$为例写出):
$$\partial_aT^b{}_c := (\diff x^\mu)_a(\partial / \partial x^\nu)^b(\diff x^\sigma)_c\partial_\mu T^\nu{}_\sigma$$
其中$T^\nu{}_\sigma$是$T^b{}_c$在该坐标系的分量,$\partial_\mu$是对坐标$x^\mu$求偏导数的符号$\partial / \partial x^\mu$的简写。
不难验证$\partial_a$满足导数算符定义的$5$个条件,可见$\partial_a$是$O$上的一个导数算符。
这是一个从定义起就依赖于坐标系的导数算符,而且只在该坐标系的坐标域上有定义,称为该坐标系的普通导数算符。
上式表明$\partial_\mu T^\nu{}_\sigma$是张量场$\partial_aT^b{}_c$在该坐标系的分量,所以$\partial_a$的定义亦可表为:
张量场$T^{b_1 \cdots b_k}{}_{c_1 \cdots c_l}$的普通导数$\partial_aT^{b_1 \cdots b_k}{}_{c_1 \cdots c_l}$的坐标分量等于该张量场的坐标分量对坐标的偏导数$\partial(T^{\nu_1 \cdots \nu_k}{}_{\sigma_1 \cdots \sigma_l}) / \partial x^\mu$。
由此易见:

\begin{enumerate}[(1)]
\item 任一坐标系的$\partial_a$作用于该系的任一坐标基矢和任一对偶坐标基矢结果为零\footnote{
坐标基矢场的坐标分量为常函数,其偏导数为零。
},即
$$\partial_a(\partial / \partial x^\nu)^b = 0, ~ \partial_a(\diff x^\nu)_b = 0$$
\item $\partial_a$满足比导数算符的无挠性条件强得多的条件,即
$$\partial_a\partial_bT^{\cdots}{}_{\cdots} = \partial_b\partial_aT^{\cdots}{}_{\cdots}, ~ \text{或} \partial_{[a}\partial_{b]}T^{\cdots}{}_{\cdots} = 0$$
其中$T^{\cdots}{}_{\cdots}$是任意型张量场。
\end{enumerate}

$\partial_a$虽可看作$\nabla_a$的特例,但其定义依赖于坐标系。
我们把与坐标系(或其他人为因素)无关的那些$\nabla_a$称为协变导数算符,$\partial_a$不在此列。

\section{矢量场沿曲线的导数和平移}

\subsection{矢量场沿曲线的平移}

\subsection{与度规相适配的导数算符}

\subsection{矢量场沿曲线的导数与沿曲线的平移的关系}

\section{测地线}

\section{黎曼曲率张量}

\subsection{黎曼曲率的定义和性质}

\subsection{由度规计算黎曼曲率}

\section{内禀曲率和外曲率}

